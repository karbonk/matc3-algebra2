% (c)~2014 Claudio Carboncini - claudio.carboncini@gmail.com
% (c)~2014 Dimitrios Vrettos - d.vrettos@gmail.com
\section{Esercizi}
\subsection{Esercizi dei singoli paragrafi}
\subsection*{8.1 - Equazioni irrazionali con un solo radicale}
Risolvi le seguenti equazioni irrazionali con un radicale.
\begin{esercizio}[\Ast]
 \label{ese:8.1}
Risolvi le seguenti equazioni irrazionali con un radicale.
\begin{multicols}{2}
 \begin{enumeratea}
 \item~$\sqrt{2x+1}=7$;
 \item~$\sqrt{4-x^2}=1$;
 \item~$\sqrt[4]{2x+1}=2$;
 \item~$\sqrt[3]{2x+1}=-2$.
 \end{enumeratea}
 \end{multicols}
\end{esercizio}

\begin{esercizio}
 \label{ese:8.2}
Risolvi le seguenti equazioni irrazionali con un radicale.
\begin{multicols}{2}
 \begin{enumeratea}
 \item~$\sqrt[3]{x+1}=-1$;
 \item~$\sqrt[3]{x^2-6x}=3$;
 \item~$\sqrt{5x-2}=-4$;
 \item~$\sqrt{2-x^2+x}=6$.
 \end{enumeratea}
 \end{multicols}
\end{esercizio}

\begin{esercizio}[\Ast]
 \label{ese:8.3}
Risolvi le seguenti equazioni irrazionali con un radicale.
\begin{multicols}{2}
 \begin{enumeratea}
 \item~$\sqrt{2x^2+9}=2$;
 \item~$\sqrt[3]{16x-64}=x-4$;
 \item~$\sqrt{3x+10}=1-\frac 3 2x$;
 \item~$\sqrt[3]{3x+10}=1-\frac 3 2x$.
 \end{enumeratea}
 \end{multicols}
\end{esercizio}

\begin{esercizio}[\Ast]
 \label{ese:8.4}
Risolvi le seguenti equazioni irrazionali con un radicale.
\begin{multicols}{2}
 \begin{enumeratea}
 \item~$-3=\sqrt{\frac 2{x+1}}$;
 \item~$x-\sqrt{x+2}=0$;
 \item~$\sqrt[3]{3x+1-3x^2}=x$;
 \item~$\sqrt{25-x^2}+x=7$.
 \end{enumeratea}
 \end{multicols}
\end{esercizio}

\begin{esercizio}[\Ast]
 \label{ese:8.5}
Risolvi le seguenti equazioni irrazionali con un radicale.
\begin{multicols}{2}
 \begin{enumeratea}
 \item~$\sqrt{x^2+1}-3+x^2=(x-2)^2$;
 \item~$\sqrt{\frac{x-1}{3-x}}=\frac 1{x-3}$;
 \item~$\sqrt{\frac{2+3x} x}=\frac 1 x+1$;
 \item~$\sqrt{3x^2+10}=3x$.
 \end{enumeratea}
 \end{multicols}
\end{esercizio}

\begin{esercizio}[\Ast]
 \label{ese:8.6}
Risolvi le seguenti equazioni irrazionali con un radicale.
\begin{multicols}{2}
 \begin{enumeratea}
 \item~$\sqrt[3]{\frac{2x^2-1} x}-x=0$;
 \item~$\sqrt[2]{\frac{2x^2-1} x}-x=0$;
 \item~$\frac{4x+1}{2x}=\sqrt{\frac 2{x+1}}$;
 \item~$\sqrt{\frac{x-1}{3-x}}=\frac 1{x-3}$.
 \end{enumeratea}
 \end{multicols}
\end{esercizio}

\begin{esercizio}[\Ast]
 \label{ese:8.7}
Risolvi le seguenti equazioni irrazionali con un radicale.
\begin{multicols}{2}
 \begin{enumeratea}
 \item~$x\sqrt{x^2-4}=x^2-4$;
 \item~$\sqrt[3]{2x^2-7x+5}=1-x$;
 \item~$\frac{\sqrt{2x-3}}{x-2}=2$;
 \item~$\sqrt{4x-1}=\sqrt 2+\sqrt 3$.
 \end{enumeratea}
 \end{multicols}
\end{esercizio}
\pagebreak
\begin{esercizio}[\Ast]
 \label{ese:8.8}
Risolvi le seguenti equazioni irrazionali con un radicale.
\begin{multicols}{2}
 \begin{enumeratea}
 \item~$\sqrt{x+2}=x-1$;
 \item~$2\sqrt{x^2-4x-33}-x=15$;
 \item~$(3-x)^2-\sqrt{x-2x^2+5}=(x-3)(x-2)$;
 \item~$4x+\frac 1 2\sqrt{25-x^2}=\frac 7 2(x+1)$.
 \end{enumeratea}
 \end{multicols}
\end{esercizio}

\begin{esercizio}[\Ast]
 \label{ese:8.9}
Risolvi le seguenti equazioni irrazionali con un radicale.
\begin{multicols}{2}
 \begin{enumeratea}
 \item~$\frac 1 3\sqrt{5x^2+4x-8}+x=2\left(x-\frac 1 3\right)$;
 \item~$1-x+\sqrt{8x^2-21x+34}=-3+2x$;
 \item~$\sqrt[4]{\frac{(x-1)^3} x}\cdot \sqrt[4]{\frac{x^3}{x-1}}=x+\frac 3 2$.
 \end{enumeratea}
 \end{multicols}
\end{esercizio}

\subsection*{8.2 - Equazioni irrazionali con più radicali}

\begin{esercizio}[\Ast]
 \label{ese:8.10}
Risolvi le seguenti equazioni irrazionali con più radicali.
\begin{multicols}{2}
 \begin{enumeratea}
 \item~$\sqrt{3x-5}=\sqrt{1-x}$;
 \item~$\sqrt{3x-2}=\sqrt{2x-3}$;
 \item~$\sqrt{x-2}=1-\sqrt{3-x}$;
 \item~$\sqrt{6-3x}=2+\sqrt{x+1}$.
 \end{enumeratea}
 \end{multicols}
\end{esercizio}

\begin{esercizio}[\Ast]
 \label{ese:8.11}
Risolvi le seguenti equazioni irrazionali con più radicali.
\begin{multicols}{2}
 \begin{enumeratea}
 \item~$\sqrt{4-3x}=\sqrt{x^2-x-1}$;
 \item~$\sqrt{3-2x}=-\sqrt{x^2+3}$;
 \item~$\sqrt{3-x}=\sqrt{x+1}-1$;
 \item~$2\sqrt{x-1}=\sqrt{1+2x}+1$.
 \end{enumeratea}
 \end{multicols}
\end{esercizio}

\begin{esercizio}[\Ast]
 \label{ese:8.12}
Risolvi le seguenti equazioni irrazionali con più radicali.
\begin{multicols}{2}
 \begin{enumeratea}
 \item~$3\sqrt{x-x^2}=2\sqrt{x-1}$;
 \item~$\sqrt{x+1}=3\sqrt{4-x}$;
 \item~$\sqrt{x^2-x-3}=2\sqrt{x+5}$;
 \item~$1+2\sqrt{1-\frac 2 3x}=\sqrt{2x+1}$.
 \end{enumeratea}
 \end{multicols}
\end{esercizio}

\begin{esercizio}[\Ast]
 \label{ese:8.13}
Risolvi le seguenti equazioni irrazionali con più radicali.
\begin{multicols}{2}
 \begin{enumeratea}
 \item~$4-\sqrt{x-2}=\sqrt{x-1}+3$;
 \item~$2-2\sqrt{2-x}=4\sqrt{1-x}$;
 \item~$\sqrt{x^3-2x^2}=3\sqrt{x^2-2x}$;
 \item~$3\sqrt{x^4-x^3}=4\sqrt{x^4+2x^3}$.
 \end{enumeratea}
 \end{multicols}
\end{esercizio}

\begin{esercizio}[\Ast]
 \label{ese:8.14}
Risolvi le seguenti equazioni irrazionali con più radicali.
\begin{multicols}{2}
 \begin{enumeratea}
 \item~$\sqrt{2x^2-4x-3}=\sqrt{x^2-1}$;
 \item~$\sqrt{x^2+8}=\sqrt{4-x^2}$;
 \item~$\sqrt{x+12}-1=\sqrt{1-x}$;
 \item~$\sqrt{2x-5}=3-\sqrt{x+1}$.
 \end{enumeratea}
 \end{multicols}
\end{esercizio}

\begin{esercizio}[\Ast]
 \label{ese:8.15}
Risolvi le seguenti equazioni irrazionali con più radicali.
\begin{multicols}{2}
 \begin{enumeratea}
 \item~$\sqrt{x^2-2x+3}=\sqrt{1-x^2+2x}$;
 \item~$\sqrt{4x-7}+\sqrt{7x-4x^2}=0$;
 \item~$\sqrt{x^2+6x+9}+2\sqrt{1-x}=0$;
 \item~$\sqrt{2x+1}+\sqrt{x-2}=4$.
 \end{enumeratea}
 \end{multicols}
\end{esercizio}

\begin{esercizio}[\Ast]
 \label{ese:8.16}
Risolvi le seguenti equazioni irrazionali con più radicali.
\begin{multicols}{2}
 \begin{enumeratea}
 \item~$\sqrt{x^2-4}=3+2\sqrt{1-x^2}$;
 \item~$\sqrt{4+x^2}=1+\sqrt{x^2-1}$;
 \item~$\sqrt{2x+1}=3+2\sqrt{x-6}$;
 \item~$\sqrt{x-2}-\sqrt{2x-1}=\sqrt{5x-1}$.
 \end{enumeratea}
 \end{multicols}
\end{esercizio}
\pagebreak
\begin{esercizio}[\Ast]
 \label{ese:8.17}
Risolvi le seguenti equazioni irrazionali con più radicali.
\begin{multicols}{2}
 \begin{enumeratea}
 \item~$\sqrt{4-x}+2\sqrt{x}=0$;
 \item~$\sqrt{16+x}-\sqrt{x}=\sqrt{x-5}$;
 \item~$5-\sqrt[3]{2(x-1)}=\sqrt[3]{37-2x}$;
 \item~$\sqrt{x-1}=\sqrt{x+1-\sqrt{x-1}}$.
 \end{enumeratea}
 \end{multicols}
\end{esercizio}

\begin{esercizio}[\Ast]
 \label{ese:8.18}
Risolvi le seguenti equazioni irrazionali con più radicali.
\begin{multicols}{2}
 \begin{enumeratea}
 \item~$\sqrt{\frac{(3-2x)}{(x-1)}}+\sqrt{2x-1}=\sqrt{\frac 1{x-1}}$;
 \item~$\sqrt{(x-1)^2+\sqrt{5x-6}}=2-x$;
 \item~$\frac{\sqrt{x-1}}{\sqrt{x+1}}=\sqrt{x-1}$;
 \item~$\sqrt{\frac x{4-x}}+3\sqrt{\frac{4-x} x}-4=0$;
 \item~$\frac 5{6-\sqrt x}+\frac 1 2=\frac 6{5-\sqrt x}$.
 \end{enumeratea}
 \end{multicols}
\end{esercizio}

\subsection*{8.3 - Disequazioni irrazionali}

\begin{esercizio}[\Ast]
 \label{ese:8.19}
Risolvi le seguenti disequazioni irrazionali.
\begin{multicols}{2}
 \begin{enumeratea}
 \item~$\sqrt{4x^2-x}\ge -2x+1$;
 \item~$\sqrt{x^2-x-2}\le 2x+6$;
 \item~$\sqrt{2x-1}\ge x-8$;
 \item~$\sqrt{x^2-3x+2}<7-5x$.
 \end{enumeratea}
 \end{multicols}
\end{esercizio}

\begin{esercizio}[\Ast]
 \label{ese:8.20}
Risolvi le seguenti disequazioni irrazionali.
\begin{multicols}{2}
 \begin{enumeratea}
 \item~$\sqrt{9x^2+2x}\ge 3x-4$;
 \item~$\sqrt{x^2-2x}\ge 5-x$;
 \item~$\sqrt{16-2x^2}<x+4$;
 \item~$\sqrt{3-2x}\le \sqrt{1-x^2}$.
 \end{enumeratea}
 \end{multicols}
\end{esercizio}

\begin{esercizio}[\Ast]
 \label{ese:8.21}
Risolvi le seguenti disequazioni irrazionali.
\begin{multicols}{2}
 \begin{enumeratea}
 \item~$\sqrt{4x^2+2x}\ge 2x-3$;
 \item~$\sqrt{1-x^2}>2x-1$;
 \item~$\sqrt{2x^2-x-1}>\sqrt{x-3}$;
 \item~$\sqrt{x^2+2x+1}\le \sqrt{1-x^2}$.
 \end{enumeratea}
 \end{multicols}
\end{esercizio}

\begin{esercizio}[\Ast]
 \label{ese:8.22}
Risolvi le seguenti disequazioni irrazionali.
\begin{multicols}{2}
 \begin{enumeratea}
 \item~$\sqrt{16x^2+2x}\ge -4x-1$;
 \item~$\sqrt{x^2-1}<x+3$;
 \item~$\sqrt{x^2-3x+2}\le \sqrt{3x^2-2x-1}$;
 \item~$\sqrt{9x^2-x}\ge -3x+6$.
 \end{enumeratea}
 \end{multicols}
\end{esercizio}

\begin{esercizio}[\Ast]
 \label{ese:8.23}
Risolvi le seguenti disequazioni irrazionali.
\begin{multicols}{2}
 \begin{enumeratea}
 \item~$\sqrt{x^2+1}\le \frac{x+\sqrt 3} 2$;
 \item~$\sqrt{x^2-5x}\ge x-4$;
 \item~$\sqrt{x^2+1}\le \frac 1 2x+1$;
 \item~$\sqrt{10x-x^2}>\sqrt{2x^2-32}$.
 \end{enumeratea}
 \end{multicols}
\end{esercizio}

\begin{esercizio}[\Ast]
 \label{ese:8.24}
Risolvi le seguenti disequazioni irrazionali.
\begin{multicols}{2}
 \begin{enumeratea}
 \item~$\sqrt{x^2+x}\ge x+3$;
 \item~$\sqrt{x^2+1}\le \frac 1 2x-1$;
 \item~$\sqrt{x^2+1}\le \sqrt{1-x^2}$;
 \item~$\sqrt{2x^2+x}\ge \sqrt{4-x^2}$.
 \end{enumeratea}
 \end{multicols}
\end{esercizio}
\pagebreak
\begin{esercizio}[\Ast]
 \label{ese:8.25}
Risolvi le seguenti disequazioni irrazionali.
\begin{multicols}{2}
 \begin{enumeratea}
 \item~$3\sqrt{3x+x^2}<2\sqrt{2x-x^3}$;
 \item~$2\sqrt{x+x^3}>\sqrt{2x^3-3x^2}$;
 \item~$\sqrt{x^5-x^3}<2\sqrt{x^4+2x^3}$.
 \end{enumeratea}
 \end{multicols}
\end{esercizio}

\begin{esercizio}[\Ast]
 \label{ese:8.26}
Risolvi le seguenti disequazioni irrazionali.
\begin{multicols}{2}
 \begin{enumeratea}
 \item~$\sqrt{3-2x}\le \sqrt{1+x^{2}}$;
 \item~$\sqrt{\frac{x^2-1}{2+x}}< \sqrt{x}$;
 \item~$\frac{1}{\sqrt{4-x^{2}}}\ge \frac{1}{x}$;
 \item~$\sqrt{\left|\frac{x+1}{x-1}\right|}> 1$.
 \end{enumeratea}
 \end{multicols}
\end{esercizio}

\subsection{Risposte}
%\begin{multicols}{2}
\paragraph{8.1.} a)~$x=24$,\quad b)~$x_{1\text{,}2}=\pm \sqrt 3$,\quad c)~$x=\frac{15} 2$,\quad d)~$x=-\frac 9 2$.

\paragraph{8.2.} a)~$x=-2$,\quad b)~$x_1=-3\vee x_2=9$,\quad c)~$\emptyset $,\quad d)~$\emptyset$.

\paragraph{8.3.} a)~$\emptyset$,\quad b)~$x_1=4\vee x_2=8$,\quad c)~$x=\frac{4-2\sqrt{13}} 3$,\quad d)~$x=-\frac 2 3$.

\paragraph{8.4.} a)~$\emptyset$,\quad b)~$x=2$,\quad c)~ $x=1$,\quad d)~$x_1=3\vee x_2=4$.

\paragraph{8.5.} a)~$x_1=\frac 4 3$,\quad b)~$\emptyset$,\quad c)~$x=\frac{\sqrt 2} 2$,\quad d)~$x=\frac{\sqrt{15}} 3$.

\paragraph{8.6.} a)~$x_1=-1\vee x_2=1$,\quad b)~$x_1=1\vee x_2=\frac{\sqrt 5+1} 2$,\quad c)~$\emptyset$,\quad d)~$\emptyset$.

\paragraph{8.7.} a)~$x_{1\text{,}2}=\mp 2$,\quad b)~$x_1=1\vee x_2=-2\vee x_3=2$,\quad c)~$x=\frac{9+\sqrt 5} 4$,\quad d)~$x=\frac{3+\sqrt 6} 2$.

\paragraph{8.8.} a)~$x=\frac{3+\sqrt{13}} 2$,\quad b)~$x_1=21\vee x_2=-\frac{17} 3$,\quad c)~$x_1=1\vee x_2=\frac 4 3$,\quad d)~$x_1=3\vee x_2=4$.

\paragraph{8.9.} a)~$x_1=1\vee x_2=3$,\quad b)~$x=6$,\quad c)~$x=-\frac 9{16}$.

\paragraph{8.10.} a)~$x=\frac 3 2$ non acc,\quad b)~$x=-1$ non acc,\quad c)~$x_1=2\vee x_2=3$,\quad d)~$x=\frac{3-2\sqrt 6} 4$.

\paragraph{8.11.} a)~$x=-1-\sqrt 6$,\quad b)~$\emptyset $,\quad c)~$x=\frac{2+\sqrt 7} 2$,\quad d)~$x=4+2\sqrt 2$.

\paragraph{8.12.} a)~$x=1$,\quad b)~$x=\frac 7 2$,\quad c)~$x_{1\text{,}2}=\frac{5\pm 3\sqrt{13}} 2$,\quad d)~$x=\frac{60}{49}$.

\paragraph{8.13.} a)~$x=2$,\quad b)~$x=1$,\quad c)~$x_1=0\vee x_2=2\vee x_3=9$,\quad d)~$x_1=0\vee x_2=-\frac{41} 7$.

\paragraph{8.14.} a)~$x=2+\sqrt 6$,\quad b)~$\emptyset $,\quad c)~$x=-3$,\quad d)~$x=3$.

\paragraph{8.15.} a)~$x=1$,\quad b)~$x=\frac{7}{4}$,\quad c)~$\emptyset$,\quad d)~$x=45-24\sqrt{3}$.

\paragraph{8.16.} a)~$\emptyset $,\quad b)~$x_{1\text{,}2}=\pm \sqrt 5$,\quad c)~$x=26-6\sqrt{11}$,\quad d)~$x=\frac{7-3\sqrt 5} 2$.

\paragraph{8.17.} a)~$\emptyset$,\quad b)~$x=9$,\quad c)~$x_1=5\vee x_2=\frac{29}{2}$,\quad d)~$x=5$.

\paragraph{8.18.} a)~$x=\frac 3 2$,\quad b)~$x=\frac 5 4$,\quad c)~$x=1$,\quad d)~$x_1=2\vee x_2=\frac{18} 5$,\quad e)~$x_1=1\vee x_2=64$.

\paragraph{8.19.} a)~$x\ge \frac 1 3$,\quad b)~$-2\le x\le -1\vee x\ge 2$,\quad c)~$\frac 1 2\le x\le 13$,\quad d)~$x\le 1$.

\paragraph{8.20.} a)~$x\le -\frac 2 9\vee x\ge 0$,\quad b)~$x\ge \frac{25} 8$,\quad c)~$-2\sqrt 2\le x<-\frac 8 3\vee 0<x\le 2\sqrt 2$,\quad d)~$ \emptyset $.

\paragraph{8.21.} a)~$x\le -\frac 1 2\vee x\ge 0$,\quad b)~$-1\le x<\frac 4 5$,\quad c)~$x\ge 3$,\quad d)~$-1\le x\le 0$.

\paragraph{8.22.} a)~$x\le -\frac 1 8\vee x\ge 0$,\quad b)~$-\frac 5 3<x\le -1\vee x\ge 1$,\quad c)~$x\le -\frac 3 2\vee x=1\vee x\ge 2$,\quad d)~$x\ge \frac{36}{35}$.

\paragraph{8.23.} a)~$x=\frac{\sqrt 3} 3$,\quad b)~$x\le 0\vee x\ge \frac{16} 3$,\quad c)~$0\le x\le \frac 4 3$,\quad d)~$4\le x<\frac{16} 3$.

\paragraph{8.24.} a)~$x\le -\frac 9 5$,\quad b)~$ \emptyset $,\quad c)~$x=0$,\quad d)~$-2\le x\le -\frac 4 3\vee 1\le x\le 2$.

\paragraph{8.25.} a)~$x\le -3$,\quad b)~$x\ge \frac 3 2$,\quad c)~$1\le x<2+\sqrt{13}$.

\paragraph{8.26.} a)~$x\le-1-\sqrt{3}\vee \sqrt{3}-1\le x\le \frac{3}{2}$,\quad b)~$x\ge 1$,\quad c)~$-2<x<0\vee \sqrt{2}\le x<2$,\quad d)~$x>0 \wedge x\neq 1$.
%\end{multicols}