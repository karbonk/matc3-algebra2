% (c)~2014 Claudio Carboncini - claudio.carboncini@gmail.com
% (c)~2014 Dimitrios Vrettos - d.vrettos@gmail.com
\section{Esercizi}
\subsection{Esercizi dei singoli paragrafi}
\subsubsection*{2.1 - Radici}

% % Radici quadrate, cubiche, n-esime

\begin{esercizio}
 \label{ese:2.1}
Determina le seguenti radici quadrate razionali (quando è possibile calcolarle).
\begin{multicols}{3}
 \begin{enumeratea}
 \item~$\sqrt 9$;
 \item~$\sqrt{36}$;
 \item~$\sqrt{-49}$;
 \item~$\sqrt{64}$;
 \item~$\sqrt{-81}$;
 \item~$\sqrt{\frac{16}{25}}$;
 \item~$\sqrt{\frac{49}{81}}$;
 \item~$\sqrt{\frac{121}{100}}$;
 \item~$\sqrt{\frac{144}{36}}$;
 \item~$\sqrt{\frac{-1} 4}$;
 \item~$\sqrt{\np{0,04}}$;
 \item~$\sqrt{\np{0,09}}$;
 \item~$\sqrt{\np{0,0001}}$;
 \item~$\sqrt{\frac{144} 9}$;
 \item~$\sqrt{\np{0,16}}$.
 \end{enumeratea}
 \end{multicols}
\end{esercizio}

\begin{esercizio}
 \label{ese:2.2}
Determina le seguenti radici quadrate razionali (quando è possibile calcolarle).
\begin{multicols}{2}
 \begin{enumeratea}
 \item~$\sqrt{\np{-0,09}}$;
 \item~$\sqrt{25\cdot 16}$;
 \item~$\sqrt{36\cdot 49}$;
 \item~$\sqrt{\np{0,04}\cdot \np{0,0121}}$;
 \item~$\sqrt{\frac 1{100}}$;
 \item~$\sqrt{13+\sqrt{7+\sqrt{1+\sqrt{6+\sqrt 9}}}}$;
 \item~$\sqrt{5+\sqrt{14+\sqrt{2+\sqrt 4}}}$.
 \end{enumeratea}
 \end{multicols}
\end{esercizio}

\begin{esercizio}
 \label{ese:2.3}
Senza usare la calcolatrice determina per ciascuna delle seguenti radici quadrate il valore approssimato a $1/10$:\quad$\sqrt 3$,\quad$\sqrt 5$,\quad$\sqrt 7$,\quad$\sqrt{11}$,\quad$\sqrt{\frac 1 2}$,\quad$\sqrt{\frac{17} 4}$.
\end{esercizio}

\begin{esercizio}
 \label{ese:2.4}
Estrai le seguenti radici di espressioni letterali, facendo attenzione al valore assoluto.
\begin{multicols}{3}
\begin{enumeratea}
 \item $\sqrt{a^2+2a+1}$;
 \item $\sqrt{4x^2+8x+4}$;
 \item $\sqrt{9-12a+4a^2}$.
\end{enumeratea}
\end{multicols}
\end{esercizio}

 \begin{esercizio}
\label{ese:2.5}
Senza usare la calcolatrice determina per ciascuna delle seguenti radici cubiche il valore approssimato a $1/10$:\quad$\sqrt[3]3$,\quad$\sqrt[3]4$,\quad$\sqrt[3]7$,\quad$\sqrt[3]{100}$,\quad$\sqrt[3]{25}$,\quad$\sqrt[3]{250}$.
\end{esercizio}

\begin{esercizio}[\Ast]
 \label{ese:2.6}
Determina le seguenti radici (se esistono).
 \begin{multicols}{3}
 \begin{enumeratea}
 \item $\sqrt[3]{27}$;
 \item $\sqrt[3]{64}$;
 \item $\sqrt[3]{-1}$;
 \item $\sqrt[3]{\np{1000}}$;
 \item $\sqrt[3]{125}$;
 \item $\sqrt[3]{-216}$;
 \item $\sqrt[3]{\frac 8{27}}$;
 \item $\sqrt[3]{-\frac{64}{125}}$;
 \item $\sqrt[3]{\frac{\np{1000}}{27}}$.
 \end{enumeratea}
 \end{multicols}
\end{esercizio}

\begin{esercizio}[\Ast]
 \label{ese:2.7}
Determina le seguenti radici (se esistono).
 \begin{multicols}{2}
 \begin{enumeratea}
 \item $\sqrt[3]{\np{0,001}}$;
 \item $\sqrt[3]{\frac 1 8}$;
 \item $\sqrt[3]{\np{-0,008}}$;
 \item $\sqrt[3]{4+\sqrt[3]{61+\sqrt[3]{25+\sqrt[3]8}}}$;
 \item $\sqrt[3]{25+\sqrt[3]{3+\sqrt[3]{122+\sqrt[3]{27}}}}$;
 \item $\sqrt[3]{27\cdot \sqrt{64}}$;
 \item $\sqrt[9]0$;
 \item $\sqrt[8]{-1}$;
 \item $\sqrt[5]{\np{-100000}}$.
 \end{enumeratea}
 \end{multicols}
\end{esercizio}

\begin{esercizio}[\Ast]
 \label{ese:2.8}
Determina le seguenti radici (se esistono).
 \begin{multicols}{2}
 \begin{enumeratea}
 \item $\sqrt[4]{\np{0,0001}}$;
 \item $\sqrt[4]{81}$;
 \item $\sqrt[6]{64}$;
 \item $\sqrt[5]{\frac{32}{243}}$;
 \item $\sqrt[4]{-4}$;
 \item $\sqrt[10]0$;
 \item $\sqrt[4]{\np{0,0081}}$;
 \item $\sqrt[5]{34-\sqrt[4]{14+\sqrt{2+\sqrt[3]8}}}$;
 \item $\sqrt{20+\sqrt[3]{121+\sqrt[4]{253+\sqrt[5]{243}}}}$.
 \end{enumeratea}
 \end{multicols}
\end{esercizio}

\begin{esercizio}[\Ast]
\label{ese:2.9}
Determina le seguenti radici (se esistono).
 \begin{multicols}{3}
 \begin{enumeratea}
 \item $\sqrt{21+\sqrt{16}}$;
 \item $\sqrt[5]{31+\sqrt[4]1}$;
 \item $\sqrt[5]{240+\sqrt 9}$;
 \item $\sqrt{\sqrt{\np{0,16}}}$;
 \item $\sqrt[5]{32\cdot 10^{-5}}$;
 \item $\sqrt{3\sqrt{37-4\sqrt{81}}\cdot 27}$;
 \item $\sqrt{72+\sqrt{80+\sqrt 1}}$;
 \item $\sqrt{\frac{25a^4} 9}$;
 \item $\sqrt[4]{620+\sqrt[4]{625}}$.
 \end{enumeratea}
 \end{multicols}
\end{esercizio}

\begin{esercizio}[\Ast]
 \label{ese:2.10}
Determina le seguenti radici (se esistono).
 \begin{multicols}{3}
 \begin{enumeratea}
 \item $\sqrt{\np{24336}}$;
 \item $\sqrt[5]{243}$;
 \item $\sqrt[4]{600+\sqrt{25}\cdot \sqrt{25}}$;
 \item $\sqrt[3]{8a^3+12a^2+6a+1}$;
 \item $\sqrt[3]{a^6+9a^4+27a^2+27}$;
 \item $\sqrt[3]{1-6x+12x^2-8x^3}$.
 \end{enumeratea}
 \end{multicols}
\end{esercizio}

\subsubsection*{2.2 - Condizioni di esistenza}

\begin{esercizio}[\Ast]
 \label{ese:2.11}
Determina le condizioni di esistenza dei seguenti radicali.
 \begin{multicols}{3}
 \begin{enumeratea}
 \item $\sqrt[3]{x+1}$;
 \item $\sqrt{1-x}$;
 \item $\sqrt{\dfrac 1{x+1}}$;
 \item $\sqrt{3x^2y}$;
 \item $\sqrt[3]{3xy}$;
 \item $\sqrt[4]{-2x^2y^2}$;
 \item $\sqrt[4]{\dfrac{x^2+1}{x-1}}$;
 \item $\sqrt[5]{\dfrac 1{x^3}}$;
 \item $\sqrt{\dfrac{4-x}{x-3}}$.
 \end{enumeratea}
 \end{multicols}
\end{esercizio}

\begin{esercizio}[\Ast]
 \label{ese:2.12}
Determina le condizioni di esistenza dei seguenti radicali.
 \begin{multicols}{3}
 \begin{enumeratea}
 \item $\sqrt{x^2(x+1)}$;
 \item $\sqrt[3]{1+a^2}$;
 \item $\sqrt[6]{2x-1}$;
 \item $\sqrt{1-x}+2\sqrt{\dfrac 1{x-1}}$;
 \item $\sqrt{1+\valass{x}}$;
 \item $\sqrt{(a-1)(a-2)}$;
 \item $\sqrt{\valass{x}+1}\cdot \sqrt[3]{x+1}$;
 \item $\sqrt{\valass{x-1}-1}$;
 \item $\sqrt[3]{\dfrac{x^2+x+1}{x^2+2x+1}}$;
 \item $\sqrt{\dfrac 1{x^2}-1}\cdot \sqrt[4]{\dfrac{x-1}{3-x}}$.
 \end{enumeratea}
 \end{multicols}
\end{esercizio}

\begin{esercizio}[\Ast]
 \label{ese:2.13}
Determina le condizioni di esistenza dei seguenti radicali.
 \begin{multicols}{3}
 \begin{enumeratea}
 \item $\sqrt{\dfrac{5-x}{x+2}}$;
 \item $\sqrt{\dfrac{2y}{(2y+1)^2}}$;
 \item $\sqrt{\dfrac{x-3}{1-x}}$;
 \item $\sqrt{\dfrac a{a^2-a-2}}$;
 \item $\sqrt{\dfrac 1{b^2-4}}$;
 \item $\sqrt{\dfrac{(x-1)^2}{(x-3)(x+2)}}$;
 \item $\sqrt{\dfrac 2 x+\dfrac x 2}$;
 \item $\sqrt[6]{\dfrac{x-1}{\left|x\right|}}$;
 \item $\sqrt[4]{\dfrac{4x^2+4+8x} 9}$.
 \end{enumeratea}
 \end{multicols}
\end{esercizio}

\begin{esercizio}[\Ast]
Determina le condizioni di esistenza dei seguenti radicali.
 \label{ese:2.14}
 \begin{multicols}{3}
 \begin{enumeratea}
 \item $\sqrt[4]{a^2 b^3}$;
 \item $\sqrt[6]{-x}$;
 \item $\sqrt[4]{x^3 - x^2}$;
 \item $\sqrt[4]{\dfrac{2}{2x+1}}$;
 \item $\sqrt[6]{\dfrac{7-x}{x^{-3}}}$.
 \end{enumeratea}
 \end{multicols}
\end{esercizio}

\begin{esercizio}[\Ast]
Determina le condizioni di esistenza dei seguenti radicali.
 \label{ese:2.15}
 \begin{multicols}{3}
 \begin{enumeratea}
 \item $\sqrt[6]{\dfrac{\left(b^2+1+2b\right)^3}{729b^6}}$;
 \item $\sqrt{\dfrac{x(x-1)}{x-4}}$;
 \item $\sqrt{\dfrac 1{x^2}+\dfrac 1{y^2}+\dfrac 2{xy}}$;
 \item $\sqrt[4]{\dfrac{m+1}{m-1}}$;
 \item $\sqrt[3]{x(x+2)^2}$;
 \item $\sqrt{\dfrac{1+a}{a^2}}$;
 \item $\sqrt{\dfrac{a+2}{a(a-4)}}$;
 \item $\sqrt{\dfrac 1{b^2-4}}$;
 \item $\sqrt{\dfrac{a^3}{a^2+6a+9}}$.
 \end{enumeratea}
 \end{multicols}
\end{esercizio}

\begin{esercizio}[\Ast]
 \label{ese:2.16}
Determina le condizioni di esistenza dei seguenti radicali.
 \begin{multicols}{3}
 \begin{enumeratea}
 \item $\sqrt{\dfrac{x^2}{x^2+1}}$;
 \item $\sqrt{\dfrac{x^2-4}{x-2}}$;
 \item $\sqrt{\dfrac x{x^2+1}}$;
 \item $\sqrt[3]{\dfrac{x^3}{x^3+1}}$;
 \item $\sqrt{2x+3}$;
 \item $\sqrt[3]{a^2-1}$;
 \item $\sqrt{x(x+1)(x+2)}$;
 \item $\sqrt{\left|x\right|+1}$;
 \item $\sqrt{\dfrac x{\left|x+1\right|}}$;
 \item $\sqrt{\dfrac 1{-x^2-1}}$;
 \item $\sqrt{\dfrac{x^2+1}{x-1}}+\sqrt{x^2-1}$.
 \end{enumeratea}
 \end{multicols}
\end{esercizio}

\subsubsection*{2.3 - Potenze a esponente razionale}
\begin{esercizio}
 \label{ese:2.17}
Calcola le seguenti potenze con esponente razionale.
 \begin{multicols}{3}
 \begin{enumeratea}
 \item $4^{\frac 3 2}$;
 \item $8^{\frac 2 3}$;
 \item $9^{-\frac 1 2}$;
 \item $16^{\frac 3 4}$;
 \item $16^{\frac 5 4}$;
 \item $\left(\dfrac 9 4\right)^{\frac 4 3}$;
 \item $125^{-\frac 2 3}$;
 \item $\left(\dfrac 1 8\right)^{-\frac 3 2}$;
 \item $25^{-\frac 3 2}$;
 \item $27^{\frac 4 3}$.
 \end{enumeratea}
 \end{multicols}
\end{esercizio}

\begin{esercizio}[\Ast]
 \label{ese:2.18}
Calcola le seguenti potenze con esponente razionale.
 \begin{multicols}{3}
 \begin{enumeratea}
 \item $32^{\frac 2 5}$;
 \item $49^{-\frac 1 2}$;
 \item $\left(\dfrac 1 4\right)^{-\frac 1 2}$;
 \item $\left(-\dfrac 1{27}\right)^{-\frac 2 3}$;
 \item $\left(\dfrac 4 9\right)^{-\frac 5 2}$;
 \item $\left(\np{0,008}\right)^{-\frac 2 3}$;
 \item $4^{\np{0,5}}$;
 \item $16^{\np{0,25}}$;
 \item $32^{\np{0,2}}$;
 \item $100^{\np{0,5}}$.
 \end{enumeratea}
 \end{multicols}
\end{esercizio}

\begin{esercizio}[\Ast]
 \label{ese:2.19}
Trasforma le seguenti espressioni in forma di potenza con esponente frazionario.
 \begin{multicols}{3}
 \begin{enumeratea}
 \item $\sqrt 2$;
 \item $\sqrt[3]{8^2}$;
 \item $\sqrt[7]{5^3}$;
 \item $\sqrt{3^3}$;
 \item $\sqrt{\left(\dfrac 1{3^3}\right)}$;
 \item $\sqrt[3]{\dfrac 1{3^2}}$;
 \item $\sqrt[3]{\dfrac 1{25}}$;
 \item $\sqrt[5]{\dfrac{4^2}{3^2}}$.
 \end{enumeratea}
 \end{multicols}
\end{esercizio}
\pagebreak
\begin{esercizio}[\Ast]
\label{ese:2.20}
 Trasforma nella forma radicale le seguenti espressioni.
 \begin{multicols}{2}
 \begin{enumeratea}
 \item $\left(\left(a^2+1\right)^{\frac 2 3}+1\right)^{\frac 1 4}$;
 \item $\left(1+\left(1+a^{\frac 2 3}\right)^{\frac 1 5}\right)^{\frac 2 3}$.
 \end{enumeratea}
 \end{multicols}
\end{esercizio}

\begin{esercizio}
 \label{ese:2.21}
Scrivi in ordine crescente i seguenti numeri:
 \[\np{0,00000001}\text{,}\quad (\np{0,1})^{10}\text{,}\quad (\np{0,1})^{\np{0,1}}\text{,}\quad 10^{-10}\text{,}\quad \sqrt{\np{0,0000000001}}.\]
\end{esercizio}

\subsubsection*{2.4 - Semplificazione di radici}
\begin{esercizio}
 \label{ese:2.22}
Trasforma i seguenti radicali applicando la proprietà invariantiva.
 \begin{multicols}{3}
 \begin{enumeratea}
 \item $\sqrt[4]4=\sqrt[8]{\vphantom{4}\ldots}$;
 \item $\sqrt[3]9=\sqrt[6]{\vphantom{4}\ldots}$;
 \item $\sqrt[5]5=\sqrt[15]{\vphantom{4}\ldots}$;
 \item $\sqrt 2=\sqrt[6]{\vphantom{4}\ldots}$;
 \item $\sqrt 2=\sqrt[\ldots]{16}$;
 \item $\sqrt[3]3=\sqrt[\ldots]{81}$.
 \end{enumeratea}
 \end{multicols}
\end{esercizio}

\begin{esercizio}
\label{ese:2.23}
Trasforma i seguenti radicali applicando la proprietà invariantiva.
 \begin{multicols}{3}
 \begin{enumeratea}
 \item $\sqrt[3]{-5}=-\sqrt[\ldots]{25}$;
 \item $\sqrt[4]{\frac 3 2}=\sqrt[\ldots]{\frac{27} 8}$;
 \item $\sqrt[21]{a^7}=\sqrt[6]{\vphantom{4}\ldots}$, $a>0$;
 \item $\sqrt[8]{a^{24}}=\sqrt[5]{\vphantom{4}\ldots}$, $a>0$;
 \item $\sqrt[3]{27}=\frac 1{\sqrt{\vphantom{4}\ldots}}$;
 \item $\sqrt{x^4+2x^2+1}=\sqrt[7]{\vphantom{4}\ldots}$
 \end{enumeratea}
 \end{multicols}
\end{esercizio}

\begin{esercizio}[\Ast]
\label{ese:2.24}
Semplifica i seguenti radicali.
 \begin{multicols}{3}
 \begin{enumeratea}
 \item $\sqrt[4]{25}$;
 \item $\sqrt[6]8$;
 \item $\sqrt[8]{16}$;
 \item $\sqrt[9]{27}$;
 \item $\sqrt[4]{100}$;
 \item $\sqrt[6]{144}$;
 \item $\sqrt[4]{169}$;
 \item $\sqrt[6]{121}$;
 \item $\sqrt[6]{125}$.
 \end{enumeratea}
 \end{multicols}
\end{esercizio}

\begin{esercizio}[\Ast]
 \label{ese:2.25}
Semplifica i seguenti radicali.
 \begin{multicols}{3}
 \begin{enumeratea}
 \item $\sqrt[4]{49}$;
 \item $\sqrt[6]{64}$;
 \item $\sqrt[12]{16}$;
 \item $\sqrt[6]{\frac{16}{121}}$;
 \item $\sqrt[4]{\frac 1{16}}$;
 \item $\sqrt[10]{\frac{25}{81}}$;
 \item $\sqrt[15]{\frac{64}{27}}$;
 \item $\sqrt[9]{-3^3}$;
 \item $\sqrt[6]{(-2)^4}$.
 \end{enumeratea}
 \end{multicols}
\end{esercizio}

\begin{esercizio}[\Ast]
 \label{ese:2.26}
Semplifica i seguenti radicali.
 \begin{multicols}{3}
 \begin{enumeratea}
 \item $\sqrt[12]{-4^6}$;
 \item $\sqrt[10]{-32}$;
 \item $\sqrt[6]{5^2-4^2}$;
 \item $\sqrt[4]{12^2+5^2}$;
 \item $\sqrt[10]{3^2+4^2}$;
 \item $\sqrt[4]{10^2-8^2}$;
 \item $\sqrt[3]{2^6\cdot 5^{15}}$;
 \item $\sqrt[4]{3^4\cdot 4^6}$;
 \item $\sqrt[5]{5^5\cdot 4^{10}\cdot 2^{15}}$.
 \end{enumeratea}
 \end{multicols}
\end{esercizio}

\begin{esercizio}[\Ast]
 \label{ese:2.27}
Semplifica i seguenti radicali.
 \begin{multicols}{3}
 \begin{enumeratea}
 \item $\sqrt[9]{27\cdot 8\cdot 125}$;
 \item $\sqrt[4]{625}$;
 \item $\sqrt[6]{\np{1000}}$;
 \item $\sqrt[4]{2+\frac{17}{16}}$;
 \item $\sqrt[6]{\left(\frac{13} 4+\frac 1 8\right)^4}$;
 \item $\sqrt[6]{\left(1+\frac{21} 4\right)^3}$;
 \item $\sqrt[16]{(-16)^4}$;
 \item $\sqrt[10]{2^{10}\cdot 3^{20}}$;
 \item $\sqrt[6]{2^8\cdot 3^6}$.
 \end{enumeratea}
 \end{multicols}
\end{esercizio}

\begin{esercizio}[\Ast]
 \label{ese:2.28}
Semplifica i seguenti radicali.
 \begin{multicols}{3}
 \begin{enumeratea}
 \item $\sqrt[12]{3^6\cdot 4^{12}}$;
 \item $\sqrt[4]{2^{10}\cdot 3^{15}\cdot 12^5}$;
 \item $\sqrt[6]{3^9\cdot 8^2}$;
 \item $\sqrt[4]{9x^2y^4}$;
 \item $\sqrt[3]{64a^6b^9}$;
 \item $\sqrt[3]{x^6y^9(x-y)^{12}}$;
 \item $\sqrt[5]{\frac{32a^{10}}{b^{20}}}$;
 \item $\sqrt[4]{\frac{20a^6}{125b^{10}}}$;
 \item $\sqrt[8]{\frac{16x^5y^8}{81x}}$.
 \end{enumeratea}
 \end{multicols}
\end{esercizio}

\begin{esercizio}[\Ast]
 \label{ese:2.29}
Semplifica i seguenti radicali.
 \begin{multicols}{3}
 \begin{enumeratea}
 \item $\left(\sqrt{a+1}\right)^6$;
 \item $\sqrt[9]{27a^6b^{12}}$;
 \item $\sqrt[12]{(2x+3)^3}$;
 \item $\sqrt[6]{\frac{\np{0,008}x^{15}y^9}{8a^{18}}}$;
 \item $\sqrt[10]{\frac{121a^5}{ab^2}}$;
 \item $\sqrt{\frac{25a^4b^8c^7}{c(a+2b)^6}}$;
 \item $\sqrt[6]{a^2+2a+1}$;
 \item $\sqrt[9]{a^3+3a^2+3a+1}$;
 \item $\sqrt{3a^2+\sqrt{a^4}}$.
 \end{enumeratea}
 \end{multicols}
\end{esercizio}

\begin{esercizio}[\Ast]
 \label{ese:2.30}
Semplifica i seguenti radicali.
 \begin{multicols}{3}
 \begin{enumeratea}
 \item $\sqrt[4]{x^4+2x^2+1}$;
 \item $\sqrt[10]{a^4+6a^2x+9x^2}$;
 \item $\sqrt[6]{8a^3-24a^2+24a-8}$;
 \item $\sqrt[6]{\frac{9x^2}{y^6}}$;
 \item $\sqrt[4]{\frac{16a^4b^6}{25x^2}}$;
 \item $\sqrt{\frac{2x^2-2}{8x^2-8}}$;
 \item $\sqrt[8]{a^4+2a^2x^2+x^4}$;
 \item $\sqrt{\frac{25a^4b^6}{a^4+4+4a^2}}$;
 \item $\sqrt[9]{x^6+3x^5+3x^4+x^3}$.
 \end{enumeratea}
 \end{multicols}
\end{esercizio}

\begin{esercizio}[\Ast]
 \label{ese:2.31}
Semplifica i seguenti radicali.
 \begin{multicols}{2}
 \begin{enumeratea}
 \item $\sqrt[4]{a^2+6a+9}$;
 \item $\sqrt[9]{8x^3-12x^2+6x+x^3}$;
 \item $\sqrt[4]{a^4(a^2-2a+1)}$;
 \item $\sqrt[4]{(x^2-6x+9)^2}$;
 \item $\sqrt[12]{(x^2+6x+9)^3}$;
 \item $\sqrt{a^2+2a+1}-\sqrt{a^2-2a+1}$;
 \item $\sqrt[18]{\frac{a^9+3a^8+3a^7+a^6}{9a^7+9a^5+18a^6}}$;
 \item $\sqrt[6]{\frac{(x^2+1-2x)^3b}{b^7\left(x^3+3x^2+3x+1\right)^2}}$;
 \item $\sqrt{\frac{\left(x^3+x^2y\right)(a+2)}{2x+2y+ax+ay}}$.
 \end{enumeratea}
 \end{multicols}
\end{esercizio}

\begin{esercizio}[\Ast]
 \label{ese:2.32}
Semplifica i seguenti radicali.
 \begin{multicols}{3}
 \begin{enumeratea}
 \item $\sqrt[2n]{16^n}$;
 \item $\sqrt[4n]{\frac{2^{3n}}{3^{2n}}}$;
 \item $\sqrt[n^2]{\frac{6^{2n}}{5^{3n}}}$;
 \item $\sqrt[3n]{27^n\cdot 64^2n}$;
 \item $\sqrt[2n^2]{16^2n\cdot 81^2n}$;
 \item $\sqrt[n+1]{16^{2n+2}}$;
 \item $\sqrt[5]{25x^3y^4}$;
 \item $\sqrt[12]{81a^6b^{12}}$;
 \item $\sqrt[5]{32x^{10}}$.
 \end{enumeratea}
 \end{multicols}
\end{esercizio}

\begin{esercizio}[\Ast]
 \label{ese:2.33}
Semplifica i seguenti radicali.
 \begin{multicols}{3}
 \begin{enumeratea}
 \item $\sqrt[3]{27 a^6 b^9}$;
 \item $\sqrt[10]{32 x^5 y^{15}}$;
 \item $\sqrt[4]{x^2 +8xy+16y^2}$;
 \item $\sqrt{\frac{16 x^4}{81 y^6}}$;
 \item $\sqrt[3]{x^9 y^3}$.
 \end{enumeratea}
 \end{multicols}
\end{esercizio}
\pagebreak
\begin{esercizio}[\Ast]
 \label{ese:2.34}
Semplifica i seguenti radicali.
 \begin{multicols}{2}
 \begin{enumeratea}
 \item $\sqrt[15]{\frac{8x^3 y^6}{125z^9}}$;
 \item $\sqrt{\frac{x^3+x^2 y-xy^2 -y^3}{x+y}}$;
 \item $\sqrt[9]{\frac{x^3 y^6}{(81x^2-18x+1)^3}}$;
 \item $\sqrt[27]{\frac{1}{x^3}-\frac{3}{x^2}+\frac{3}{x}-1}$;
 \item $\sqrt[8]{\left(\frac{a^2-2a+1}{a^2+2a+1}\right)^2}$;
 \item $\sqrt[12]{\frac{81(a^2-2a+1)^2}{625 x^4 y^{12}}}$.
 \end{enumeratea}
 \end{multicols}
\end{esercizio}

\subsubsection*{2.5 - Moltiplicazione e divisione di radici}
\begin{esercizio}[\Ast]
 \label{ese:2.35}
Esegui le seguenti moltiplicazioni e divisioni di radicali.
 \begin{multicols}{3}
 \begin{enumeratea}
 \item $\sqrt{45}\cdot \sqrt 5$;
 \item $\sqrt 2\cdot \sqrt{18}$;
 \item $\sqrt[3]{16}\cdot \sqrt[3]4$;
 \item $\sqrt{75}\cdot \sqrt{12}$;
 \item $\sqrt[3]{20}\cdot \sqrt{50}$;
 \item $\sqrt{40}:\left(\sqrt 2\cdot \sqrt 5\right)$;
 \item $\sqrt{\frac 1 5}\cdot \sqrt{45}$;
 \item $\sqrt[3]3:\sqrt[3]9$;
 \item $\sqrt[5]2\cdot \sqrt[5]6:\sqrt[5]{12}$.
 \end{enumeratea}
 \end{multicols}
\end{esercizio}

\begin{esercizio}[\Ast]
 \label{ese:2.36}
Esegui le seguenti moltiplicazioni e divisioni di radicali.
 \begin{multicols}{3}
 \begin{enumeratea}
 \item $\sqrt[4]{9}\cdot\sqrt[4]{27}$;
 \item $\sqrt[4]{16}\cdot\sqrt[4]{128}$;
 \item $\sqrt[4]{\frac{16}{7}}\cdot\sqrt[4]{\frac{7}{12}}$;
 \item $\sqrt[7]{\frac{15}{2}}\cdot\sqrt[7]{\frac{28}{5}}$;
 \item $\sqrt[3]{\frac{21}{4}}\cdot\sqrt[3]{\frac{12}{7}}$;
 \item $\sqrt[5]{\frac{a}{b}}\cdot\sqrt[5]{\frac{b}{a}}$;
 \item $\sqrt[3]{\frac{a}{b}}\cdot\sqrt[3]{\frac{b^2}{a^2}}$.
 \end{enumeratea}
 \end{multicols}
\end{esercizio}

\begin{esercizio}[\Ast]
 \label{ese:2.37}
Esegui le seguenti moltiplicazioni e divisioni di radicali.
 \begin{multicols}{2}
 \begin{enumeratea}
 \item $\sqrt[6]{81}\cdot \sqrt[6]{81}:\sqrt[6]9$;
 \item $\sqrt[4]{1+\frac 1 2}\cdot \sqrt[4]{2-\frac 1 2}\cdot \sqrt[4]{1+\frac 5 4}$;
 \item $\sqrt 3\cdot \sqrt[3]9$;
 \item $\sqrt[3]2\cdot \sqrt 8$;
 \item $\sqrt[6]{81}\cdot \sqrt 3$;
 \item $\sqrt 2\cdot \sqrt 2\cdot\sqrt 2$;
 \item $\sqrt{\frac{10} 2}\cdot \sqrt[3]{\frac 6 3}:\sqrt[6]{\frac 4 9}$;
 \item $\sqrt{2^3\cdot 3}\cdot \sqrt 2\cdot \sqrt{3^3}$.
 \end{enumeratea}
 \end{multicols}
\end{esercizio}

\begin{esercizio}[\Ast]
 \label{ese:2.38}
Esegui le seguenti moltiplicazioni e divisioni di radicali.
 \begin{multicols}{3}
 \begin{enumeratea}
 \item $\left(\sqrt[3]{\frac{42}{13}}:\sqrt[3]{\frac{91}{36}}\right):\sqrt[3]{13}$;
 \item $\sqrt[3]{\frac 3 4}\cdot \sqrt[3]{\frac{25}{24}}\cdot \sqrt[3]{\frac 5 2}$;
 \item $\sqrt[3]{5+\frac 1 3}\cdot \sqrt[3]{\frac 4 3}$;
 \item $\sqrt[5]{2^3}\cdot \sqrt[10]{2^4}$;
 \item $\sqrt{15}\cdot \sqrt{30}\cdot \sqrt 8$;
 \item $\sqrt{1+\frac 1 2}\cdot \sqrt[4]{2+\frac 1 4}$.
 \end{enumeratea}
 \end{multicols}
\end{esercizio}

\begin{esercizio}
 \label{ese:2.39}[\Ast]
Esegui le seguenti operazioni (le lettere rappresentano numeri reali positivi).
 \begin{multicols}{3}
 \begin{enumeratea}
 \item $\sqrt[3]{4a}\cdot \sqrt[3]{9a}\cdot \sqrt[3]{12a}$;
 \item $\sqrt{3a}:\sqrt{\frac 1 5a}$;
 \item $\sqrt[3]{2ab}\cdot \sqrt[3]{4a^2b^2}$;
 \item $\sqrt x\cdot \sqrt[3]{x^2}:\sqrt[6]x$;
 \item $\sqrt{\frac 1{a^4}}\cdot \sqrt{\frac{a^6b} 2}:\sqrt{\frac{2b} a}$;
 \item $\sqrt{\frac 4 9}\cdot \sqrt{\frac 3 2a}:\sqrt[6]{3a}$.
 \end{enumeratea}
 \end{multicols}
\end{esercizio}

\begin{esercizio}[\Ast]
 \label{ese:2.40}
Esegui le seguenti operazioni (le lettere rappresentano numeri reali positivi).
 \begin{multicols}{3}
 \begin{enumeratea}
 \item $\sqrt[3]{ax}\cdot \sqrt{xy}\cdot \sqrt[5]{ay}$;
 \item $\sqrt[3]{(x+1)^2}:\sqrt{x-1}$;
 \item $\sqrt{a^2-b^2}:\sqrt{a+b}$;
 \item $\sqrt{a^2-3a}\cdot \sqrt[3]{a^2}\cdot \sqrt[6]{a^5}$;
 \item $\sqrt{\frac{1-x}{1+x}}\cdot \sqrt[3]{\frac{1-x^2}{1+x^2}}$;
 \item $\sqrt{\frac{a+b}{a-b}}:\sqrt[3]{\frac{a+b}{a-b}}$.
 \end{enumeratea}
 \end{multicols}
\end{esercizio}

\begin{esercizio}[\Ast]
 \label{ese:2.41}
Esegui le seguenti operazioni (le lettere rappresentano numeri reali positivi).
 \begin{multicols}{2}
 \begin{enumeratea}
 \item $\sqrt{\frac{a^2+2a+1}{2a}}\cdot \sqrt{\frac{1+a}{a^2}}:\sqrt{\frac 2 a}$;
 \item $\sqrt{\frac{a+1}{a-3}}\cdot \sqrt[3]{\frac{a^2-9}{a^2-1}}$;
 \item $\sqrt{\frac{x+1}{x-2}}\cdot \sqrt{\frac{x-1}{x+3}}:\sqrt[3]{\frac{x^2-1}{x^2+x-6}}$;
 \item $\sqrt{a^4b}\cdot \sqrt[6]{\frac{a^2} b}$;
 \item $\sqrt[3]{\frac{a^2-4}{a+3}}\cdot \sqrt[4]{\frac{a+3}{a-2}}$;
 \item $\sqrt{\frac x y-\frac y x}:\sqrt{x+y}$.
 \end{enumeratea}
 \end{multicols}
\end{esercizio}

\begin{esercizio}[\Ast]
 \label{ese:2.42}
Esegui le seguenti operazioni (le lettere rappresentano numeri reali positivi).
 \begin{multicols}{2}
 \begin{enumeratea}
 \item $\sqrt{\frac 1{b^2}-\frac 1{a^2}}:\sqrt{\frac 1 b-\frac 1 a}$;
 \item $\frac{\sqrt{4a^2-9}\cdot \sqrt{2a-3}}{\sqrt[3]{2a+3}}$;
 \item $\sqrt{\frac{9-a^2}{(a+3)^2}}\cdot \sqrt{\frac{27+9a}{3-a}}$;
 \item $\sqrt{\frac{a+2}{a-1}}:\sqrt[3]{\frac{(a-1)^2}{a^2+4a+4}}$;
 \item $\sqrt{\frac{x^2-4}{x+1}}\cdot \sqrt[3]{\frac 1{x^3-2x^2}}$;
 \item $\sqrt[4]{\frac{a+b}{a^2-b^2}}\cdot \sqrt[3]{\frac{a-2b}{a+2b}}\cdot \sqrt[6]{a^2-4b^2}$.
 \end{enumeratea}
 \end{multicols}
\end{esercizio}

\begin{esercizio}[\Ast]
 \label{ese:2.43}
Esegui le seguenti operazioni (le lettere rappresentano numeri reali positivi).
 \begin{enumeratea}
 \item $\sqrt{\frac{a^2b+ab^2}{xy}}\cdot \sqrt[6]{\frac{(a+b)^2}{x^2}}\cdot \sqrt[6]{\frac{x^2y^3}{(a+b)^2}}\cdot \sqrt[4]{\frac x{a^3b^2+a^2b^3}}$;
 \item $\frac{\sqrt{\frac x y+\frac y x}:\sqrt[3]{\frac x y-\frac 1 x}}{\sqrt{\frac{xy}{x+y}}}$;
 \item $\sqrt{a-1}\cdot\sqrt{3x+2}\cdot\sqrt{\frac{a-1}{3x+2}}$;
 \item $\sqrt{\frac{x+y}{x-y}}\cdot\sqrt[3]{\frac{x-y}{x+y}}$;
 \item $\sqrt{\frac{a+2}{a+5}}\cdot\sqrt{\frac{a+2}{3a}}\cdot\sqrt{\frac{3a}{a+5}}$;
 \item $\sqrt[4]{\frac{x^2+2xy+y^2}{3x+2y}}\cdot\sqrt[4]{\frac{9x^2+12xy+4y^2}{x+y}}$.
 \end{enumeratea}
\end{esercizio}

\begin{esercizio}[\Ast]
 \label{ese:2.44}
Esegui le seguenti operazioni (le lettere rappresentano numeri reali positivi).
 \begin{enumeratea}
 \item $\sqrt[6]{\frac{a^2-ab}{b}}\cdot\sqrt[4]{\frac{b^2}{a^2-ab}}\cdot\sqrt[3]{\frac{a^2-ab}{b}}$;
 \item $\sqrt{\frac{x^2-xy}{xy+y^2}}\cdot\sqrt[4]{\frac{x^3+2x^2y+xy^2}{x^3-2x^2y+xy^2}}$;
 \item $(x-2)\cdot\sqrt{\frac{x+2}{x^2+5x+6}}\cdot\sqrt{\frac{x-2}{x^2+x-6}}$;
 \item $\sqrt[8]{\frac{(x^2-y^2)^4}{x^2}}\cdot\sqrt[4]{\frac{xy^2}{(x+y)^2}}\cdot\sqrt{\frac{x^2y}{x-y}}$;
 \item $\sqrt{\frac{a+b}{a-b}}\cdot\sqrt[3]{\frac{a-b}{a+b}}\cdot\sqrt[10]{\frac{a+b}{a-b}}$;
 \item $\sqrt{1-\frac{b}{a+2b}}\cdot\sqrt[4]{1+\frac{b}{a+b}}\cdot\sqrt[8]{\frac{a+2b}{2a+2b}}$.
 \end{enumeratea}
\end{esercizio}

\begin{esercizio}[\Ast]
 \label{ese:2.45}
Esegui le seguenti operazioni (le lettere rappresentano numeri reali positivi).
 \begin{enumeratea}
 \item $\sqrt[5]{\frac{4x-9}{4x}}\cdot\sqrt[5]{\frac{1}{16x^2-81}}\cdot\sqrt[5]{4x(4x+9)}$;
 \item $\sqrt[3]{\frac{1}{b}-\frac{1}{a}}\cdot\sqrt[4]{\frac{ab}{a^2+b^2-2ab}}$;
 \item $\sqrt[6]{\frac{x^2+xy}{xy-y^2}}\cdot\sqrt[3]{\frac{x-y}{x+y}}\cdot\sqrt{\frac{xy+y^2}{x^2-xy}}$;
 \item $\sqrt[6]{\frac{1}{a}+4a-4}\cdot\sqrt[3]{\frac{1}{a}+4a+4}\cdot\sqrt{\frac{a}{4a^2-1}}$;
 \item $\sqrt[5]{\frac{a+2b}{a-2b}}\cdot\sqrt[3]{\frac{a^2-2ab}{ab+2b^2}}\cdot\sqrt[15]{\frac{a^2+4ab+4b^2}{a^2-4ab+4b^2}}$;
 \item $\sqrt[5]{\frac{x^2-xy}{xy+y^2}}\cdot\sqrt[6]{\frac{x+y}{x-y}}\cdot\sqrt[15]{\frac{x^5+2x^4y+x^3y^2}{x^2y^3-2xy^4+y^5}}$.
 \end{enumeratea}
\end{esercizio}

\begin{esercizio}[\Ast]
 \label{ese:2.46}
Esegui le seguenti operazioni (le lettere rappresentano numeri reali positivi).
 \begin{enumeratea}
 \item $\sqrt[3]{\frac{x}{x^2+xy}+\frac{x+y}{xy-y^2}}\cdot\sqrt[3]{\frac{x^2-y^2}{x+3y}}$;
 \item $\sqrt[3]{a^2+2ab+b^2}\cdot\sqrt{\frac{a}{a+b}}$;
 \item $\sqrt{x^2-y^2}\cdot\sqrt[3]{\frac{x}{x-y}}\cdot\sqrt[4]{\frac{1}{x+y}}$;
 \item $\sqrt[6]{\frac{x^6}{y^2(x^2-xy)^3}}\cdot\sqrt[4]{\frac{(xy-y^2)^2}{x}}\cdot\sqrt[12]{x^9y^{10}}$;
 \item $\sqrt[3]{\frac{4xy}{x-y}}\cdot\sqrt{\frac{(x+y)^2}{4xy}-1}$;
 \item $\sqrt{\frac{x-3}{x+3}}\cdot\sqrt[4]{\frac{x^2}{3-x}}\cdot\sqrt[8]{\frac{(x+3)^3}{x^2}}$.
 \end{enumeratea}
\end{esercizio}

\begin{esercizio}[\Ast]
 \label{ese:2.47}
Esegui le seguenti operazioni (le lettere rappresentano numeri reali positivi).
 \begin{enumeratea}
 \item $\sqrt[6]{\frac{a-2}{a+3}}\cdot\sqrt{\frac{a+2}{a-3}}\cdot\sqrt[6]{\frac{a^2-5a+6}{a^2+5a+6}}\cdot\sqrt[3]{\frac{a^2-9}{a^2-4}}$;
 \item $\sqrt[3]{\frac{x-y}{x+y}+\frac{x+y}{x-y}}\cdot\sqrt[3]{\frac{x^2+y^2}{2xy}+1}\cdot\sqrt[3]{\frac{xy}{x^2+y^2}}$;
 \item $\sqrt[3]{\frac{(a^3-1)^4}{(a+1)^2}}\cdot\sqrt[4]{\frac{a+1}{(a^2+a+1)^3}}\cdot\sqrt[12]{\frac{(a^2-1)^5}{(a^2+a+1)^7}}$;
 \item $\sqrt[4]{\frac{a^2-1}{a+b}}\cdot\sqrt[3]{\frac{a^2+ab}{a^2+a}}\cdot\sqrt[6]{\frac{a^3+a^2}{a^2(a-1)}}$;
 \item $\sqrt[4]{\left(\frac{a}{b}+\frac{b}{a}\right)\left(\frac{a}{b}-\frac{b}{a}\right)}\sqrt{\frac{ab}{(a-b)(a+b)}}$;
 \item $\sqrt{\frac{x^2+xy}{xy-y^2}}\cdot\sqrt[3]{\frac{x^2y^2-2xy^3+y^4}{x^4+2x^3y+x^2y^2}}\cdot\sqrt[6]{x^4y^2-x^3y^3-x^2y^4+xy^5}$.
 \end{enumeratea}
\end{esercizio}

\subsubsection*{2.6 - Portare un fattore sotto il segno di radice}

\begin{esercizio}[\Ast]
 \label{ese:2.48}
Trasporta dentro la radice i fattori esterni.
 \begin{multicols}{4}
 \begin{enumeratea}
 \item $2\sqrt 2$;
 \item $3\sqrt 3$;
 \item $2\sqrt 3$;
 \item $3\sqrt 2$;
 \item $\frac 1 2\sqrt 2$;
 \item $\frac 1 3\sqrt 3$;
 \item $\frac 1 2\sqrt 6$;
 \item $\frac 2 3\sqrt 6$;
 \item $\frac 3 4\sqrt{\frac 3 2}$;
 \item $2\sqrt[3]2$;
 \item $\frac 1 3\sqrt[3]3$;
 \item $4\sqrt[3]{\frac 1 2}$;
 \item $-3\sqrt 3$;
 \item $-2\sqrt[3]2$;
 \item $\frac{-1} 2\sqrt[3]4$;
 \item $\frac{-1} 5\sqrt 5$;
 \item $-\frac 1 3\sqrt[3]9$;
 \item $\left(1+\frac 1 2\right)\sqrt 2$.
 \end{enumeratea}
 \end{multicols}
\end{esercizio}

\begin{esercizio}[\Ast]
 \label{ese:2.49}
Trasporta dentro la radice i fattori esterni, discutendo i casi letterali.
 \begin{multicols}{3}
 \begin{enumeratea}
 \item $x\sqrt{\frac 1 5}$;
 \item $x^2\sqrt[3]x$;
 \item $a\sqrt 2$;
 \item $x^2\sqrt[3]3$;
 \item $2a\sqrt 5$;
 \item $a\sqrt{-a}$;
 \item $(a-1)\sqrt a$;
 \item $(x-2)\sqrt{\frac 1{2x-4}}$;
 \item $x\sqrt{\frac 1{x^2+x}}$;
 \item $\frac{a+1}{a+2}\sqrt{\frac{a^2+3a+2}{a^2+4a+3}}$;
 \item $\frac 2 x\sqrt{\frac{x^2+x}{x-1}-x}$;
 \item $\frac 1{x-1}\sqrt{x^2-1}$.
 \end{enumeratea}
 \end{multicols}
\end{esercizio}
\pagebreak
\begin{esercizio}[\Ast]
 \label{ese:2.50}
Trasporta dentro la radice i fattori esterni, discutendo i casi letterali.
 \begin{multicols}{3}
 \begin{enumeratea}
 \item $\frac{x}{x+1}\sqrt{\frac{x+1}{x}}$;
 \item $(x+1)\sqrt{\frac{x+2}{x+1}}$;
 \item $3ab\sqrt[4]{\frac{ab^2}{3c^2}}$;
 \item $x\sqrt[3]{1-\frac{1}{x^2}}$;
 \item $\frac{1}{a+b}\sqrt{a^2-b^2}$;
 \item $\frac{x}{x-1}\sqrt[3]{\left(1-\frac{1}{x}\right)^2}$.
 \end{enumeratea}
 \end{multicols}
\end{esercizio}

\begin{esercizio}[\Ast]
 \label{ese:2.51}
Trasporta dentro la radice i fattori esterni, discutendo i casi letterali.
 \begin{multicols}{3}
 \begin{enumeratea}
 \item $(a+b)\sqrt{\frac{a-b}{a+b}}$;
 \item $\frac{x-y}{xy}\sqrt{\frac{xy}{x^2-y^2}}$;
 \item $\frac{3a-2}{3a+2}\sqrt{\frac{3a+2}{3a-2}}$;
 \item $\frac{a-1}{a}\sqrt[3]{\frac{a^3}{(a^2-1)^2}}$;
 \item $\frac{a+1}{b}\sqrt[5]{\frac{ab^2-b^2}{a^2+a}}$;
 \item $\left(a+\frac{1}{3}\right)\sqrt[5]{\frac{27}{(3a+1)^4}}$;
 \item $\frac{a^2b}{c}\sqrt[5]{\frac{c^2}{a^6b^3}}$;
 \item $\left(2-\frac{3}{5}\right)\sqrt{4-\frac{3}{7}}$.
 \end{enumeratea}
 \end{multicols}
\end{esercizio}

\subsubsection*{2.7 - Portare un fattore fuori dal segno di radice}

\begin{esercizio}[\Ast]
 \label{ese:2.52}
Semplifica i radicali portando fuori i fattori possibili (attenzione al valore assoluto).
 \begin{multicols}{4}
 \begin{enumeratea}
 \item $\sqrt{250}$;
 \item $\sqrt{486}$;
 \item $\sqrt{864}$;
 \item $\sqrt{\np{3456}}$;
 \item $\sqrt{20}$;
 \item $\sqrt{\np{0,12}}$;
 \item $\sqrt{45}$;
 \item $\sqrt{48}$;
 \item $\sqrt{98}$;
 \item $\sqrt{50}$;
 \item $\sqrt{300}$;
 \item $\sqrt{27}$;
 \item $\sqrt{75}$;
 \item $\sqrt{40}$;
 \item $\sqrt{12}$;
 \item $\sqrt{80}$.
 \end{enumeratea}
 \end{multicols}
\end{esercizio}

\begin{esercizio}[\Ast]
 \label{ese:2.53}
Semplifica i radicali portando fuori i fattori possibili (attenzione al valore assoluto).
 \begin{multicols}{3}
 \begin{enumeratea}
 \item $\sqrt{\frac{18}{80}}$;
 \item $\sqrt{\frac 9 4+\frac 4 9}$;
 \item $\sqrt{1-\frac 9{25}}$;
 \item $\sqrt{\frac{10} 3+\frac 2 9}$;
 \item $\frac 2 5\sqrt{\frac{50} 4}$;
 \item $\frac 3 2\sqrt{\frac 8{27}}$;
 \item $\frac 5 7\sqrt{\frac{98}{75}}$;
 \item $\frac 1 5\sqrt{\frac{\np{1000}}{81}}$;
 \item $\sqrt[3]{250}$;
 \item $\sqrt[3]{24}$;
 \item $\sqrt[3]{108}$;
 \item $\sqrt[4]{32}$;
 \item $\sqrt[4]{48}$;
 \item $\sqrt[4]{250}$;
 \item $\sqrt[5]{96}$;
 \item $\sqrt[5]{160}$.
 \end{enumeratea}
 \end{multicols}
\end{esercizio}

\begin{esercizio}[\Ast]
 \label{ese:2.54}
Semplifica i radicali portando fuori i fattori possibili (attenzione al valore assoluto).
 \begin{multicols}{3}
 \begin{enumeratea}
 \item $\sqrt{x^2y}$;
 \item $\sqrt{\frac{a^5}{b^2}}$;
 \item $\sqrt{\frac{a^2b^3c^3}{d^9}}$;
 \item $\sqrt{4ax^2}$;
 \item $\sqrt{9a^2b}$;
 \item $\sqrt{2a^2x}$;
 \item $\sqrt{x^3}$;
 \item $\sqrt{a^7}$;
 \item $\sqrt[3]{16a^3x^4}$;
 \item $\sqrt[3]{4a^4b^5}$;
 \item $\sqrt[3]{27a^7b^8}$;
 \item $\sqrt{18a^6b^5c^7}$.
 \end{enumeratea}
 \end{multicols}
\end{esercizio}


\begin{esercizio}[\Ast]
 \label{ese:2.55}
Semplifica i radicali portando fuori i fattori possibili (attenzione al valore assoluto).
 \begin{multicols}{3}
 \begin{enumeratea}
 \item $\sqrt[15]{\np{729}a^{6}b^{9}c^{3}}$;
 \item $\sqrt[5]{3x^{10}y^{7}z^{5}}$;
 \item $\sqrt[6]{\frac{x^{2}(x-y)^{4}}{\np{1296}}}$;
 \item $\sqrt[3]{\frac{125a^{3}b^{2}}{(a-b)^6}}$;
 \item $\sqrt[4]{\frac{a^{2}+b^{2}+2ab}{b^{2}-2ab+a^{2}}}$.
 \end{enumeratea}
 \end{multicols}
\end{esercizio}
\pagebreak
\begin{esercizio}[\Ast]
 \label{ese:2.56}
Semplifica i radicali portando fuori i fattori possibili (attenzione al valore assoluto).
 \begin{multicols}{3}
 \begin{enumeratea}
 \item $\sqrt[3]{\frac{128}{125}x^{2}}$;
 \item $\sqrt[3]{(a-b)^{4}}$;
 \item $\sqrt[5]{64a^{5}+32a^{7}}$;
 \item $\sqrt{x^{3}y+2x^{2}y^{2}+xy^{3}}$;
 \item $\sqrt[4]{16a^{4}+32}$.
 \end{enumeratea}
 \end{multicols}
\end{esercizio}

\begin{esercizio}[\Ast]
 \label{ese:2.57}
Semplifica i radicali portando fuori i fattori possibili (attenzione al valore assoluto).
 \begin{multicols}{3}
 \begin{enumeratea}
 \item $\sqrt{a^2+a^3}$;
 \item $\sqrt{4x^4-4x^2}$;
 \item $\sqrt{25x^7-25x^5}$;
 \item $\sqrt[3]{3a^5b^2c^9}$;
 \item $\sqrt[4]{16a^4b^5c^7x^6}$;
 \item $\sqrt[5]{64a^4b^5c^6d^7}$;
 \item $\sqrt[6]{a^{42}b^{57}}$;
 \item $\sqrt[7]{a^{71}b^{82}}$;
 \item $\sqrt{a^3}+\sqrt{a^5}+\sqrt{a^7}$.
 \end{enumeratea}
 \end{multicols}
\end{esercizio}

\subsubsection*{2.8 - Potenza di radice e radice di radice}

\begin{esercizio}[\Ast]
 \label{ese:2.58}
Esegui le seguenti potenze di radici.
 \begin{multicols}{4}
 \begin{enumeratea}
 \item $\left(\sqrt 3\right)^2$;
 \item $\left(\sqrt[3]2\right)^3$;
 \item $\left(\sqrt 4\right)^2$;
 \item $\left(\sqrt[4]2\right)^6$;
 \item $\left(2\sqrt 3\right)^2$;
 \item $\left(3\sqrt 5\right)^2$;
 \item $\left(5\sqrt 2\right)^2$;
 \item $\left(-2\sqrt 5\right)^2$;
 \item $\left(\frac 1 2\sqrt 2\right)^2$;
 \item $\left(\frac 2 3\sqrt[4]{\frac 2 3}\right)^2$;
 \item $\left(a\sqrt{2a}\right)^2$;
 \item $\left(\frac 1 a\sqrt a\right)^2$;
 \item $\left(2\sqrt[3]3\right)^3$;
 \item $\left(3\sqrt[3]3\right)^3$;
 \item $\left(\frac 1 3\sqrt[3]3\right)^3$;
 \item $\left(\frac 1 9\sqrt[3]9\right)^3$.
 \end{enumeratea}
 \end{multicols}
\end{esercizio}

\begin{esercizio}[\Ast]
 \label{ese:2.59}
Esegui le seguenti potenze di radici.
 \begin{multicols}{4}
 \begin{enumeratea}
 \item $\left(\sqrt 3\right)^3$;
 \item $\left(2\sqrt 5\right)^3$;
 \item $\left(3\sqrt 2\right)^3$;
 \item $\left(\sqrt[3]2\right)^6$;
 \item $\left(\sqrt[3]3\right)^6$;
 \item $\left(\sqrt[3]5\right)^5$;
 \item $\left(\sqrt[3]2\right)^6$;
 \item $\left(\sqrt[6]3\right)^4$;
 \item $\left(\sqrt[6]{3ab^2}\right)^4$;
 \item $\left(\sqrt[4]{16a^2b^3}\right)^2$;
 \item $\left(\sqrt[3]{6a^3b^2}\right)^4$;
 \item $\left(\sqrt[3]{81ab^4}\right)^4$.
 \end{enumeratea}
 \end{multicols}
\end{esercizio}

\begin{esercizio}[\Ast]
 \label{ese:2.60}
Esegui le seguenti radici di radici.
 \begin{multicols}{4}
 \begin{enumeratea}
 \item $\sqrt[3]{\sqrt 2}$;
 \item $\sqrt[3]{\sqrt[3]{16}}$;
 \item $\sqrt[3]{\sqrt[4]{15}}$;
 \item $\sqrt[5]{\sqrt{a^5}}$;
 \item $\sqrt{\sqrt{16}}$;
 \item $\sqrt{\sqrt{\sqrt 3}}$;
 \item $\sqrt[5]{\sqrt{a^{10}}}$;
 \item $\sqrt[3]{\sqrt{\sqrt[3]{a^{12}}}}$.
 \end{enumeratea}
 \end{multicols}
\end{esercizio}

\begin{esercizio}[\Ast]
 \label{ese:2.61}
Esegui le seguenti radici di radici.
\begin{multicols}{2}
 \begin{enumeratea}
 \item $\sqrt{\sqrt[3]{3a}}$;
 \item $\sqrt{\sqrt[4]{3ab}}$;
 \item $\sqrt[3]{\sqrt{(a+1)^5}}$;
 \item $\sqrt[4]{\sqrt{(2a)^5}}$;
 \item $\sqrt{2(a-b)}\cdot \sqrt{\sqrt[3]{\frac 1{4a-4b}}}$;
 \item $\sqrt{3(a+b)}\cdot \sqrt{\sqrt[3]{\frac 1{3a+3b}}}$.
 \end{enumeratea}
 \end{multicols}
\end{esercizio}
\pagebreak
\begin{esercizio}[\Ast]
 \label{ese:2.62}
Esegui le seguenti radici di radici.
\begin{multicols}{2}
 \begin{enumeratea}
 \item $\sqrt[5]{y^{2}\sqrt[3]{y}}$;
 \item $\sqrt{\frac{2x+1}{\sqrt[5]{(2x+1)^{4}}}}$;
 \item $\sqrt{x\sqrt[3]{x\sqrt{\frac{1}{x^{7}}}}}$;
 \item $\sqrt{(a+1)\sqrt{\frac{1}{a^{2}-1}}}$;
 \item $\sqrt{(a-b)\sqrt[3]{a^{2}-2ab+b^{2}}}$;
 \item $\sqrt{x+y+2\sqrt{xy}}\cdot\sqrt{x+y-2\sqrt{xy}}$.
 \end{enumeratea}
 \end{multicols}
\end{esercizio}

\subsubsection*{2.9 - Somma di radicali}

\begin{esercizio}[\Ast]
 \label{ese:2.63}
Esegui le seguenti operazioni con i radicali.
 \begin{multicols}{2}
 \begin{enumeratea}
 \item $3\sqrt 2+\sqrt 2$;
 \item $\sqrt 3-3\sqrt 3$;
 \item $8\sqrt 6-3\sqrt 6$;
 \item $\sqrt 5-3\sqrt 5+7\sqrt 5$;
 \item $3\sqrt 2+2\sqrt 2-3\sqrt 2$;
 \item $2\sqrt 7-7\sqrt 7+4\sqrt 7$;
 \item $11\sqrt 5+6\sqrt 2-(8\sqrt 5+3\sqrt 2)$;
 \item $5\sqrt 3+3\sqrt 7-[2\sqrt 3-(4\sqrt 7-3\sqrt 3)]$.
 \end{enumeratea}
 \end{multicols}
\end{esercizio}

\begin{esercizio}[\Ast]
 \label{ese:2.64}
Esegui le seguenti operazioni con i radicali.
 \begin{multicols}{2}
 \begin{enumeratea}
 \item $\sqrt 2+\frac 1 2\sqrt 2-\frac 3 4\sqrt 2$;
 \item $\frac{\sqrt 3} 2-\frac{\sqrt 3} 3+\frac{\sqrt 3} 4$;
 \item $3\sqrt 5+\frac 2 3\sqrt 2-\frac 5 6\sqrt 2$;
 \item $5\sqrt{10}-\left(6+4\sqrt{19}\right)+2-\sqrt{10}$;
 \item $\sqrt 5+\sqrt 2+3\sqrt 2-2\sqrt 2$;
 \item $-3\sqrt 7+4\sqrt 2+\sqrt 3-5\sqrt 7+8\sqrt 3$;
 \item $3\sqrt 3+5\sqrt 5+6\sqrt 6-7\sqrt 3-8\sqrt 5-9\sqrt 6$;
 \item $\sqrt[3]2+3\sqrt[3]2-2\sqrt 2+3\sqrt 2$;
 \item $5\sqrt 6+3\sqrt[4]6-2\sqrt[4]6+3\sqrt[3]6-2\sqrt 6$;
 \item $\sqrt{75}+3\sqrt{18}-2\sqrt{12}-2\sqrt{50}$.
 \end{enumeratea}
 \end{multicols}
\end{esercizio}

\begin{esercizio}[\Ast]
 \label{ese:2.65}
Esegui le seguenti operazioni con i radicali.
 \begin{multicols}{2}
 \begin{enumeratea}
 \item $3\sqrt{128}-2\sqrt{72}-(2\sqrt{50}+\sqrt 8)$;
 \item $3\sqrt{48}+2\sqrt{32}+\sqrt{98}-(4\sqrt{27}+\sqrt{450})$;
 \item $\sqrt[4]{162}-\sqrt[4]{32}+5\sqrt[3]{16}-\sqrt[3]{54}+\sqrt[3]{250}$;
 \item $2\sqrt[3]{54}-\sqrt[4]{243}+3\sqrt[4]{48}-\sqrt[3]{250}$;
 \item $\sqrt{\frac{32}{25}}-\sqrt{\frac{108}{25}}+\sqrt{\frac{27}{49}}+\frac 2 5\sqrt{\frac 3 4}-\sqrt{\frac 8 9}$;
 \item $2\sqrt{\frac{27} 8}+5\sqrt{\frac 3{50}}+7\sqrt{\frac{27}{98}}-5\sqrt{\frac{147}{50}}$.
 \end{enumeratea}
 \end{multicols}
\end{esercizio}

\begin{esercizio}[\Ast]
 \label{ese:2.66}
Esegui le seguenti operazioni con i radicali.
 \begin{multicols}{2}
 \begin{enumeratea}
 \item $\frac 1 2\sqrt a-\frac 4 5\sqrt b-\sqrt a+\np{0,4}\sqrt b$;
 \item $\sqrt[3]{a-b}+\sqrt[3]{a^4-a^3b}-\sqrt[3]{{ab}^3-b^4}$;
 \item $3\sqrt x-5\sqrt x$;
 \item $2\sqrt[3]{x^2}+3\sqrt x+3\sqrt[3]{x^2}-2\sqrt x$;
 \item $\sqrt{a-b}+\sqrt{a+b}-\sqrt{a-b}+2\sqrt{a+b}$;
 \item $\frac 1 3\sqrt x-\frac 4 5\sqrt x+\np{0,4}\sqrt a-\frac 1 2\sqrt a$;
 \item $2a\sqrt{2a}-7a\sqrt{2a}+3a\sqrt{2a}-\frac 1 2\sqrt a$;
 \end{enumeratea}
 \end{multicols}
\end{esercizio}

\begin{esercizio}[\Ast]
 \label{ese:2.67}
Esegui le seguenti operazioni con i radicali.
 \begin{enumeratea}
 \item $6\sqrt{{ab}}-3\sqrt a-7\sqrt{{ab}}+2\sqrt a+9\sqrt b+\sqrt a$;
 \item $3\sqrt{xy}+3\sqrt x-3\sqrt y+2\sqrt{xy}-3(\sqrt x+\sqrt y)$;
 \item $3y^{2}\sqrt{x^{2}z}+\frac{2}{z}\sqrt{x^{5}z^{3}}-z^{4}\sqrt{\frac{xz}{y^{2}}}$;
 \item $\sqrt{25+25a^{2}}-\sqrt{64+64a^{2}}+\sqrt{9+9a^{2}}$;
 \item $\sqrt{x^{4}-x^{3}y}-\sqrt[3]{xy^{3}-y^{4}}+\sqrt[3]{x-y}$.
 \end{enumeratea}
\end{esercizio}

\begin{esercizio}[\Ast]
 \label{ese:2.68}
Esegui le seguenti operazioni con i radicali.
 \begin{enumeratea}
 \item $\sqrt{\frac{3}{4}xy}-\sqrt{3xy}-\frac{1}{2}\sqrt{\frac{9xy}{48}}$;
 \item $\sqrt{\frac{a^{2}}{a+b}}+\sqrt{\frac{4b^{2}}{9a-9b}}+\sqrt{\frac{a^{2}-4ab+4b^{2}}{4a+4b}}$;
 \item $\sqrt[3]{x^{7}}-\sqrt[3]{64y^{3}x^{4}}+\sqrt[3]{64y^{6}x}$;
 \item $\sqrt[3]{2x^{3}}-\frac{1}{5}\sqrt[3]{250}-\sqrt[3]{54x^{3}}+\frac{2}{3}\sqrt[3]{54}+\frac{1}{4}\sqrt[3]{16}$;
 \item $\frac{1}{10}\sqrt{\frac{3}{5}}+\sqrt{\frac{12}{5}}-\sqrt{\frac{147}{125}}+\sqrt{\frac{27}{500}}$;
 \item $\sqrt[9]{x^{3}y^{6}}+\sqrt[9]{x^{3}y^{22}}-2\sqrt[9]{x^{3}y^{13}}$.
 \end{enumeratea}
\end{esercizio}

\begin{esercizio}[\Ast]
 \label{ese:2.69}
Esegui le seguenti operazioni con i radicali.
 \begin{multicols}{3}
 \begin{enumeratea}
 \item $(\sqrt 2+1)(\sqrt 2+2)$;
 \item $(3\sqrt 2-1)(2\sqrt 2-3)$;
 \item $(\sqrt 2-1)(\sqrt 2+1)$;
 \item $(\sqrt 2-3\sqrt 3)(3\sqrt 3-\sqrt 2)$;
 \item $(\sqrt 3+1)^2$;
 \item $(\sqrt 3-2)^2$;
 \item $(2+\sqrt 5)^2$;
 \item $(4-\sqrt 3)^2$;
 \item $(6+2\sqrt 3)^2$;
 \item $(\sqrt 6-\frac 1 2\sqrt 3)^2$;
 \item $(\sqrt 2-1)^2$;
 \item $(2\sqrt 2-1)^2$.
 \end{enumeratea}
 \end{multicols}
\end{esercizio}

\begin{esercizio}[\Ast]
 \label{ese:2.70}
Esegui le seguenti operazioni con i radicali.
 \begin{multicols}{3}
 \begin{enumeratea}
 \item $(\sqrt 3+1)^2$;
 \item $(\sqrt 3-3)^2$;
 \item $(\sqrt 5-2)^2$;
 \item $(2\sqrt 5+3)^2$;
 \item $(2\sqrt 7-\sqrt 5)^2$;
 \item $(3\sqrt 2-2\sqrt 3)^2$;
 \item $(\sqrt 2-3\sqrt 3)^2$;
 \item $(1+\sqrt 2+\sqrt 3)^2$;
 \item $(\sqrt 2-1-\sqrt 5)^2$;
 \item $(\sqrt 3-2\sqrt 2+1)^2$;
 \item $(\sqrt 2+\sqrt 3+\sqrt 6)^2$;
 \item $(\sqrt[3]2-1)^3$.
 \end{enumeratea}
 \end{multicols}
\end{esercizio}


\begin{esercizio}[\Ast]
 \label{ese:2.71}
Esegui le seguenti operazioni con i radicali.
 \begin{multicols}{2}
 \begin{enumeratea}
 \item $(\sqrt[3]3+1)^3$;
 \item $(\sqrt[3]2-2)^3$;
 \item $(\sqrt[3]3+\sqrt[3]2)^3$;
 \item $(\sqrt[3]3+\sqrt[3]2)(\sqrt[3]9-\sqrt[3]4)$;
 \item $\left[(\sqrt[4]2+1)(\sqrt[4]2-1)\right]^2$;
 \item $(\sqrt[3]2+\sqrt[3]3)(\sqrt[3]4-\sqrt[3]6+\sqrt[3]9)$;
 \item $(\sqrt 3+\sqrt 3)\sqrt 3 \sqrt 3$;
 \item $3\sqrt 3+\sqrt 3:\sqrt 3-(1+\sqrt 3)^2$;
 \item $6\sqrt 5+2\sqrt 5\cdot \sqrt{20}-3\sqrt 5+\sqrt{25}$;
 \item $(\sqrt[3]a-\sqrt[3]2)(\sqrt[3]{a^2}+\sqrt[3]{2a}+\sqrt[3]4)$;
 \item $(1+\sqrt 2)^2$;
 \item $(2-\sqrt 2)^2$.
 \end{enumeratea}
 \end{multicols}
\end{esercizio}

\begin{esercizio}[\Ast]
 \label{ese:2.72}
Esegui le seguenti operazioni con i radicali.
 \begin{multicols}{3}
 \begin{enumeratea}
 \item $(\sqrt 2+\sqrt 3)^2$;
 \item $(2\sqrt 2-1)^2$;
 \item $(3\sqrt 3+2\sqrt 2)^2$;
 \item $\left(\sqrt 3-2\sqrt 2\right)^2$;
 \item $(4\sqrt 3-3\sqrt 7)^2$;
 \item $(2\sqrt 2-3\sqrt 3)^2$;
 \item $(\sqrt x-1)^2$;
 \item $(2x+\sqrt x)^2$;
 \item $(x+\sqrt[3]x)^3$;
 \item $(2x+\sqrt x)(2x-\sqrt x)$;
 \item $\left(\sqrt a+\frac 1{\sqrt a}\right)^2$;
 \item $\left(\sqrt a+\frac 1 a\right)\left(\sqrt a-\frac 1 a\right)$.
 \end{enumeratea}
 \end{multicols}
\end{esercizio}

\begin{esercizio}[\Ast]
 \label{ese:2.73}
Esegui le seguenti operazioni con i radicali.
 \begin{enumeratea}
 \item $(\sqrt x+\sqrt y)(\sqrt x-\sqrt y)$;
 \item $(\sqrt 2-1)^2-(2\sqrt 2-1)^2+(\sqrt 2-1)(\sqrt 2+1)$;
 \item $(\sqrt 3+1)^2+\sqrt 3(\sqrt 3-3)-2(\sqrt 3+3)(\sqrt 3-3)$;
 \item $(\sqrt 3-3)^2+(\sqrt 3-3)^3+2\sqrt{27}-\sqrt 3(2\sqrt 3-2)$;
 \item $(\sqrt 5-2)^2-(2\sqrt 5+3)^2+\left[(\sqrt 5-\sqrt 2)^2+1\right](\sqrt 5+\sqrt 2)$;
 \item $(2\sqrt 7-\sqrt 5)^2+2(\sqrt 7+\sqrt 5+1)^2-\sqrt{35}$;
 \item $(\sqrt 2+1)^2+(\sqrt 2-1)^2$;
 \item $(2\sqrt 2-3\sqrt 3)(3\sqrt 2+2\sqrt 3)$.
 \end{enumeratea}
\end{esercizio}

\begin{esercizio}
 \label{ese:2.74}
Esegui le seguenti operazioni con i radicali.
 \begin{multicols}{2}
 \begin{enumeratea}
 \item $(\sqrt x-1)^2+(2\sqrt x+1)(\sqrt x-2)$;
 \item $(\sqrt 2-1)^3+(\sqrt 2-1)^2\sqrt 2-1$;
 \item $2\sqrt{54}-\sqrt[4]{243}+3\sqrt[4]{48}-\sqrt[3]{250}$;
 \item $(\sqrt{10}-\sqrt 7)(2\sqrt{10}+3\sqrt 7)$;
 \item $\sqrt{48x^2y}+5x\sqrt{27y}$;
 \item $\sqrt 5\sqrt{15}-4\sqrt 3$;
 \item $(\sqrt 7-\sqrt 5)(2\sqrt 7+3\sqrt 5)$;
 \item $\sqrt{27ax^4}+5x^2\sqrt{75a}$.
 \end{enumeratea}
 \end{multicols}
\end{esercizio}

\begin{esercizio}[\Ast]
 \label{ese:2.75}
Esegui le seguenti operazioni con i radicali.
 \begin{enumeratea}
 \item $\sqrt{125}+3\sqrt[6]{27}-\sqrt{45}-2\sqrt[4]9+\sqrt{20}+7\sqrt[8]{81}$;
 \item $\sqrt[3]{a\sqrt a}\cdot \sqrt{a\sqrt[3]a}\cdot \sqrt[3]{a\sqrt[3]a}\cdot \sqrt[3]{a\sqrt a}\cdot \sqrt[9]{a^8}$;
 \item $\sqrt[5]{b\sqrt[3]{b^2}}\cdot \sqrt{b^2\sqrt{b\sqrt{b^2}}}:\sqrt[5]{b^4\sqrt[3]{b^2}}\cdot \sqrt b$;
 \item $\sqrt[3]{\frac x{y^3}-\frac 1{y^2}}+\sqrt[3]{xy^3-y^4}-\sqrt[3]{8x-8y}$;
 \item $(\sqrt 2+3)\cdot (1-\sqrt 3)^2$;
 \item $(\sqrt[3]2+3)\cdot (1-\sqrt[3]3)^2$;
 \item $\frac{\sqrt a}{\sqrt a+1}\cdot \frac{\sqrt a}{\sqrt a-1}$;
 \item $\sqrt[5]{b\sqrt[3]{b^2}}\cdot \sqrt{b\sqrt{b\sqrt{b^2}}}:\left(\sqrt[5]{b\sqrt[3]{b^2}}\cdot \sqrt b\right)$.
 \end{enumeratea}
\end{esercizio}

\begin{esercizio}[\Ast]
 \label{ese:2.76}
Esegui le seguenti operazioni con i radicali.
 \begin{multicols}{2}
 \begin{enumeratea}
 \item $3\sqrt[6]{\frac{4x^{2}}{y^{2}}}-2\sqrt[9]{\frac{8x^{3}}{y^{3}}}+2\sqrt[3]{\frac{2x}{y}}$;
 \item $(3\sqrt{2}-2)^{2}-\sqrt{2}(6+2\sqrt{6})+(1+2\sqrt{3})^{2}$;
 \item $a\sqrt[3]{\frac{b^{2}c^{2}}{a^{2}}}\sqrt{\frac{b}{ac}}$;
 \item $\sqrt[20]{a^{9}b^{2}}\sqrt[12]{a^{3}b^{4}}\sqrt[15]{a^{5}b}$;
 \item $\left(\sqrt{1+x}+\sqrt{x-2}\right)\left(\sqrt{1+x}-\sqrt{x-2}\right)$;
 \item $\left(5+3\sqrt{2}\right)\left(2\sqrt{3}-\sqrt{2}+1\right)$;
 \item $\sqrt[3]{\frac{x+y}{x-y}}\sqrt[4]{\frac{x-y}{x+y}}:\sqrt[6]{\frac{x+y}{x-y}}$;
 \item $\left(x+\sqrt{y}\right)^{3}-\left(x-\sqrt{y}\right)^{3}$;
 \item $\sqrt[9]{\frac{32x^{5}}{27y^{3}}\sqrt[4]{\frac{27y^{3}}{4x^{2}}}}$;
 \item $\sqrt{\frac{1}{2ab+1}}\sqrt{9a^{2}b^{2}x^{4}18a^{3}b^{3}x^{4}}$.
 \end{enumeratea}
 \end{multicols}
\end{esercizio}

\begin{esercizio}[\Ast]
 \label{ese:2.77}
Esegui le seguenti operazioni con i radicali.
 \begin{multicols}{2}
 \begin{enumeratea}
 \item $\sqrt{\frac{4a^2-b^2}{a^2-b^2}}\sqrt{\frac{a-b}{2a+b}}$;
 \item $\sqrt{\frac {9a}{b}}\sqrt{\frac{b^2-2b}{3ab-6a}}$;
 \item $\sqrt{\frac{9a^2-6ab+b^2}{a^2-b^2}}\sqrt{\frac{a+b}{3a-b}}$;
 \item $\sqrt{\frac{x-y}{x+y}}\sqrt{\frac{x^2+2xy+y^2}{x^2-y^2}}$;
 \item $\sqrt[3]{\frac a{a+3}\sqrt{\frac a{a+3}\sqrt{\frac a{a+3}}}}:\sqrt{\frac a{a+3}}$;
 \item $\sqrt{\frac{x-1}{x+1}\sqrt{\frac{x-1}{x+1}\sqrt{\frac 1{x-1}}}}\cdot \sqrt[4]{x+1}$.
 \end{enumeratea}
 \end{multicols}
\end{esercizio}
\pagebreak
\begin{esercizio}[\Ast]
 \label{ese:2.78}
Esegui le seguenti operazioni con i radicali.
 \begin{enumeratea}
 \item $\sqrt{\frac{a^2-2a+1}{a(a+1)^3}}\cdot \sqrt[4]{\frac{a^2}{(a+1)^2}}\cdot \sqrt[3]{\frac{(a+1)^3}{(a-1)^2}}$;
 \item $\left(\sqrt{\frac 1{b^4}+\frac 1{b^3}}+\sqrt{\frac{ab^5+ab^4} a}-2\sqrt{b+1}\right)\cdot \frac{b^2}{(b+1)^2}$;
 \item $\left(\sqrt[3x]{y^x\sqrt[4x]y}+\sqrt[6]{y^2\sqrt[2x^2]y}\right)\cdot \sqrt[3]{y\sqrt[4x^2]{\frac 1 y}}$;
 \item $\sqrt[4]{\frac{b^2-1} b}\cdot \sqrt[3]{\frac{3b-3}{6b^2}}:\sqrt[6]{\frac{(b-1)^4}{4b^5}}$;
 \item $\sqrt[3]{\frac{a^2+2a+1}{ab-b}}\cdot \sqrt[6]{\frac{a^2-2a+1}{ab+b}}\cdot \sqrt[4]{\frac{b^2(a-1)^2}{2a^2+4a+2}}$;
 \item $\sqrt[3]{\frac{x^2+2xy+y^2}{x+3}}\cdot \sqrt[3]{\frac{5x}{x^2+6x+9}}\cdot \sqrt[3]{\frac{x+y}{5x}}$.
 \end{enumeratea}
\end{esercizio}

\begin{esercizio}[\Ast]
 \label{ese:2.79}
Esegui le seguenti operazioni con i radicali.
 \begin{enumeratea}
 \item $\sqrt[3]{\frac{x^2-x}{x+1}}\cdot \sqrt[15]{\frac{x^2+2x+1}{x^2-2x+1}}:\sqrt[5]{\frac{x-1}{x+1}}$;
 \item $\sqrt{\frac{25x^3+25x^2}{y^3-y^2}}+\sqrt{\frac{x^3+x^2}{y^3-y^2}}-x\sqrt{\frac{4x+4}{y^3-y^2}}$;
 \item $\left(\sqrt{\frac 1{y^4}+\frac 1{y^3}}+\sqrt{\frac{xy^5+xy^4} x}-2\sqrt{y+1}\right):\frac{(y+1)^2}{y^2}$;
 \item $\sqrt[4]{\frac{a^2-a}{(a+1)^2}}\cdot \sqrt[12]{\frac{a^2-2a+1}{(a-1)^7}}:\sqrt[3]{\frac{2a^2-2a+1}{a^3-a^2}-\frac 1{a-1}}$;
 \item $\sqrt{\frac{a^2b+ab^2}{xy}}\cdot \sqrt[6]{\frac{(a+b)^2}{x^2}}\cdot \sqrt[6]{\frac{x^2y^3}{(a+b)^2}\cdot \sqrt[4]{\frac x{a^3b^2+a^2b^3}}}$;
 \item $\sqrt[6]{\frac 1 x+4x-4}\cdot \sqrt[3]{\frac 1 x+4x+4}\cdot \sqrt{\frac x{4x^2-1}}$.
 \end{enumeratea}
\end{esercizio}

\begin{esercizio}[\Ast]
 \label{ese:2.80}
Esegui le seguenti operazioni con i radicali.
 \begin{enumeratea}
 \item $\sqrt{\frac{a^2-2a+1}{a(a+1)^3}}\cdot \sqrt[4]{\frac{a^2}{(a+1)^2}}\cdot \sqrt[3]{\frac{(a+1)^2}{(a-1)^2}}$;
 \item $\left(\sqrt[3]{\frac a 3-2+\frac 3 a}\cdot \sqrt[6]{\frac{9a^2(a+3)^3}{(a-3)^2}}\right):\sqrt{\frac{a^2-9}{3a}}$;
 \item $\sqrt[4]{\frac{a^3-a^2}{(a+1)^2}}\cdot \sqrt[12]{\frac{a^2-2a+1}{(a-1)^7}}\cdot \sqrt[3]{\frac{2a^2-2a+1}{a^3-a^2}-\frac 1{a-1}}$;
 \item $\sqrt{1-\frac 1 y+\frac 1{4y^2}}:\left(\sqrt[6]{\frac 1{8y^3+12y^2+6y+1}}\cdot \sqrt{1-\frac 1{4y^2}}\right)$;
 \item $\sqrt[3]{1-\frac 1 a+\frac 1{4a^2}}:\left(\sqrt{1-\frac 1{4a^2}}\cdot \sqrt[6]{\frac 1{8a^3+12a^2+6a+1}}\right)$;
 \item $\sqrt{\frac 1{5a}+\frac 1{25a^2}}+\sqrt{\frac{25a^2-1}{20a^3-4a^2}}-\sqrt{\frac{5a+1}{100a^2}}$.
 \end{enumeratea}
\end{esercizio}

\begin{esercizio}[\Ast]
 \label{ese:2.81}
Esegui le seguenti operazioni con i radicali.
 \begin{enumeratea}
 \item $\sqrt[3]{\frac x{y^3}-\frac 1{y^2}}+\sqrt[3]{xy^3-y^4}-\sqrt[3]{8x-8y}$;
 \item $\sqrt{\frac{x^2+xy+y^2}{4x^2}}+\sqrt{\frac{4x^3-4y^3}{x-y}}+\sqrt{4x^4+4x^3y+4x^2y^2}$;
 \item $\sqrt{\frac{a^3+2a^2+a}{a^2+6a+9}}+\sqrt{\frac{a^3+4a^2+4a}{a^2+6a+9}}-\sqrt{\frac{a^3}{a^2+6a+9}}$;
 \item $\sqrt{4x-12y}+\sqrt{\frac{x^3-3x^2y}{y^2}}+\sqrt{\frac{xy^2-3y^3}{x^2}}$;
 \item $\left(\sqrt[6]{\frac 1{x^2-2x+1}}+\sqrt[6]{\frac{64a^6}{x^2-2x+1}}+\sqrt[6]{\frac{a^{12}}{x^2-2x+1}}\right)\cdot \sqrt[3]{x-1}$;
 \item $\left(\sqrt[3x]{y^x\sqrt[4x]y}+\sqrt[6]{y^2\sqrt[2x^2]y}\right)\cdot \sqrt[4x^2]{\frac 1 y}$.
 \end{enumeratea}
\end{esercizio}

\begin{esercizio}[\Ast]
 \label{ese:2.82}
Esegui le seguenti operazioni con i radicali.
 \begin{enumeratea}
 \item $\sqrt{\frac{x^{3}-xy^{2}+x^{2}y-y^{3}}{x+y}}\cdot\sqrt{\frac{x-y}{(x-y)^{2}}}$;
 \item $\sqrt{\frac{a^{2}-b^{2}}{3a^{2}+5b^{2}}}\cdot\sqrt{\frac{2a+b}{a-b}+\frac{a-3b}{a+b}}$;
 \item $\frac{\sqrt{2}+2\sqrt{a}}{2\sqrt{a}-\sqrt{2}}-\frac{2\sqrt{a}-\sqrt{2}}{2\sqrt{a}+\sqrt{2}}-\frac{8\sqrt{2a}}{4a-2}$;
 \item $\sqrt[3]{\frac{(x+y)^{2}}{x^{2}+y^{2}-2xy}}\cdot\sqrt{\frac{x-y}{x+y}}$;
 \item $\left(\sqrt[3]{\frac{x+1}{\sqrt{3x+2}+\sqrt{2x+1}}+\sqrt{2x+1}}\right)^{2}:\sqrt{\frac{1}{3x+2}\sqrt[3]{9x^{2}+x+4}}$;
 \item $\frac{\sqrt{x}-3\sqrt[4]{x}+2}{\sqrt{x}-4\sqrt[4]{x}+4}\cdot\frac{\sqrt{x}-\sqrt[4]{x}-2}{\sqrt{x}-1}$.
 \end{enumeratea}
\end{esercizio}

\begin{esercizio}[\Ast]
 \label{ese:2.83}
Esegui trasformando i radicali in potenze con esponente frazionario.
 \begin{multicols}{2}
 \begin{enumeratea}
 \item $\sqrt{a\sqrt[3]{a\sqrt[3]{a^2}}}\cdot \sqrt[3]{a\sqrt[3]{\frac 1 a}}:\sqrt{\frac 1 a}$;
 \item $\sqrt[5]{a\sqrt{a^3}}\cdot \sqrt{a\sqrt[7]{\frac 1{a^2}}}:\sqrt[7]{a^4\sqrt a}$;
 \item $\sqrt[3]{a\sqrt a}\cdot \sqrt[3]{a\sqrt[3]a}\cdot \sqrt{a\sqrt[3]a}\cdot \sqrt[3]{a\sqrt a}$;
 \item $\sqrt[5]{b\sqrt[3]{b^2}}\cdot \sqrt{b^2\sqrt{b\sqrt{b^2}}}:\sqrt[5]{b^4\sqrt[3]{b^2}}\cdot \sqrt b$.
 \end{enumeratea}
 \end{multicols}
\end{esercizio}

\subsubsection*{2.10 - Razionalizzazione del denominatore di una frazione}

\begin{esercizio}[\Ast]
 \label{ese:2.84}
Razionalizza i seguenti radicali.
 \begin{multicols}{4}
 \begin{enumeratea}
 \item $\frac 1{\sqrt 3}$;
 \item $\frac 2{\sqrt 2}$;
 \item $\frac 5{\sqrt{10}}$;
 \item $\frac{10}{\sqrt 5}$;
 \item $-\frac 2{\sqrt 3}$;
 \item $\frac 4{2\sqrt 2}$;
 \item $\frac 3{\sqrt{27}}$;
 \item $\frac 4{\sqrt 8}$;
 \item $-\frac{10}{5\sqrt 5}$;
 \item $\frac 2{3\sqrt 6}$;
 \item $-\frac 3{4\sqrt 5}$;
 \item $\frac 1{\sqrt{50}}$.
 \end{enumeratea}
 \end{multicols}
\end{esercizio}

\begin{esercizio}
 \label{ese:2.85}
Razionalizza i seguenti radicali.
 \begin{multicols}{4}
 \begin{enumeratea}
 \item $\frac 9{\sqrt{18}}$;
 \item $\frac 7{\sqrt{48}}$;
 \item $\frac 3{\sqrt{45}}$;
 \item $\frac 5{\sqrt{125}}$;
 \item $\frac 6{5\sqrt{120}}$;
 \item $\frac 1{3\sqrt{20}}$;
 \item $\frac{\sqrt 2}{5\sqrt{50}}$;
 \item $3\frac{\sqrt 3}{2\sqrt{324}}$;
 \item $\frac 2{\sqrt{2\sqrt 2}}$;
 \item $\frac a{\sqrt a}$;
 \item $\frac x{\sqrt x}$;
 \item $\frac{ax}{\sqrt{2a}}$.
 \end{enumeratea}
 \end{multicols}
\end{esercizio}

\begin{esercizio}[\Ast]
 \label{ese:2.86}
Razionalizza i seguenti radicali.
 \begin{multicols}{4}
 \begin{enumeratea}
 \item $\frac{2a}{\sqrt 2}$;
 \item $\frac a{2\sqrt a}$;
 \item $\frac x{3\sqrt{2x}}$;
 \item $\frac{x^2}{a\sqrt x}$;
 \item $\frac{3x}{\sqrt{12x}}$;
 \item $\frac{1+\sqrt 2}{\sqrt 2}$;
 \item $\frac{2-\sqrt 2}{\sqrt 2}$;
 \item $\frac{\sqrt 2+\sqrt 3}{\sqrt 3}$;
 \item $\frac{\sqrt 2-\sqrt 3}{\sqrt 6}$;
 \item $\frac{\sqrt 3+2}{2\sqrt 3}$;
 \item $\frac{\sqrt 3-1}{3\sqrt 3}$;
 \item $\frac{\sqrt 6+2\sqrt 3}{\sqrt 3}$.
 \end{enumeratea}
 \end{multicols}
\end{esercizio}
\pagebreak
\begin{esercizio}[\Ast]
 \label{ese:2.87}
Razionalizza i seguenti radicali.
 \begin{multicols}{4}
 \begin{enumeratea}
 \item $\frac{\sqrt 5-5\sqrt 2}{\sqrt{10}}$;
 \item $\frac{\sqrt{16}+\sqrt{40}}{\sqrt 8}$;
 \item $\frac{\sqrt{10}+\sqrt{20}}{2\sqrt 5}$;
 \item $\frac{9-\sqrt 2}{\sqrt 2}$;
 \item $\frac{3a-\sqrt 3}{2\sqrt 5}$;
 \item $\frac{a^2-b^2}{\sqrt{a+b}}$;
 \item $\frac{\sqrt{x-y}}{\sqrt{x^2-y^2}}$;
 \item $\frac x{\sqrt{2x+1}}$;
 \item $\frac 1{\sqrt[3]2}$;
 \item $\frac 2{\sqrt[3]4}$;
 \item $\frac 3{\sqrt[3]5}$;
 \item $\frac 4{\sqrt[3]6}$.
 \end{enumeratea}
 \end{multicols}
\end{esercizio}

\begin{esercizio}
 \label{ese:2.88}
Razionalizza i seguenti radicali.
 \begin{multicols}{4}
 \begin{enumeratea}
 \item $\frac 1{\sqrt[3]2}$;
 \item $\frac 2{\sqrt[3]4}$;
 \item $\frac 3{\sqrt[3]5}$;
 \item $\frac 4{\sqrt[3]6}$;
 \item $\frac 2{3\sqrt[3]2}$;
 \item $\frac 6{5\sqrt[3]{100}}$;
 \item $\frac 2{\sqrt[5]9}$;
 \item $\frac 3{2\sqrt[6]{27}}$;
 \item $\frac{10}{\sqrt[5]{125}}$;
 \item $\frac{16}{\sqrt[3]{36}}$;
 \item $\frac 9{\sqrt[4]{\np{2025}}}$;
 \item $\frac 1{\sqrt[5]{144}}$.
 \end{enumeratea}
 \end{multicols}
\end{esercizio}

\begin{esercizio}[\Ast]
 \label{ese:2.89}
Razionalizza i seguenti radicali.
 \begin{multicols}{4}
 \begin{enumeratea}
 \item $\frac{ab}{\sqrt[3]{a^2b}}$;
 \item $\frac{ab^2}{\sqrt[3]{ab^2}}$;
 \item $\frac{3a^2b}{\sqrt[4]{9ab^3}}$;
 \item $\frac{2\sqrt a}{\sqrt[4]{27ab^2c^5}}$;
 \item $\frac{5x}{\sqrt[3]{x\sqrt 5}}$;
 \item $\frac{2\sqrt 2}{\sqrt[5]{16a^2b^3c^4}}$;
 \item $\frac{\sqrt[3]{x^2y}+\sqrt[3]{xy^2}}{\sqrt[3]{xy}}$;
 \item $\frac{3-a\sqrt[3]9}{\sqrt[3]{9a}}$;
 \item $\frac{1-\sqrt[3]a}{\sqrt[3]{4a^2x}}$;
 \item $\frac 1{\sqrt 3+\sqrt 2}$;
 \item $\frac 1{\sqrt 2-\sqrt 3}$;
 \item $\frac 2{\sqrt 3+\sqrt 5}$.
 \end{enumeratea}
 \end{multicols}
\end{esercizio}

\begin{esercizio}[\Ast]
 \label{ese:2.90}
Razionalizza i seguenti radicali.
 \begin{multicols}{4}
 \begin{enumeratea}
 \item $\frac{2\sqrt 2}{\sqrt 5+\sqrt 7}$;
 \item $\frac 3{\sqrt 2+1}$;
 \item $\frac 2{\sqrt 2-1}$;
 \item $\frac{\sqrt 3+1}{\sqrt 3-1}$;
 \item $\frac{2+\sqrt 3}{\sqrt 3+\sqrt 2}$;
 \item $\frac 3{2+3\sqrt 3}$;
 \item $\frac x{\sqrt x+1}$;
 \item $\frac 1{\sqrt x+\sqrt y}$;
 \item $\frac{\sqrt x}{\sqrt x-\sqrt y}$;
 \item $\frac{a+b}{\sqrt a+\sqrt{ab}}$;
 \item $\frac x{\sqrt y-\sqrt{x+y}}$;
 \item $\frac{\sqrt 2-1}{\sqrt{3-\sqrt 3}}$.
 \end{enumeratea}
 \end{multicols}
\end{esercizio}

\begin{esercizio}
 \label{ese:2.91}
Razionalizza i seguenti radicali.
 \begin{multicols}{4}
 \begin{enumeratea}
 \item $\frac 1{\sqrt{\sqrt 2}+1}$;
 \item $\frac 7{\sqrt{7+2\sqrt 6}}$;
 \item $\frac{a-2}{\sqrt a-2}$;
 \item $\frac{a-x}{\sqrt a-2\sqrt x}$;
 \item $\frac{x+1}{\sqrt{x(x+1)}}$;
 \item $\frac 4{\sqrt 5+\sqrt 3-\sqrt 2}$;
 \item $\frac{-3}{\sqrt 2-\sqrt 3+1}$;
 \item $\frac 2{2\sqrt 3-3\sqrt 2+2}$;
 \item $\frac{(a+b)^2}{\sqrt a+\sqrt b-\sqrt{ab}}$;
 \item $\frac 3{\sqrt[3]2+\sqrt[3]9}$;
 \item $\frac 6{\sqrt[3]3-\sqrt[3]5}$;
 \item $\frac{\sqrt 6}{\sqrt[3]4+\sqrt[3]9}$.
 \end{enumeratea}
 \end{multicols}
\end{esercizio}

\begin{esercizio}[\Ast]
 \label{ese:2.92}
Razionalizza i seguenti radicali.
 \begin{multicols}{3}
 \begin{enumeratea}
 \item $\frac{\sqrt 2}{2\sqrt[3]2-3\sqrt[3]3}$;
 \item $\frac{\sqrt 2+1}{\sqrt[3]2-1}$;
 \item $\frac 3{\sqrt[3]4-\sqrt[3]2}$;
 \item $\frac{a-4b^2}{\sqrt a-2b}$;
 \item $\frac 2{\sqrt[3]2-1}$;
 \item $\frac{\sqrt a}{\sqrt a+1}$;
 \item $\frac{a-b}{\sqrt a+\sqrt b}$;
 \item $\frac 1{\sqrt a-\sqrt b}+\frac{3\sqrt a-\sqrt b}{a-b}$;
 \item $\frac{\sqrt 5}{\sqrt 5+\sqrt 2+\sqrt 3}$;
 \item $\frac{1-\sqrt 2}{1+\sqrt 2-\sqrt 3}$;
 \item $\frac{\sqrt 2+\sqrt 3+\sqrt 5}{\sqrt 5-\sqrt 2+\sqrt 3}$;
 \item $\frac{a+2\sqrt{\mathit{ab}}+b}{\sqrt a+\sqrt b}$.
 \end{enumeratea}
 \end{multicols}
\end{esercizio}
\pagebreak
\begin{esercizio}[\Ast]
 \label{ese:2.93}
Razionalizza i seguenti radicali.
 \begin{multicols}{3}
 \begin{enumeratea}
 \item $\frac{18\sqrt{2}}{\sqrt{6}}$;
 \item $\frac{3}{2\sqrt[3]{3}}$;
 \item $\frac{7}{3\sqrt{2}+2}$;
 \item $\frac{a}{\sqrt{a}+a}$;
 \item $\frac{30}{3+\sqrt[3]{3}}$;
 \item $\frac{2\sqrt{6}+1}{\sqrt{3}+\sqrt{2}-2}$.
 \end{enumeratea}
 \end{multicols}
\end{esercizio}

\begin{esercizio}[\Ast]
 \label{ese:2.94}
Razionalizza i seguenti radicali.
 \begin{multicols}{3}
 \begin{enumeratea}
 \item $\frac{1}{\sqrt[3]{(x+3)^{2}}}$;
 \item $\frac{x+y}{\sqrt[6]{x^{2}}+\sqrt[6]{y^{2}}}$;
 \item $\frac{4\sqrt{4+x^{2}}}{\sqrt{4+x^{2}}-\sqrt{x^{2}-4}}$;
 \item $\frac{\sqrt{x}}{\sqrt{x^{2}-\sqrt{x}}-x}$;
 \item $\frac{2}{\sqrt{2+x}-\sqrt{x}}$;
 \item $\frac{6+\sqrt{x^{2}-9}}{6-\sqrt{x^{2}-9}}$.
 \end{enumeratea}
 \end{multicols}
\end{esercizio}

\subsubsection*{2.11 - Radicali doppi}

\begin{esercizio}[\Ast]
 \label{ese:2.95}
Riduci a radicali semplici i seguenti radicali doppi (ove possibile).
 \begin{multicols}{4}
 \begin{enumeratea}
 \item $\sqrt{12-\sqrt{23}}$;
 \item $\sqrt{12+2\sqrt 5}$;
 \item $\sqrt{15+\sqrt{29}}$;
 \item $\sqrt{3+\sqrt 5}$;
 \item $\sqrt{3-\sqrt 8}$;
 \item $\sqrt{4+2\sqrt 3}$;
 \item $\sqrt{4-\sqrt 7}$;
 \item $\sqrt{5+\sqrt{21}}$;
 \item $\sqrt{6+4\sqrt 2}$;
 \item $\sqrt{6-3\sqrt 3}$;
 \item $\sqrt{6+2\sqrt 5}$;
 \item $\sqrt{6-\sqrt{11}}$.
 \end{enumeratea}
 \end{multicols}
\end{esercizio}

\begin{esercizio}[\Ast]
 \label{ese:2.96}
Riduci a radicali semplici i seguenti radicali doppi (ove possibile).
 \begin{multicols}{4}
 \begin{enumeratea}
 \item $\sqrt{7+3\sqrt 5}$;
 \item $\sqrt{7+2\sqrt{10}}$;
 \item $\sqrt{7-\sqrt{33}}$;
 \item $\sqrt{7+2\sqrt 6}$;
 \item $\sqrt{7-\sqrt{13}}$;
 \item $\sqrt{8+2\sqrt{15}}$;
 \item $\sqrt{8-\sqrt{55}}$;
 \item $\sqrt{8+4\sqrt 3}$.
 \end{enumeratea}
 \end{multicols}
\end{esercizio}

\begin{esercizio}
 \label{ese:2.97}
Riduci a radicali semplici i seguenti radicali doppi (ove possibile).
 \begin{multicols}{4}
 \begin{enumeratea}
 \item $\sqrt{8-\sqrt{39}}$;
 \item $\sqrt{8-4\sqrt 7}$;
 \item $\sqrt{8+\sqrt{15}}$;
 \item $\sqrt{5+2\sqrt 6}$;
 \item $\sqrt{\frac{15} 2-\sqrt{\frac{86} 9}}$;
 \item $\sqrt{\frac 5 2-\sqrt 6}$;
 \item $\sqrt{\frac 8 5-\sqrt{\frac 7 4}}$;
 \item $\sqrt{10+\sqrt{19}}$.
 \end{enumeratea}
 \end{multicols}
\end{esercizio}

\begin{esercizio}[\Ast]
 \label{ese:2.98}
Riduci a radicali semplici i seguenti radicali doppi (ove possibile).
 \begin{multicols}{2}
 \begin{enumeratea}
 \item $\sqrt{x+3+2\sqrt{x+2}}$;
 \item $\sqrt{a+1-2\sqrt{a}}$;
 \item $\sqrt{a+b+1+\sqrt{4(a+b)}}$;
 \item $\sqrt{a+2-2\sqrt{a+1}}$;
 \item $\sqrt{2a+1+\sqrt{1+4a}}$;
 \item $\sqrt{a+\sqrt{ab-\frac{b^{2}}{4}}}$.
 \end{enumeratea}
 \end{multicols}
\end{esercizio}

\subsubsection*{2.12 - Equazioni, disequazioni, sistemi}

\begin{esercizio}[\Ast]
 \label{ese:2.99}
Semplifica le seguenti espressioni a coefficienti irrazionali.
 \begin{enumeratea}
 \item $\frac{\sqrt{x+2y+\frac{y^{2}}{x}}}{\sqrt{\frac{1}{x}+x+2}}$;
 \item $\left(x\sqrt{\frac{y}{x}}+y\sqrt{\frac{x}{y}}\right)\left(x\sqrt{\frac{y}{x}}-y\sqrt{\frac{x}{y}}\right)\sqrt{xy}$;
 \item $\frac{\sqrt[4]{x+y+2\sqrt{xy}}\cdot\sqrt[4]{x+y-2\sqrt{xy}}}{\sqrt{x-y}}$;
 \item $\frac{2-\sqrt{x}}{2+\sqrt{x}}-\frac{2+\sqrt{x}}{2-\sqrt{x}}$;
 \item $\frac{\sqrt{x-\sqrt{x^{2}-1}}}{\sqrt{x+\sqrt{x^{2}-1}}}+\frac{x^{2}-1}{\sqrt{x^{2}-1}}$;
 \item $\sqrt{2b}\left(\sqrt{\frac{2}{b}}-\sqrt{\frac{b}{2}} \right)+\sqrt{2b}\left(\sqrt{\frac{b}{2}}-\sqrt{\frac{2}{b}} \right)$.
 \end{enumeratea}
\end{esercizio}

\begin{esercizio}[\Ast]
 \label{ese:2.100}
Risolvi le seguenti equazioni a coefficienti irrazionali.
 \begin{multicols}{2}
 \begin{enumeratea}
 \item $\sqrt 2x=2$;
 \item $\sqrt 2x=\sqrt{12}$;
 \item $2x=\sqrt 6$;
 \item $\sqrt 2x=\sqrt 6+\sqrt{14}$;
 \item $x-\sqrt 3=2\left(x-\sqrt 3\right)$;
 \item $2\sqrt 3x-\sqrt 2=\sqrt 2$;
 \item $2x+\sqrt 5=\sqrt 5x+2$;
 \item $(1+\sqrt 2)x=\sqrt 2(1-\sqrt 2)$.
 \end{enumeratea}
 \end{multicols}
\end{esercizio}

\begin{esercizio}[\Ast]
 \label{ese:2.101}
Risolvi le seguenti equazioni a coefficienti irrazionali.
 \begin{multicols}{2}
 \begin{enumeratea}
 \item $\frac{1-x}{\sqrt 2}-\frac x{\sqrt 8}=x-\sqrt 2$;
 \item $2x-\left(x+\sqrt 3\right)\sqrt 2=2x+3\sqrt 5$;
 \item $\frac{x+1}{\sqrt 2}+\frac{x+\sqrt 2}{\sqrt 2}=\frac{x-1} 2$;
 \item $\frac{x+\sqrt 2}{x-\sqrt 2}+\frac{x-\sqrt 2}{x+\sqrt 2}=2$;
 \item $(x+\sqrt 2)^2-(x+\sqrt 3)^2=6$.
 \end{enumeratea}
 \end{multicols}
\end{esercizio}

\begin{esercizio}[\Ast]
 \label{ese:2.102}
Risolvi le seguenti equazioni a coefficienti irrazionali.
 \begin{multicols}{2}
 \begin{enumeratea}
 \item $\frac{x-\sqrt 3} 2-\frac{\sqrt 2-3x} 4=2x$;
 \item $2(x-1)^2-\sqrt 2x=1+2x(x-2)$;
 \item $\frac{\sqrt 3}{3x-6}-\frac 1{20-10x}=\sqrt 3+2$;
 \item $\frac{3x-2}{\sqrt 8x-\sqrt{32}}+\frac{5x}{4\sqrt 3x-8\sqrt 3}=0$.
 \end{enumeratea}
 \end{multicols}
\end{esercizio}

\begin{esercizio}[\Ast]
 \label{ese:2.103}
Risolvi le seguenti disequazioni a coefficienti irrazionali.
 \begin{multicols}{2}
 \begin{enumeratea}
 \item $4x+\sqrt 2<2x-\sqrt 2$;
 \item $(\sqrt 3+1)-(\sqrt 3+\sqrt 2x)<3\sqrt 2$;
 \item $x\sqrt 2+\sqrt 5>\sqrt{10}$;
 \item $3(x-\sqrt 3)<2(x+\sqrt 3)-\sqrt 6$;
 \item $\frac{x-\sqrt 2} 2\le \frac{2x-\sqrt 3}{\sqrt 2}$.
 \end{enumeratea}
 \end{multicols}
\end{esercizio}

\begin{esercizio}[\Ast]
 \label{ese:2.104}
Risolvi i seguenti sistemi di disequazioni a coefficienti irrazionali.
 \begin{multicols}{2}
 \begin{enumeratea}
 \item $\left\{\begin{array}{l}\sqrt 2x\ge 2\\
 (3-\sqrt 2)x<\sqrt 2 \end{array}\right.;$
 \item $\left\{\begin{array}{l}2(x-\sqrt 2)>3x-\sqrt 3\\
 (x-\sqrt 2)^2>(x-\sqrt 3)^2-\sqrt 3 \end{array}\right..$
 \end{enumeratea}
 \end{multicols}
\end{esercizio}

\begin{esercizio}[\Ast]
 \label{ese:2.105}
Risolvi i seguenti sistemi di equazioni a coefficienti irrazionali.
 \begin{multicols}{2}
 \begin{enumeratea}
 \item $\left\{\begin{array}{l}{\sqrt 2x+\sqrt 3y=5}\\
 {\sqrt 3x+\sqrt 2y=2\sqrt 6} \end{array}\right.;$
 \item $\left\{\begin{array}{l}{x-\sqrt 3=2-y}\\
 {x+2=y+\sqrt 3} \end{array}\right.;$
 \item $\left\{\begin{array}{l}{x+2y=\sqrt 2-1}\\
 {2x-2y=2\sqrt 2} \end{array}\right.;$
 \item $\left\{\begin{array}{l}{\frac{2\left(x+\sqrt 3\right)}{\sqrt 2+2\sqrt 3}=\frac y{\sqrt 2}}\\
 {\frac{2x-y}{2\sqrt 6}=\frac{\sqrt 2} 2} \end{array}\right..$
 \end{enumeratea}
 \end{multicols}
\end{esercizio}
\pagebreak
\begin{esercizio}[\Ast]
 \label{ese:2.106}
Risolvi i seguenti sistemi di equazioni a coefficienti irrazionali.
 \begin{multicols}{2}
 \begin{enumeratea}
 \item $\left\{\begin{array}{l}x+\sqrt 3y=2\\
 \sqrt 3x-4y=1 \end{array}\right.;$
 \item $\left\{\begin{array}{l}\sqrt 2x-y=1\\
 2x+\sqrt 2y=0 \end{array}\right.;$
 \item $\left\{\begin{array}{l}4x-2\sqrt 5y=\sqrt 2\\
 \sqrt 2x+y=-2 \end{array}\right.;$
 \item $\left\{\begin{array}{l}\sqrt 3x+4\sqrt 2y=4\\
 \sqrt{12}x+8\sqrt 2y=8 \end{array}\right.;$
 \item $\left\{\begin{array}{l}2x+3\sqrt 2y=2\\
 \sqrt 3x-y=-\sqrt 8 \end{array}\right..$
 \end{enumeratea}
 \end{multicols}
\end{esercizio}

\begin{esercizio}[\Ast]
 \label{ese:2.107}
Risolvi i seguenti sistemi di equazioni a coefficienti irrazionali.
 \begin{multicols}{2}
 \begin{enumeratea}
 \item $\left\{\begin{array}{l}x+y=3\sqrt 5\\
 \sqrt 8x+2\sqrt 2y=-5\sqrt{11} \end{array}\right.;$
 \item $\left\{\begin{array}{l}x-3\sqrt 3y=\sqrt{27}\\
 -\sqrt 3x+\sqrt{243}y=0 \end{array}\right.;$
 \item $\left\{\begin{array}{l}\sqrt 2x+2y=4\\
 2x+\sqrt{32}y=-1 \end{array}\right.;$
 \item $\left\{\begin{array}{l}x-y\sqrt 3=2\\
 x\sqrt 3-y=1 \end{array}\right..$
 \end{enumeratea}
 \end{multicols}
\end{esercizio}

\begin{esercizio}[\Ast]
 \label{ese:2.108}
Risolvi i seguenti sistemi di equazioni a coefficienti irrazionali.
 \begin{multicols}{2}
 \begin{enumeratea}
 \item $\left\{\begin{array}{l}x-2y\sqrt 2=\sqrt 2\\
 x\sqrt 2+y=\sqrt 2 \end{array}\right.;$
 \item $\left\{\begin{array}{l}x\sqrt 2+y=1\\
 x+y\sqrt 2=0 \end{array}\right.;$
 \item $\left\{\begin{array}{l}2x+3y\sqrt 2=0\\
 x+y=\sqrt 8 \end{array}\right.;$
 \item $\left\{\begin{array}{l}x\sqrt 3+4y\sqrt 2=4\\
 x\sqrt{12}+8y\sqrt 2=-4 \end{array}\right..$
 \end{enumeratea}
 \end{multicols}
\end{esercizio}

\begin{esercizio}[\Ast]
 \label{ese:2.109}
Risolvi i seguenti sistemi di equazioni a coefficienti irrazionali.
 \begin{multicols}{2}
 \begin{enumeratea}
 \item $\left\{\begin{array}{l}x-3y\sqrt 3=0\\
 -x\sqrt 3+9y=0 \end{array}\right.;$
 \item $\left\{\begin{array}{l}x+y=3\sqrt 5\\
 2x-y=\sqrt 5 \end{array}\right.;$
 \item $\left\{\begin{array}{l}x\sqrt 2-2y=-1\\
 x\sqrt 8-y=0 \end{array}\right..$
 \end{enumeratea}
 \end{multicols}
\end{esercizio}

\subsection*{Esercizi di riepilogo}

\begin{esercizio}%2.110
Vero o Falso? È dato un quadrato di lato $3\sqrt 2$.

\TabPositions{11.5cm}
 \begin{enumeratea}
 \item Il suo perimetro è in numero irrazionale \tab\boxV\quad\boxF
 \item La sua area è un numero irrazionale\tab\boxV\quad\boxF
 \end{enumeratea}
\end{esercizio}

\begin{esercizio}%2.111
Vero o Falso? È dato un rettangolo di base $\sqrt{12}$ e altezza $14$.

\TabPositions{11.5cm}
 \begin{enumeratea}
 \item il suo perimetro è un numero irrazionale \tab\boxV\quad\boxF
 \item la sua area è un numero razionale \tab\boxV\quad\boxF
 \item il perimetro non esiste perché non si sommano razionali con irrazionali \tab\boxV\quad\boxF
 \item la misura del perimetro è un numero sia razionale che irrazionale \tab\boxV\quad\boxF
 \end{enumeratea}
\end{esercizio}

\begin{esercizio}%2.112
Vero o Falso? Un triangolo rettangolo ha i cateti lunghi rispettivamente $\sqrt 3\unit{cm}$ e $\sqrt{13}\unit{cm}$.

\TabPositions{11.5cm}
 \begin{enumeratea}
 \item l’ipotenusa ha come misura un numero razionale \tab\boxV\quad\boxF
 \item il perimetro è un numero irrazionale \tab\boxV\quad\boxF
 \item l'area è un numero irrazionale \tab\boxV\quad\boxF
 \end{enumeratea}
\end{esercizio}

\begin{esercizio}%2.113
Vero o Falso? È dato un quadrato di lato $1+\sqrt 5$.

\TabPositions{11.5cm}
 \begin{enumeratea}
 \item la misura della diagonale è un numero irrazionale \tab\boxV\quad\boxF
 \item l'area è un numero irrazionale \tab\boxV\quad\boxF
 \end{enumeratea}
\end{esercizio}

\begin{esercizio}%2.114
Vero o Falso? È dato un rettangolo di base $\sqrt{12}$ e altezza $\sqrt 3$.

\TabPositions{11.5cm}
 \begin{enumeratea}
 \item il perimetro è un numero irrazionale \tab\boxV\quad\boxF
 \item l’area è un numero irrazionale \tab\boxV\quad\boxF
 \item la misura della diagonale è un numero irrazionale \tab\boxV\quad\boxF
 \item il quadrato della misura del perimetro è un numero irrazionale \tab\boxV\quad\boxF
 \end{enumeratea}
\end{esercizio}

\begin{esercizio}%2.115
Un triangolo rettangolo ha un cateto lungo $7\unit{cm}$. Determina, se esiste, una possibile misura dell’altro cateto in modo che questa sia un numero irrazionale e che l’ipotenusa sia, invece, un numero razionale.
\end{esercizio}

\begin{esercizio}%2.116
Perché l'uguaglianza $\sqrt{(-5)^2}=-5$ è falsa?
\end{esercizio}

\begin{esercizio}%2.117
Determina il valore di verità delle seguenti affermazioni.
\begin{enumeratea}
 \item la radice terza del triplo di $a$ è uguale ad $a$;
 \item dati due numeri reali positivi, il quoziente delle loro radici quadrate è uguale alla radice quadrata del quoziente;
 \item il doppio della radice quadrata di $a$ è uguale alla radice quadrata del quadruplo di $a$;
 \item dati due numeri reali positivi, la somma delle loro radici cubiche è uguale alla radice cubica della loro somma;
 \item la radice cubica di $2$ è la metà della radice cubica di $8$;
 \item dati un numero reale positivo, la radice quadrata della sua radice cubica è uguale alla radice cubica della sua radice quadrata;
 \item sommando due radicali letterali simili si ottiene un radicale che ha la stessa parte letterale dei radicali dati.
\end{enumeratea}
\end{esercizio}

\begin{esercizio}%2.118
Riscrivi in ordine crescente i radicali $\sqrt 5$, $4\sqrt 2$, $2\sqrt 3$.
\end{esercizio}

\begin{esercizio}%2.119
Verifica che il numero irrazionale $\sqrt{7-2\sqrt 6}$ appartiene all'intervallo $(1$, $2)$ e rappresentalo sull'asse dei numeri reali.
\end{esercizio}

\begin{esercizio}%2.120
Dati i numeri\quad $\alpha =\sqrt[3]{(\sqrt{30}-\sqrt 3)\cdot (\sqrt{30}+\sqrt 3)}+\sqrt[4]{(7\sqrt 2-\sqrt{17})\cdot (7\sqrt 2-\sqrt{17})}$\quad e $\beta =(3+\sqrt 5)\cdot (3-\sqrt 5)-\frac 3{2+\sqrt 5}$, quali affermazioni sono vere?
\begin{multicols}{2}
\begin{enumeratea}
 \item sono entrambi irrazionali;
 \item solo $\alpha$ è irrazionale;
 \item $\alpha$ è minore di $\beta$;
 \item $\alpha$ è maggiore di $\beta$;
 \item $\beta$ è irrazionale negativo.
\end{enumeratea}
\end{multicols}
\end{esercizio}

\begin{esercizio}%2.121
Le misure rispetto al cm dei lati di un rettangolo sono i numeri reali $l_1=\sqrt[3]{1-\frac 1 8}\cdot \sqrt[3]{1-\frac 2 7}\cdot \sqrt[3]{25}$ e $l_2=\sqrt{\sqrt 2}\cdot \sqrt[4]3\cdot (\sqrt[8]6)^3:\sqrt[4]{\sqrt 6}$. Determinare la misura del perimetro e della diagonale del rettangolo.
\end{esercizio}
\pagebreak
\begin{esercizio}%2.122
Se $x$ è positivo e diverso da $1$, l'espressione $E=\sqrt[4]{\frac 4{\sqrt x-1}-\frac 4{\sqrt x+1}}:\sqrt[4]{\frac 4{\sqrt x-1}+\frac 4{\sqrt x+1}}$ è uguale a:
\begin{multicols}{5}
\begin{enumeratea}
 \item $\sqrt[4]{\frac 1 x}$;
 \item $\sqrt[8]{\frac 1 x}$;
 \item $\frac 1 x$;
 \item $\sqrt[8]x$;
 \item $0$.
\end{enumeratea}
\end{multicols}
\end{esercizio}

\begin{esercizio}%2.123
Stabilire se la seguente affermazione è vera o falsa. Per tutte le coppie $(a;b)$ di numeri reali positivi con $a=3b$, l'espressione $E=\frac{\sqrt a+\sqrt b}{\sqrt a-\sqrt b}+\frac{\sqrt a-\sqrt b}{\sqrt a+\sqrt b}-\frac{a+b}{a-b}$ ha il numeratore doppio del denominatore.
\end{esercizio}

\begin{esercizio}%2.124
Calcola il valore delle seguenti espressioni letterali per i valori indicati delle lettere.
\begin{multicols}{3}
\begin{enumeratea}
\item $x+2\sqrt 3$ per $x=\sqrt 3$
\item $\sqrt 2x+3\sqrt 6$ per $x=\sqrt{3}$
\item $x^2+x-1$ per $x=\sqrt 2$
\item $x^2+\sqrt 5x-1$ per $x=\sqrt 5$
\item $(x+2\sqrt 2)^2$ per $x=\sqrt 2$
\end{enumeratea}
\end{multicols}
\end{esercizio}

\begin{esercizio}%2.125
Trasforma in un radicale di indice $9$ il seguente radicale $\sqrt[3]{\frac{\sqrt{\frac a b-\frac b a}}{\sqrt{\frac a b+\frac b a+2}}:\sqrt{\frac{a+b}{a-b}}+1}$.
\end{esercizio}

\begin{esercizio}[\Ast]%2.126
Risolvi le seguenti equazioni.
\begin{multicols}{2}
 \begin{enumeratea}
 \item $\frac{x\sqrt 2-\sqrt 3}{\sqrt 2+\sqrt 3}+\frac{x\sqrt 2+\sqrt 3}{\sqrt 3-\sqrt 2}=\frac{3x+3}{\sqrt 3}$;
 \item $\frac{\sqrt 3+x}{x-\sqrt 3}+\frac{x+\sqrt 2}{x-\sqrt 2}=2$.
 \end{enumeratea}
\end{multicols}
\end{esercizio}

\begin{esercizio}%2.127
Per quale valore di $k$ il sistema lineare è determinato?
$\left\{\begin{array}{l}{x\sqrt 3+(k-\sqrt 3)y=1}\\
 {-2x+y\sqrt 6=-k} \end{array}\right..$
\end{esercizio}

\begin{esercizio}%2.128
L’insieme di soluzioni della disequazione $(\sqrt 2-\sqrt 3)x<0$ è:
\begin{multicols}{5}
 \begin{enumeratea}
 \item $x\ge 0$;
 \item $x\le 0$;
 \item $x>0$;
 \item $x<0$;
 \item $\insR$.
 \end{enumeratea}
\end{multicols}
\end{esercizio}

\begin{esercizio}%2.129
Data l'espressione $E=\frac{2a-2\sqrt 2}{\sqrt 2}+\frac{(a+2)\cdot \sqrt 2} 2+\frac 4{\sqrt 2}-1$, stabilire se esistono valori di $a$ che la rendono positiva.
\end{esercizio}

\begin{esercizio}%2.130
Data la funzione $f(x)=\frac{\sqrt{x+1}}{\sqrt{x+1}-\sqrt{x-1}}$
 \begin{enumeratea}
 \item determina il suo dominio;
 \item riscrivi la funzione razionalizzando il denominatore;
 \item calcola $f(2)$;
 \item per quali valori di $x$ si ha $f(x)>0$?;
 \item risolvi l'equazione $f(x)=0$.
 \end{enumeratea}
\end{esercizio}


\subsection{Risposte}

\paragraph{2.6.}
b)~$4$,\quad h)~$-\frac{4}{5}$,\quad i)~$\frac{10}{3}$.

\paragraph{2.7.}
e)~$3$,\quad h)~$\emptyset$.

\paragraph{2.8.}
b)~$3$,\quad d)~$\frac{2}{3}$,\quad h)~$2$.

\paragraph{2.9.}
c)~$3$,\quad e)~$\np{0,2}$,\quad i)~$5$.

\paragraph{2.10.}
d)~$2a+1$,\quad e)~$a^2+3$,\quad f)~$1-2x$.

\paragraph{2.11.}
a)~$\forall x\in \insR$,\quad b)~$x\le 1$,\quad c)~$x>-1$,\quad d)~$y\ge 0$,\quad f)$x>1$.

\paragraph{2.12.}
a)~$x\ge -1$,\quad d)~$\emptyset$,\quad h)~$x\le 0 \vee x\ge 2$.

\paragraph{2.13.}
a)~$-2<x\le 5$,\quad e)~$b<-2\vee b>2$.

\paragraph{2.14.}
a)~$b\ge 0$,\quad b)~$x\le 0$,\quad c)~$x=0\vee x\ge 1$,\quad d)~$x>-\frac{1}{2}$,\quad e)~$3<x\le 7$.

\paragraph{2.15.}
b)~$0\le x\le 1\vee x>4$,\quad e)~$-2<a<0\;\vee \;a>4$.

\paragraph{2.16.}
a)~$\forall x\in \insR$,\quad d)~$\forall x\in \insR$,\quad g)~$-2<x<-1 \vee x>0$,\quad i)~$x>0$,\quad f)~$\emptyset$\quad k)~$x\ge 1$.

\paragraph{2.18.}
a)~$4$,\quad f)~$25$,\quad i)~$2$.

\paragraph{2.19.}
c)~$5^{\frac 3 7}$,\quad g)~$25^{-\frac 1 3}$.

\paragraph{2.20.}
a)~$\sqrt[4]{\sqrt[3]{(a^2+1)^2+1}}$.

\paragraph{2.24.}
c)~$\sqrt 2$,\quad e)~$\sqrt{10}$,\quad i)~$\sqrt 5$.

\paragraph{2.25.}
b)~$2$,\quad d)~$\sqrt[3]{\frac 4{11}}$,\quad h)$\sqrt[3]{-3}$.

\paragraph{2.26.}
a)~$\emptyset$,\quad e)~$\sqrt[5]5$,\quad g)~12.500.

\paragraph{2.27.}
b)~$5$,\quad d)~,\quad e)~$\frac 9 4$,\quad g)$2$.

\paragraph{2.28.}
a)~$4\cdot \sqrt 3$,\quad e)~$4a^2b^3$,\quad i)~$\valass y \cdot \sqrt{\frac{2\cdot \valass x} 3}$.

\paragraph{2.29.}
a)~$\sqrt{(2x+3)}$,\quad e)~$\sqrt[5]{\frac{11a^2} b}$,\quad i)~$2\cdot \valass a$.

\paragraph{2.30.}
b)~$\sqrt[5]{\left|a^2+3x\right|}$,\quad f)~$\frac 1 2$,\quad h)~$\frac{5a^2\valass b^3}{a^2+2}$.

\paragraph{2.31.}
c)~$\left|a\right|\sqrt{\left|a-1\right|}$,\quad d)~$\valass{x-3}$,\quad h)~$\frac{\valass{x-1}}{\valass b \valass{x+1}}$,\quad f)~$-2$ se $a\le-1$; $2a$ se $-1<a\le 1$; $2$ se $a>1$.

\paragraph{2.32.}
b)~$\sqrt[4]{\frac 8 9}$,\quad e)~$\sqrt[n]{6^4}$,\quad i)$2x^2$.

\paragraph{2.33.}
a)~$3a^2b^3$,\quad b)~$\sqrt{2xy^3}$,\quad c)~$\sqrt{x+4y}$,\quad d)~$\frac{4x^2}{3y^2}$,\quad e)~$x^3y$.

\paragraph{2.34.}
a)~$\sqrt[5]{\frac{2xy^2}{5z^3}}$,\quad b)~$\valass{x^2-y^2}$,\quad c)~$\sqrt[3]{\frac{xy^2}{(9x-1)^2}}$,\quad d)~$\sqrt[9]{\frac{1-x}{x}}$ se~$x\neq 0$,\quad e)~$\sqrt{\frac{\valass{a-1}}{a+1}}$,\quad f)~$\sqrt[3]{\frac{3(a-1)}{5xy^3}}$.

\paragraph{2.35.}
a)~$15$,\quad d)~$30$,\quad i)~$1$.

\paragraph{2.36.}
a)~$3\sqrt[4]{3}$,\quad b)~$4\sqrt[4]{8}$,\quad c)~$\sqrt[4]{\frac{4}{3}}$,\quad d)~$\sqrt[7]{42}$,\quad e)~$\sqrt[3]{9}$,\quad f)~$1$,\quad g)~$\sqrt[3]{\frac{b}{a}}$.

\paragraph{2.37.}
c)~$\sqrt[6]{3^7}$,\quad e)~$\sqrt[6]{3^7}$,\quad h)~$\sqrt[6]{\frac{3^2\cdot 5^3}{4^2}}$.

\paragraph{2.38.}
a)~$\frac 6 13$,\quad b)~$\frac 5 4$,\quad c)~$\frac{4}{\sqrt[3]{3^2}}$,\quad d)~$2$,\quad e)~$60$.

\paragraph{2.39.}
b)~$\sqrt{15}$,\quad c)$2ab$~,\quad e)~$\sqrt[6]{\frac{2^3a^2}{3^4}}$.

\paragraph{2.40.}
b)~$\sqrt[6]{\frac{(x+1)^4}{(x-1)^3}}$,\quad c)~$\sqrt{a-b}$,\quad e)~$\sqrt[6]{\frac{(1-x)^4}{(1+x)(1+x^2)^2}}$.

\paragraph{2.41.}
b)~$\sqrt[6]{\frac{(a+1)(a+3)^2}{(a-3)(a-1)^2}}$,\quad c)~$\sqrt[6]{\frac{(x-1)(x+1)}{(x-2)(x+3)}}$,\quad f)~$\sqrt{\frac{x-y}{xy}}$.

\paragraph{2.42.}
a)~$\sqrt{\frac{a+b}{ab}}$,\quad d)~$\sqrt[6]{\frac{(a+2)^7}{(a-1)^7}}$,\quad e)~$\sqrt[6]{\frac{x+2}{x^2(x+1)}}$.

\paragraph{2.43.}
a)~$\sqrt[4]{\frac{a+b} x}$,\quad c)~$a-1$,\quad d)~$\sqrt[6]{\frac{x+y}{x-y}}$,\quad e)~$\frac{a+2}{a+5}$,\quad f)~$\sqrt[4]{(x+y)(3x+2y)}$.

\paragraph{2.44.}
a)~$\sqrt[4]{a(a-b)}$,\quad b)~$\sqrt{\frac{x}{y}}$,\quad c)~$\frac{x-2}{x+3}$,\quad d)~$xy$,\quad e)~$\sqrt[15]{\left(\frac{a+b}{a-b}\right)^4}$,\quad f)~$\sqrt[8]{\frac{a+b}{2(a+2b)}}$.

\paragraph{2.45.}
a)~$1$,\quad b)~$\frac{1}{\sqrt[12]{ab(a-b)^2}}$,\quad c)~$\sqrt[3]{\frac{y(x+y)}{x(x-y)}}$,\quad d)~$\sqrt[6]{\frac{2a+1}{2a-1}}$,\quad e)~$\sqrt[3]{\frac{a}{b}}$,\quad f)~$\sqrt[10]{\frac{x^4(x+y)}{y^4(x-y)}}$.

\paragraph{2.46.}
a)~$\sqrt[3]{\frac{x}{y}}$,\quad b)~$\sqrt[6]{a^3(a+b)}$,\quad c)~$\sqrt[12]{x^4(x+y)^3(x-y)^2}$,\quad d)~$xy$,\quad e)~$\sqrt[6]{\frac{(x-y)^4}{4xy}}$. %\quad f)~$\sqrt[8]{\frac{x^2(x-3)^2}{x+3}}$.

\paragraph{2.47.}
a)~$1$,\quad b)~$\sqrt[3]{\frac{x+y}{x-y}}$,\quad c)~$(a-1)\sqrt[4]{(a-1)^3}$,\quad d)~$\sqrt[12]{(a^2-1)(a+b)}$,\quad e)~$\sqrt[4]{\frac{a^2+b^2}{a^2-b^2}}$. %,\quad f)~$y(x-y)$.

\paragraph{2.48.}
a)~$\sqrt{2^3}$,\quad g)~$\sqrt{\frac 3 4}$,\quad o)~$-\sqrt[3]{\frac 1 2}$.

\paragraph{2.49.}
b)~$\sqrt[3]{x^7}$,\quad g)~$\sqrt{(a-1)^2a}$.

\paragraph{2.50.}
a)~$\sqrt{\frac{x}{x+1}}$,\quad b)~$\sqrt{(x+1)(x+2)}$,\quad c)~$\sqrt[4]{\frac{27a^5b^6}{c^2}}$,\quad d)~$\sqrt[3]{x(x^2-1)}$,\quad e)~$\sqrt{\frac{a-b}{a+b}}$,\quad f)~$\sqrt[3]{\frac{x}{x-1}}$.

\paragraph{2.51.}
a)~$\sqrt{a^2-b^2}$,\quad b)~$\sqrt{\frac{x-y}{xy(x+y)}}$,\quad c)~$\sqrt{\frac{3a-2}{3a+2}}$,\quad d)~$\sqrt[3]{\frac{a-1}{(a+1)^2}}$,\quad e)~$\sqrt{\frac{a^2-1}{a}}$,\quad f)~$\sqrt[5]{\frac{3a+1}{9}}$,\protect\\
\quad g)~$\sqrt[5]{\frac{a^4b^2}{c^3}}$,\quad h)~$\sqrt{8}$.

\paragraph{2.52.}
a)~$5\sqrt{10}$,\quad b)~$9\sqrt 6$,\quad c)~$12\sqrt 6$,\quad d)~$24\sqrt 6$,\quad k)~$10\sqrt 3$.

\paragraph{2.53.}
b)~$\frac 1 6\sqrt{97}$,\quad g)~$\sqrt{\frac 2 3}$.

\paragraph{2.54.}
e)~$3\valass a\sqrt b, \CE b>=0$.

\paragraph{2.55.}
a)~$\sqrt[5]{9a^2b^3c}$,\quad b)~$x^2yz\sqrt[5]{3y^2}$,\quad c)~$\sqrt[3]{\frac{x(x-y)^2}{36}}$,\quad d)~$\frac{5a\sqrt[3]{b^2}}{(a-b)^2}$,\quad e)~$\sqrt{\frac{a+b}{a-b}}$.

\paragraph{2.56.}
a)~$xy\sqrt[4]{24xy^2}$,\quad b)~$(a-b)\sqrt[3]{a-b}$,\quad c)~$2a\sqrt[5]{2+a^2}$,\quad d)~$(x+y)\sqrt{xy}$,\quad e)~$2\sqrt[4]{a^4+2}$.

\paragraph{2.57.}
b)~$\valass{2x}\sqrt{x^2-1}, \CE x\le 1\vee x\ge 1$,\quad i)~$(a+a^2+a^3)\sqrt a$.

\paragraph{2.58.}
d)~$\sqrt{2^3}$,\quad l)~$2a^3$,\quad p)~$\frac 1 9$.

\paragraph{2.59.}
j)~$\sqrt{2^4a^2\left|b^3\right|}$.

\paragraph{2.60.}
h)~$\sqrt[3]{a^2}$.

\paragraph{2.61.}
f)~$\sqrt[3]{3(a+b)}, \CE a>b$.

\paragraph{2.62.}
a)~$\sqrt[15]{y^7}$,\quad b)~$\sqrt[10]{2x+1}$,\quad c)~$\sqrt[12]{x}$,\quad d)~$\sqrt[4]{\frac{a+1}{a-1}}$,\quad e)~$\sqrt[6]{(a-b)^5}$,\quad f)~$x-y$.

\paragraph{2.63.}
c)~$5\sqrt 6$,\quad f)~$-\sqrt 7$,\quad g)~$3(\sqrt 5+3\sqrt 2)$,\quad h)~$7\sqrt 7$.

\paragraph{2.64.}
c)~$\sqrt 5-\frac 1 6\sqrt 2$,\quad j)~$\sqrt 3-\sqrt 2$.

\paragraph{2.65.}
a)~$0$,\quad b)~$0$,\quad c)~$\sqrt[4]2+12\sqrt[3]2$,\quad d)~$\sqrt[3]2+3\sqrt[4]3$,\quad e)~$\frac 2{15}\sqrt 2-\frac 4 7\sqrt 3$,\quad f)~$0$.

\paragraph{2.66.}
a)~$-\frac 1 2\sqrt a-\frac 2 5\sqrt b$,\quad b)~$(1+a-b)\sqrt[3]{a-b}$.

\paragraph{2.67.}
a)~$9\sqrt b-\sqrt{ab}$,\quad c)~$\left(3xy^2+2x^2-\frac{z^4}{y}\right)\sqrt{xz}$,\quad d)~$0$,\quad e)~$(x-y+1)\sqrt[3]{x-y}$.

\paragraph{2.68.}
a)~$-\frac{3}{8}\sqrt{3xy}$,\quad b)~$\frac{9a-2b}{6\sqrt{a}+b}$,\quad c)~$(x-2y)^2\sqrt[3]{x}$,\quad d)~$\frac{3-4x}{2}\sqrt[3]{2}$,\quad e)~$\sqrt{\frac{3}{5}}$,\quad f)~$(1-y)^2\sqrt[9]{x^3y^4}$.

\paragraph{2.69.}
e)~$4+2\sqrt 3$,\quad f)~$7-4\sqrt 3$,\quad g)~$9+4\sqrt 5$,\quad h)~$19-8\sqrt 3$,\quad i)~$48+24\sqrt 3$,\quad j)~$\frac{27} 4-\sqrt{18}$.

\paragraph{2.70.}
i)~$8-2\sqrt 2-2\sqrt{10}+2\sqrt 5$,\quad l)~$1-3\sqrt[3]4+3\sqrt[3]2$.

\paragraph{2.71.}
i)~$3\sqrt 5+25$.

\paragraph{2.72.}
f)~$-19-12\sqrt 6$,\quad k)~$a+2+\frac 1 a$.

\paragraph{2.73.}
a)~$x-y$,\quad g)~$6$.

\paragraph{2.74.}
b)~$8\sqrt{2}-12$.

\paragraph{2.75.}
c)~$\sqrt[5]{b^7}$,\quad h)~$\sqrt b$.

\paragraph{2.76.}
a)~$3\sqrt[3]{\frac{2x}{y}}$,\quad b)~$35-18\sqrt{2}$,\quad c)~$b\sqrt[6]{\frac{bc}{a}}$,\quad d)~$a\sqrt[20]{ab^{15}}$,\quad e)~$3$,\quad f)~$10\sqrt{3}-6\sqrt{6}-2\sqrt{2}-1$,\quad g)~$\sqrt[12]{\frac{x-y}{x+y}}$,\quad h)~$2(3x^2+y)\sqrt{y}$,\quad i)~$\sqrt[4]{\frac{4x^2}{3y}}$,\quad j)~$3abx^2$.

\paragraph{2.77.}
e)~$\sqrt[12]{\frac a{a+3}}$,\quad f)~$\sqrt[8]{\left(\frac{x-1}{x+1}\right)^5}$.

\paragraph{2.78.}
a)~$\sqrt[3]{\frac{a-1}{(a+1)^3}}$,\quad b)~$(b-1)^2\sqrt{b+1}$,\quad c)~$2\sqrt[3]{y^2}$,\quad d)~$\sqrt[12]{\frac{(b+1)^3}{b(b-1)}}$,\quad e)~$\sqrt[4]{\frac{(a-1)^2} 2}$,\quad f)~$\frac{x+y}{x+3}$.

\paragraph{2.79.}
a)~$\sqrt[3]x$,\quad c)~$(y-1)^2\sqrt{y+1}$,\quad d)~$\sqrt[12]{\frac{a^{11}}{(a^2-1)^6}}$,\quad e)~$\sqrt[24]{\frac{a^{10}b^{10}(a+b)^{11}}{x^{11}}}$,\quad f)~$\sqrt[6]{\frac{2x+1}{2x-1}}$.

\paragraph{2.80.}
a)~$\sqrt[3]{\frac{\valass{a-1}}{(a+1)^4}}$,\; b)~$\sqrt[6]{\frac{27a^3}{a-3}}$,\; c)~$\sqrt[6]{\frac{a-1}{a(a+1)^3}}$,\; d)~$\sqrt{2y-1}$,\; e)~$\sqrt[6]{4a^2(2a-1)}$,\; f)~$\frac 3{5a}\sqrt{5a+1}$.

\paragraph{2.81.}
a)~$\frac{(1-y)^2} y\sqrt[3]{x-y}$,\; b)~$\frac{(1+2x)^2}{2x}\sqrt{x^2+xy+y^2}$,\; c)~$\sqrt a$,\; d)~$\frac{(x+y)^2}{xy}\sqrt{x-3y}$,\; e)~$(1+a)^2$.

\paragraph{2.82.}
a)~$\sqrt{x+y}$,\quad b)~$1$,\quad c)~$0$,\quad d)~$\sqrt[6]{\frac{x+y}{x-y}}$,\quad e)~$\sqrt{3x+2}$,\quad f)~$1$.

\paragraph{2.83.}
a)~$\sqrt{a^3}$,\quad b)~$\sqrt[14]{a^3}$,\quad c)~$\sqrt[9]{a^{19}}$,\quad d)~$\sqrt[5]{b^7}$.

\paragraph{2.84.}
d)~$2\sqrt 5$,\quad h)~$\sqrt 2$,\quad j)~$\frac{\sqrt 6} 9$,\quad d)~,\quad e)~,\quad f)~.

\paragraph{2.86.}
c)~$\frac{\sqrt{2x}} 6$.

\paragraph{2.87.}
c)~$\frac{\sqrt 2+2} 2$,\quad l)~$\frac 2 3\sqrt[3]{36}$.

\paragraph{2.89.}
b)~$\sqrt[3]{a^2b}$.

\paragraph{2.90.}
d)~$3-2\sqrt 2+2\sqrt 3-\sqrt 6$.

\paragraph{2.93.}
a)~$6\sqrt{3}$,\quad b)~$\frac{\sqrt[3]{9}}{2}$,\quad c)~$\frac{3\sqrt{2}-2}{2}$,\quad d)~$\frac{\sqrt{a}-a}{1-a}$,\quad e)~$\sqrt[3]{9}-3\sqrt[3]{3}+9$,\quad f)~$\sqrt{3}+\sqrt{2}+2$.

\paragraph{2.94.}
a)~$\frac{\sqrt[3]{x+3}}{x+3}$,\quad b)~$\sqrt[3]{x^2}+\sqrt[3]{y^2}-\sqrt[3]{xy}$,\quad c)~$1$,\quad d)~$-\sqrt{x^2-\sqrt{x}}-x$,\quad e)~$\sqrt{2+x}+\sqrt{x}$,\protect\\
\quad f)~$\frac{27+x^2+12\sqrt{x^2-9}}{45-x^2}$.

\paragraph{2.95.}
d)~$\frac{\sqrt{10}} 2+\frac{\sqrt 2} 2$.

\paragraph{2.96.}
d)~$\sqrt 6+1$.

\paragraph{2.98.}
a)~$\sqrt{x+2}+1$,\quad b)~$\sqrt{a}-1$,\quad c)~$\sqrt{x+y}+1$,\quad d)~$\sqrt{a+1}-1$,\quad e)~$\frac{\sqrt{8a+2}+\sqrt{2}}{2}$,\protect\\\quad f)~$\frac{\sqrt{4a-b}+\sqrt{b}}{2}$.

\paragraph{2.99.}
a)~$\frac{x+y}{x+1}$,\quad b)~$0$,\quad c)~$1$,\quad d)~$\frac{8\sqrt{x}}{x-4}$,\quad e)~$x$,\quad f)~$0$.

\paragraph{2.100.}
a)~$\sqrt{2}$,\quad e)~$\sqrt{3}$,\quad f)~$\frac{\sqrt 6} 3$,\quad g)~$1$,\quad h)~$4-3\sqrt 2$.

\paragraph{2.101.}
a)~$18-12\sqrt 2$,\quad b)~$-\frac{2\sqrt 3+3\sqrt{10}} 2$,\quad c)~$-(1+\sqrt 2)$,\quad e)~$-\frac{7}{2}(\sqrt{2}+\sqrt{3})$,\quad f)~$\frac{-7(\sqrt 2+\sqrt 3)} 2$.

\paragraph{2.102.}
a)~$-\frac{\sqrt 2+2\sqrt 3} 3$,\quad b)~$\frac{\sqrt 2} 2$,\quad c)~$\frac{36+17\sqrt 3}{30}$,\quad d)~$\frac{36-10\sqrt 6}{29}$.
%-------------2.81
%(301) a R. $-\frac{\sqrt 2+2\sqrt 3} 3$ (302) b R. $\frac{\sqrt 2} 2$ (303) c R. $\frac{36+17\sqrt 3}{30}$ (304) d R. $\frac{36-10\sqrt 6}{29}$

\paragraph{2.103.}
a)~$x<-\sqrt 2$,\, b)~$x>\frac{\sqrt 2-6} 2$,\, c)~$x>\frac{\sqrt{10}(\sqrt 2-1)} 2$,\, d)~$x<5\sqrt 3-\sqrt 6$,\, e)~$x\ge \frac{4\sqrt 3-4+\sqrt 6-\sqrt 2} 7$.
%---------------2.83
%(305) a R. $x<-\sqrt 2$ (306) b R. $x>\frac{\sqrt 2-6} 2$ (307) c R. $x>\frac{\sqrt{10}(\sqrt 2-1)} 2$ (308) d R. $x<5\sqrt 3-\sqrt 6$
%(309) e R. $x\ge \frac{4\sqrt 3-4+\sqrt 6-\sqrt 2} 7$

\paragraph{2.104.}
a)~$\emptyset$,\quad b)~$\frac{\sqrt 3-3+\sqrt 2-\sqrt 6} 2<x<\sqrt 3-2\sqrt 2$.
%---------------2.84
%(310) impossibile
%(311) R. $\frac{\sqrt 3-3+\sqrt 2-\sqrt 6} 2<x<\sqrt 3-2\sqrt 2$

\paragraph{2.105.}
a)~$(\sqrt 2;\sqrt 3)$,\quad b)~$(\sqrt 3;2)$,\quad c)~$(\sqrt 2+\frac 1 3-\frac 1 3)$,\quad d)~$(\sqrt 2+\sqrt 3;2\sqrt 2)$.

\paragraph{2.106.}
a)~$(\frac{\sqrt 3+8} 7;\frac{2\sqrt 3-1} 7)$,\quad b)~$(\frac{\sqrt 2} 4;-\frac 1 2)$,\quad c)~$(\frac{5\sqrt 5-11\sqrt 2} 6\frac{10-5\sqrt{10}} 6$,\quad d)~$\insR$,\quad e)~$(\frac{2-3\sqrt 6} 5; \frac{\sqrt 2+2\sqrt 3} 5)$.

\paragraph{2.107.}
a)~$\emptyset$,\quad b)~$(\frac{9+9\sqrt 3} 2;\frac{1+\sqrt 3} 2)$,\quad c)~$(\frac 1 2+4\sqrt 2;-2-\frac{\sqrt 2} 4)$,\quad d)~$(\frac{\sqrt 3} 2-1;\frac 1 2-\sqrt 3)$.
%---------------2.87
%(318) a impossibile
%(319) b R. $\left(\frac{9+9\sqrt 3} 2;\frac{1+\sqrt 3} 2\right)$
%(320) c R. $\left(\frac 1 2+4\sqrt 2;-2-\frac{\sqrt 2} 4\right)$
%(321) d R. $\left(\frac{\sqrt 3} 2-1;\frac 1 2-\sqrt 3\right)$

\paragraph{2.108.}
a)~$(\frac{\sqrt 2+4} 5;\frac{\sqrt 2-2} 5)$,\quad b)~$(\sqrt 2;-1)$,\quad c)~$(-\frac{4\sqrt 2+12} 7;\frac{18\sqrt 2+12} 7)$,\quad d)~$\emptyset$.
%---------------2.88
%(322) a R. $\left(\frac{\sqrt 2+4} 5;\frac{\sqrt 2-2} 5\right)$
%(323) b R. $\left(\sqrt 2;-1\right)$
%(324) c R. $\left(-\frac{4\sqrt 2+12} 7;\frac{18\sqrt 2+12} 7\right)$
%(325) d R. impossibile

\paragraph{2.109.}
a)~$\insR$,\quad b)~$(\frac{4\sqrt 5} 3;\frac{5\sqrt 5} 3)$,\quad c)~$(\frac{\sqrt 2} 6;\frac 2 3)$.
%---------------2.89
%(326) a R. indeterminato
%(327) b R. $\left(\frac{4\sqrt 5} 3;\frac{5\sqrt 5} 3\right)$
%(328) c R. $\left(\frac{\sqrt 2} 6;\frac 2 3\right)$
\paragraph{2.126.}
a)~$-1$,\quad b)~$2\cdot (3\sqrt 2-2\sqrt 3)$.
%--------------2.106
%Risolvi le equazioni
%(345) a \ \ \ \ $\mathit{R.}[-1]$
%(346) b \ \ \ \ $\mathit{R.}[2\cdot (3\sqrt 2-2\sqrt 3)]$