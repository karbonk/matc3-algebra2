% (c)~2014 Claudio Carboncini - claudio.carboncini@gmail.com
% (c)~2014 Dimitrios Vrettos - d.vrettos@gmail.com
\section{Esercizi}
\subsection{Esercizi dei singoli paragrafi}
\subsection*{5.2 - Equazioni riconducibili al prodotto di due o più fattori}

\begin{esercizio}[\Ast]
 \label{ese:5.1}
Trovare gli zeri dei seguenti polinomi.
\begin{multicols}{2}
 \begin{enumeratea}
 \item~$x^3+5x^2-2x-24$;
 \item~$6x^3+23x^2+11x-12$;
 \item~$8x^3-40x^2+62x-30$;
 \item~$x^3+10x^2-7x-196$;
 \item~$x^3+\frac 4 3x^2-\frac{17} 3x-2$;
 \item~$x^3-\frac 1 3x^2-\frac{38} 3x+\frac{56} 3$.
 \end{enumeratea}
 \end{multicols}
\end{esercizio}

\begin{esercizio}[\Ast]
\label{ese:5.2}
Trovare gli zeri dei seguenti polinomi.
\begin{multicols}{2}
 \begin{enumeratea}
 \item~$3x^3-\frac 9 2x^2+\frac 3 2x$;
 \item~$3x^3-9x^2-9x-12$;
 \item~$\frac 6 5x^3+\frac{42} 5x^2+\frac{72} 5x+12$;
 \item~$4x^3-8x^2-11x-3$;
 \item~$\frac 3 2x^3-4x^2-10x+8$;
 \item~$\frac 3 2x^3-4x^2-10x+8$.
 \end{enumeratea}
 \end{multicols}
\end{esercizio}

\begin{esercizio}[\Ast]
 \label{ese:5.3}
Trovare gli zeri dei seguenti polinomi.
\begin{multicols}{2}
 \begin{enumeratea}
 \item~$-3x^3+9x-6$;
 \item~$\frac 1 2x^3-3x^2+6x-4$;
 \item~$4x^3+4x^2-4x-4$;
 \item~$\frac 2 5x^3+\frac 8 5x^2+\frac{14} 5x-4$;
 \item~$-6x^3-30x^2+192x-216$;
 \item~$x^3-2x^2-x+2$.
 \end{enumeratea}
 \end{multicols}
\end{esercizio}

\begin{esercizio}[\Ast]
 \label{ese:5.4}
Trovare gli zeri dei seguenti polinomi.
\begin{multicols}{2}
 \begin{enumeratea}
 \item~$9x^3-7x+2$;
 \item~$x^3-7x^2+4x+12$;
 \item~$ x^3+10x^2-7x-196 $;
 \item~$ 400x^3-\np{1600}x^2$;
 \item~$x^6-5x^5+6x^4+4x^3-24x^2+16x+32 $;
 \item~$ 8x^3-14{ax}^2-5a^2x+2a^3 $.
 \end{enumeratea}
 \end{multicols}
\end{esercizio}

\begin{esercizio}
\label{ese:5.5}
Trovare gli zeri dei seguenti polinomi.
\begin{multicols}{2}
 \begin{enumeratea}
 \item~$ x^4-x^3-x^2-x-2 $;
 \item~$ 3x^5-19x^4+42x^3-42x^2+19x-3 $;
 \item~$ {ax}^3-(a^2+1-a)x^2-(a^2+1-a)x+a $.
 \end{enumeratea}
 \end{multicols}
\end{esercizio}

\begin{esercizio}[\Ast]
\label{ese:5.6}
Determinare l'insieme soluzione delle seguenti equazioni.
\begin{multicols}{2}
 \begin{enumeratea}
 \item~$x^3-3x+2=0$;
 \item~$x^3+2x^2+2x+1=0$;
 \item~$x^3-6x+9=0$;
 \item~$x^4-2x^2+1=0$;
 \item~$x^3+3x^2-x-3=0$;
 \item~$6x^3-7x^2-x+2=0$.
 \end{enumeratea}
 \end{multicols}
\end{esercizio}

\begin{esercizio}[\Ast]
 \label{ese:5.7}
Determinare l'insieme soluzione delle seguenti equazioni.
\begin{multicols}{2}
 \begin{enumeratea}
 \item~$x^3-6x^2+11x-6=0$;
 \item~$x^3-2x^4=0$;
 \item~$x^4-5x^3+2x^2+20x-24=0$;
 \item~$x^5+1=x\cdot \left(x^3+1\right)$;
 \item~$\frac{x^3+2-x\cdot (2x+1)}{2x-1}=0$;
 \item~$2x^2-2x+3(x-1)=2x\left(2x^2-1\right)$.
 \end{enumeratea}
 \end{multicols}
\end{esercizio}
%\newpage
\pagebreak
\begin{esercizio}[\Ast]
 \label{ese:5.8}
Determinare l'insieme soluzione delle seguenti equazioni.
 \begin{enumeratea}
 \item~$(3x+1)^2=x\left(9x^2+6x+1\right)$;
 \item~$(x+1)\left(x^2-1\right)=\left(x^2+x\right)\left(x^2-2x+1\right)$;
 \item~$(x-1)(x^2+x+1)=x(2-3x)+5$;
 \item~$x^3+4x^2+4x=x^2-4$;
 \item~$\sqrt 3 x^4-\sqrt{27}\;x^2=0$;
 \item~$ (x+1)^3-(x-1)^3=8 $.
 \end{enumeratea}
\end{esercizio}

\begin{esercizio}[\Ast]
 \label{ese:5.9}
Determinare l'insieme soluzione delle seguenti equazioni.
\begin{multicols}{2}
 \begin{enumeratea}
 \item~$\sqrt 2x^3-(1-2\sqrt 2)x^2-x=0$;
 \item~$64x^7=27x^4$;
 \item~$(x^2-4x)^{\np{2011}}=-(4x-x^2)^{\np{2011}}$;
 \item~$(x^2-4x)^{\np{2012}}=-(4x-x^2)^{\np{2011}}$;
 \item~$x^7-x^6+\sqrt{27}x^5=0$;
 \item~$3x^4-14x^3+20x^2-8x=0$.
 \end{enumeratea}
 \end{multicols}
\end{esercizio}

\begin{esercizio}[\Ast]
 \label{ese:5.10}
Determinare l'insieme soluzione delle seguenti equazioni.
\begin{multicols}{2}
 \begin{enumeratea}
 \item~$\frac{3x-1}{x^2}=1-2x+\frac 1 x$;
 \item~$\frac{x-1}{x^2+5x+4}-\frac{2x+1}{x-1}-\frac 3{2\left(x^2-1\right)}=0$;
 \item~$\frac{x^2-3x}{2x}-\frac{x-2}{x-1}=0$;
 \item~$\frac{x(x-1)}{x+1}=\frac{x-1}{x^2+2x+1}$;
 \item~$\frac 1{x^4-4}=\frac 3{x^4-16}$.
 \item~$\frac{x^2}{x^2+1}-\frac 1{4-x^2}+\frac 1{x^4-3x^2-4}=0$;
 \end{enumeratea}
 \end{multicols}
\end{esercizio}


\begin{esercizio}[\Ast]
 \label{ese:5.11}
Determinare l'insieme soluzione delle seguenti equazioni.
 \begin{enumeratea}
 \item~$\frac{x^4-4x^2+9}{x^4-3x^2+2}-\frac{x^2-1}{x^2-2}=\frac{x^2-2}{x^2-1}$;
 \item~$(x^2-1)^3+7x^3=3x(4-x-x^3)-(x-2)^3$;
 \item~$\frac{x^2-1}{x^2-3}-\frac{x^2-3}{1-x^2}=\frac{10} 3$.
 \end{enumeratea}
\end{esercizio}

\subsection*{5.3 - Equazioni binomie}

\begin{esercizio}
 \label{ese:5.12}
Determinare l'insieme soluzione delle seguenti equazioni binomie.
\begin{multicols}{3}
 \begin{enumeratea}
 \item~$-2x^3+16=0$;
 \item~$x^5+15=0$;
 \item~$x^4+16=0$;
 \item~$-2x^4+162=0$;
 \item~$-3x^6+125=0$;
 \item~$81x^4-1=0$.
 \end{enumeratea}
 \end{multicols}
\end{esercizio}

\begin{esercizio}[\Ast]
 \label{ese:5.13}
Determinare l'insieme soluzione delle seguenti equazioni binomie.
\begin{multicols}{3}
 \begin{enumeratea}
 \item~$27x^3+1=0$;
 \item~$81x^4+1=0$;
 \item~$81x^8-1=0$;
 \item~$\frac{16}{x^4}-1=0$;
 \item~$x^6-1=0$;
 \item~$8x^3-27=0$.
 \end{enumeratea}
 \end{multicols}
\end{esercizio}

\begin{esercizio}[\Ast]
 \label{ese:5.14}
Determinare l'insieme soluzione delle seguenti equazioni binomie.
\begin{multicols}{3}
 \begin{enumeratea}
 \item~$x^5-1=0$;
 \item~$x^4+81=0$;
 \item~$x^4-4=0$;
 \item~$3x^5+96=0$;
 \item~$49x^6-25=0$;
 \item~$\frac 1{x^3}=27$.
 \end{enumeratea}
 \end{multicols}
\end{esercizio}
%\newpage
\pagebreak
\begin{esercizio}[\Ast]
 \label{ese:5.15}
Determinare l'insieme soluzione delle seguenti equazioni binomie.
\begin{multicols}{3}
 \begin{enumeratea}
 \item~$x^4-\np{10000}=0$;
 \item~$\np{100000}x^5+1=0$;
 \item~$x^6-\np{64000000}=0$;
 \item~$x^4+625=0$;
 \item~$8x^3-27=0$;
 \item~$8x^3+9=0$.
 \end{enumeratea}
 \end{multicols}
\end{esercizio}

\begin{esercizio}[\Ast]
 \label{ese:5.16}
Determinare l'insieme soluzione delle seguenti equazioni binomie.
\begin{multicols}{3}
 \begin{enumeratea}
 \item~$81x^4-16=0$;
 \item~$16x^4-9=0$;
 \item~$\frac 8{x^3}-125=0$;
 \item~$\frac{81}{x^3}=27$;
 \item~$81x^4=1$;
 \item~$x^3-\frac 1{27}=0$.
 \end{enumeratea}
 \end{multicols}
\end{esercizio}

\begin{esercizio}
 \label{ese:5.17}
Determinare l'insieme soluzione delle seguenti equazioni binomie.
\begin{multicols}{3}
 \begin{enumeratea}
 \item~$\frac{x^6}{64}-1=0$;
 \item~$\frac{64}{x^6}=1$;
 \item~$x^6=6$;
 \item~$x^{10}+10=0$;
 \item~$x^{100}=0$;
 \item~$10x^5-10=0$.
 \end{enumeratea}
 \end{multicols}
\end{esercizio}

\begin{esercizio}[\Ast]
 \label{ese:5.18}
Determinare l'insieme soluzione delle seguenti equazioni binomie.
\begin{multicols}{3}
 \begin{enumeratea}
 \item~$\frac 1{81}x^4-1=0$;
 \item~$\frac 1{x^4}-81=0$;
 \item~$\sqrt[3]2\;x^6=\sqrt[3]{24}$;
 \item~$\frac 3 5x^3=\frac{25} 9$;
 \item~$x^8-256=0$;
 \item~$x^{21}+1=0$.
 \end{enumeratea}
 \end{multicols}
\end{esercizio}

\begin{esercizio}[\Ast]
 \label{ese:5.19}
Determinare l'insieme soluzione delle seguenti equazioni binomie.
\begin{multicols}{3}
 \begin{enumeratea}
 \item~$\frac 1{243}x^5+1=0$;
 \item~$x^3+3\sqrt 3=0$;
 \item~$6x^{12}-12=0$;
 \item~$\frac{x^3}{\sqrt 2}-\frac{\sqrt[3]2}{\sqrt 3}=0$;
 \item~$\sqrt 3\;x^3-3\sqrt[3]3=0$;
 \item~$\frac{x^4} 9-\frac 9{25}=0$.
 \end{enumeratea}
 \end{multicols}
\end{esercizio}

\begin{esercizio}
\label{ese:5.20}
Determinare l'insieme soluzione delle seguenti equazioni binomie.
\begin{multicols}{2}
 \begin{enumeratea}
 \item~$(x-1)^4=16$;
 \item~$(x^2-1)^3-27=0$;
 \item~$ \frac 3{x^4-1}=\frac 5{x^4+1} $;
 \item~$ \frac{x^4(x^2+2)-5}{x^2-1}=2(x^2+1) $.
 \end{enumeratea}
 \end{multicols}
\end{esercizio}

\subsection*{5.4 - Equazioni trinomie}

\begin{esercizio}[\Ast]
\label{ese:5.21}
Determinare l'insieme soluzione delle seguenti equazioni biquadratiche.
\begin{multicols}{3}
 \begin{enumeratea}
 \item~$x^4-13x^2+36=0$;
 \item~$2x^4-20x^2+18=0$;
 \item~$x^4-\frac{37} 9x^2+\frac 4 9=0$;
 \item~$x^4-\frac{13} 3x^2+\frac 4 3=0$;
 \item~$-x^4+\frac{17} 4x^2-1=0$;
 \item~$-2x^4+\frac{65} 2x^2-8=0$.
 \end{enumeratea}
\end{multicols}
\end{esercizio}

\begin{esercizio}[\Ast]
 \label{ese:5.22}
Determinare l'insieme soluzione delle seguenti equazioni biquadratiche.
\begin{multicols}{3}
 \begin{enumeratea}
 \item~$-2x^4+82x^2-800=0$;
 \item~$-3x^4+\frac{85} 3x^2-12=0$;
 \item~$x^4-\frac{16} 3x^2+\frac{16} 3=0$;
 \item~$x^4-7x^2+6=0$;
 \item~$x^4-10x^2+16=0$;
 \item~$-3x^4+9x^2+12=0$.
 \end{enumeratea}
\end{multicols}
\end{esercizio}
%\newpage
\pagebreak
\begin{esercizio}[\Ast]
\label{ese:5.23}
Determinare l'insieme soluzione delle seguenti equazioni biquadratiche.
\begin{multicols}{3}
 \begin{enumeratea}
 \item~$-\frac 1 2x^4+\frac 5 2x^2+18=0$;
 \item~$x^4+\frac{15} 4x^2-1=0$;
 \item~$-8x^4-\frac 7 2x^2+\frac 9 2=0$;
 \item~$-16x^4-63x^2+4=0$;
 \item~$x^4-2x^2-15=0$;
 \item~$x^4-2x^2-3=0$.
 \end{enumeratea}
\end{multicols}
\end{esercizio}

\begin{esercizio}[\Ast]
\label{ese:5.24}
Determinare l'insieme soluzione delle seguenti equazioni biquadratiche.
\begin{multicols}{2}
 \begin{enumeratea}
 \item~$ x^4-8x^2+16=0 $;
 \item~$8x^2+\frac{6x^2+x-4}{x^2-1}=4-\frac{3+4x}{1+x}$.
 \end{enumeratea}
\end{multicols}
\end{esercizio}

\begin{esercizio}
\label{ese:5.25}
È vero che l'equazione $4x^4-4=0$ ha quattro soluzioni reali a due a due coincidenti? Rispondi senza risolvere l'equazione.
\end{esercizio}

\begin{esercizio}
 \label{ese:5.26}
È vero che l'equazione $-x^4+2x^2-1=0$ ha quattro soluzioni reali a due a due coincidenti? Rispondi senza risolvere l'equazione.
\end{esercizio}

\begin{esercizio}
 \label{ese:5.27}
Perché le seguenti equazioni non hanno soluzioni reali?

\boxA\; $x^4+\frac{37} 4x^2+\frac 9 4=0$\quad \boxB\; $x^4-x^2+3=0$\quad\boxC\; $-2x^4-x^2-5=0$\quad\boxD\; $-x^4-5x^2-4=0$
\end{esercizio}

\begin{esercizio}[\Ast]
 \label{ese:5.28}
Senza risolvere le seguenti equazioni, dire se ammettono soluzioni reali:

\boxA\; $2x^4+5x^2-4=0$\quad \boxB\; $2x^4-5x^2+4=0$\quad\boxC\; $x^4-5x^2+1=0$\quad\boxD\; $-4x^4+5x^2-1=0$
\end{esercizio}

\begin{esercizio}[\Ast]
 \label{ese:5.29}
Data l'equazione $x^2\cdot \left(x^2-2a+1\right)=a\cdot (1-a)$ determinare per quali valori del parametro $a$ si hanno quattro soluzioni reali.
\end{esercizio}

\begin{esercizio}
 \label{ese:5.30}
È vero che la somma delle radici dell'equazione $ax^4+bx^2+c=0$ è nulla?
\end{esercizio}

\begin{esercizio}
 \label{ese:5.31}
Data l'equazione $ax^4+bx^2+c=0$ verifica le seguenti uguaglianze relative alle soluzioni reali:

\boxA\quad $x_1^2+x_2^2+x_3^2+x_4^2=-\frac{2b} a$\qquad \boxB\quad $x_1^2\cdot x_2^2\cdot x_3^2\cdot x_4^2=\frac c a$
\end{esercizio}

\subsection*{5.5 - Equazioni che si risolvono con sostituzioni}

\begin{esercizio}
 \label{ese:5.32}
Determinare l'insieme soluzione delle seguenti equazioni trinomie.
\begin{multicols}{2}
 \begin{enumeratea}
 \item~$x^6+13x^3+40=0$;
 \item~$x^8-4x^4+3=0$;
 \item~$-x^6+29x^3-54=0$;
 \item~$\frac 1 2x^{10}-\frac 3 2x^5+1=0$;
 \item~$-3x^{12}-3x^6+6=0$;
 \item~$2x^8+6x^4+4=0$.
 \end{enumeratea}
\end{multicols}
\end{esercizio}

\begin{esercizio}[\Ast]
 \label{ese:5.33}
Determinare l'insieme soluzione delle seguenti equazioni trinomie.
\begin{multicols}{2}
 \begin{enumeratea}
 \item~$-x^8-6x^4+7=0$;
 \item~$-2x^6+\frac{65} 4x^3-2=0$;
 \item~$-\frac 3 2x^{10}+\frac{99} 2x^5-48=0$;
 \item~$-\frac 4 3x^{14}-\frac 8 9x^7+\frac 4 9=0$.
 \end{enumeratea}
\end{multicols}
\end{esercizio}

\begin{esercizio}[\Ast]
 \label{ese:5.34}
Risolvi con le opportune sostituzioni le seguenti equazioni.
\begin{multicols}{2}
 \begin{enumeratea}
 \item~$\left(x^3+1\right)^3-8=0$;
 \item~$2\left(\frac{x+1}{x-1}\right)^2-3\left(\frac{x+1}{x-1}\right)-1=0$;
 \item~$\left(x^2+1\right)^2-6\left(x^2+1\right)+8=0$;
 \item~$\left(x+\frac 1 x\right)^2=\frac{16} 9$;
 \item~$\left(x+\frac 1 x\right)^2-16\left(x+\frac 1 x\right)=0$;
 \item~$\left(x^2-\frac 1 3\right)^2-12\left(x^2-\frac 1 3\right)+27=0$.
 \end{enumeratea}
\end{multicols}
\end{esercizio}

\begin{esercizio}[\Ast]
\label{ese:5.35}
Risolvi con le opportune sostituzioni le seguenti equazioni.
 \begin{enumeratea}
 \item~$(2x-1)^3=8$;
 \item~$(x+1)^3+6(x+1)^2-(x+1)-30=0$;
 \item~$(x^2+1)^3-4(x^2+1)^2-19(x^2+1)-14=0$;
 \item~$\frac{3x}{x+1}-\left(\frac{3x}{x+1}\right)^3=0$;
 \item~$\left(x-1\right)^2+\frac{x-3}{\left(x-1\right)^2}=\frac{x+6}{(1-x)^2}$;
 \item~$\left(\frac{x+1}{x-1}\right)^4-5\left(\frac{x+1}{x-1}\right)^2+4=0$.
 \end{enumeratea}
\end{esercizio}

\begin{esercizio}
 \label{ese:5.36}
Risolvi con le opportune sostituzioni le seguenti equazioni.
\begin{multicols}{2}
 \begin{enumeratea}
 \item~$ (x^3+2)^5=1 $;
 \item~$ \left(\frac x{x-1}\right)^4-13\left(\frac x{x-1}\right)^2+36=0 $;
 \item~$ \left(\frac{x+1}{x+2}\right)^4-10\left(\frac{x+1}{x+2}\right)^2+9=0 $;
 \item~$ \left(x-\sqrt 2\right)^6-4\left(x-\sqrt 2\right)^3+3=0 $;
 \item~$ \left(\frac{x+1} x\right)^{10}-33\left(\frac{x+1} x\right)^5+32=0 $;
 \item~$ \left(\frac x{x+1}\right)^2-13+36\left(\frac{x+1} x\right)^2=0 $.
 \end{enumeratea}
\end{multicols}
\end{esercizio}

\begin{esercizio}
\label{ese:5.37}
Risolvi con le opportune sostituzioni le seguenti equazioni.
 \begin{enumeratea}
 \item~$ \frac{x-3}{x+3}+2=15\left(\frac{x+3}{x-3}\right) $;
 \item~$ \left(x^2-1\right)^3+\frac 8{\left(x^2-1\right)^3}=9 $;
 \item~$ \left(\frac 1{x^2-1}\right)^3-3\left(\frac 1{x^2-1}\right)^3-4\left(\frac 1{x^2-1}\right)^3+12=0 $.
 \end{enumeratea}
\end{esercizio}

\subsection*{5.6 - Equazioni reciproche}

\begin{esercizio}[\Ast]
\label{ese:5.38}
Risolvi le seguenti equazioni reciproche di prima specie.
\begin{multicols}{2}
\begin{enumeratea}
\item $3x^3+13x^2+13x+3=0$;
\item $2x^3-3x^2-3x+2=0$;
\item $5x^3-21x^2-21x+5=0$;
\item $12x^3+37x^2+37x+12=0$;
\item $10x^3-19x^2-19x+10=0$;
\item $15x^3-19x^2-19x+15=0$.
\end{enumeratea}
\end{multicols}
\end{esercizio}

\begin{esercizio}[\Ast]
\label{ese:5.39}
Risolvi le seguenti equazioni reciproche di prima specie.
\begin{enumeratea}
\item $4x^3+13x^2-13x=4$;
\item $4x^3-13x^2=13x-4$;
\item $3x(10x-19)+9x(x-2)=10(x+1)(x^2-x+1)$;
\item $2x^3-(3\sqrt 2+2)x^2-(3\sqrt 2+2)x+2=0$.
\end{enumeratea}
\end{esercizio}

\begin{esercizio}[\Ast]
\label{ese:5.40}
Risolvi le seguenti equazioni reciproche di prima specie.
\begin{enumeratea}
\item $x^3+x^2(2\sqrt 2+1)+x(2\sqrt 2+1)+1=0$;
\item $x^3-3x^2-3x+1=0$;
\item ${ax}^3+(a^2+a+1)x^2+(a^2+a+1)x+1=0$.
\end{enumeratea}
\end{esercizio}

 \begin{esercizio}[\Ast]
\label{ese:5.41}
Dopo aver verificato che $x=3$ è radice dell'equazione $3x^3-13x^2+13x-3=0$, verificate che l'equazione ammette come soluzione $x=\frac 1 3$.
 \end{esercizio}

\begin{esercizio}
\label{ese:5.42}
Determina il valore di verità delle seguenti proposizioni:
\begin{enumeratea}
\item l'equazione $ax^3+bx^2+cx+d=0$ ammette sempre $x=-1$ come soluzione;
\item se nell'equazione $ax^3+bx^2+cx+d=0$ si ha $a=d$ e $b=c$ allora $x=-1$ è una soluzione;
\item in una equazione reciproca di terzo grado la somma dei coefficienti è nulla;
\item se in $ax^3+bx^2+cx+d=0$ si ha $a+d=0$ e $b+c=0$ allora $x=1$ appartiene all'$\IS$
\end{enumeratea}
\end{esercizio}

\begin{esercizio}[\Ast]
\label{ese:5.43}
Risolvi le seguenti equazioni reciproche di seconda specie.
\begin{multicols}{2}
\begin{enumeratea}
\item $6x^3-19x^2+19x-6=0$;
\item $7x^3-57x^2+57x-7=0$;
\item $3x^3+7x^2-7x-3=0$;
\item $12x^3+13x^2-13x-12=0$;
\item $10x^3+19x^2-19x-10=0$;
\item $x^3+3x^2-3x-1=0$.
\end{enumeratea}
\end{multicols}
\end{esercizio}

\begin{esercizio}[\Ast]
\label{ese:5.44}
Risolvi le seguenti equazioni reciproche di seconda specie.
\begin{multicols}{2}
\begin{enumeratea}
\item $\frac{x^3-1} x=\frac{21} 4\cdot (1-x)$;
\item $5x^3+(6\sqrt 5-5)x^2+x(5-6\sqrt 5)-5=0$;
\item $x^3+13x^2-13x-1=0$;
\item $4x^3+(5\sqrt 5-1)x^2+x(1-5\sqrt 5)=4$.
\end{enumeratea}
\end{multicols}
\end{esercizio}

\begin{esercizio}
\label{ese:5.45}
Data l'equazione $a_0\left(t^2-2\right)+a_1t+a_2=0\;\to \;a_0t^2+a_1t+a_2-2a_0=0$ stabilire quale condizione deve sussistere tra i coefficienti affinché esistano valori reali dell'incognita $ t $.
\end{esercizio}

\begin{esercizio}[\Ast]
\label{ese:5.46}
Risolvi le seguenti equazioni di quarto grado reciproche di prima specie.
\begin{multicols}{2}
\begin{enumeratea}
\item $x^4-5x^3+8x^2-5x+1=0$;
\item $x^4+5x^3-4x^2+5x+1=0$;
\item $x^4+2x^3-13x^2+2x+1=0$;
\item $x^4-\frac 5 6x^3-\frac{19} 3x^2-\frac 5 6x+1=0$.
\end{enumeratea}
\end{multicols}
\end{esercizio}

\begin{esercizio}[\Ast]
 \label{ese:5.47}
Risolvi le seguenti equazioni di quarto grado reciproche di seconda specie.
\begin{multicols}{2}
\begin{enumeratea}
\item $x^4-3x^3+3x-1=0$;
\item $4x^4-5x^3+5x-4=0$;
\item $3x^4+7x^3-7x-3=0$;
\item $x^4-7x^3+7x-1=0$.
\end{enumeratea}
\end{multicols}
\end{esercizio}

\begin{esercizio}[\Ast]
 \label{ese:5.48}
Risolvi le seguenti equazioni di quarto grado reciproche di seconda specie.
\begin{multicols}{2}
\begin{enumeratea}
\item $5x^4-11x^3+11x-5=0$;
\item $6x^4-13x^3+13x-6=0$;
\item $7x^4-15x^3+15x-7=0$;
\item $x^3(x-4)=1-4x$.
\end{enumeratea}
\end{multicols}
\end{esercizio}

\begin{esercizio}[\Ast]
 \label{ese:5.49}
Risolvi le seguenti equazioni di quarto grado reciproche di seconda specie.
\begin{multicols}{2}
\begin{enumeratea}
\item $\frac{x-5}{5x-1}+\frac 1{x^3}=0$;
\item $x^4-x^3+x-1=0$;
\item $\frac{x^4+2x-1}{8x^3}-\frac{1+8x^2}{4x^2}+\frac{x-1} x+\frac{1+x}{x^2}=0$.
\end{enumeratea}
\end{multicols}
\end{esercizio}

\begin{esercizio}
 \label{ese:5.50}
Quale condizione deve sussistere tra i coefficienti dell'equazione $a_0x^2+a_1x+a_0=0$ affinché siano reali le sue soluzioni?
\end{esercizio}

\begin{esercizio}[\Ast]
 \label{ese:5.51}
Determinare per quale valore di $k$, l'equazione $\left(2k-\sqrt 2\right)x^4+5x^3-5x-2\sqrt 2=0$ è reciproca. È vero che $\IS=\{+1\text{, }-1\}$?
\end{esercizio}
%\newpage
\pagebreak
\subsection{Esercizi riepilogativi}

\begin{esercizio}[\Ast] %5.52
Risolvi le seguenti equazioni di grado superiore al secondo.
\begin{multicols}{2}
\begin{enumeratea}
\item $6x^3+7x^2-7x-6=0$;
\item $2x^3+5x^2+5x+2=0$;
\item $x^3-3x^2+3x-1=0$;
\item $3x^3-4x^2+4x-3=0$;
\item $2x^4-5x^3+5x-2=0$;
\item $-5x^4+3x^3-3x+5=0$.
\end{enumeratea}
\end{multicols}
\end{esercizio}

\begin{esercizio}[\Ast] %5.53
Risolvi le seguenti equazioni di grado superiore al secondo.
\begin{multicols}{2}
\begin{enumeratea}
\item $2x^5-3x^4+4x^3-4x^2+3x-2=0$;
\item $-2x^4+8x^3-8x+2=0$;
\item $2x^3-5x^2-5x+2=0$;
\item $3x^3-6x^2-6x+3=0$;
\item $5x^3-7x^2+7x-5=0$;
\item $4x^3-20x^2+20x-4=0$.
\end{enumeratea}
\end{multicols}
\end{esercizio}

\begin{esercizio}[\Ast] %5.54
Risolvi le seguenti equazioni di grado superiore al secondo.
\begin{multicols}{2}
\begin{enumeratea}
\item $5x^3-5x^2-5x+5=0$;
\item $4x^3-9x^2+9x-4=0$;
\item $\frac 3 2x^3+\frac 7 4x^2-\frac 7 4x-\frac 3 2=0$;
\item $3x^3-2x^2+2x-3=0$;
\item $-2x^3+10x^2+10x-2=0$;
\item $x^4-\frac 9 4x^3-\frac{13} 2x^2-\frac 9 4x+1=0$.
\end{enumeratea}
\end{multicols}
\end{esercizio}

\begin{esercizio}[\Ast] %5.55
Risolvi le seguenti equazioni di grado superiore al secondo.
\begin{multicols}{2}
\begin{enumeratea}
\item $x^4-4x^3+6x^2-4x+1=0$;
\item $x^4+\frac{10} 3x^3+2x^2+\frac{10} 3x+1=0$;
\item $x^4-4x^3+2x^2-4x+1=0$;
\item $x^4-x^3+x-1=0$;
\item $x^4-6x^3+6x-1=0$;
\item $x^4-3x^3+2x^2-3x+1=0$.
\end{enumeratea}
\end{multicols}
\end{esercizio}

\begin{esercizio}[\Ast] %5.56
Risolvi le seguenti equazioni di grado superiore al secondo.
\begin{multicols}{2}
\begin{enumeratea}
\item $x^4-5x^3-12x^2-5x+1=0$;
\item $3x^4-x^3+x-3=0$;
\item $2x^4-5x^3+4x^2-5x+2=0$;
\item $2x^4-x^3+4x^2-x+2=0$;
\item $3x^4-7x^3+7x-3=0$;
\item $3x^4-6x^3+6x-3=0$.
\end{enumeratea}
\end{multicols}
\end{esercizio}

\begin{esercizio}[\Ast] %5.57
Risolvi le seguenti equazioni di grado superiore al secondo.
\begin{multicols}{2}
\begin{enumeratea}
\item $2x^4-6x^3+4x^2-6x+2=0$;
\item $x^4+8x^3-8x-1=0$;
\item $6x^4-37x^3+37x-6=0$;
\item $x^5-3x^4+2x^3+2x^2-3x+1=0$;
\item $x^5-2x^4-5x^3-5x^2-2x+1=0$;
\item $x^5+3x^4+x^3-x^2-3x-1=0$.
\end{enumeratea}
\end{multicols}
\end{esercizio}

\begin{esercizio}[\Ast] %5.58
Risolvi le seguenti equazioni di grado superiore al secondo.
\begin{multicols}{2}
\begin{enumeratea}
\item $x^5+x^4+x^3-x^2-x-1=0$;
\item $x^5-2x^4+x^3-x^2+2x-1=0$;
\item $x^5-5x^3-5x^2+1=0$;
\item $x^5+3x^4-2x^3+2x^2-3x-1=0$;
\item $2x^5-2x^4+2x^3+2x^2-2x+2=0$;
\item $x^6-x^5-5x^4+5x^2+x-1=0$.
\end{enumeratea}
\end{multicols}
\end{esercizio}
%\newpage
\pagebreak
\begin{esercizio}[\Ast] %5.59
Risolvi le seguenti equazioni di grado superiore al secondo.
\begin{multicols}{2}
\begin{enumeratea}
\item $x^6-x^5-x^4+2x^3-x^2-x+1=0$;
\item $x^5-2x^4+x^3+x^2-2x+1=0$;
\item $x^5-\frac{11} 4x^4-\frac{55} 8x^3+\frac{55} 8x^2+\frac{11} 4x-1=0$;
\item $x^5-4x^4+\frac{13} 4x^3+\frac{13} 4x^2-4x+1=0$;
\item $x^6+\frac{13} 6x^5+x^4-x^2-\frac{13} 6x-1=0$;
\item $x^6+\frac{16} 3x^5+\frac{23} 3x^4-\frac{23} 3x^2-\frac{16} 3x-1=0$.
\end{enumeratea}
\end{multicols}
\end{esercizio}

\begin{esercizio}[\Ast] %5.60
Risolvi le seguenti equazioni di grado superiore al secondo.
\begin{multicols}{2}
\begin{enumeratea}
\item $x^6+x^4-x^2-1=0$;
\item $x^6-4x^5-x^4+8x^3-x^2-4x+1=0$;
\item $x^6+2x^4+2x^2+1=0$;
\item $3(2x-2)^3+(10x-5)^2-25=0$.
\end{enumeratea}
\end{multicols}
\end{esercizio}

\begin{esercizio}[\Ast] %5.61
Risolvi le seguenti equazioni di grado superiore al secondo.
\begin{multicols}{2}
\begin{enumeratea}
\item $\frac{6x^2-2}{x^2+1}=\frac 6{x^4-5x^2-6}+\frac{5x}{6-x^2}$;
\item $x^4+5x(x+1)^2+(1-2x)(1+2x)=0$;
\item $\frac{9x^2(x+4)}{9x+1+\sqrt{10}}=\frac{9x+1-\sqrt{10}}{x-6}$;
\item $\frac{x^2(x+4)}{x-1}-\frac{8x+1}{x+1}-\frac{2x}{x^2-1}=0$.
\end{enumeratea}
\end{multicols}
\end{esercizio}

\begin{esercizio}[\Ast] %5.62
Nell'equazione $(2-a)x^5-x^4+(3+a)x^3+2bx^2+x+5b=0$ determinare $a$ e $b$ in modo che l'equazione sia reciproca.
\end{esercizio}

\subsection{Risposte}
\paragraph{5.1.} a)~$-4;-3;2$,\quad b)~$\frac 1 2;-3;-\frac 4 3$,\quad c)~$\frac 5 2;1;\frac 3 2$,\quad d)~$4;-7$,\quad e)~$-3;-\frac 1 3;+2$,\quad f)~$-4;+\frac 7 3;+2$.

\paragraph{5.2.} a)~$0;+\frac 1 2;+1$,\quad b)~$+4$,\quad c)~$-5$,\quad d)~$3;-\frac 1 2$,\quad e)~$4;\frac 2 3;-2$,\quad f)~$2;1;-\frac 1 2$.

\paragraph{5.3.} a)~$1;-2$,\quad b)~$2$,\quad c)~$1;-1$,\quad d)~$5;1;-2$,\quad e)~$2;-9$,\quad f)~$1;-1;2$.

\paragraph{5.4.} a)~$-1;\frac 1 2;\frac 2 3$,\quad b)~$-1;6;2$.

\paragraph{5.6.} a)~$\{-2;1\}$,\quad b)~$\{-1\}$,\quad c)~$\{-3\}$,\quad d)~$\{-1;1\}$,\quad e)~$\{-3;-1;1\}$,\quad f)~$\left\{-\frac1 2;\frac 2 3;1\right\}$.

\paragraph{5.7.} a)~$\{1;2;3\}$,\quad b)~$\left\{0;\frac1 2\right\}$,\quad c)~$\{2;-2;3\}$,\quad d)~$\{-1;+1\}$,\quad e)~$\{-1;1;2\}$,\quad f)~$\{-1\}$.

\paragraph{5.8.} a)~$\left\{-\frac 1 3;1\right\}$,\quad b)~$\left\{\pm 1;1\pm \sqrt 2\right\}$,\quad c)~$\left\{-3;\pm \sqrt 2\right\}$,\quad d)~$\{-2\}$,\quad e)~$\left\{-\sqrt 3;0;+\sqrt 3\right\}$.

\paragraph{5.9.} a)~$\left\{0;\sqrt 2-1;-\left(\frac{\sqrt 2} 2+1\right)\right\}$,\quad b)~$\left\{0;\frac 3 4\right\}$,\quad c)~$\insR$,\quad d)~$\{0;4;2\pm \sqrt 5\}$,\quad e)~$\{0\}$,\quad f)~$\left\{0;\frac 2 3;2\right\}$.

\paragraph{5.10.} a)~$\left\{\frac 1 2\right\}$,\quad b)~$\left\{-\frac 3 2;-2\right\}$,\quad c)~$\{0;3\pm \sqrt 2\}$,\quad d)~$\left\{1;\frac{-1\pm \sqrt 5} 2\right\}$,\quad e)~$\emptyset $,\quad f)~$\{\pm 1;\pm \sqrt 2\}$.

\paragraph{5.11.} b)~$\left\{\pm \sqrt{1+\sqrt 5}\right\}$,\quad c)~$\{1;-\sqrt[3]9\}$,\quad d)~$\{0;\pm 2\}$.

\paragraph{5.12.} a)~$\{2\}$,\quad b)~$\{-\sqrt[5]{15}\}$,\quad c)~$\emptyset $,\quad d)~$\{-3;+3\}$,\quad e)~$\left\{\pm \frac{\sqrt 5}{\sqrt[6]3}\right\}$,\quad f)~$\left\{\pm \frac 1 3\right\}$.

\paragraph{5.13.} a)~$\left\{-\frac 1 3\right\}$,\quad b)~$\emptyset $,\quad c)~$\left\{-\frac{\sqrt 3} 3;+\frac{\sqrt 3} 3\right\}$,\quad d)~$\left\{-2;+2\right\}$,\quad e)~$\{-1;1\}$,\quad f)~$\left\{\frac 3 2\right\}$.

\paragraph{5.14.} a)~$\{1\}$,\quad b)~ $\emptyset $,\quad c)~$\left\{-\sqrt 2;\sqrt 2\right\}$,\quad d)~$\left\{-2\right\}$,\quad e)~$\left\{-\sqrt[3]{\frac 5 7};\sqrt[3]{\frac 5 7}\right\}$,\quad f)~$\left\{\frac 1 3\right\}$.

\paragraph{5.15.} a)~$\{\pm 10\}$,\quad b)~ $\left\{-\frac 1{10}\right\}$,\quad c)~$\{\pm 20\}$,\quad d)~$\emptyset $,\quad e)~$\left\{\frac 3 2\right\}$,\quad f)~${I.S}=\left\{-\frac 1 2\sqrt[3]9\right\}$.

\paragraph{5.16.} a)~$\left\{\pm \frac 2 3\right\}$,\quad b)~$\left\{\pm \frac{\sqrt 3} 2\right\}$.

\paragraph{5.18.} c)~$\left\{\pm 2^{\frac 1 9}3^{\frac 1{18}}\right\}$,\quad d)~${I.S}=\left\{\frac 5 3\right\}$.

\paragraph{5.19.} e)~$\left\{3^{\frac 5{18}}\right\}$,\quad f)~$\left\{\pm 3\frac{\sqrt 5} 5\right\}$.

\paragraph{5.21.} a)~$\{\pm 3;\pm 2\}$,\quad b)~$\{\pm 1; \pm 3\}$,\quad c)~$\left\{\pm 2;\pm \frac 1 3\right\}$,\quad d)~ $\left\{\pm 2;\pm \frac{\sqrt 3} 3\right\}$,\quad e)~$\left\{\pm 2;\pm \frac 1 2\right\}$,\quad f)~$\left\{\pm 4;\pm \frac 1 2\right\}$.

\paragraph{5.22.} a)~$\{\pm 4;\pm 5\}$,\; b)~$\left\{\pm 3;\pm \frac 2 3\right\}$,\; c)~$\left\{\pm 2;\pm \frac{2\sqrt 3} 3\right\}$,\; d)~$\left\{\pm 1;\pm \sqrt 6\right\}$,\; e)~$\left\{\pm \sqrt 2;\pm 2\sqrt 2\right\}$,\; f)~$\{-2;2\}$.

\paragraph{5.23.} a)~$\{\pm 3\}$,\quad b)~$\left\{\pm \frac 1 2\right\}$,\quad c)~$\left\{\pm \frac 3 4\right\}$,\quad d)~$\left\{\pm \frac 1 4\right\}$,\quad e)~$\left\{\pm \sqrt 5\right\}$,\quad f)~$\left\{\pm \sqrt 3\right\}$.

\paragraph{5.24.} b)~$\left\{\pm \frac{\sqrt 3} 2\right\}$.

\paragraph{5.28.} sì, no, sì, no.

\paragraph{5.29.} $a>1$.

\paragraph{5.32.} a)~$\left\{-2;-\sqrt[3]5\right\}$,\quad b)~$\left\{\pm 1;\pm \sqrt[4]3\right\}$,\quad c)~$\left\{3;\sqrt[3]2\right\}$,\quad d)~$\left\{1;\sqrt[5]2\right\}$,\quad e)~$\{\pm 1\}$,\quad f)~$\emptyset $.

\paragraph{5.33.} a)~$\{\pm 1\}$,\quad b)~$\left\{2;\frac 1 2\right\}$,\quad c)~$\left\{1;2\right\}$,\quad d)~$\left\{-1;\sqrt[7]{\frac 1 3}\right\}$.

\paragraph{5.34.} a)~$\{1\}$,\quad b)~$\left\{\frac{3\pm\sqrt{17}} 2\right\}$,\quad c)~$\left\{\pm\sqrt 3;\pm 1\right\}$,\quad d)~$\emptyset $,\quad e)~$\left\{8\pm3\sqrt 7\right\}$,\quad f)~$\left\{\pm \frac{2\sqrt{21}} 3;\pm \frac{\sqrt{30}} 3\right\}$.

\paragraph{5.35.} a)~$\left\{\frac 3 2\right\}$,\quad b)~$\left\{-6;-4;1\right\}$,\quad c)~$\left\{\pm\sqrt 6\right\}$,\quad d)~$\left\{-\frac 1 4;0;\frac 1 2\right\}$,\quad e)~$\left\{1\pm \sqrt 3\right\}$,\quad f)~$\left\{0;3;\frac 1 3\right\}$.

\paragraph{5.38.} a)~$\left\{-1;-\frac 1 3;-3\right\}$,\quad b)~$\left\{-1;2;\frac 1 2 \right\}$,\quad c)~$\left\{-1;5;\frac 1 5\right\}$,\quad d)~$\left\{-\frac 4 3;-1;-\frac 3 4\right\}$,\quad e)~$\left\{-1;\frac 2 5;\frac 5 2\right\}$,\protect \\
f)~$\left\{-1;\frac 3 5;\frac 5 3\right\}$.

\paragraph{5.39.} a)~$\left\{-4;-\frac 1 4;1\right\}$,\quad b)~$\left\{-1;4;\frac 1 4\right\}$,\quad c)~$\left\{-1;\frac{49\pm \sqrt{\np{2001}}}{20}\right\}$,\quad d)~$\left\{-1;-\sqrt 2;-\frac{\sqrt 2} 2\right\}$.

\paragraph{5.40.} a)~$\{-1;\sqrt 2-1;-\sqrt 2+1\}$,\quad b)~$\{-1;\sqrt 3+2;2-\sqrt 3\}$,\quad c)~$\left\{-1;-a;-\frac 1 a\right\}$.

\paragraph{5.43.} a)~$\left\{1;\frac 2 3;\frac 3 2\right\}$,\quad b)~$\left\{1;7;\frac 1 7\right\}$,\quad c)~$\left\{-3;-\frac 1 3;1\right\}$,\quad d)~$\left\{-\frac 4 3;-\frac 3 4;1\right\}$,\quad e)~$\left\{-\frac 5 2;-\frac 2 5;1\right\}$,\protect\\ f)~$\left\{2\pm\sqrt 3;-1\right\}$.

\paragraph{5.44.} a)~$\left\{1;\frac{-25\pm \sqrt{561}} 8\right\}$,\quad b)~$\left\{1;-\sqrt 5;-\frac{\sqrt 5} 5\right\}$,\quad c)~$\left\{1;-7\pm 4\sqrt 3\right\}$,\quad d)~$\left\{1;-\sqrt 5-1;\frac 1 4-\frac{\sqrt 5} 4\right\}$.

\paragraph{5.46.} a)~$\left\{1;\frac{3\pm \sqrt 5} 2\right\}$,\quad b)~$\left\{-3\pm 2\sqrt 2\right\}$,\quad c)~$\left\{\frac{-5\pm \sqrt{21}} 2;\frac{3\pm \sqrt 5} 2\right\}$,\quad d)~$\left\{3;\frac 1 3;2-\frac 1 2\right\}$.

\paragraph{5.47.} a)~$\left\{\pm 1;\frac{3\pm \sqrt 5} 2\right\}$,\quad b)~$\left\{\pm 1\right\}$,\quad c)~$\left\{\pm 1;\frac{-7\pm \sqrt{13}} 6\right\}$,\quad d)~$\left\{\pm 1;\frac{7\pm 3\sqrt 5} 2\right\}$.

\paragraph{5.48.} a)~$\left\{\pm 1;\frac{11\pm \sqrt{21}}{10}\right\}$,\quad b)~$\left\{\pm 1;\frac 2 3;\frac 3 2\right\}$,\quad c)~$\left\{\pm 1;\frac{15\pm \sqrt{29}}{14}\right\}$,\quad d)~$\left\{\pm 1;2\pm \sqrt 3\right\}$.

\paragraph{5.49.} a)~$\left\{\pm 1;\frac{5\pm \sqrt{21}} 2\right\}$,\quad b)~$\left\{\pm 1\right\}$,\quad c)~$\{\pm 1;4\pm \sqrt{15}\}$.

\paragraph{5.51.} a)~$k=\frac 3 2\sqrt 2$.

\paragraph{5.52.} a)~$\left\{-\frac 3 2;-\frac 2 3;1\right\}$,\quad b)~$\left\{-1\right\}$,\quad c)~$\left\{1\right\}$,\quad d)~$\left\{1\right\}$,\quad e)~$\left\{\frac 1 2;2;\pm 1\right\}$,\quad f)~$\left\{\pm 1\right\}$.

\paragraph{5.53.} a)~$\left\{1\right\}$,\quad b)~$\left\{2\pm \sqrt 3;\pm 1\right\}$,\quad c)~$\left\{-1;\frac{7\pm\sqrt{33}} 4\right\}$,\quad d)~$\left\{-1;\frac{3\pm\sqrt 5} 2 \right\}$,\quad e)~$\{1\}$,\protect\\ \quad f)~$\left\{x=1;x=2\pm\sqrt 3\right\}$.

\paragraph{5.54.} a)~$\{\pm 1\}$,\quad b)~$\{1\}$,\quad c)~$\left\{1;\pm\frac 2 3\right\}$,\quad d)~$\{1\}$,\quad e)~$\left\{1;3\pm 2\sqrt 2\right\}$,\quad f)~$\left\{-1;4;\frac 1 4\right\}$.

\paragraph{5.55.} a)~$\{\pm 1\}$,\quad b)~$\left\{-3;-\frac 1 3\right\}$,\quad c)~$\left\{2\pm\sqrt 3\right\}$,\quad d)~$\{\pm 1\}$,\quad e)~$\left\{\pm 1;3\pm 2\sqrt 2\right\}$,\quad f)~$\left\{\frac{3\pm\sqrt 5} 2\right\}$.

\paragraph{5.56.} a)~$\left\{-1;\frac{7\pm 3\sqrt 5} 2\right\}$,\quad b)~$\{\pm 1\}$,\quad c)~$\left\{2;\frac 1 2\right\}$,\quad d)~$\emptyset$,\quad e)~$\left\{\pm 1;\frac{7\pm \sqrt{13}} 6\right\}$,\quad f)~$\{\pm 1\}$.

\paragraph{5.57.} a)~$\left\{\frac{3\pm\sqrt 5} 2\right\}$,\quad b)~$\left\{\pm 1;-4\pm \sqrt{15}\right\}$,\quad c)~$\left\{\pm 1;+6;\frac 1 6\right\}$,\quad d)~$\{\pm 1\}$,\quad e)~$\left\{-1;2\pm\sqrt 3\right\}$,\protect\\ \quad f)~$\left\{1;\frac{-3\pm\sqrt 5} 2\right\}$.

\paragraph{5.58.} a)~$\{1\}$,\quad b)~$\{1\}$,\quad c)~$\left\{-1;\frac{3\pm\sqrt 5} 2\right\}$,\quad d)~$\left\{1;-2\pm\sqrt 3\right\}$,\quad e)~$\{-1\}$,\quad f)~$\left\{\pm 1;\frac{3\pm \sqrt 5} 2\right\}$.

\paragraph{5.59.} a)~$\{\pm 1\}$,\quad b)~$\{\pm 1\}$,\quad c)~$\left\{1;4;\frac 1 4;-2;-\frac 1 2\right\}$,\quad d)~$\left\{-1;2;\frac 1 2\right\}$,\quad e)~$\left\{\pm 1;x=\pm\frac 3 2\right\}$,\protect\\ \quad f)~$\left\{\pm 1-3;-\frac 1 3\right\}$.

\paragraph{5.60.} a)~$\{\pm 1\}$,\quad b)~$\left\{\pm 1;2\pm \sqrt 3\right\}$,\quad c)~$\emptyset$,\quad d)~$\left\{1;\pm\frac 3 2\right\}$.

\paragraph{5.61.} a)~$\left\{2;\frac 1 2;-3;-\frac 1 3\right\}$,\quad b)~$\left\{-2\pm \sqrt 3\right\}$,\quad c)~$\left\{\frac{7\pm 3\sqrt 5} 2;\frac{-5\pm \sqrt{21}} 2\right\}$,\quad d)~$\left\{-3\pm 2\sqrt 2\right\}$.

%\paragraph{5.62.} $a=-\frac{11} 7$ e $b=-\frac 5 7$.