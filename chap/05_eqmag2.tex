% (c)~2014 Claudio Carboncini - claudio.carboncini@gmail.com
% (c)~2014 Dimitrios Vrettos - d.vrettos@gmail.com
\chapter{Equazioni di grado superiore al secondo}
\section{L'equazione di terzo grado, un po' di storia}
Problema: ``Trovare un numero il cui cubo, insieme con due suoi quadrati e dieci volte il numero stesso, dia come somma 20''.

Il problema enunciato venne posto da Giovanni Panormita, astronomo e filosofo alla corte di Federico II, a Leonardo Pisano, detto Fibonacci, che ne tentò la soluzione nella sua opera ``Flos''.

Con il linguaggio matematico attuale il problema si formalizza nell'equazione di terzo grado $x^3+2x^2+10x=20$; Fibonacci pervenne al valore approssimato $x=\np{1,3688}$ come soluzione al problema, senza indicare la via seguita per la sua determinazione. Pur tuttavia egli riuscì a dimostrare che le soluzioni di un'equazione di terzo grado non possono mai esprimersi mediante radicali quadratici neanche se sovrapposti.

Solo tra il 1540 e il 1545, ad opera dei matematici italiani Niccolò Fontana, detto Tartaglia\footnote{soprannome dovuto al suo linguaggio balbettante (1499 ca. - 1557).}, e Gerolamo Cardano\footnote{matematico, medico, astrologo e filosofo (1501 - 1576).}, fu scoperta la formula risolutiva dell'equazione generale di terzo grado.
Cardano dimostrò che ogni equazione di terzo grado $ax^3+bx^2+cx+d=0$ è riconducibile alla forma $y^3+py+q=0$ operando la sostituzione $x=y-\frac b{3a}$, per la quale si ricava la formula risolutiva: 
\[y=\sqrt[3]{-\frac q 2+\sqrt{\left(\frac p 3\right)^3+\left(\frac q 2\right)^2}}+\sqrt[3]{-\frac q 2-\sqrt{\left(\frac p 3\right)^3+\left(\frac q 2\right)^2}}\] 
da cui poi si risale alla soluzione in $x$ dell'equazione assegnata.
\begin{exrig}
\begin{esempio}
Risolvere l'equazione: $x^3+3x^2+6x+5=0$.

Operiamo la sostituzione $x=y-\frac b{3a}$ che in questo caso è $x=y-1$; l'equazione diventa $(y-1)^3+3(y-1)^2+6(y-1)+5=0$ ed eseguendo i calcoli si ha $y^3+3y+1=0$, quindi $p=3$ e $q=1$.
Applicando la formula risolutiva si ha 
\[y=\sqrt[3]{-\frac 1 2+\sqrt{\frac 1 4+1}}+\sqrt[3]{-\frac 1 2-\sqrt{\frac 1 4+1}}=\sqrt[3]{\frac{\sqrt 5-1} 2}+\sqrt[3]{\frac{-\sqrt 5-1} 2}\] 
e quindi, sostituendo $y=x+1$, si ha $x=\sqrt[3]{\dfrac{\sqrt 5-1} 2}+\sqrt[3]{\dfrac{-\sqrt 5-1} 2}-1$.
\end{esempio}

\begin{esempio}
Risolvere l'equazione $x^3=15x+4$ applicando la formula di Cardano.

Notiamo che è $p=-15$ e $q=-4$ e dunque sotto la radice quadrata della formula 
si ha $\left(\frac p 3\right)^3+\left(\frac q 2\right)^2=(-5)^3+(-2)^2=-121$ pertanto non un numero
reale, mentre è evidente la soluzione reale $x=4$, infatti $4^3=64=15\cdot 4+4$. Questa circostanza ha spinto il matematico Raffaele Bombelli\footnote{matematico ed ingegnere italiano, noto anche con il nome di Rafael o Raphael (1526 - 1572).}, ad elaborare nella sua opera ``Algebra'' del 1572, calcoli con radici quadrate di numeri negativi (numeri) che troveranno una sistemazione coerente nella teoria dei \emph{numeri complessi} sviluppata da Friedrich Gauss.\footnote{Johann Carl Friedrich Gauss è stato un matematico, astronomo e fisico tedesco (1777 - 1855).}

Vediamo come possiamo determinare l'$IS$ dell'equazione di Bombelli con le nostre conoscenze. Scriviamo l'equazione nella forma canonica $p(x)=0\:\Rightarrow\: x^3-15x-4=0$; sappiamo che uno zero intero è $x=4$ dunque scomponiamo dividendo $p(x)=x^3-15x-4$ per il binomio $x-4$. Potete verificare che si ottiene $x^3-15x-4=(x-4)\cdot (x^2+4x+1)=0$ da cui, per la legge di annullamento del prodotto, 
\[x-4=0\:\Rightarrow\: x=4\quad\vee\quad x^2+4x+1=0\:\Rightarrow\: x_{1\text{,}2}=-2\pm \sqrt 3.\]

Poco dopo la scoperta della formula risolutiva per le equazioni di terzo grado, il matematico italiano Ferrari\footnote{Lodovico Ferrari (1522 - 1565).} trovò anche la formula per risolvere le equazioni di quarto grado. Le ricerche per trovare la formula che risolvesse l'equazione di quinto grado furono invece vane, non perché i matematici non furono abbastanza ``ingegnosi'' bensì per il fatto che, come dimostrò Galois\footnote{\'{E}variste Galois è stato un matematico francese (1811 - 1832).} non esistono formule che per mezzo di radici ed altre operazioni algebriche possano risolvere le 
equazioni dal quinto grado in poi. In altre parole, esistono solo formule per le equazioni di primo, secondo, terzo e quarto grado.

Oggigiorno, tuttavia, si preferisce non approfondire le applicazioni di queste formule. Lo studio è generalmente rivolto soltanto alle equazioni di primo e secondo grado e per quelle di grado superiore al secondo si applicano i metodi che vedremo in questo capitolo, oppure si utilizzano metodi di calcolo numerico che forniscono soluzioni per approssimazioni successive.
\end{esempio}
\end{exrig}

\section{Equazioni riconducibili al prodotto di due o più fattori}

In questo capitolo ci proponiamo di determinare l'insieme soluzione di equazioni algebriche di grado superiore al secondo.

Ricordiamo che un'equazione algebrica si presenta nella forma $p(x)=0$ dove $p(x)$ è un polinomio nella variabile $x$, di grado $n$, a coefficienti reali: \[a_nx^n+a_{n-1}x^{n-1}+ \ldots +a_2x^2+a_1x+a_0=0.\]

\begin{exrig}
\begin{esempio}
Determinare le radici reali dell'equazione $4x^3+x^2-4x-1=0$.

Scomponiamo in fattori il polinomio al primo membro mediante raccoglimento parziale: 
\[p(x)=4x^3+x^2-4x-1=4x \left(x^2-1\right)+\left(x^2-1\right)=\left(x^2-1\right) (4x+1).\] 
Per la legge dell'annullamento del prodotto si ottiene 
\[x^2-1=0 \quad\Rightarrow\quad x_1=-1 \;\vee\;x_2=1\text{,}\]
\[4x+1=0 \quad\Rightarrow\quad x=-\frac{1}{4}.\]
L'equazione ha dunque tre soluzioni reali distinte e $\IS=\left\{-1\text{, }1\text{, }-\frac 1 4\right\}$.
\end{esempio}
\begin{esempio}
Determinare le radici reali dell'equazione fratta $\dfrac{2x+3}{2x+1}+\dfrac{x^2}{x+1}=5x+3$.

Riduciamo allo stesso denominatore 
\[\frac{2x^2+5x+3+2x^3+x^2-10x^3-15x^2-5x-6x^2-9x-3}{(2x+1)\cdot (x+1)}=0.\]
Poniamo le condizioni d'esistenza: $x\neq -\frac 1 2\wedge x\neq -1$. Eliminiamo il denominatore e sommiamo i monomi simili; otteniamo un'equazione di terzo grado $8x^3+18x^2+9x=0$, che, scomponendo il polinomio, può essere scritta come $x\cdot \left(8x^2+18x+9\right)=0$. Per la legge di annullamento del prodotto si ha $x=0\;\vee\;x^2+18x+9=0$. Risolvendo anche l'equazione di secondo grado con la formula risolutiva si ottengono le soluzioni $x_1=0\;\vee\; x_2=-\frac 3 4\;\vee\; x_3=-\frac 3 2$.

\osservazione
Si dimostra che un'equazione ammette tante soluzioni, che possono essere reali (distinte o coincidenti) o non reali, quante ne indica il suo grado.

Ricordiamo che uno zero di un polinomio è il valore che assegnato alla variabile rende il polinomio uguale a zero. Quindi trovare gli zeri di un polinomio equivale a trovare le soluzioni dell'equazione che si ottiene ponendo il polinomio uguale a zero, come nell'esempio seguente.
\end{esempio}

\begin{esempio}
Trovare gli zeri del seguente polinomio di terzo grado $p(x)=x^3-7x^2+4x+12$.

Scriviamo l'equazione $x^3-7x^2+4x+12=0$ e cerchiamo di scomporre con il metodo di Ruffini. Sostituendo $x=-1$ si ottiene $(-1)^3-7(-1)^2+4(-1)+12=-1-7-4+12=0$. Possiamo allora dividere il polinomio $p(x)$ per il binomio $x+1$. Applicando la regola di Ruffini si ha:
\begin{center}
% (c) 2013 Claudio Carboncini - claudio.carboncini@gmail.com

\begin{tikzpicture}[x=5mm,y=5mm]
\matrix (a)[matrix of nodes, nodes in empty cells,nodes={ text width=8mm, text depth=1mm, text centered}]{
&$1$&$-7$&$4$&$ 12 $\\
$ -1 $&&$ -1 $&$ 8 $&$-12$\\
&$ 1 $&$ -8 $&$12$&//\\
};  
\begin{scope}[blue]
\draw(a-1-2.north west)--(a-3-1.south east);
\draw(a-2-1.south west)--(a-2-5.south east);
\draw(a-1-4.north east)--(a-3-4.south east);
      \end{scope}
\end{tikzpicture}

\end{center}
Il polinomio si scompone in $(x+1)(x^2-8x+12)$. Per la legge di annullamento del prodotto $x+1=0\;\vee\; x^2-8x+12=0$. L'equazione $x+1=0$ dà come soluzione $x=-1$. L'equazione $x^2-8x+12=0$ si può risolvere con la formula risolutiva ridotta $x_{1\text{,}2}=4\pm \sqrt{16-12}=4\pm 2$. Il polinomio assegnato ha tre zeri distinti $x_1=-1\;\vee\; x_2=2\;\vee\; x_3=6$.
\end{esempio}
\end{exrig}
\ovalbox{\risolvii \ref{ese:5.1}, \ref{ese:5.2}, \ref{ese:5.3}, \ref{ese:5.4}, \ref{ese:5.5}, \ref{ese:5.6}, \ref{ese:5.7}, \ref{ese:5.8}, \ref{ese:5.9}, \ref{ese:5.10}, \ref{ese:5.11}}

\section{Equazioni binomie}

\begin{definizione}
Un'\emph{equazione binomia} è un'equazione del tipo $ax^n+b=0$ con $a\neq 0$ e con $n\in \insN_0$.
\end{definizione}

L'equazione scritta come $ax^n+b=0$ è detta in \emph{forma normale} o \emph{canonica}.

Dobbiamo distinguere i casi:
\begin{itemize*}
\item $n$ è pari e $ a\cdot b< 0 $. I coefficienti $ a $ e $ b $ hanno segno discorde. L'equazione ammette due sole soluzioni reali ed opposte: $x_1=\sqrt[n]{-\frac b a}\;\vee\; x_2=-\sqrt[n]{-\frac b a}$;
\item $n$ è pari e $ a\cdot b> 0 $. I coefficienti $ a $ e $ b $ hanno lo stesso segno. L'equazione non ammette soluzioni reali;
\item $n$ è dispari e $ b\neq 0 $. L'equazione ha un'unica soluzione reale $x_1=\sqrt[n]{-\frac b a}$;
\item $b=0$. L'equazione è $ax^n=0$ e le $n$ soluzioni sono coincidenti nell'unica soluzione $x=0$. In questo caso si dice che l'unica soluzione $x=0$ ha molteplicità $n$.
\end{itemize*}
%\newpage
\begin{exrig}
 \begin{esempio}
Risolvere le seguenti equazioni binomie
\begin{itemize}
\item $3x^4-8=0$.

L'esponente $n$ è pari, i coefficienti sono discordi: l'equazione ammette due soluzioni reali distinte: $x_1=\sqrt[4]{\frac 8 3}\;\vee\; x_2=-\sqrt[4]{\frac 8 3}$.

Osserviamo che l'equazione proposta può essere risolta col metodo della scomposizione in fattori: $3x^4-8=0 \:\Rightarrow\: \left(\sqrt 3x^2+\sqrt 8\right)\cdot \left(\sqrt 3x^2-\sqrt 8\right)=0$ e per la legge di annullamento del prodotto $\left(\sqrt 3x^2+\sqrt 8\right)=0\;\vee\; \left(\sqrt 3x^2-\sqrt 8\right)=0$. La prima equazione non ha soluzioni reali, mentre per la seconda si ha $x^2=\sqrt{\frac 8 3}\:\Rightarrow\: x=\pm \sqrt{\sqrt{\frac 8 3}}\:\Rightarrow\: x=\pm \sqrt[4]{\frac 8 3}$.

\item $-6x^4+9=13$.

Riducendo alla forma normale troviamo $-6x^4-4=0$; moltiplicando ambo i membri per $-1$ si ottiene $6x^4+4=0$ in cui il primo membro è una somma di numeri sempre positivi sempre maggiore di zero, quindi in $\insR$ l'equazione è impossibile e $\IS=\emptyset $.

\item $8x^3+3=4$.

Riduciamo l'equazione alla forma normale $8x^3-1=0$. Essendo di grado dispari, l'unica soluzione è $x=\sqrt[3]{\frac 1 8}=\frac 1 2$.

Allo stesso risultato perveniamo scomponendo in fattori la differenza di due cubi: $8x^3-1=0 \:\Rightarrow\: (2x-1)\cdot \left(4x^2+2x+1\right)=0$. Per la legge di annullamento del prodotto $2x-1=0 \:\Rightarrow\: x=\frac 1 2\ \:\vee\: \ 4x^2+2x+1=0$ ma l'equazione di secondo grado non ha soluzioni reali essendo $\Delta <0$. Pertanto l'unica soluzione è $x=\frac 1 2$, quindi $\IS=\left\{\frac 1 2\right\}$.

\item $ 2x^7+3=2 $.

In forma normale $2x^7+1=0$. Si trova così l'unica soluzione reale $x=\sqrt[7]{-\frac 1 2}=-\sqrt[7]{\frac 1 2}$.

\item $3x\cdot \left(x^4-1\right)=4\cdot (1+x)-(7x+4)$.

Sviluppando i calcoli si ottiene $3x^5=0 \:\Rightarrow\: x^5=0 \:\Rightarrow\: x=0$, cioè una sola soluzione reale con molteplicità $5$.

\item $x^3+3=0$.

L'equazione ha l'unica soluzione reale $x=-\sqrt[3]3$. Spieghiamo il risultato scomponendo la somma di cubi $\left(x\right)^3+\left(\sqrt[3]3\right)^3=0 \:\Rightarrow\: \left(x+\sqrt[3]3\right) \left(x^2-x\sqrt[3]3+\sqrt[3]{3^2}\right)=0$. Per la legge di annullamento del prodotto otteniamo: $\left(x+\sqrt[3]3\right)=0 \:\Rightarrow\: x=-\sqrt[3] 3$ e $x^2-\sqrt[3]3 x+\sqrt[3]{3^2}=0$ che non ha soluzioni reali essendo $\Delta <0$.
\end{itemize}
 \end{esempio}
\end{exrig}

\ovalbox{\risolvii \ref{ese:5.12}, \ref{ese:5.13}, \ref{ese:5.14}, \ref{ese:5.15}, \ref{ese:5.16}, \ref{ese:5.17}, \ref{ese:5.18}, \ref{ese:5.19}, \ref{ese:5.20}}

\section{Equazioni trinomie}

\begin{definizione}
Un'\emph{equazione trinomia} è un'equazione del tipo $ax^{2n}+bx^n+c=0$ dove $n \in \insN_0$~~e~~$a\neq 0$,~~$b\neq 0$.
\end{definizione}

Sono esempi di equazioni trinomie $x^4-5x^2+4=0$,\ \  $x^6-4x^3+3=0$,\ \ $x^{10}-x^5+6=0$.

Per risolvere queste equazioni è opportuno fare un cambio di incognita: ponendo $t=x^n$ l'equazione trinomia diventa di secondo grado: $at^2+bt+c=0$ e da questa, detta \emph{equazione risolvente}, si ricavano i valori di $t$. Successivamente, dalla relazione $t=x^n$, si ricavano i valori di $x$.

\subsection{Equazione biquadratica}

Se $n=2$ l'equazione è detta \emph{biquadratica} e si presenta nella forma~~$ax^4+bx^2+c=0$.

\begin{exrig}
\begin{esempio}
Risolvere le seguenti equazioni biquadratiche.
\begin{itemize}
\item $ x^4-5x^2+4=0 $.

L'equazione è biquadratica; facciamo un cambio di incognita ponendo $x^2=t$; l'equazione diventa $t^2-5t+4=0$ che ha due soluzioni reali distinte $t_1=1\;\vee \;t_2=4$. Per determinare le soluzioni dell'equazione assegnata teniamo conto della sostituzione fatta. Da $t_1=1$ otteniamo $x^2=1$ con soluzioni $x_1=-1\;\vee\; x_2=+1$ e da $t_2=4$ otteniamo $x^2=4$ con soluzioni $x_1=-2\;\vee\; x_2=+2$. Pertanto l'equazione assegnata ha quattro soluzioni reali distinte e $\IS=\{-1\text{, }+1\text{, }-2\text{, }+2\}$.

\item $ 2x^4+3x^2-2=0 $.

L'equazione è biquadratica, ponendo $x^2=t$ diventa $2t^2+3t-2=0$ che ha per soluzioni $t_1=-2\;\vee\; t_2=\frac 1 2$. Ritornando alla sostituzione iniziale, da $t_1=-2$ otteniamo $x^2=-2\:\Rightarrow\: \IS=\emptyset $ e da $t_2=\frac 1 2$ otteniamo $x^2=\frac 1 2$ con soluzioni $x_1=-\sqrt{\frac 1 2}\;\vee\; x_2=+\sqrt{\frac 1 2}$ e razionalizzando $x_1=-\frac{\sqrt 2} 2\;\vee\; x_2=+\frac{\sqrt 2} 2$.

\item $ x^4-\frac{16} 9x^2=0 $.

L'equazione è biquadratica incompleta; si può determinare l'insieme soluzione raccogliendo $x^2$ a fattore comune, ottenendo così $x^2\left(x^2-\frac{16} 9\right)=0$ e per la legge di annullamento del prodotto possiamo concludere $x^2=0\;\vee\; x^2=\frac{16} 9$ da cui $\IS=\left\{0\text{, }-\frac 4 3\text{, }+\frac 4 3\right\}$.
 \end{itemize}

\end{esempio}
\end{exrig}
\conclusione
L'equazione biquadratica ${ax}^4+{bx}^2+c=0$
\begin{itemize*}
\item ha quattro soluzioni reali distinte se il discriminante dell'equazione risolvente è positivo e se risultano positivi anche i rapporti $-\frac b a$ e $\frac c a$ che indicano rispettivamente la somma e il prodotto delle sue soluzioni. Infatti \dotfill;
\item ha due soluzioni reali distinte se il discriminante dell'equazione risolvente è positivo e se risulta negativo il rapporto $\frac c a$ che indica il prodotto delle sue soluzioni. Infatti \dotfill;
\item non ha soluzioni reali se il discriminante dell'equazione risolvente è positivo e se risulta positivo il rapporto $\frac c a$ e negativo il rapporto $-\frac b a$. Infatti \dotfill;
\item non ha soluzioni reali se il discriminante dell'equazione risolvente è negativo.
\end{itemize*}
Per stabilire il numero di soluzioni di un'equazione biquadratica si può anche utilizzare la regola dei segni di Cartesio:
\begin{itemize*}
\item $\Delta >0$ e due variazioni si hanno 4 soluzioni reali;
\item $\Delta >0$ una permanenza e una variazione si hanno 2 soluzioni reali;
\item $\Delta =0$ e~~~$-\frac b{2a}>0$ si hanno due soluzioni reali; $-\frac b{2a}<0$ nessuna soluzione reale;
\item $\Delta <0$ nessuna soluzione reale.
\end{itemize*}

\ovalbox{\risolvii \ref{ese:5.21}, \ref{ese:5.22}, \ref{ese:5.23}, \ref{ese:5.24}, \ref{ese:5.25}, \ref{ese:5.26}, \ref{ese:5.27}, \ref{ese:5.28}, \ref{ese:5.29}, \ref{ese:5.30}, \ref{ese:5.31}}

\subsection{Equazioni trinomie con n maggiore di 2}

\begin{exrig}
\begin{esempio}
Risolvere le seguenti equazioni trinomie.
\begin{itemize}
\item $ x^6-4x^3+3=0 $.

Ponendo $t=x^3$ abbiamo l'equazione risolvente $t^2-4t+3=0$, le cui soluzioni reali sono $t_1=1$, $t_2=3$; per ricavare i valori di $x$ è sufficiente risolvere le due equazioni binomie $x^3=1$ e $x^3=3$, trovando così le soluzioni reali per l'equazione assegnata $x_1=1\;\vee\; x_2=\sqrt[3]3$;

\item $ x^8-x^4-2=0 $.

Ponendo $t=x^4$ arriviamo all'equazione $t^2-t-2=0$ da cui $t_1=2$ e $t_2=-1$; pertanto le due equazioni binomie da risolvere sono: $x^4=2$ e $x^4=-1$. Quindi $x^4=2\:\Rightarrow\: x^2=-\sqrt 2\;\vee\; x^2=\sqrt 2$ e di queste due, solo la seconda ha soluzioni reali e precisamente $x_1=\sqrt[4]2 \;\vee\; x_2=-\sqrt[4]2$; la prima equazione binomia $x^4=-1$ non ha invece soluzioni reali. Concludendo: $\IS=\left\{-\sqrt[4]2\text{, }\sqrt[4]2\right\}$.
\end{itemize}
\end{esempio}
\end{exrig}
\ovalbox{\risolvii \ref{ese:5.32}, \ref{ese:5.33}}

\section{Equazioni che si risolvono con sostituzioni}

Molte altre equazioni si possono risolvere con opportune sostituzioni.
\begin{exrig}
\begin{esempio}
Risolvere la seguente equazione $ \left(x^2-4\right)^4-1=0 $.

Sostituendo $t=x^2-4$ l'equazione diventa $t^4-1=0$. È un'equazione binomia che ha per soluzioni $t_1=-1$, $t_2=+1$. Sostituendo questi valori nella relazione $t=x^2-4$ si ha: 
\[x^2-4=-1\:\Rightarrow\: x^2=3\:\Rightarrow\: x_{1\text{,}2}=\pm \sqrt 3\quad\text{ e }\quad x^2-4=+1\:\Rightarrow\: x^2=5\:\Rightarrow\: x_{3\text{,}4}=\pm \sqrt 5.\]
\end{esempio}
\end{exrig}
\ovalbox{\risolvii \ref{ese:5.34}, \ref{ese:5.35}, \ref{ese:5.36}, \ref{ese:5.37}}

\section{Equazioni reciproche}

\begin{definizione}
Un'equazione è detta \emph{reciproca di prima specie} se, posta nella forma canonica $p(x)=0$, il polinomio $p(x)$ ha i coefficienti dei termini estremi e quelli dei termini equidistanti dagli estremi \emph{uguali}.
\end{definizione}

\begin{definizione}
Un'equazione è detta \emph{reciproca di seconda specie} se, posta nella forma canonica $p(x)=0$, il polinomio $p(x)$ ha i coefficienti dei termini estremi e quelli dei termini equidistanti dagli estremi \emph{opposti}. In particolare, se $p(x)$ ha grado $2k$ (pari), il coefficiente di $x^k$ è nullo.
\end{definizione}

Dalle definizioni si ha che:

\begin{itemize*}
\item $x^3-2x^2-2x+1=0$ è un'equazione di terzo grado reciproca di prima specie;
\item $3x^4+5x^3-4x^2+5x+3=0$ è un'equazione di quarto grado reciproca di prima specie;
\item $-7x^4+5x^3-5x+7=0$ è un'equazione di quarto grado reciproca di seconda specie;
\item $3x^5+2x^4+6x^3-6x^2-2x-3=0$ è un'equazione di quinto grado reciproca di seconda specie;
\item $-2x^4+8x^3+3x^2-8x+2=0$ è un'equazione di quarto grado, ma non è reciproca di seconda specie, in quanto il coefficiente di secondo grado dovrebbe essere nullo.
\end{itemize*}

Il seguente teorema mette in luce una importante proprietà di cui godono queste equazioni.
\begin{teorema}[delle radici reciproche]
Se $\lambda$ è una radice non nulla di un'equazione reciproca di qualunque grado, allora anche $\dfrac 1{\lambda}$ è radice dell'equazione.
\end{teorema}

Consideriamo il polinomio dell'equazione reciproca di prima specie
\[p(x)=a_0x^n+a_1x^{n-1}+a_2x^{n-2}+ \ldots +a_2x^2+a_1x+a_0.\]

\emph{Ipotesi}. $x=\lambda$ è una radice dell'equazione $p(x)=0$;

\emph{Tesi}. $x=\frac 1{\lambda}$ è una radice dell'equazione $p(x)=0$.
\begin{proof}
Sappiamo che se $x=\lambda$ è una radice di $p(x)=0$ allora $p(\lambda)=0$, cioè
\[p(\lambda)=a_0\lambda ^n+a_1\lambda ^{n-1}+\ldots +a_1\lambda +a_0=0.\]

Scriviamo il polinomio con $x=\frac 1{\lambda }$
\begin{align*}
p\left(\frac 1{\lambda }\right)&=a_0\left(\frac 1{\lambda }\right)^n+a_1\left(\frac 1{\lambda }\right)^{n-1}+\ldots +a_1\left(\frac 1{\lambda }\right)+a_0\\
&=\frac 1{\lambda ^n} \left(a_0+a_1\lambda +\ldots +a_1\lambda ^{n-1}+a_0\lambda ^n \right)\\
\end{align*}
dove nell'ultimo passaggio abbiamo messo in evidenza il termine $ \left(\frac 1{\lambda }\right)^n $ (è consentito perché per ipotesi $\lambda$ non è nullo). Confrontando le scritture di $p(\lambda)$ e $p\left(\frac 1{\lambda }\right)$ risulta
\[
p\left(\frac 1{\lambda }\right)=\frac 1{\lambda ^n} p(\lambda)=0
\]
Quindi anche $\dfrac 1{\lambda }$ è una radice di $p(x)=0$.
\end{proof}
Dimostra tu il teorema per le equazioni di seconda specie.

\subsection{Equazioni di terzo grado reciproche di prima specie}

Queste equazioni hanno la seguente struttura: \[a_0x^3+a_1x^2+a_1x+a_0=0.\]
Una radice dell'equazione è $x=-1$, infatti sostituendo tale valore al posto della $x$ nel polinomio al primo membro si ottiene: \[p(-1)=a_0(-1)^3+a_1(-1)^2+a_1(-1)+a_0=-a_0+a_1-a_1+a_0=0.\]

Ricordiamo che secondo la regola del resto, il valore trovato (zero) ci assicura che il polinomio al primo membro è divisibile per $x+1$; con la divisione polinomiale o con la regola di Ruffini possiamo scrivere $a_0x^3+a_1x^2+a_1x+a_0=(x+1)\cdot \left(a_0x^2+(a_1-a_0)x+a_0\right)=0$ da cui, per la legge di annullamento del prodotto, possiamo determinare le soluzioni dell'equazione assegnata.

Un modo alternativo per determinare l'insieme soluzione dell'equazione reciproca di prima specie consiste nel raccogliere parzialmente i due coefficienti $a_0$ e $a_1$ in modo da ottenere $a_0\left(x^3+1\right)+a_1\left(x^2+x\right)=0$ da cui $a_0(x+1)\left(x^2-x+1\right)+a_1x(x+1)=0$ e raccogliendo il binomio $(x+1)$ ritroviamo la fattorizzazione precedente: $(x+1) \left(a_0x^2+(a_1-a_0)x+a_0\right)=0$.
\begin{exrig}
 \begin{esempio}
 Risolvere le seguenti equazioni di terzo grado reciproche di prima specie.

 \begin{itemize}
 \item $x^3-5x^2-5x+1=0$.

 Si tratta di un'equazione di terzo grado reciproca di prima specie. Una radice è $x=-1$, per cui possiamo fattorizzare il polinomio al primo membro eseguendo la divisione polinomiale e ottenere $(x+1)\left(x^2-6x+1\right)=0$. Per la legge di annullamento del prodotto otteniamo la radice $x=-1$ già nota e, risolvendo l'equazione $x^2-6x+1=0$ troviamo le altre radici $x_2=3+2\sqrt 2$~~e~~$x_3=3-2\sqrt 2$. Quindi $\IS=\left\{-1\text{, }3+2\sqrt 2\text{, }3-2\sqrt 2\right\}$.

 Notiamo che $x_2$ e $x_3$ sono tra loro reciproche: $x_2 \cdot x_3=1$ cioè $(3+2\sqrt 2)\cdot (3-2\sqrt 2)=1$.

 \item $3x^3-5x^2-5x+3=0$

 L'equazione assegnata è reciproca di terzo grado e di prima specie; ammette dunque $x=-1$ come radice. Infatti $p(-1)=3(-1)^3-5(-1)^2-5(-1)+3=\ldots$ Il polinomio al primo membro si può scomporre con la regola di Ruffini cioè $(x+1)\left(3x^2-8x+3\right)=0$; per la legge di annullamento del prodotto avremo $x+1=0\:\Rightarrow\: x=-1$, come già noto, e $3x^2-8x+3=0$ con soluzioni $x_2=\ldots $ e $x_3=\ldots$ Quindi $\IS=\{\dotfill \}$.
 \end{itemize}
 \end{esempio}
\end{exrig}
\ovalbox{\risolvii \ref{ese:5.38}, \ref{ese:5.39}, \ref{ese:5.40}, \ref{ese:5.41}}

\subsection{Equazioni di terzo grado reciproche di seconda specie}

Queste equazioni hanno la seguente struttura:
\[a_0x^3+a_1x^2-a_1x-a_0=0.\]
Una radice dell'equazione è $x=1$, basta verificare sostituendo tale valore al posto della $x$ nel polinomio al primo membro. Si ottiene: \[p(1)=a_0(1)^3+a_1(1)^2-a_1(1)-a_0=a_0+a_1-a_1-a_0=0.\]

Procedendo come nel caso precedente si può ottenere la scomposizione in fattori del polinomio al primo membro: $(x-1)\cdot \left(a_0x^2+(a_0+a_1)x+a_0\right)=0$ e quindi determinare l'$\IS$ dell'equazione assegnata applicando la legge di annullamento del prodotto.
\begin{exrig}
 \begin{esempio}
 Risolvere le seguenti equazioni di terzo grado reciproche di seconda specie

 \begin{itemize}
 \item $ 2x^3-7x^2+7x-2=0 $.

 \`E un'equazione di terzo grado reciproca di seconda specie, i coefficienti infatti sono opposti a due a due. Una radice è $x_1=1$ infatti $p(1)=2-7+7-2=0$. Applicando la regola di Ruffini per scomporre il polinomio di terzo grado si ottiene $(x-1)\left(2x^2-5x+2\right)=0$. Per la legge di annullamento del prodotto abbiamo la radice $x=1$, già nota, e risolvendo $\left(2x^2-5x+2\right)=0$ si ricavano le altre due radici: $x_2=2$~~e~~$x_3=\frac 1 2$. Dunque $\IS=\left\{1\text{, }2\text{, }\frac 1 2\right\}$.

 \item $ 2x^3-9x^2+9x-2=0 $.

 L'equazione assegnata è reciproca di terzo grado e di seconda specie perché ha i coefficienti opposti a due a due, quindi ammette $x=+1$ come radice. Infatti $p(1)=\ldots$
 Applichiamo la regola di Ruffini per scomporre in fattori il polinomio di terzo grado. Il polinomio si scompone in $(x-1)\left(2x^2- \ldots \ldots\right)=0$. Per la legge di annullamento del prodotto avremo $x-1=0\:\Rightarrow\: x=1$, come già noto, e $2x^2-7x+2=0$ con soluzioni $x_2=\ldots $ e $x_3=\ldots $. Quindi $\IS=\{\ldots \ldots \ldots \}$. L'equazione assegnata ha tre soluzioni reali di cui le due irrazionali sono l'una il reciproco dell'altra.
 \end{itemize}
 \end{esempio}
\end{exrig}
\ovalbox{\risolvii \ref{ese:5.42}, \ref{ese:5.43}, \ref{ese:5.44}}

\subsection{Equazioni di quarto grado reciproche di prima specie}

Rientrano in questa classificazione le equazioni del tipo: \[a_0x^4+a_1x^3+a_2x^2+a_1x+a_0=0.\]

Prima di tutto osserviamo che $x=0$ non può essere una radice in quanto, se lo fosse, sarebbe nullo il termine noto, cioè $a_0=0$ e di conseguenza sarebbe nullo anche il coefficiente di $x^{4}$ quindi il grado dell'equazione diventerebbe 3 o inferiore. Questa premessa ci consente di dividere per $x^2$ ottenendo l'equazione equivalente $a_0 x^2+a_1x+a_2+\dfrac{a_1} x+\dfrac{a_0}{x^2}=0$ e raccogliendo a fattori parziali si ha $a_0\left(x^2+\dfrac 1{x^2}\right)+a_1\left(x+\dfrac 1 x\right)+a_2=0$.

Ponendo~~$t=x+\dfrac 1 x$~~si ha~~$t^2=\left(x+\dfrac 1 x\right)^2\:\Rightarrow\: t^2=x^2+\dfrac 1{x^2}+2 \:\Rightarrow\: x^2+\dfrac 1{x^2}=t^2-2$.

Sostituendo nell'equazione otteniamo la seguente equazione di secondo grado equivalente a quella data: $a_0\left(t^2-2\right)+a_1t+a_2=0$, ovvero
\[a_0t^2+a_1t+a_2-2a_0=0.\]

Una volta trovate, se esistono (reali), le radici $t_1$ e $t_2$ di questa equazione, possiamo determinare le corrispondenti radici dell'equazione iniziale risolvendo le due equazioni fratte $x+\dfrac 1 x=t_1$~~e~~$x+\dfrac 1 x=t_2$~~nell'incognita $x$, rispettivamente equivalenti a
\[x^2-t_1x+1=0\qquad\text{e}\qquad {x}^2-t_2x+1=0.\]

Queste ultime equazioni hanno soluzioni reali se e solo se $|t|\ge 2$. Infatti, risolvendo rispetto a $x$ l'equazione $x+\dfrac 1 x=t$, troviamo: $x^2-tx+1=0$ e calcolando il discriminante $\Delta =t^2-4$ vediamo che ci sono soluzioni reali se e solo se $t^2-4\ge 0$ ovvero se e solo se $t\le -2 \;\vee\; t\ge 2$ cioè $|t|\ge 2$.
%\newpage
\begin{exrig}
 \begin{esempio}
 Risolvere le seguenti equazioni di quarto grado reciproche di prima specie.
 \begin{itemize}
 \item $x^4-4x^3+5x^2-4x+1=0$.

 Si tratta di un'equazione di quarto grado reciproca di prima specie. Dividiamo per $x^2$, otteniamo $x^2-4x+5-4\frac 1 x+\frac 1{x^2}=0$. Raccogliendo in fattori comuni come nella regola abbiamo $\left(x^2+\frac 1{x^2}\right)-4\left(x+\frac 1 x\right)+5=0$. Ponendo $t=x+\frac 1 x$ otteniamo l'equazione $ \left(t^2-2\right)-4t+5=0$ ovvero $t^2-4t+3=0$ da cui $t_1=1$ e $t_2=3$. Il primo valore $t_1$ non dà soluzioni reali poiché l'equazione $x+\frac 1 x=1$ ha il discriminante negativo mentre l'equazione $x+\frac 1 x=3$ ha due soluzioni reali distinte $x_1=\frac{3+\sqrt 5} 2$ e $x_2=\frac{3-\sqrt 5} 2$.

 \item $ 2x^4+3x^3-16x^2+3x+2=0 $.

 Dividiamo ambo i membri dell'equazione per $x^2$, certamente diverso da zero e otteniamo: $2x^2+3x-16+3\cdot \frac 1 x+2\cdot \frac 1{x^2}=0$. Mettiamo in evidenza $ 2 $ nel primo e quinto addendo e $ 3 $ tra il secondo e quarto addendo: $2\cdot \left(x^2+\frac 1{x^2}\right)+3\cdot \left(x+\frac 1 x\right)-16=0$. Ponendo $x+\frac 1 x=t$ otteniamo l'equazione $2(t^2-2)+3t-16=0\Rightarrow 2t^2+3t-20=0$ che ha come soluzioni $t_1=-4\vee t_2=\frac 5 2$. Poiché $|t|\ge 2$ le equazioni $x+\frac 1 x=t_1$ e $x+\frac 1 x=t_2$ hanno entrambe soluzioni reali distinte, pertanto $\IS=\left\{-2\sqrt{3}\text{, }2\sqrt{3}\text{, }\frac 1 2\text{, }2\right\}$.
 \end{itemize}
 \end{esempio}
\end{exrig}
\ovalbox{\risolvii \ref{ese:5.45}, \ref{ese:5.46}}

\subsection{Equazioni di quarto grado reciproche di seconda specie}

Fanno parte di questa classe le equazioni del tipo: \[a_0x^4+a_1x^3-a_1x-a_0=0\] in cui il coefficiente di $x^2$ è nullo. Per risolvere questa equazione, raccogliamo a fattore parziale $a_0$ e $a_1$ ottenendo: $a_0\left(x^4-1\right)+a_1\left(x^3-x\right)=0 \:\Rightarrow\: a_0\left(x^2-1\right)\left(x^2+1\right)+a_1x\left(x^2-1\right)=0$.

Raccogliendo a fattore comune totale si ha:
\[\left(x^2-1\right)\left[a_0\left(x^2+1\right)+a_1x\right]=0\quad\Rightarrow \quad\left(x-1\right)\left(x+1\right)\left(a_0x^2+a_1x+a_0\right)=0.\]
Per la legge di annullamento del prodotto si hanno quindi le due radici $x_1=1$, $x_2=-1$ e le eventuali radici reali dell'equazione di secondo grado $a_0x^2+a_1x+a_0=0$.
\begin{exrig}
 \begin{esempio}
 Risolvere l'equazione $x^4-8x^3+8x-1=0$.

 Si tratta di un'equazione di quarto grado reciproca di seconda specie, si osservi che il coefficiente di secondo grado è nullo e che gli altri coefficienti sono a due a due opposti.
 Raccogliendo a fattore comune parziale abbiamo 
\[(x^4-1)-8x(x^2-1)=0\quad\Rightarrow\quad (x^2-1)(x^2+1)-8x(x^2-1)=0.\]
 Mettendo poi in evidenza il binomio $\left(x^2-1\right)$ abbiamo $\left(x^2-1\right)\left(x^2-8x+1\right)$. Risolvendo le equazioni $x^2-1=0$ e $x^2-8x+1=0$, otteniamo tutte le radici: 
\[x_1=1\;\vee\; x_2=-1\;\vee\; x_3=4+\sqrt{15}\;\vee\; x_4=4-\sqrt{15}\] 
e quindi 
\[\IS=\left\{-1\text{, }1\text{, }4+\sqrt{15}\text{, }4-\sqrt{15}\right\}.\]
 \end{esempio}
\end{exrig}
\ovalbox{\risolvii \ref{ese:5.47}, \ref{ese:5.48}, \ref{ese:5.49}, \ref{ese:5.50}, \ref{ese:5.51}}

\subsection{Equazioni di quinto grado reciproche di prima specie}

Fanno parte di questa classe le equazioni del tipo: \[a_0x^5+a_1x^4+a_2x^3+a_2x^2+a_1x+a_0=0.\]

Con il raccoglimento parziale otteniamo: 
\[a_0\left(x^5+1\right)+a_1\left(x^4+x\right)+a_2\left(x^3+x^2\right)=0.\]

Applichiamo ora la formula per la scomposizione della somma di potenze ottenendo
\[a_0(x+1)\left(x^4-x^3+x^2-x+1\right)+a_1x(x+1)\left(x^2-x+1\right)+a_2x^2(x+1)=0.\]
Raccogliendo $(x+1)$ ricaviamo: 
\[(x+1)\left[a_0\left(x^4-x^3+x^2-x+1\right)+a_1x\left(x^2-x+1\right)+a_2x^2\right]=0\]
e quindi l'equazione diventa:
\[(x+1)\left[a_0x^4+(a_1-a_0)x^3+(a_2-a_1+a_0)x^2+(a_1-a_0)x+a_0\right]=0.\]

Per la legge di annullamento del prodotto si determina la soluzione reale $x=-1$ e con i metodi analizzati in precedenza si risolve l'equazione di quarto grado reciproca di prima specie: 
\[\left(a_0x^4+(a_1-a_0)x^3+(a_2-a_1+a_0)x^2+(a_1-a_0)x+a_0\right)=0.\]
\pagebreak
\begin{exrig}
 \begin{esempio}
 Risolvere l'equazione $6x^5+x^4-43x^3-43x^2+x+6=0$.

 L'equazione è di quinto grado reciproca di prima specie. Una radice è $x_1=-1$ e l'equazione può essere scritta come: 
\[(x+1)\left(6x^4-5x^3-38x^2-5x+6\right)=0.\]
 Risolvendo l'equazione di quarto grado reciproca di prima specie 
\[6x^4-5x^3-38x^2-5x+6=0\text{,}\] 
si trovano le altre quattro radici: 
\[x_2=-2\text{,}\quad x_3=-\frac 1 2\text{,}\quad x_4=3\text{,}\quad x_5=\frac{1}{3}\] quindi 
\[\IS=\left\{-1\text{, }-2\text{, }-\frac 1 2\text{, }3\text{, }\frac 1 3\right\}.\]
 \end{esempio}
\end{exrig}

\subsection{Equazioni di quinto grado reciproche di seconda specie}
Fanno parte di questa classe le equazioni del tipo:
\[a_0x^5+a_1x^4+a_2x^3-a_2x^2-a_1x-a_0=0.\]

Con il raccoglimento parziale si ottiene 
\[a_0\left(x^5-1\right)+a_1\left(x^4-x\right)+a_2\left(x^3-x^2\right)=0.\]

 Applichiamo ora la formula per la scomposizione della differenza di potenze ottenendo: 
\[a_0(x-1)\left(x^4+x^3+x^2+x+1\right)+a_1x(x-1)\left(x^2+x+1\right)+a_2x^2(x-1)=0.\] 
Raccogliendo il binomio $(x-1)$ si ottiene 
\[(x-1)\left[a_0\left(x^4+x^3+x^2+x+1\right)+a_1x\left(x^2+x+1\right)+a_2x^2\right]=0\] e quindi 
\[(x+1)\left[a_0x^4+(a_1-a_0)x^3+(a_2-a_1+a_0)x^2+(a_1-a_0)x+a_0\right]=0.\]

Una radice è $x=1$ e le altre provengono dall'equazione di quarto grado reciproca di prima specie: \[a_0x^4+(a_1+a_0)x^3+(a_2+a_1+a_0)x^2+(a_1+a_0)x+a_0=0.\]
\pagebreak
\begin{exrig}
 \begin{esempio}
 Risolvere l'equazione $x^5+2x^4-5x^3+5x^2-2x-1=0$.

 È un'equazione di quinto grado reciproca di seconda specie. Una radice è $x_1=1$ e l'equazione può essere scritta come: 
\[(x-1)\left(x^4+3x^3-2x^2+3x+1\right)=0.\] 
Risolvendo l'equazione di quarto grado reciproca di prima specie 
\[x^4+3x^3-2x^2+3x+1=0\text{,}\] si trovano altre due radici reali: 
\[x_2=-2+\sqrt 3\quad\text{e}\quad x_3=-2-\sqrt 3.\] 
Pertanto 
\[\IS=\left\{+1\text{, }-2+\sqrt 3\text{, }-2-\sqrt 3\right\}.\]
 \end{esempio}
\end{exrig}

\subsection{Equazioni reciproche di sesto grado}
\begin{exrig}
 \begin{esempio}
 Risolvere l'equazione $-x^6+6x^5+6x^4-6x^2-6x+1=0$.

 Si tratta di un'equazione di sesto grado reciproca di seconda specie (si osservi che il termine di terzo grado è nullo); l'equazione ammette per radici $x_1=1$ e $x_2=-1$.
Possiamo quindi dividere il polinomio per il binomio $\left(x^2-1\right)$, ottenendo come quoziente $-x^4+6x^3+5x^2+6x-1$. Si tratta allora di risolvere un'equazione di quarto grado reciproca di prima specie. Si trovano in questo modo altre due radici reali: $x_3=\frac{7+\sqrt 5}{2}$~~e~~$x_4=\frac{7-\sqrt 5} 2$.
 \end{esempio}
\end{exrig}
\newpage
% (c)~2014 Claudio Carboncini - claudio.carboncini@gmail.com
% (c)~2014 Dimitrios Vrettos - d.vrettos@gmail.com
\section{Esercizi}
\subsection{Esercizi dei singoli paragrafi}
\subsection*{5.2 - Equazioni riconducibili al prodotto di due o più fattori}

\begin{esercizio}[\Ast]
 \label{ese:5.1}
Trovare gli zeri dei seguenti polinomi.
\begin{multicols}{2}
 \begin{enumeratea}
 \item~$x^3+5x^2-2x-24$;
 \item~$6x^3+23x^2+11x-12$;
 \item~$8x^3-40x^2+62x-30$;
 \item~$x^3+10x^2-7x-196$;
 \item~$x^3+\frac 4 3x^2-\frac{17} 3x-2$;
 \item~$x^3-\frac 1 3x^2-\frac{38} 3x+\frac{56} 3$.
 \end{enumeratea}
 \end{multicols}
\end{esercizio}

\begin{esercizio}[\Ast]
\label{ese:5.2}
Trovare gli zeri dei seguenti polinomi.
\begin{multicols}{2}
 \begin{enumeratea}
 \item~$3x^3-\frac 9 2x^2+\frac 3 2x$;
 \item~$3x^3-9x^2-9x-12$;
 \item~$\frac 6 5x^3+\frac{42} 5x^2+\frac{72} 5x+12$;
 \item~$4x^3-8x^2-11x-3$;
 \item~$\frac 3 2x^3-4x^2-10x+8$;
 \item~$\frac 3 2x^3-4x^2-10x+8$.
 \end{enumeratea}
 \end{multicols}
\end{esercizio}

\begin{esercizio}[\Ast]
 \label{ese:5.3}
Trovare gli zeri dei seguenti polinomi.
\begin{multicols}{2}
 \begin{enumeratea}
 \item~$-3x^3+9x-6$;
 \item~$\frac 1 2x^3-3x^2+6x-4$;
 \item~$4x^3+4x^2-4x-4$;
 \item~$\frac 2 5x^3+\frac 8 5x^2+\frac{14} 5x-4$;
 \item~$-6x^3-30x^2+192x-216$;
 \item~$x^3-2x^2-x+2$.
 \end{enumeratea}
 \end{multicols}
\end{esercizio}

\begin{esercizio}[\Ast]
 \label{ese:5.4}
Trovare gli zeri dei seguenti polinomi.
\begin{multicols}{2}
 \begin{enumeratea}
 \item~$9x^3-7x+2$;
 \item~$x^3-7x^2+4x+12$;
 \item~$ x^3+10x^2-7x-196 $;
 \item~$ 400x^3-1600x^2$;
 \item~$x^6-5x^5+6x^4+4x^3-24x^2+16x+32 $;
 \item~$ 8x^3-14{ax}^2-5a^2x+2a^3 $.
 \end{enumeratea}
 \end{multicols}
\end{esercizio}

\begin{esercizio}
\label{ese:5.5}
Trovare gli zeri dei seguenti polinomi.
\begin{multicols}{2}
 \begin{enumeratea}
 \item~$ x^4-x^3-x^2-x-2 $;
 \item~$ 3x^5-19x^4+42x^3-42x^2+19x-3 $;
 \item~$ {ax}^3-(a^2+1-a)x^2-(a^2+1-a)x+a $.
 \end{enumeratea}
 \end{multicols}
\end{esercizio}

\begin{esercizio}[\Ast]
\label{ese:5.6}
Determinare l'insieme soluzione delle seguenti equazioni.
\begin{multicols}{2}
 \begin{enumeratea}
 \item~$x^3-3x+2=0$;
 \item~$x^3+2x^2+2x+1=0$;
 \item~$x^3-6x+9=0$;
 \item~$x^4-2x^2+1=0$;
 \item~$x^3+3x^2-x-3=0$;
 \item~$6x^3-7x^2-x+2=0$.
 \end{enumeratea}
 \end{multicols}
\end{esercizio}

\begin{esercizio}[\Ast]
 \label{ese:5.7}
Determinare l'insieme soluzione delle seguenti equazioni.
\begin{multicols}{2}
 \begin{enumeratea}
 \item~$x^3-6x^2+11x-6=0$;
 \item~$x^3-2x^4=0$;
 \item~$x^4-5x^3+2x^2+20x-24=0$;
 \item~$x^5+1=x\cdot \left(x^3+1\right)$;
 \item~$\frac{x^3+2-x\cdot (2x+1)}{2x-1}=0$;
 \item~$2x^2-2x+3(x-1)=2x\left(2x^2-1\right)$.
 \end{enumeratea}
 \end{multicols}
\end{esercizio}
\newpage
\begin{esercizio}[\Ast]
 \label{ese:5.8}
Determinare l'insieme soluzione delle seguenti equazioni.
 \begin{enumeratea}
 \item~$(3x+1)^2=x\left(9x^2+6x+1\right)$;
 \item~$(x+1)\left(x^2-1\right)=\left(x^2+x\right)\left(x^2-2x+1\right)$;
 \item~$(x-1)(x^2+x+1)=x(2-3x)+5$;
 \item~$x^3+4x^2+4x=x^2-4$;
 \item~$\sqrt 3 x^4-\sqrt{27}\;x^2=0$;
 \item~$ (x+1)^3-(x-1)^3=8 $.
 \end{enumeratea}
\end{esercizio}

\begin{esercizio}[\Ast]
 \label{ese:5.9}
Determinare l'insieme soluzione delle seguenti equazioni.
\begin{multicols}{2}
 \begin{enumeratea}
 \item~$\sqrt 2x^3-(1-2\sqrt 2)x^2-x=0$;
 \item~$64x^7=27x^4$;
 \item~$(x^2-4x)^{2011}=-(4x-x^2)^{2011}$;
 \item~$(x^2-4x)^{2012}=-(4x-x^2)^{2011}$;
 \item~$x^7-x^6+\sqrt{27}x^5=0$;
 \item~$3x^4-14x^3+20x^2-8x=0$.
 \end{enumeratea}
 \end{multicols}
\end{esercizio}

\begin{esercizio}[\Ast]
 \label{ese:5.10}
Determinare l'insieme soluzione delle seguenti equazioni.
\begin{multicols}{2}
 \begin{enumeratea}
 \item~$\frac{3x-1}{x^2}=1-2x+\frac 1 x$;
 \item~$\frac{x-1}{x^2+5x+4}-\frac{2x+1}{x-1}-\frac 3{2\left(x^2-1\right)}=0$;
 \item~$\frac{x^2-3x}{2x}-\frac{x-2}{x-1}=0$;
 \item~$\frac{x(x-1)}{x+1}=\frac{x-1}{x^2+2x+1}$;
 \item~$\frac 1{x^4-4}=\frac 3{x^4-16}$.
 \item~$\frac{x^2}{x^2+1}-\frac 1{4-x^2}+\frac 1{x^4-3x^2-4}=0$;
 \end{enumeratea}
 \end{multicols}
\end{esercizio}


\begin{esercizio}[\Ast]
 \label{ese:5.11}
Determinare l'insieme soluzione delle seguenti equazioni.
 \begin{enumeratea}
 \item~$\frac{x^4-4x^2+9}{x^4-3x^2+2}-\frac{x^2-1}{x^2-2}=\frac{x^2-2}{x^2-1}$;
 \item~$(x^2-1)^3+7x^3=3x(4-x-x^3)-(x-2)^3$;
 \item~$\frac{x^2-1}{x^2-3}-\frac{x^2-3}{1-x^2}=\frac{10} 3$.
 \end{enumeratea}
\end{esercizio}

\subsection*{5.3 - Equazioni binomie}

\begin{esercizio}
 \label{ese:5.12}
Determinare l'insieme soluzione delle seguenti equazioni binomie.
\begin{multicols}{3}
 \begin{enumeratea}
 \item~$-2x^3+16=0$;
 \item~$x^5+15=0$;
 \item~$x^4+16=0$;
 \item~$-2x^4+162=0$;
 \item~$-3x^6+125=0$;
 \item~$81x^4-1=0$.
 \end{enumeratea}
 \end{multicols}
\end{esercizio}

\begin{esercizio}[\Ast]
 \label{ese:5.13}
Determinare l'insieme soluzione delle seguenti equazioni binomie.
\begin{multicols}{3}
 \begin{enumeratea}
 \item~$27x^3+1=0$;
 \item~$81x^4+1=0$;
 \item~$81x^8-1=0$;
 \item~$\frac{16}{x^4}-1=0$;
 \item~$x^6-1=0$;
 \item~$8x^3-27=0$.
 \end{enumeratea}
 \end{multicols}
\end{esercizio}

\begin{esercizio}[\Ast]
 \label{ese:5.14}
Determinare l'insieme soluzione delle seguenti equazioni binomie.
\begin{multicols}{3}
 \begin{enumeratea}
 \item~$x^5-1=0$;
 \item~$x^4+81=0$;
 \item~$x^4-4=0$;
 \item~$3x^5+96=0$;
 \item~$49x^6-25=0$;
 \item~$\frac 1{x^3}=27$.
 \end{enumeratea}
 \end{multicols}
\end{esercizio}
\newpage
\begin{esercizio}[\Ast]
 \label{ese:5.15}
Determinare l'insieme soluzione delle seguenti equazioni binomie.
\begin{multicols}{3}
 \begin{enumeratea}
 \item~$x^4-10000=0$;
 \item~$100000x^5+1=0$;
 \item~$x^6-64000000=0$;
 \item~$x^4+625=0$;
 \item~$8x^3-27=0$;
 \item~$8x^3+9=0$.
 \end{enumeratea}
 \end{multicols}
\end{esercizio}

\begin{esercizio}[\Ast]
 \label{ese:5.16}
Determinare l'insieme soluzione delle seguenti equazioni binomie.
\begin{multicols}{3}
 \begin{enumeratea}
 \item~$81x^4-16=0$;
 \item~$16x^4-9=0$;
 \item~$\frac 8{x^3}-125=0$;
 \item~$\frac{81}{x^3}=27$;
 \item~$81x^4=1$;
 \item~$x^3-\frac 1{27}=0$.
 \end{enumeratea}
 \end{multicols}
\end{esercizio}

\begin{esercizio}
 \label{ese:5.17}
Determinare l'insieme soluzione delle seguenti equazioni binomie.
\begin{multicols}{3}
 \begin{enumeratea}
 \item~$\frac{x^6}{64}-1=0$;
 \item~$\frac{64}{x^6}=1$;
 \item~$x^6=6$;
 \item~$x^{10}+10=0$;
 \item~$x^{100}=0$;
 \item~$10x^5-10=0$.
 \end{enumeratea}
 \end{multicols}
\end{esercizio}

\begin{esercizio}[\Ast]
 \label{ese:5.18}
Determinare l'insieme soluzione delle seguenti equazioni binomie.
\begin{multicols}{3}
 \begin{enumeratea}
 \item~$\frac 1{81}x^4-1=0$;
 \item~$\frac 1{x^4}-81=0$;
 \item~$\sqrt[3]2\;x^6=\sqrt[3]{24}$;
 \item~$\frac 3 5x^3=\frac{25} 9$;
 \item~$x^8-256=0$;
 \item~$x^{21}+1=0$.
 \end{enumeratea}
 \end{multicols}
\end{esercizio}

\begin{esercizio}[\Ast]
 \label{ese:5.19}
Determinare l'insieme soluzione delle seguenti equazioni binomie.
\begin{multicols}{3}
 \begin{enumeratea}
 \item~$\frac 1{243}x^5+1=0$;
 \item~$x^3+3\sqrt 3=0$;
 \item~$6x^{12}-12=0$;
 \item~$\frac{x^3}{\sqrt 2}-\frac{\sqrt[3]2}{\sqrt 3}=0$;
 \item~$\sqrt 3\;x^3-3\sqrt[3]3=0$;
 \item~$\frac{x^4} 9-\frac 9{25}=0$.
 \end{enumeratea}
 \end{multicols}
\end{esercizio}

\begin{esercizio}
\label{ese:5.20}
Determinare l'insieme soluzione delle seguenti equazioni binomie.
\begin{multicols}{2}
 \begin{enumeratea}
 \item~$(x-1)^4=16$;
 \item~$(x^2-1)^3-27=0$;
 \item~$ \frac 3{x^4-1}=\frac 5{x^4+1} $;
 \item~$ \frac{x^4(x^2+2)-5}{x^2-1}=2(x^2+1) $.
 \end{enumeratea}
 \end{multicols}
\end{esercizio}

\subsection*{5.4 - Equazioni trinomie}

\begin{esercizio}[\Ast]
\label{ese:5.21}
Determinare l'insieme soluzione delle seguenti equazioni biquadratiche.
\begin{multicols}{3}
 \begin{enumeratea}
 \item~$x^4-13x^2+36=0$;
 \item~$2x^4-20x^2+18=0$;
 \item~$x^4-\frac{37} 9x^2+\frac 4 9=0$;
 \item~$x^4-\frac{13} 3x^2+\frac 4 3=0$;
 \item~$-x^4+\frac{17} 4x^2-1=0$;
 \item~$-2x^4+\frac{65} 2x^2-8=0$.
 \end{enumeratea}
\end{multicols}
\end{esercizio}

\begin{esercizio}[\Ast]
 \label{ese:5.22}
Determinare l'insieme soluzione delle seguenti equazioni biquadratiche.
\begin{multicols}{3}
 \begin{enumeratea}
 \item~$-2x^4+82x^2-800=0$;
 \item~$-3x^4+\frac{85} 3x^2-12=0$;
 \item~$x^4-\frac{16} 3x^2+\frac{16} 3=0$;
 \item~$x^4-7x^2+6=0$;
 \item~$x^4-10x^2+16=0$;
 \item~$-3x^4+9x^2+12=0$.
 \end{enumeratea}
\end{multicols}
\end{esercizio}
\newpage
\begin{esercizio}[\Ast]
\label{ese:5.23}
Determinare l'insieme soluzione delle seguenti equazioni biquadratiche.
\begin{multicols}{3}
 \begin{enumeratea}
 \item~$-\frac 1 2x^4+\frac 5 2x^2+18=0$;
 \item~$x^4+\frac{15} 4x^2-1=0$;
 \item~$-8x^4-\frac 7 2x^2+\frac 9 2=0$;
 \item~$-16x^4-63x^2+4=0$;
 \item~$x^4-2x^2-15=0$;
 \item~$x^4-2x^2-3=0$.
 \end{enumeratea}
\end{multicols}
\end{esercizio}

\begin{esercizio}[\Ast]
\label{ese:5.24}
Determinare l'insieme soluzione delle seguenti equazioni biquadratiche.
\begin{multicols}{2}
 \begin{enumeratea}
 \item~$ x^4-8x^2+16=0 $;
 \item~$8x^2+\frac{6x^2+x-4}{x^2-1}=4-\frac{3+4x}{1+x}$.
 \end{enumeratea}
\end{multicols}
\end{esercizio}

\begin{esercizio}
\label{ese:5.25}
È vero che l’equazione $4x^4-4=0$ ha quattro soluzioni reali a due a due coincidenti? Rispondi senza risolvere l'equazione.
\end{esercizio}

\begin{esercizio}
 \label{ese:5.26}
È vero che l’equazione $-x^4+2x^2-1=0$ ha quattro soluzioni reali a due a due coincidenti? Rispondi senza risolvere l'equazione.
\end{esercizio}

\begin{esercizio}
 \label{ese:5.27}
Perché le seguenti equazioni non hanno soluzioni reali?

\boxA\; $x^4+\frac{37} 4x^2+\frac 9 4=0$\quad \boxB\; $x^4-x^2+3=0$\quad\boxC\; $-2x^4-x^2-5=0$\quad\boxD\; $-x^4-5x^2-4=0$
\end{esercizio}

\begin{esercizio}[\Ast]
 \label{ese:5.28}
Senza risolvere le seguenti equazioni, dire se ammettono soluzioni reali:

\boxA\; $2x^4+5x^2-4=0$\quad \boxB\; $2x^4-5x^2+4=0$\quad\boxC\; $x^4-5x^2+1=0$\quad\boxD\; $-4x^4+5x^2-1=0$
\end{esercizio}

\begin{esercizio}[\Ast]
 \label{ese:5.29}
Data l'equazione $x^2\cdot \left(x^2-2a+1\right)=a\cdot (1-a)$ determinare per quali valori del parametro $a$ si hanno quattro soluzioni reali.
\end{esercizio}

\begin{esercizio}
 \label{ese:5.30}
È vero che la somma delle radici dell’equazione $ax^4+bx^2+c=0$ è nulla?
\end{esercizio}

\begin{esercizio}
 \label{ese:5.31}
Data l’equazione $ax^4+bx^2+c=0$ verifica le seguenti uguaglianze relative alle soluzioni reali:

\boxA\quad $x_1^2+x_2^2+x_3^2+x_4^2=-\frac{2b} a$\qquad \boxB\quad $x_1^2\cdot x_2^2\cdot x_3^2\cdot x_4^2=\frac c a$
\end{esercizio}

\subsection*{5.5 - Equazioni che si risolvono con sostituzioni}

\begin{esercizio}
 \label{ese:5.32}
Determinare l'insieme soluzione delle seguenti equazioni trinomie.
\begin{multicols}{2}
 \begin{enumeratea}
 \item~$x^6+13x^3+40=0$;
 \item~$x^8-4x^4+3=0$;
 \item~$-x^6+29x^3-54=0$;
 \item~$\frac 1 2x^{10}-\frac 3 2x^5+1=0$;
 \item~$-3x^{12}-3x^6+6=0$;
 \item~$2x^8+6x^4+4=0$.
 \end{enumeratea}
\end{multicols}
\end{esercizio}

\begin{esercizio}[\Ast]
 \label{ese:5.33}
Determinare l'insieme soluzione delle seguenti equazioni trinomie.
\begin{multicols}{2}
 \begin{enumeratea}
 \item~$-x^8-6x^4+7=0$;
 \item~$-2x^6+\frac{65} 4x^3-2=0$;
 \item~$-\frac 3 2x^{10}+\frac{99} 2x^5-48=0$;
 \item~$-\frac 4 3x^{14}-\frac 8 9x^7+\frac 4 9=0$.
 \end{enumeratea}
\end{multicols}
\end{esercizio}

\begin{esercizio}[\Ast]
 \label{ese:5.34}
Risolvi con le opportune sostituzioni le seguenti equazioni.
\begin{multicols}{2}
 \begin{enumeratea}
 \item~$\left(x^3+1\right)^3-8=0$;
 \item~$2\left(\frac{x+1}{x-1}\right)^2-3\left(\frac{x+1}{x-1}\right)-1=0$;
 \item~$\left(x^2+1\right)^2-6\left(x^2+1\right)+8=0$;
 \item~$\left(x+\frac 1 x\right)^2=\frac{16} 9$;
 \item~$\left(x+\frac 1 x\right)^2-16\left(x+\frac 1 x\right)=0$;
 \item~$\left(x^2-\frac 1 3\right)^2-12\left(x^2-\frac 1 3\right)+27=0$.
 \end{enumeratea}
\end{multicols}
\end{esercizio}

\begin{esercizio}[\Ast]
\label{ese:5.35}
Risolvi con le opportune sostituzioni le seguenti equazioni.
 \begin{enumeratea}
 \item~$(2x-1)^3=8$;
 \item~$(x+1)^3+6(x+1)^2-(x+1)-30=0$;
 \item~$(x^2+1)^3-4(x^2+1)^2-19(x^2+1)-14=0$;
 \item~$\frac{3x}{x+1}-\left(\frac{3x}{x+1}\right)^3=0$;
 \item~$\left(x-1\right)^2+\frac{x-3}{\left(x-1\right)^2}=\frac{x+6}{(1-x)^2}$;
 \item~$\left(\frac{x+1}{x-1}\right)^4-5\left(\frac{x+1}{x-1}\right)^2+4=0$.
 \end{enumeratea}
\end{esercizio}

\begin{esercizio}
 \label{ese:5.36}
Risolvi con le opportune sostituzioni le seguenti equazioni.
\begin{multicols}{2}
 \begin{enumeratea}
 \item~$ (x^3+2)^5=1 $;
 \item~$ \left(\frac x{x-1}\right)^4-13\left(\frac x{x-1}\right)^2+36=0 $;
 \item~$ \left(\frac{x+1}{x+2}\right)^4-10\left(\frac{x+1}{x+2}\right)^2+9=0 $;
 \item~$ \left(x-\sqrt 2\right)^6-4\left(x-\sqrt 2\right)^3+3=0 $;
 \item~$ \left(\frac{x+1} x\right)^{10}-33\left(\frac{x+1} x\right)^5+32=0 $;
 \item~$ \left(\frac x{x+1}\right)^2-13+36\left(\frac{x+1} x\right)^2=0 $.
 \end{enumeratea}
\end{multicols}
\end{esercizio}

\begin{esercizio}
\label{ese:5.37}
Risolvi con le opportune sostituzioni le seguenti equazioni.
 \begin{enumeratea}
 \item~$ \frac{x-3}{x+3}+2=15\left(\frac{x+3}{x-3}\right) $;
 \item~$ \left(x^2-1\right)^3+\frac 8{\left(x^2-1\right)^3}=9 $;
 \item~$ \left(\frac 1{x^2-1}\right)^3-3\left(\frac 1{x^2-1}\right)^3-4\left(\frac 1{x^2-1}\right)^3+12=0 $.
 \end{enumeratea}
\end{esercizio}

\subsection*{5.6 - Equazioni reciproche}

\begin{esercizio}[\Ast]
\label{ese:5.38}
Risolvi le seguenti equazioni reciproche di prima specie.
\begin{multicols}{2}
\begin{enumeratea}
\item $3x^3+13x^2+13x+3=0$;
\item $2x^3-3x^2-3x+2=0$;
\item $5x^3-21x^2-21x+5=0$;
\item $12x^3+37x^2+37x+12=0$;
\item $10x^3-19x^2-19x+10=0$;
\item $15x^3-19x^2-19x+15=0$.
\end{enumeratea}
\end{multicols}
\end{esercizio}

\begin{esercizio}[\Ast]
\label{ese:5.39}
Risolvi le seguenti equazioni reciproche di prima specie.
\begin{enumeratea}
\item $4x^3+13x^2-13x=4$;
\item $4x^3-13x^2=13x-4$;
\item $3x(10x-19)+9x(x-2)=10(x+1)(x^2-x+1)$;
\item $2x^3-(3\sqrt 2+2)x^2-(3\sqrt 2+2)x+2=0$.
\end{enumeratea}
\end{esercizio}

\begin{esercizio}[\Ast]
\label{ese:5.40}
Risolvi le seguenti equazioni reciproche di prima specie.
\begin{enumeratea}
\item $x^3+x^2(2\sqrt 2+1)+x(2\sqrt 2+1)+1=0$;
\item $x^3-3x^2-3x+1=0$;
\item ${ax}^3+(a^2+a+1)x^2+(a^2+a+1)x+1=0$.
\end{enumeratea}
\end{esercizio}

 \begin{esercizio}[\Ast]
\label{ese:5.41}
Dopo aver verificato che $x=3$ è radice dell’equazione $3x^3-13x^2+13x-3=0$, verificate che l’equazione ammette come soluzione $x=\frac 1 3$.
 \end{esercizio}

\begin{esercizio}
\label{ese:5.42}
Determina il valore di verità delle seguenti proposizioni:
\begin{enumeratea}
\item l’equazione $ax^3+bx^2+cx+d=0$ ammette sempre $x=-1$ come soluzione;
\item se nell’equazione $ax^3+bx^2+cx+d=0$ si ha $a=d$ e $b=c$ allora $x=-1$ è una soluzione;
\item in una equazione reciproca di terzo grado la somma dei coefficienti è nulla;
\item se in $ax^3+bx^2+cx+d=0$ si ha $a+d=0$ e $b+c=0$ allora $x=1$ appartiene all’$\IS$
\end{enumeratea}
\end{esercizio}

\begin{esercizio}[\Ast]
\label{ese:5.43}
Risolvi le seguenti equazioni reciproche di seconda specie.
\begin{multicols}{2}
\begin{enumeratea}
\item $6x^3-19x^2+19x-6=0$;
\item $7x^3-57x^2+57x-7=0$;
\item $3x^3+7x^2-7x-3=0$;
\item $12x^3+13x^2-13x-12=0$;
\item $10x^3+19x^2-19x-10=0$;
\item $x^3+3x^2-3x-1=0$.
\end{enumeratea}
\end{multicols}
\end{esercizio}

\begin{esercizio}[\Ast]
\label{ese:5.44}
Risolvi le seguenti equazioni reciproche di seconda specie.
\begin{multicols}{2}
\begin{enumeratea}
\item $\frac{x^3-1} x=\frac{21} 4\cdot (1-x)$;
\item $5x^3+(6\sqrt 5-5)x^2+x(5-6\sqrt 5)-5=0$;
\item $x^3+13x^2-13x-1=0$;
\item $4x^3+(5\sqrt 5-1)x^2+x(1-5\sqrt 5)=4$.
\end{enumeratea}
\end{multicols}
\end{esercizio}

\begin{esercizio}
\label{ese:5.45}
Data l’equazione $a_0\left(t^2-2\right)+a_1t+a_2=0\;\to \;a_0t^2+a_1t+a_2-2a_0=0$ stabilire quale condizione deve sussistere tra i coefficienti affinché esistano valori reali dell'incognita $ t $.
\end{esercizio}

\begin{esercizio}[\Ast]
\label{ese:5.46}
Risolvi le seguenti equazioni di quarto grado reciproche di prima specie.
\begin{multicols}{2}
\begin{enumeratea}
\item $x^4-5x^3+8x^2-5x+1=0$;
\item $x^4+5x^3-4x^2+5x+1=0$;
\item $x^4+2x^3-13x^2+2x+1=0$;
\item $x^4-\frac 5 6x^3-\frac{19} 3x^2-\frac 5 6x+1=0$.
\end{enumeratea}
\end{multicols}
\end{esercizio}

\begin{esercizio}[\Ast]
 \label{ese:5.47}
Risolvi le seguenti equazioni di quarto grado reciproche di seconda specie.
\begin{multicols}{2}
\begin{enumeratea}
\item $x^4-3x^3+3x-1=0$;
\item $4x^4-5x^3+5x-4=0$;
\item $3x^4+7x^3-7x-3=0$;
\item $x^4-7x^3+7x-1=0$.
\end{enumeratea}
\end{multicols}
\end{esercizio}

\begin{esercizio}[\Ast]
 \label{ese:5.48}
Risolvi le seguenti equazioni di quarto grado reciproche di seconda specie.
\begin{multicols}{2}
\begin{enumeratea}
\item $5x^4-11x^3+11x-5=0$;
\item $6x^4-13x^3+13x-6=0$;
\item $7x^4-15x^3+15x-7=0$;
\item $x^3(x-4)=1-4x$.
\end{enumeratea}
\end{multicols}
\end{esercizio}

\begin{esercizio}[\Ast]
 \label{ese:5.49}
Risolvi le seguenti equazioni di quarto grado reciproche di seconda specie.
\begin{multicols}{2}
\begin{enumeratea}
\item $\frac{x-5}{5x-1}+\frac 1{x^3}=0$;
\item $x^4-x^3+x-1=0$;
\item $\frac{x^4+2x-1}{8x^3}-\frac{1+8x^2}{4x^2}+\frac{x-1} x+\frac{1+x}{x^2}=0$.
\end{enumeratea}
\end{multicols}
\end{esercizio}

\begin{esercizio}
 \label{ese:5.50}
Quale condizione deve sussistere tra i coefficienti dell’equazione $a_0x^2+a_1x+a_0=0$ affinché siano reali le sue soluzioni?
\end{esercizio}

\begin{esercizio}[\Ast]
 \label{ese:5.51}
Determinare per quale valore di $k$, l’equazione $\left(2k-\sqrt 2\right)x^4+5x^3-5x-2\sqrt 2=0$ è reciproca. È vero che $\IS=\{+1;-1\}$
\end{esercizio}
\newpage
\subsection{Esercizi riepilogativi}

\begin{esercizio}[\Ast] %5.52
Risolvi le seguenti equazioni di grado superiore al secondo.
\begin{multicols}{2}
\begin{enumeratea}
\item $6x^3+7x^2-7x-6=0$;
\item $2x^3+5x^2+5x+2=0$;
\item $x^3-3x^2+3x-1=0$;
\item $3x^3-4x^2+4x-3=0$;
\item $2x^4-5x^3+5x-2=0$;
\item $-5x^4+3x^3-3x+5=0$.
\end{enumeratea}
\end{multicols}
\end{esercizio}

\begin{esercizio}[\Ast] %5.53
Risolvi le seguenti equazioni di grado superiore al secondo.
\begin{multicols}{2}
\begin{enumeratea}
\item $2x^5-3x^4+4x^3-4x^2+3x-2=0$;
\item $-2x^4+8x^3-8x+2=0$;
\item $2x^3-5x^2-5x+2=0$;
\item $3x^3-6x^2-6x+3=0$;
\item $5x^3-7x^2+7x-5=0$;
\item $4x^3-20x^2+20x-4=0$.
\end{enumeratea}
\end{multicols}
\end{esercizio}

\begin{esercizio}[\Ast] %5.54
Risolvi le seguenti equazioni di grado superiore al secondo.
\begin{multicols}{2}
\begin{enumeratea}
\item $5x^3-5x^2-5x+5=0$;
\item $4x^3-9x^2+9x-4=0$;
\item $\frac 3 2x^3+\frac 7 4x^2-\frac 7 4x-\frac 3 2=0$;
\item $3x^3-2x^2+2x-3=0$;
\item $-2x^3+10x^2+10x-2=0$;
\item $x^4-\frac 9 4x^3-\frac{13} 2x^2-\frac 9 4x+1=0$.
\end{enumeratea}
\end{multicols}
\end{esercizio}

\begin{esercizio}[\Ast] %5.55
Risolvi le seguenti equazioni di grado superiore al secondo.
\begin{multicols}{2}
\begin{enumeratea}
\item $x^4-4x^3+6x^2-4x+1=0$;
\item $x^4+\frac{10} 3x^3+2x^2+\frac{10} 3x+1=0$;
\item $x^4-4x^3+2x^2-4x+1=0$;
\item $x^4-x^3+x-1=0$;
\item $x^4-6x^3+6x-1=0$;
\item $x^4-3x^3+2x^2-3x+1=0$.
\end{enumeratea}
\end{multicols}
\end{esercizio}

\begin{esercizio}[\Ast] %5.56
Risolvi le seguenti equazioni di grado superiore al secondo.
\begin{multicols}{2}
\begin{enumeratea}
\item $x^4-5x^3-12x^2-5x+1=0$;
\item $3x^4-x^3+x-3=0$;
\item $2x^4-5x^3+4x^2-5x+2=0$;
\item $2x^4-x^3+4x^2-x+2=0$;
\item $3x^4-7x^3+7x-3=0$;
\item $3x^4-6x^3+6x-3=0$.
\end{enumeratea}
\end{multicols}
\end{esercizio}

\begin{esercizio}[\Ast] %5.57
Risolvi le seguenti equazioni di grado superiore al secondo.
\begin{multicols}{2}
\begin{enumeratea}
\item $2x^4-6x^3+4x^2-6x+2=0$;
\item $x^4+8x^3-8x-1=0$;
\item $6x^4-37x^3+37x-6=0$;
\item $x^5-3x^4+2x^3+2x^2-3x+1=0$;
\item $x^5-2x^4-5x^3-5x^2-2x+1=0$;
\item $x^5+3x^4+x^3-x^2-3x-1=0$.
\end{enumeratea}
\end{multicols}
\end{esercizio}

\begin{esercizio}[\Ast] %5.58
Risolvi le seguenti equazioni di grado superiore al secondo.
\begin{multicols}{2}
\begin{enumeratea}
\item $x^5+x^4+x^3-x^2-x-1=0$;
\item $x^5-2x^4+x^3-x^2+2x-1=0$;
\item $x^5-5x^3-5x^2+1=0$;
\item $x^5+3x^4-2x^3+2x^2-3x-1=0$;
\item $2x^5-2x^4+2x^3+2x^2-2x+2=0$;
\item $x^6-x^5-5x^4+5x^2+x-1=0$.
\end{enumeratea}
\end{multicols}
\end{esercizio}
\newpage
\begin{esercizio}[\Ast] %5.59
Risolvi le seguenti equazioni di grado superiore al secondo.
\begin{multicols}{2}
\begin{enumeratea}
\item $x^6-x^5-x^4+2x^3-x^2-x+1=0$;
\item $x^5-2x^4+x^3+x^2-2x+1=0$;
\item $x^5-\frac{11} 4x^4-\frac{55} 8x^3+\frac{55} 8x^2+\frac{11} 4x-1=0$;
\item $x^5-4x^4+\frac{13} 4x^3+\frac{13} 4x^2-4x+1=0$;
\item $x^6+\frac{13} 6x^5+x^4-x^2-\frac{13} 6x-1=0$;
\item $x^6+\frac{16} 3x^5+\frac{23} 3x^4-\frac{23} 3x^2-\frac{16} 3x-1=0$.
\end{enumeratea}
\end{multicols}
\end{esercizio}

\begin{esercizio}[\Ast] %5.60
Risolvi le seguenti equazioni di grado superiore al secondo.
\begin{multicols}{2}
\begin{enumeratea}
\item $x^6+x^4-x^2-1=0$;
\item $x^6-4x^5-x^4+8x^3-x^2-4x+1=0$;
\item $x^6+2x^4+2x^2+1=0$;
\item $3(2x-2)^3+(10x-5)^2-25=0$.
\end{enumeratea}
\end{multicols}
\end{esercizio}

\begin{esercizio}[\Ast] %5.61
Risolvi le seguenti equazioni di grado superiore al secondo.
\begin{multicols}{2}
\begin{enumeratea}
\item $\frac{6x^2-2}{x^2+1}=\frac 6{x^4-5x^2-6}+\frac{5x}{6-x^2}$;
\item $x^4+5x(x+1)^2+(1-2x)(1+2x)=0$;
\item $\frac{9x^2(x+4)}{9x+1+\sqrt{10}}=\frac{9x+1-\sqrt{10}}{x-6}$;
\item $\frac{x^2(x+4)}{x-1}-\frac{8x+1}{x+1}-\frac{2x}{x^2-1}=0$.
\end{enumeratea}
\end{multicols}
\end{esercizio}

\begin{esercizio}[\Ast] %5.62
Nell’equazione $(2-a)x^5-x^4+(3+a)x^3+2bx^2+x+5b=0$ determinare $a$ e $b$ in modo che l’equazione sia reciproca.
\end{esercizio}

\subsection{Risposte}
\paragraph{5.1.} a)~$-4;-3;2$,\quad b)~$\frac 1 2;-3;-\frac 4 3$,\quad c)~$\frac 5 2;1;\frac 3 2$,\quad d)~$4;-7$,\quad e)~$-3,-\frac 1 3,+2$,\quad f)~$-4,+\frac 7 3,+2$.

\paragraph{5.2.} a)~$0,+\frac 1 2,+1$,\quad b)~$+4$,\quad c)~$-5$,\quad d)~$3;-\frac 1 2$,\quad e)~$4,\frac 2 3,-2$,\quad f)~$2;1;-\frac 1 2$.

\paragraph{5.3.} a)~$1,-2$,\quad b)~$2$,\quad c)~$1,-1$,\quad d)~$5,1,-2$,\quad e)~$2,-9$,\quad f)~$1;-1;2$.

\paragraph{5.4.} a)~$-1;\frac 1 2;\frac 2 3$,\quad b)~$-1;6;2$.

\paragraph{5.6.} a)~$\{-2;1\}$,\quad b)~$\{-1\}$,\quad c)~$\{-3\}$,\quad d)~$\{-1;1\}$,\quad e)~$\{-3;-1;1\}$,\quad f)~$\left\{-\frac1 2;\frac 2 3;1\right\}$.

\paragraph{5.7.} a)~$\{1;2;3\}$,\quad b)~$\left\{0;\frac1 2\right\}$,\quad c)~$\{2;-2;3\}$,\quad d)~$\{-1;+1\}$,\quad e)~$\{-1;1;2\}$,\quad f)~$\{-1\}$.

\paragraph{5.8.} a)~$\left\{-\frac 1 3;1\right\}$,\quad b)~$\left\{\pm 1;1\pm \sqrt 2\right\}$,\quad c)~$\left\{-3;\pm \sqrt 2\right\}$,\quad d)~$\{-2\}$,\quad e)~$\left\{-\sqrt 3;0;+\sqrt 3\right\}$.

\paragraph{5.9.} a)~$\left\{0;\sqrt 2-1;-\left(\frac{\sqrt 2} 2+1\right)\right\}$,\quad b)~$\left\{0,\frac 3 4\right\}$,\quad c)~$\insR$,\quad d)~$\{0;4;2\pm \sqrt 5\}$,\quad e)~$\{0\}$,\quad f)~$\left\{0;\frac 2 3;2\right\}$.

\paragraph{5.10.} a)~$\left\{\frac 1 2\right\}$,\quad b)~$\left\{-\frac 3 2;-2\right\}$,\quad c)~$\{0;3\pm \sqrt 2\}$,\quad d)~$\left\{1;\frac{-1\pm \sqrt 5} 2\right\}$,\quad e)~$\emptyset $,\quad f)~$\{\pm 1;\pm \sqrt 2\}$.

\paragraph{5.11.} b)~$\left\{\pm \sqrt{1+\sqrt 5}\right\}$,\quad c)~$\{1;-\sqrt[3]9\}$,\quad d)~$\{0;\pm 2\}$.

\paragraph{5.12.} a)~$\{2\}$,\quad b)~$\{-\sqrt[5]{15}\}$,\quad c)~$\emptyset $,\quad d)~$\{-3;+3\}$,\quad e)~$\left\{\pm \frac{\sqrt 5}{\sqrt[6]3}\right\}$,\quad f)~$\left\{\pm \frac 1 3\right\}$.

\paragraph{5.13.} a)~$\left\{-\frac 1 3\right\}$,\quad b)~$\emptyset $,\quad c)~$\left\{-\frac{\sqrt 3} 3,+\frac{\sqrt 3} 3\right\}$,\quad d)~$\left\{-2;+2\right\}$,\quad e)~$\{-1;1\}$,\quad f)~$\left\{\frac 3 2\right\}$.

\paragraph{5.14.} a)~$\{1\}$,\quad b)~ $\emptyset $,\quad c)~$\left\{-\sqrt 2;\sqrt 2\right\}$,\quad d)~$\left\{-2\right\}$,\quad e)~$\left\{-\sqrt[3]{\frac 5 7},\sqrt[3]{\frac 5 7}\right\}$,\quad f)~$\left\{\frac 1 3\right\}$.

\paragraph{5.15.} a)~$\{\pm 10\}$,\quad b)~ $\left\{-\frac 1{10}\right\}$,\quad c)~$\{\pm 20\}$,\quad d)~$\emptyset $,\quad e)~$\left\{\frac 3 2\right\}$,\quad f)~${I.S}=\left\{-\frac 1 2\sqrt[3]9\right\}$.

\paragraph{5.16.} a)~$\left\{\pm \frac 2 3\right\}$,\quad b)~$\left\{\pm \frac{\sqrt 3} 2\right\}$.

\paragraph{5.18.} c)~$\left\{\pm 2^{\frac 1 9}3^{\frac 1{18}}\right\}$,\quad d)~${I.S}=\left\{\frac 5 3\right\}$.

\paragraph{5.19.} e)~$\left\{3^{\frac 5{18}}\right\}$,\quad f)~$\left\{\pm 3\frac{\sqrt 5} 5\right\}$.

\paragraph{5.21.} a)~$\{\pm 3;\pm 2\}$,\quad b)~$\{\pm 1; \pm 3\}$,\quad c)~$\left\{\pm 2;\pm \frac 1 3\right\}$,\quad d)~ $\left\{\pm 2;\pm \frac{\sqrt 3} 3\right\}$,\quad e)~$\left\{\pm 2;\pm \frac 1 2\right\}$,\quad f)~$\left\{\pm 4;\pm \frac 1 2\right\}$.

\paragraph{5.22.} a)~$\{\pm 4;\pm 5\}$,\; b)~$\left\{\pm 3;\pm \frac 2 3\right\}$,\; c)~$\left\{\pm 2;\pm \frac{2\sqrt 3} 3\right\}$,\; d)~$\left\{\pm 1;\pm \sqrt 6\right\}$,\; e)~$\left\{\pm \sqrt 2;\pm 2\sqrt 2\right\}$,\; f)~$\{-2;2\}$.

\paragraph{5.23.} a)~$\{\pm 3\}$,\quad b)~$\left\{\pm \frac 1 2\right\}$,\quad c)~$\left\{\pm \frac 3 4\right\}$,\quad d)~$\left\{\pm \frac 1 4\right\}$,\quad e)~$\left\{\pm sqrt 5\right\}$,\quad f)~$\left\{\pm \sqrt 3\right\}$.

\paragraph{5.24.} b)~$\left\{\pm \frac{\sqrt 3} 2\right\}$.

\paragraph{5.28.} si, no, si, no.

\paragraph{5.29.} $a>1$.

\paragraph{5.32.} a)~$\left\{-2;-\sqrt[3]5\right\}$,\quad b)~$\left\{\pm 1;\pm \sqrt[4]3\right\}$,\quad c)~$\left\{3;\sqrt[3]2\right\}$,\quad d)~$\left\{1;\sqrt[5]2\right\}$,\quad e)~$\{\pm 1\}$,\quad f)~$\emptyset $.

\paragraph{5.33.} a)~$\{\pm 1\}$,\quad b)~$\left\{2;\frac 1 2\right\}$,\quad c)~$\left\{1;2\right\}$,\quad d)~$\left\{-1;\sqrt[7]{\frac 1 3}\right\}$.

\paragraph{5.34.} a)~$\{1\}$,\quad b)~$\left\{\frac{3\pm\sqrt{17}} 2\right\}$,\quad c)~$\left\{\pm\sqrt 3;\pm 1\right\}$,\quad d)~$\emptyset $,\quad e)~$\left\{8\pm3\sqrt 7\right\}$,\quad f)~$\left\{\pm \frac{2\sqrt{21}} 3;\pm \frac{\sqrt{30}} 3\right\}$.

\paragraph{5.35.} a)~$\left\{\frac 3 2\right\}$,\quad b)~$\left\{-6;-4;1\right\}$,\quad c)~$\left\{\pm\sqrt 6\right\}$,\quad d)~$\left\{-\frac 1 4;0;\frac 1 2\right\}$,\quad e)~$\left\{1\pm \sqrt 3\right\}$,\quad f)~$\left\{0;3,\frac 1 3\right\}$.

\paragraph{5.38.} a)~$\left\{-1;-\frac 1 3;-3\right\}$,\quad b)~$\left\{-1;2;\frac 1 2 \right\}$,\quad c)~$\left\{-1;5;\frac 1 5\right\}$,\quad d)~$\left\{-\frac 4 3;-1;-\frac 3 4\right\}$,\quad e)~$\left\{-1;\frac 2 5;\frac 5 2\right\}$,\protect \\
f)~$\left\{-1;\frac 3 5;\frac 5 3\right\}$.

\paragraph{5.39.} a)~$\left\{-4;-\frac 1 4;1\right\}$,\quad b)~$\left\{-1;4;\frac 1 4\right\}$,\quad c)~$\left\{-1;\frac{49\pm \sqrt{2001}}{20}\right\}$,\quad d)~$\left\{-1;-\sqrt 2;-\frac{\sqrt 2} 2\right\}$.

\paragraph{5.40.} a)~$\{-1;\sqrt 2-1;-\sqrt 2+1\}$,\quad b)~$\{-1;\sqrt 3+2;2-\sqrt 3\}$,\quad c)~$\left\{-1;-a;-\frac 1 a\right\}$.

\paragraph{5.43.} a)~$\left\{1;\frac 2 3;\frac 3 2\right\}$,\quad b)~$\left\{1;7;\frac 1 7\right\}$,\quad c)~$\left\{-3;-\frac 1 3;1\right\}$,\quad d)~$\left\{-\frac 4 3;-\frac 3 4;1\right\}$,\quad e)~$\left\{-\frac 5 2;-\frac 2 5;1\right\}$,\protect\\ f)~$\left\{2\pm\sqrt 3;-1\right\}$.

\paragraph{5.44.} a)~$\left\{1;\frac{-25\pm \sqrt{561}} 8\right\}$,\quad b)~$\left\{1;-\sqrt 5;-\frac{\sqrt 5} 5\right\}$,\quad c)~$\left\{1;-7\pm 4\sqrt 3\right\}$,\quad d)~$\left\{1;-\sqrt 5-1;\frac 1 4-\frac{\sqrt 5} 4\right\}$.

\paragraph{5.46.} a)~$\left\{1;\frac{3\pm \sqrt 5} 2\right\}$,\quad b)~$\left\{-3\pm 2\sqrt 2\right\}$,\quad c)~$\left\{\frac{-5\pm \sqrt{21}} 2;\frac{3\pm \sqrt 5} 2\right\}$,\quad d)~$\left\{3;\frac 1 3;2-\frac 1 2\right\}$.

\paragraph{5.47.} a)~$\left\{\pm 1;\frac{3\pm \sqrt 5} 2\right\}$,\quad b)~$\left\{\pm 1\right\}$,\quad c)~$\left\{\pm 1;\frac{-7\pm \sqrt{13}} 6\right\}$,\quad d)~$\left\{\pm 1;\frac{7\pm 3\sqrt 5} 2\right\}$.

\paragraph{5.48.} a)~$\left\{\pm 1;\frac{11\pm \sqrt{21}}{10}\right\}$,\quad b)~$\left\{\pm 1;\frac 2 3;\frac 3 2\right\}$,\quad c)~$\left\{\pm 1;\frac{15\pm \sqrt{29}}{14}\right\}$,\quad d)~$\left\{\pm 1;2\pm \sqrt 3\right\}$.

\paragraph{5.49.} a)~$\left\{\pm 1;\frac{5\pm \sqrt{21}} 2\right\}$,\quad b)~$\left\{\pm 1\right\}$,\quad c)~$\{\pm 1;4\pm \sqrt{15}\}$.

\paragraph{5.51.} a)~$k=\frac 3 2\sqrt 2$.

\paragraph{5.52.} a)~$\left\{-\frac 3 2;-\frac 2 3;1\right\}$,\quad b)~$\left\{-1\right\}$,\quad c)~$\left\{1\right\}$,\quad d)~$\left\{1\right\}$,\quad e)~$\left\{\frac 1 2;2;\pm 1\right\}$,\quad f)~$\left\{\pm 1\right\}$.

\paragraph{5.53.} a)~$\left\{1\right\}$,\quad b)~$\left\{2\pm \sqrt 3;\pm 1\right\}$,\quad c)~$\left\{-1;\frac{7\pm\sqrt{33}} 4\right\}$,\quad d)~$\left\{-1;\frac{3\pm\sqrt 5} 2 \right\}$,\quad e)~$\{1\}$,\protect\\ \quad f)~$\left\{x=1;x=2\pm\sqrt 3\right\}$.

\paragraph{5.54.} a)~$\{\pm 1\}$,\quad b)~$\{1\}$,\quad c)~$\left\{1;\pm\frac 2 3\right\}$,\quad d)~$\{1\}$,\quad e)~$\left\{1;3\pm 2\sqrt 2\right\}$,\quad f)~$\left\{-1;4;\frac 1 4\right\}$.

\paragraph{5.55.} a)~$\{\pm 1\}$,\quad b)~$\left\{-3;-\frac 1 3\right\}$,\quad c)~$\left\{2\pm\sqrt 3\right\}$,\quad d)~$\{\pm 1\}$,\quad e)~$\left\{\pm 1;3\pm 2\sqrt 2\right\}$,\quad f)~$\left\{\frac{3\pm\sqrt 5} 2\right\}$.

\paragraph{5.56.} a)~$\left\{-1;\frac{7\pm 3\sqrt 5} 2\right\}$,\quad b)~$\{\pm 1\}$,\quad c)~$\left\{2;\frac 1 2\right\}$,\quad d)~$\emptyset$,\quad e)~$\left\{\pm 1;\frac{7\pm \sqrt{13}} 6\right\}$,\quad f)~$\{\pm 1\}$.

\paragraph{5.57.} a)~$\left\{\frac{3\pm\sqrt 5} 2\right\}$,\quad b)~$\left\{\pm 1;-4\pm \sqrt{15}\right\}$,\quad c)~$\left\{\pm 1;+6;\frac 1 6\right\}$,\quad d)~$\{\pm 1\}$,\quad e)~$\left\{-1;2\pm\sqrt 3\right\}$,\protect\\ \quad f)~$\left\{1;\frac{-3\pm\sqrt 5} 2\right\}$.

\paragraph{5.58.} a)~$\{1\}$,\quad b)~$\{1\}$,\quad c)~$\left\{-1;\frac{3\pm\sqrt 5} 2\right\}$,\quad d)~$\left\{1;-2\pm\sqrt 3\right\}$,\quad e)~$\{-1\}$,\quad f)~$\left\{\pm 1;\frac{3\pm \sqrt 5} 2\right\}$.

\paragraph{5.59.} a)~$\{\pm 1\}$,\quad b)~$\{\pm 1\}$,\quad c)~$\left\{1;4;\frac 1 4;-2;-\frac 1 2\right\}$,\quad d)~$\left\{-1;2;\frac 1 2\right\}$,\quad e)~$\left\{\pm 1;x=\pm\frac 3 2\right\}$,\protect\\ \quad f)~$\left\{\pm 1-3;-\frac 1 3\right\}$.

\paragraph{5.60.} a)~$\{\pm 1\}$,\quad b)~$\left\{\pm 1;2\pm \sqrt 3\right\}$,\quad c)~$\emptyset$,\quad d)~$\left\{1;\pm\frac 3 2\right\}$.

\paragraph{5.61.} a)~$\left\{2;\frac 1 2;-3;-\frac 1 3\right\}$,\quad b)~$\left\{-2\pm \sqrt 3\right\}$,\quad c)~$\left\{\frac{7\pm 3\sqrt 5} 2;\frac{-5\pm \sqrt{21}} 2\right\}$,\quad d)~$\left\{-3\pm 2\sqrt 2\right\}$.

%\paragraph{5.62.} $a=-\frac{11} 7$ e $b=-\frac 5 7$.
\cleardoublepage
