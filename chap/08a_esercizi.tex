% (c) 2014 Daniele Masini - d.masini.it@gmail.com
\section{Esercizi}
\subsection{Esercizi dei singoli paragrafi}
\subsubsection*{\thechapter.1 - Successioni numeriche}

\begin{esercizio}
\label{ese:8a_succ.1}
Le due scritture
\[1\text{, }\frac{1}{2}\text{, }\frac{1}{3}\text{, }\frac{1}{4}\text{, }\frac{1}{5}\text{, }\frac{1}{6}\text{, }\ldots \qquad
1\text{, }\frac{1}{3}\text{, }\frac{1}{5}\text{, }\frac{1}{4}\text{, }\frac{1}{2}\text{, }\frac{1}{6}\text{, }\ldots\]
rappresentano la stessa successione? Perché?
\end{esercizio}

\begin{esercizio}
\label{ese:8a_succ.2}
Scrivere il termine generico $a_n$ della successione $\dfrac{1}{3}$, $\dfrac{1}{6}$, $\dfrac{1}{9}$, $\dfrac{1}{12}$, \ldots{}
\end{esercizio}

\begin{esercizio}
\label{ese:8a_succ.3}
Scrivere il termine generico $a_n$ della successione $\dfrac{1}{2}$, $-\dfrac{1}{2}$, $\dfrac{1}{2}$, $-\dfrac{1}{2}$, \ldots{}
\end{esercizio}

\begin{esercizio}
\label{ese:8a_succ.4}
Scrivere il termine generico $a_n$ della successione $\dfrac{1}{2}$, $\dfrac{3}{5}$, $\dfrac{5}{8}$, $\dfrac{7}{11}$, $\dfrac{9}{14}$, \ldots{}
\end{esercizio}

\begin{esercizio}
\label{ese:8a_succ.5}
Scrivere il termine generico $a_n$ della successione $\sqrt{2}-1$, $\sqrt{6}-2$, $2\sqrt{3}-3$, $2\sqrt{5}-4$, \ldots{}
\end{esercizio}

\begin{esercizio}
\label{ese:8a_succ.6}
Si determini il dominio della successione $\left\{\dfrac{2n-1}{3n-1}\right\}$ e si rappresentino i suoi primi 5 termini sul piano cartesiano.
\end{esercizio}

\begin{esercizio}
\label{ese:8a_succ.7}
Si determini il dominio della successione $\left\{\dfrac{4n-5}{2n-4}\right\}$ e si rappresentino i suoi primi 5 termini sul piano cartesiano.
\end{esercizio}

\begin{esercizio}
\label{ese:8a_succ.8}
Si determini il dominio della successione $\left\{\dfrac{n-3}{2n-3}\right\}$ e si rappresentino i suoi primi 5 termini sul piano cartesiano.
\end{esercizio}

\begin{esercizio}
\label{ese:8a_succ.9}
Si determini il dominio della successione $\left\{\dfrac{n^2-1}{n}\right\}$ e si rappresentino i suoi primi 5 termini sul piano cartesiano.
\end{esercizio}

\subsubsection*{\thechapter.2 - Proprietà delle successioni}

\begin{esercizio}
\label{ese:8a_psucc.1}
Determina l'andamento della successione $\left\{3-\sqrt{n}\right\}$ e rappresenta i suoi primi 5 termini sul piano cartesiano.
\end{esercizio}

\begin{esercizio}
\label{ese:8a_psucc.2}
Determina l'andamento della successione $\left\{\sqrt{1+n}\right\}$ e rappresenta i suoi primi 5 termini sul piano cartesiano.
\end{esercizio}

\begin{esercizio}
\label{ese:8a_psucc.3}
Determina l'andamento della successione $\left\{\dfrac{n^2-1}{n}\right\}$ e rappresenta i suoi primi 5 termini sul piano cartesiano.
\end{esercizio}

\begin{esercizio}
\label{ese:8a_psucc.4}
Determina l'andamento della successione $\left\{(-1)^{2n}\right\}$ e rappresenta i suoi primi 5 termini sul piano cartesiano.
\end{esercizio}

\begin{esercizio}
\label{ese:8a_psucc.5}
Determina l'andamento della successione $\left\{(-1)^{3n}\right\}$ e rappresenta i suoi primi 5 termini sul piano cartesiano.
\end{esercizio}


\subsubsection*{\thechapter.3 - Limite di una successione}

\begin{esercizio}
\label{ese:8a_lsucc.1}
Determina, se esiste, il limite della successione $\left\{\dfrac{3n-1}{2n-1}\right\}$.
\end{esercizio}

\begin{esercizio}
\label{ese:8a_lsucc.2}
Determina, se esiste, il limite della successione $\left\{\dfrac{n^2}{n^2-1}\right\}$.
\end{esercizio}

\begin{esercizio}
\label{ese:8a_lsucc.3}
Determina, se esiste, il limite della successione $\left\{\dfrac{n+1}{2n^2}\right\}$.
\end{esercizio}

\begin{esercizio}
\label{ese:8a_lsucc.4}
Determina, se esiste, il limite della successione $\left\{\dfrac{5n^3-4}{n^2-n^3}\right\}$.
\end{esercizio}

\begin{esercizio}
\label{ese:8a_lsucc.5}
Determina, se esiste, il limite della successione $\left\{(-1)^{3n}\right\}$.
\end{esercizio}

\begin{esercizio}
\label{ese:8a_lsucc.6}
Determina, se esiste, il limite della successione $\left\{\sqrt{n^2+5n}-n\right\}$.
\end{esercizio}

\begin{esercizio}
\label{ese:8a_lsucc.7}
Determina, se esiste, il limite della successione $\left\{\dfrac{n^2+1}{n-\sqrt{n}}\right\}$.
\end{esercizio}


\subsubsection*{\thechapter.4 - Progressioni}

\begin{esercizio}
\label{ese:8a_progr.1}
Scrivi i primi 5 termini di una progressione aritmetica con $a_1=4$ e ragione $q=3$. Quindi rappresentali sul piano cartesiano.
\end{esercizio}

\begin{esercizio}
\label{ese:8a_progr.2}
Scrivi i primi 5 termini di una progressione aritmetica con $a_1=\frac{3}{2}$ e ragione $q=\frac{1}{2}$. Quindi rappresentali sul piano cartesiano.
\end{esercizio}

\begin{esercizio}
\label{ese:8a_progr.3}
Calcolare la ragione $q$ e scrivere altri 3 termini della seguente successione aritmetica: $-4$, $-1$, $2$, $5$, \ldots{} Quindi li si rappresenti sul piano cartesiano.
\end{esercizio}

\begin{esercizio}
\label{ese:8a_progr.4}
Calcolare la ragione $q$ e scrivere altri 3 termini della seguente successione aritmetica: $\frac{4}{5}$, 1, $\frac{6}{5}$, $\frac{7}{5}$, \ldots{} Quindi li si rappresenti sul piano cartesiano.
\end{esercizio}

\begin{esercizio}
\label{ese:8a_progr.5}
Calcolare la ragione $q$ e scrivere altri 3 termini della seguente successione aritmetica: $\frac{2\sqrt{3}}{5}$, $\frac{7\sqrt{3}}{5}$, $\frac{4\cdot3^{\frac{3}{2}}}{5}$, $\frac{17\sqrt{3}}{5}$, \ldots{} Quindi li si rappresenti sul piano cartesiano.
\end{esercizio}

\begin{esercizio}
\label{ese:8a_progr.6}
Di una progressione aritmetica sono assegnati $a_1=3$ e la ragione $q=4$. Calcolare $a_9$ e $a_{15}$.
\end{esercizio}

\begin{esercizio}
\label{ese:8a_progr.7}
Di una progressione aritmetica sono assegnati $a_1=-\frac{3}{5}$ e $a_8=\frac{18}{5}$. Calcolare la ragione $q$.
\end{esercizio}

\begin{esercizio}
\label{ese:8a_progr.8}
Di una progressione aritmetica si conoscono $a_2=7$ e la ragione $q=\frac{3}{4}$. Calcolare $a_8$.
\end{esercizio}

\begin{esercizio}
\label{ese:8a_progr.9}
Di una progressione aritmetica si conoscono $a_3=\frac{3}{5}$ e $a_8=\frac{8}{5}$. Calcolare la ragione $q$ e $a_1$.
\end{esercizio}

\begin{esercizio}
\label{ese:8a_progr.10}
Di una progressione aritmetica sono dati $a_1=3$ e la ragione $q=\frac{1}{3}$. Calcolare la somma dei primi 12 termini $S_{12}$.
\end{esercizio}

\begin{esercizio}
\label{ese:8a_progr.11}
Di una progressione aritmetica sono dati $a_1=3\frac{\sqrt{3}}{4}$ e la ragione $q=\frac{\sqrt{2}}{2}$. Calcolare la somma dei primi 10 termini $S_{10}$.
\end{esercizio}

\begin{esercizio}
\label{ese:8a_progr.12}
Calcolare la somma dei primi 500 numeri naturali.
\end{esercizio}

\begin{esercizio}
\label{ese:8a_progr.13}
Calcolare la somma dei primi $\np{1000}$ numeri dispari.
\end{esercizio}

\begin{esercizio}
\label{ese:8a_progr.14}
Sommare i numeri pari minori di 700.
\end{esercizio}

\begin{esercizio}
\label{ese:8a_progr.15}
Considera la generica progressione aritmetica $\{a_n\} = a_1$, $a_2$, \ldots{} di ragione $q$. Se ad ogni termine aggiungiamo un valore costante $k$, quale sarà la ragione della nuova progressione ottenuta?
\end{esercizio}

\begin{esercizio}
\label{ese:8a_progr.16}
Considera la generica progressione aritmetica $\{a_n\} = a_1$, $a_2$, \ldots{} di ragione $q$. Se moltiplichiamo ogni termine per un valore costante $k$, quale sarà la ragione della nuova progressione ottenuta?
\end{esercizio}

\begin{esercizio}
\label{ese:8a_progr.17}
Calcola la ragione di una progressione aritmetica ottenuta sommando termine a termine gli elementi di due progressioni aritmetiche di ragione $q_1$ e $q_2$.
\end{esercizio}

\begin{esercizio}
\label{ese:8a_progr.18}
Calcola la ragione di una progressione aritmetica ottenuta sottraendo termine a termine gli elementi di due progressioni aritmetiche di ragione $q_1$ e $q_2$.
\end{esercizio}

\begin{esercizio}
\label{ese:8a_progr.19}
Di una progressione aritmetica sono dati $a_1=\frac{2}{3}$ e la ragione $q=\frac{1}{3}$. Si sa anche che la somma dei primi $n$ termini è $S_n=\frac{44}{3}$. Quanti sono gli $n$ termini considerati? Quanto vale l'ultimo termine considerato $a_n$?
\end{esercizio}

\begin{esercizio}
\label{ese:8a_progr.20}
Di una progressione aritmetica sono dati $a_5=14$, $a_n=38$ e la ragione $q=\frac{3}{2}$. Qual è la posizione $n$ dell'elemento $a_n$ nella successione?
\end{esercizio}

\begin{esercizio}
\label{ese:8a_progr.21}
Di una progressione aritmetica sono dati $a_2=4$ e la ragione $q=-1$. Calcola $a_{15}$.
\end{esercizio}

\begin{esercizio}
\label{ese:8a_progr.22}
Scrivi i primi 5 termini di una progressione geometrica con $a_1=1$ e ragione $q=2$. Quindi rappresentali sul piano cartesiano.
\end{esercizio}

\begin{esercizio}
\label{ese:8a_progr.23}
Scrivi i primi 5 termini di una progressione geometrica con $a_1=7$ e ragione $q=\frac{1}{2}$. Quindi rappresentali sul piano cartesiano.
\end{esercizio}

\begin{esercizio}
\label{ese:8a_progr.24}
Calcolare la ragione $q$ e scrivere altri 3 termini della seguente successione geometrica: 2, $\np{2,4}$, $\np{2,88}$, $\np{3,456}$, \ldots{} Quindi li si rappresenti sul piano cartesiano.
\end{esercizio}

\begin{esercizio}
\label{ese:8a_progr.25}
Calcolare la ragione $q$ e scrivere altri 3 termini della seguente successione geometrica: $\frac{1}{3}$, $\frac{7}{9}$, $\frac{49}{27}$, $\frac{343}{81}$, \ldots{} Quindi li si rappresenti sul piano cartesiano.
\end{esercizio}

\begin{esercizio}
\label{ese:8a_progr.26}
Calcolare la ragione $q$ e scrivere altri 3 termini della seguente successione geometrica: $\frac{25\sqrt{3}}{2}$, $\frac{5\sqrt{3}}{2}$, $\frac{\sqrt{3}}{2}$, \ldots{} Quindi li si rappresenti sul piano cartesiano.
\end{esercizio}

\begin{esercizio}
\label{ese:8a_progr.27}
Di una progressione geometrica sono assegnati $a_1=\frac{1}{2}$ e la ragione $q=\frac{3}{2}$. Calcolare $a_9$ e $a_{15}$.
\end{esercizio}

\begin{esercizio}
\label{ese:8a_progr.28}
Di una progressione geometrica sono assegnati $a_1=-\frac{9}{4}$ e $a_4=\frac{2}{3}$. Calcolare la ragione $q$.
\end{esercizio}

\begin{esercizio}
\label{ese:8a_progr.29}
Di una progressione geometrica si conoscono $a_2=7$ e la ragione $q=\frac{3}{4}$. Calcolare $a_8$.
\end{esercizio}

\begin{esercizio}
\label{ese:8a_progr.30}
Di una progressione geometrica si conoscono $a_3=\frac{3}{5}$ e $a_5=\frac{6}{5}$. Calcolare la ragione $q$ e $a_1$.
\end{esercizio}

\begin{esercizio}
\label{ese:8a_progr.31}
Di una progressione geometrica sono dati $a_1=3$ e la ragione $q=\frac{1}{3}$. Calcolare la somma dei primi 12 termini $S_{12}$.
\end{esercizio}

\begin{esercizio}
\label{ese:8a_progr.32}
Di una progressione geometrica sono dati $a_1=3\frac{\sqrt{3}}{4}$ e la ragione $q=\frac{\sqrt{2}}{2}$. Calcolare la somma dei primi 10 termini $S_{10}$.
\end{esercizio}

\begin{esercizio}
\label{ese:8a_progr.33}
Scrivere la formula per calcolare la somma dei primi $n$ elementi della progressione geometrica $\left\{\frac{1}{2^n}\right\}$.
\end{esercizio}

\subsection{Risposte}
\begin{multicols}{2}

\paragraph{\thechapter.6.} $\Dom=\insN_0$.

\paragraph{\thechapter.7.} $\Dom=\insN_0-\{2\}$.

\paragraph{\thechapter.8.} $\Dom=\insN_0$.

\paragraph{\thechapter.9.} $\Dom=\insN_0$.

\paragraph{\thechapter.15.} $\frac{3}{2}$.

\paragraph{\thechapter.16.} $1$.

\paragraph{\thechapter.17.} $0$.

\paragraph{\thechapter.18.} $-5$.

\paragraph{\thechapter.19.} $\nexists$.

\paragraph{\thechapter.20.} $\frac{5}{2}$.

\paragraph{\thechapter.21.} $1$.

\paragraph{\thechapter.22.} 3, 5, 7, 9, 11.

\paragraph{\thechapter.23.} $\frac{3}{2}$, 2, $\frac{5}{2}$, 3, $\frac{7}{2}$.

\paragraph{\thechapter.24.} $q=3$;\quad 8, 11, 14.

\paragraph{\thechapter.25.} $q=\frac{1}{5}$;\quad $\frac{8}{5}$, 2, $\frac{9}{5}$, 2.

\paragraph{\thechapter.26.} $q=1$;\quad $\frac{22\sqrt{3}}{5}$, $\frac{27\sqrt{3}}{5}$, $\frac{32\sqrt{3}}{5}$.

\paragraph{\thechapter.27.} $a_9=35$;\quad $a_{15}=59$.

\paragraph{\thechapter.28.} $q=\frac{3}{5}$.

\paragraph{\thechapter.29.} $a_8=\frac{23}{2}$.

\paragraph{\thechapter.30.} $q=\frac{1}{5}$;\quad $a_1=\frac{1}{5}$.

\paragraph{\thechapter.31.} $S_{12}=58$.

\paragraph{\thechapter.32.} $S_{10}=\frac{15}{2}\left(3\sqrt{2}+\sqrt{3}\right)$.

\paragraph{\thechapter.33.} $\np{125250}$.

\paragraph{\thechapter.34.} $\np{1000000}$.

\paragraph{\thechapter.35.} $\np{122150}$.

\paragraph{\thechapter.36.} $q$.

\paragraph{\thechapter.37.} $kq$.

\paragraph{\thechapter.38.} $q_1+q_2$.

\paragraph{\thechapter.39.} $q_1-q_2$.

\paragraph{\thechapter.40.} $n=8$;\quad $a_n=3$.

\paragraph{\thechapter.41.} $n=21$.

\paragraph{\thechapter.42.} $a_{15}=-9$.

\paragraph{\thechapter.43.} 1, 2, 4, 8, 16.

\paragraph{\thechapter.44.} 7, $\frac{7}{2}$, $\frac{7}{4}$, $\frac{7}{8}$, $\frac{7}{16}$.

\paragraph{\thechapter.45.} $q=\np{1,2}$;\quad $\np{4,1472}$, $\np{4,97664}$, $\np{5,971968}$.

\paragraph{\thechapter.46.} $q=\frac{7}{3}$;\quad $\frac{\np{2041}}{243}$, $\frac{\np{16807}}{729}$, $\frac{\np{117649}}{\np{2187}}$.

\paragraph{\thechapter.47.} $q=\frac{1}{5}$;\quad $\frac{\sqrt{3}}{10}$, $\frac{\sqrt{3}}{50}$, $\frac{\sqrt{3}}{250}$.

\paragraph{\thechapter.48.} $a_9=\frac{3^8}{2^9}$;\quad $a_{15}=\frac{3^{14}}{2^{15}}$.

\paragraph{\thechapter.49.} $q=\frac{2}{3}$.

\paragraph{\thechapter.50.} $a_8=\frac{7\cdot3^6}{2^{12}}$.

\paragraph{\thechapter.51.} $q=\sqrt{2}$;\quad $a_1=\frac{3}{10}$.

\paragraph{\thechapter.52.} $S_{12}=12\left(1-\frac{3^{12}}{2^{24}}\right)$.

\paragraph{\thechapter.53.} $S_{10}=\frac{3\sqrt{3}}{2^6\left(2-\sqrt{2}\right)}\left(2^5-1\right)$.

\paragraph{\thechapter.54.} $S_{n}=1-\frac{1}{2^n}$.

\end{multicols}

