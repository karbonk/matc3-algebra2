% (c)~2014 Claudio Carboncini - claudio.carboncini@gmail.com
% (c)~2014 Dimitrios Vrettos - d.vrettos@gmail.com
\section{Esercizi}
\subsection{Esercizi dei singoli paragrafi}
\subsubsection*{1.1 - Dai numeri naturali ai numeri irrazionali}

\begin{esercizio}
\label{ese:1.1}
Dimostra, con un ragionamento analogo a quello fatto per $\sqrt 2$, che $\sqrt 3$ non è razionale.
\end{esercizio}

\subsubsection*{1.2 - I numeri reali}


\begin{esercizio}
\label{ese:1.2}
Per ciascuno dei seguenti numeri reali scrivi una sequenza di sei numeri razionali che lo approssimano per difetto e sei numeri razionali che lo approssimano per eccesso. Esempio:
$\sqrt{3}$: $A=\{1$, $\np{1,7}$, $\np{1,73}$, $\np{1,732}$, $\np{1,7320}$, $\np{1,73205}\}$,$\quad B=\{2$, $\np{1,8}$, $\np{1,74}$, $\np{1,733}$, $\np{1,7321}$, $\np{1,73206}\}$.
\begin{enumeratea}
 \item$\sqrt{5}$:\, $A=\{\ldots\ldots\ldots\ldots\ldots\ldots\ldots\ldots\}$,$\quad B=\{\ldots\ldots\ldots\ldots\ldots\ldots\ldots\ldots\}$;
 \item$\dfrac{6}{7}$:\, $A=\{\ldots\ldots\ldots\ldots\ldots\ldots\ldots\ldots\}$,$\quad B=\{\ldots\ldots\ldots\ldots\ldots\ldots\ldots\ldots\}$;
 \item$\dfrac{1}{7}$:\, $A=\{\ldots\ldots\ldots\ldots\ldots\ldots\ldots\ldots\}$,$\quad B=\{\ldots\ldots\ldots\ldots\ldots\ldots\ldots\ldots\}$;
 \item$\sqrt{2}+\sqrt{3}$:\, $A=\{\ldots\ldots\ldots\ldots\ldots\ldots\ldots\ldots\}$,$\quad B=\{\ldots\ldots\ldots\ldots\ldots\ldots\ldots\ldots\}$;
 \item$\sqrt{2}\cdot\sqrt{3}$:\, $A=\{\ldots\ldots\ldots\ldots\ldots\ldots\ldots\ldots\}$,$\quad B=\{\ldots\ldots\ldots\ldots\ldots\ldots\ldots\ldots\}$.
 \end{enumeratea}
\end{esercizio}

\begin{esercizio}[\Ast]
\label{ese:1.3}
Determina per ciascuno dei seguenti numeri irrazionali i numeri interi tra i quali è compreso. Esempio: $5<\sqrt{30}<6$.
\begin{multicols}{3}
\begin{enumeratea}
 \item~$\sqrt{50}$;
 \item~$\sqrt{47}$;
 \item~$\sqrt{91}$;
 \item~$\sqrt{73}$;
 \item~$\sqrt{107}$;
 \item~$\sqrt{119}$;
 \item~$\sqrt{5}+\sqrt{3}$;
 \item~$2\sqrt{7}$;
 \item~$2+\sqrt{7}$;
 \item~$\sqrt{20}-\sqrt{10}$;
 \item~$\sqrt{\frac{7}{10}}$;
 \item~$7+\sqrt{\frac{1}{2}}$.
\end{enumeratea}
\end{multicols}
\end{esercizio}

\begin{esercizio}
\label{ese:1.4}
 Disponi in ordine crescente i seguenti numeri reali:
 \begin{enumeratea}
 \item $\sqrt{2}$,\quad $1$,\quad $\dfrac{2}{3}$,\quad $\np{2,0}\overline{13}$,\quad $\sqrt{5}$,\quad $\dfrac{3}{2}$,\quad $\np{0,75}$.
 \item $\pi$,\quad $\sqrt{3}$,\quad $\dfrac{11}{5}$,\quad $\np{0,}\overline{9}$,\quad $\sqrt{10}$,\quad $\np{3,1}\overline{4}$,\quad $\sqrt[3]{25}$.
 \end{enumeratea}
\end{esercizio}

\begin{esercizio}
\label{ese:1.5}
 Rappresenta con un diagramma di Eulero-Venn l'insieme dei numeri reali $\insR$, suddividilo nei seguenti sottoinsiemi: l'insieme dei numeri naturali $\insN$, l'insieme dei numeri interi relativi~$\insZ$, l'insieme dei numeri razionali $\insQ$, l'insieme $\insJ$ dei numeri irrazionali. Disponi in maniera opportuna i seguenti numeri: $\sqrt{3}$,\quad $\sqrt[3]{5}$,\quad$\pi$,\quad $\np{0,}\overline{3}$,\quad $\np{3,14}$,\quad $\frac{3}{2}$,\quad$-2$.
\end{esercizio}

%\newpage
\begin{esercizio}[\Ast]
\label{ese:1.6}
Indica il valore di verità delle seguenti affermazioni:
\begin{enumeratea}
\item un numero decimale finito è sempre un numero razionale;
\item un numero decimale illimitato è sempre un numero irrazionale;
\item un numero decimale periodico è un numero irrazionale;
\item la somma algebrica di due numeri razionali è sempre un numero razionale;
\item la somma algebrica di due numeri irrazionali è sempre un numero irrazionale;
\item il prodotto di due numeri razionali è sempre un numero razionale;
\item il prodotto di due numeri irrazionali è sempre un numero irrazionale.
\end{enumeratea}
\end{esercizio}

\subsubsection*{1.3 - Valore assoluto}
\begin{esercizio}[\Ast]
\label{ese:1.7}
 Calcola il valore assoluto dei seguenti numeri:
\begin{multicols}{3}
 \begin{enumeratea}
 \item~$\valass{-5}$
 \item~$\valass{+2}$
 \item~$\valass{-1}$
 \item~$\valass{0}$
 \item~$\valass{-10}$
 \item~$\valass{3-5\cdot(2)}$
 \item~$\valass{-3+5}$
 \item~$\left|{(-1)^3}\right|$
 \item~$\valass{-1-2-3}$
% \item~$\valass{3(-2)-5}$
 \end{enumeratea}
 \end{multicols}
\end{esercizio}

\begin{esercizio}
\label{ese:1.8}
Dati due numeri reali $x$ ed $y$ entrambi non nulli e di segno opposto, verifica le seguenti relazioni con gli esempi numerici riportati sotto.
Quali delle relazioni sono vere in alcuni casi e false in altri, quali sono sempre vere, quali sono sempre false?
\begin{center}
 \begin{tabular}{lcccc}
\toprule
Relazione & $x=-3$, $y=5$&$x=-2$, $y=2$ &$x=-10$, $y=1$&$x=1$, $y=-5$\\
\midrule
$\valass{x}<\valass{y}$& \boxV\qquad\boxF& \boxV\qquad\boxF&\boxV\qquad\boxF&\boxV\qquad\boxF\\
$\valass{x}=\valass{y}$& \boxV\qquad\boxF& \boxV\qquad\boxF&\boxV\qquad\boxF&\boxV\qquad\boxF\\
$\valass{x}<y$& \boxV\qquad\boxF& \boxV\qquad\boxF&\boxV\qquad\boxF&\boxV\qquad\boxF\\
$\valass{x+y}<\valass{x}+\valass{y}$& \boxV\qquad\boxF& \boxV\qquad\boxF&\boxV\qquad\boxF&\boxV\qquad\boxF\\
$\valass{x-y}=\valass{x}-\valass{y}$& \boxV\qquad\boxF& \boxV\qquad\boxF&\boxV\qquad\boxF&\boxV\qquad\boxF\\
$\left|{\valass{x}-\valass{y}}\right|=\valass{x-y}$& \boxV\qquad\boxF& \boxV\qquad\boxF&\boxV\qquad\boxF&\boxV\qquad\boxF\\
\bottomrule
\end{tabular}
\end{center}
\end{esercizio}

\begin{esercizio}[\Ast]
\label{ese:1.9}
 Elimina il valore assoluto sostituendo le espressioni con una funzione definita per casi:
 \begin{multicols}{2}
 \begin{enumeratea}
 \item~$f(x)=\left|x+1\right|$;
 \item~$f(x)=\left|x-1\right|$;
 \item~$f(x)=\left|x^2+1\right|$;
 \item~$f(x)=\left|(x+1)^2\right|$;
 \item~$f(x)=\left|x^2-1\right|$;
 \item~$f(x)=\left|x^3-1\right|$;
 \item~$f(x)=\left|x^2-6x+8\right|$;
 \item~$f(x)=\left|x^2+5x+4\right|$.
 \end{enumeratea}
 \end{multicols}
\end{esercizio}

\begin{esercizio}[\Ast]
\label{ese:1.10}
 Elimina il segno di valore assoluto dalle seguenti espressioni sostituendole con una funzione definita per casi:
 \begin{multicols}{2}
 \begin{enumeratea}
 \item~$f(x)=\dfrac{\left|x+1\right|}{\left|x+2\right|}$;
 \item~$f(x)=\left|\dfrac{x+1}{x-1}\right|$;
 \item~$f(x)=\left|x+1\right|+\left|x-2\right|$;
 \item~$f(x)=\left|x+2\right|+\left|x-2\right|$;
 \item~$f(x)=\left|x-2\right|+\left|x-3\right|$;
 \item~$f(x)=\left|x+1\right|\cdot \left|x+2\right|$;
 \item~$f(x)=\left|\dfrac{x+1} 4\right|+\left|\dfrac{x+2}{x+1}\right|$;
 \item~$f(x)=\left|\dfrac{x+1}{x+2}\right|+\left|\dfrac{x+2}{x+1}\right|$.
 \end{enumeratea}
 \end{multicols}
\end{esercizio}

\subsection{Risposte}

\paragraph{1.3.}
a)~$7<\sqrt{50}<8$,\quad g)~$3<\sqrt{5}+\sqrt{3}<4$,\quad h)~$5<2\sqrt{7}<6$,\quad i)~$4<2+\sqrt{7}<5$.

\paragraph{1.6.}
a)~V,\quad b)~F,\quad c)~F,\quad d)~V,\quad e)~V,\quad f)~F,\quad g)~F.

\paragraph{1.7.}
a)~5,\quad b)~0,\quad c)~2,\quad d)~2,\quad e)~10,\quad f)~1,\quad g)~1,\quad h)~7,\quad i)~6.

\paragraph{1.9.}
a)~${x+1}\text{ se }x\ge-1\text{; }-x-1\text{ se }x<-1$,\quad b)~${x-1}\text{ se }x\ge 1\text{; }1-x\text{ se }x<1$.

\paragraph{1.10.}
a)~$\frac{x+1}{x+2}\text{ se }x<-2 \vee x>-1\text{; }-\frac{x+1}{x+2}\text{ se }-2<x<-1\text{; }0\text{ se }x=-1\text{; senza significato se }x=-2$.
