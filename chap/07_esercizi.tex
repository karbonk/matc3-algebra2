% (c)~2014 Claudio Carboncini - claudio.carboncini@gmail.com
% (c)~2014 Dimitrios Vrettos - d.vrettos@gmail.com
\section{Esercizi}
\subsection{Esercizi dei singoli paragrafi}
\subsection*{7.1 - Valore assoluto}

\begin{esercizio}
 \label{ese:7.1}
Scrivi l'espressione algebrica che descrive i casi della funzione.
\begin{multicols}{3}
 \begin{enumeratea}
 \item~$ f(x)=\left|-2x+5\right| $;
 \item~$ f(x)=\left|x-1\right| $;
 \item~$ f(x)=\left|-x\right| $;
 \item~$ f(x)=\left|-x^2+4\right| $;
 \item~$ f(x)=\left|x^2+1\right| $;
 \item~$ f(x)=\left|x^2-3x+1\right| $.
 \end{enumeratea}
 \end{multicols}
\end{esercizio}

\begin{esercizio}
 \label{ese:7.2}
Scrivi l'espressione algebrica che descrive i casi della funzione.
\begin{multicols}{3}
 \begin{enumeratea}
 \item~$ f(a)=\left|2a-2\right| $;
 \item~$ f(p)=\left|3p^2-\frac 1 2\right| $;
 \item~$ f(a)=\left|-2a^2-1\right| $;
 \item~$ f(x)=\left|\frac 1{x-1}\right| $;
 \item~$ f(x)=\left|\frac{2x}{x-2}\right| $;
 \item~$ f(x)=\left|\frac{x+1}{2x-1}\right| $.
 \end{enumeratea}
 \end{multicols}
\end{esercizio}

\begin{esercizio}[\Ast]
 \label{ese:7.3}
Scrivi l'espressione algebrica che descrive i casi della funzione.
\begin{multicols}{2}
 \begin{enumeratea}
 \item~$ f(x)=\left|x+1\right|+\left|x-1\right| $;
 \item~$ f(x)=\left|3x-2\right|-\left|7x+1\right| $;
 \item~$ f(x)=-\left|x+2\right|+\left|x-2\right|-x $;
 \item~$f(x)=\left|x^2+1\right|-\left|x^2-1\right|$;
 \item~$f(x)=\left|\frac 1 x\right|-\left|x\right|$;
 \item~$f(x)=\left|\frac{x+2}{x-1}\right|+\left|x^2+4x+3\right|+1$.
 \end{enumeratea}
 \end{multicols}
\end{esercizio}

\subsection*{7.2 - Equazioni in una incognita in valore assoluto}

\begin{esercizio}[\Ast]
 \label{ese:7.4}
Risolvi le seguenti equazioni che hanno l'incognita solo nel valore assoluto.
\begin{multicols}{2}
 \begin{enumeratea}
 \item~$\left|x-2x^2\right|=1$;
 \item~$\left|-x^2-4\right|=9$;
 \item~$\left|x^2-x\right|=-3$;
 \item~$\left|x^2+1\right|=0$.
 \end{enumeratea}
 \end{multicols}
\end{esercizio}

\begin{esercizio}[\Ast]
 \label{ese:7.5}
Risolvi le seguenti equazioni che hanno l'incognita solo nel valore assoluto.
\begin{multicols}{2}
 \begin{enumeratea}
 \item~$\left|2x+1\right|=2$;
 \item~$\left|x^2-3x+1\right|=1$;
 \item~$\left|x^2+1\right|=3$;
 \item~$\left|x^2-1\right|=3$.
 \end{enumeratea}
 \end{multicols}
\end{esercizio}

\begin{esercizio}[\Ast]
 \label{ese:7.6}
Risolvi le seguenti equazioni che hanno l'incognita solo nel valore assoluto.
\begin{multicols}{2}
 \begin{enumeratea}
 \item~$\left|x^2-7\right|=3$;
 \item~ $6\left|x^2-1\right|=0$;
 \item~$\left|\frac 1 3-\frac 1{x^2}\right|=-1$;
 \item~$\left|\frac 1 3-\frac 1{x^2}\right|=1$.
 \end{enumeratea}
 \end{multicols}
\end{esercizio}

\begin{esercizio}[\Ast]
 \label{ese:7.7}
Risolvi le seguenti equazioni che hanno l'incognita solo nel valore assoluto.
\begin{multicols}{2}
 \begin{enumeratea}
 \item~$\frac 5{\left|x^2-1\right|}=1$;
 \item~$\left|\frac{x^2-5x+1}{2x^2+3x-1}\right|=1$;
 \item~$4\left|x^2-x\right|=1$;
 \item~$\frac 4{\left|x^2-x\right|}=1$.
 \end{enumeratea}
 \end{multicols}
\end{esercizio}
%\newpage
\pagebreak
\begin{esercizio}[\Ast]
 \label{ese:7.8}
Risolvi le seguenti equazioni con valore assoluto.
\begin{multicols}{2}
 \begin{enumeratea}
 \item~$\left|x-1\right|=x$;
 \item~$\left|x^2-4\right|=3x-1$;
 \item~$\left|2-x\right|=4-x^2$;
 \item~$\left|x^2+2\right|=1-x^2$.
 \end{enumeratea}
 \end{multicols}
\end{esercizio}

\begin{esercizio}[\Ast]
 \label{ese:7.9}
Risolvi le seguenti equazioni con valore assoluto.
\begin{multicols}{2}
 \begin{enumeratea}
 \item~$\left|-x^2+2x-3\right|=x+1$;
 \item~$\left|-x^2+4x-7\right|=3-2x$;
 \item~$\left|2-4x\right|=4(x-1)(x+2)$;
 \item~$\left|x^2-4x+3\right|=4x-6$.
 \end{enumeratea}
 \end{multicols}
\end{esercizio}

\begin{esercizio}[\Ast]
 \label{ese:7.10}
Risolvi le seguenti equazioni con valore assoluto.
\begin{multicols}{2}
 \begin{enumeratea}
 \item~$\left|1-2x\right|=5x-7$;
 \item~$\left|x^3-x^2\right|=x-1$;
 \item~$\left|x^2-3x+2\right|=x+1$;
 \item~$\left|x^2+1\right|=3+x$.
 \end{enumeratea}
 \end{multicols}
\end{esercizio}

\begin{esercizio}[\Ast]
 \label{ese:7.11}
Risolvi le seguenti equazioni con valore assoluto.
\begin{multicols}{2}
 \begin{enumeratea}
 \item~$\left|-x^2-4x-8\right|=3x-2-x^2$;
 \item~$\left|2x^2-3x\right|=-x$;
 \item~$\left|x^3-4x^2\right|=1-4x$;
 \item~$\left|x^4-3x^2\right|=x^2-2$.
 \end{enumeratea}
 \end{multicols}
\end{esercizio}

\begin{esercizio}[\Ast]
 \label{ese:7.12}
Risolvi le seguenti equazioni con valore assoluto.
\begin{multicols}{2}
 \begin{enumeratea}
 \item~$\left|x^4-5x^2\right|=5-x^2$;
 \item~$\left|9-x^2\right|=x^2-3x+4$;
 \item~$\left|x^2-2x-5\right|=4-\frac 1 4x^2$;
 \item~$\left|x^2-3x+2\right|=2x-4$.
 \end{enumeratea}
 \end{multicols}
\end{esercizio}

\begin{esercizio}[\Ast]
 \label{ese:7.13}
Risolvi le seguenti equazioni con valore assoluto.
\begin{multicols}{2}
 \begin{enumeratea}
 \item~$\left|x+5\right|=x^2-1$;
 \item~$\left|2x-6\right|=7-2x^2$;
 \item~$\left|x^2-4\right|=x+8$;
 \item~$\left|x^2+1\right|=5-x$.
 \end{enumeratea}
 \end{multicols}
\end{esercizio}

\begin{esercizio}[\Ast]
 \label{ese:7.14}
Risolvi le seguenti equazioni con valore assoluto.
\begin{multicols}{2}
 \begin{enumeratea}
 \item~$\left|x^4-x^2\right|=x^2+8$;
 \item~$\left|x^4-9\right|=x^2$;
 \item~$\left|1-x^2\right|=4x^2+x$;
 \item~$\left|x^2-3x+2\right|=2x-4$.
 \end{enumeratea}
 \end{multicols}
\end{esercizio}

\begin{esercizio}[\Ast]
 \label{ese:7.15}
Risolvi le seguenti equazioni con valore assoluto.
\begin{multicols}{2}
 \begin{enumeratea}
 \item~$\left|x^2-1\right|=x^2-1$;
 \item~$\left|x^2-5x+6\right|=3x^2-x$;
 \item~$\left|x^2-3\right|=x^2-6x+9$;
 \item~$\left|1-3x\right|=\frac{(x-3)^2}{1-2x}$;
 \item~$\left|\frac{1-3x}{1-2x}\right|=\frac{x^2-3x+2}{1-2x}$.
 \end{enumeratea}
 \end{multicols}
\end{esercizio}
%\newpage
\subsection*{7.3 - Equazioni con più espressioni in valore assoluto}

\begin{esercizio}[\Ast]
 \label{ese:7.16}
Risolvi le seguenti equazioni con più valori assoluti.
\begin{multicols}{2}
 \begin{enumeratea}
 \item~$\left|x-2\right|+\left|5-2x\right|=x-1$;
 \item~$\left|x^2-4x+3\right|=1-2\left|4-x^2\right|$;
 \item~$\left|x-1\right|=x^2-x+\left|3-x^2\right|$;
 \item~$\left|3x-2\right|=x^2-\left|x^2-x\right|+3$.
 \end{enumeratea}
 \end{multicols}
\end{esercizio}
\pagebreak
\begin{esercizio}[\Ast]
 \label{ese:7.17}
Risolvi le seguenti equazioni con più valori assoluti.
\begin{multicols}{2}
 \begin{enumeratea}
 \item~$\left|3x-x^2-2\right|=\frac 1 2+x^2-x-2\left|1-x^2\right|$;
 \item~$\left|2x-5\right|+\left|x^2-1\right|=x-2$;
 \item~$\left|x-2\right|=\left|x^2-4\right|$;
 \item~$\left|x-2\right|=\left|x^2-4\right|+1$.
 \end{enumeratea}
 \end{multicols}
\end{esercizio}

\begin{esercizio}[\Ast]
 \label{ese:7.18}
Risolvi le seguenti equazioni con più valori assoluti.
\begin{multicols}{2}
 \begin{enumeratea}
 \item~$\left|x-2\right|=\left|x^2-4\right|+4$;
 \item~$\left|x-2\right|=\left|x^2-4\right|+5$;
 \item~$\left|x^2-3x\right|=x\left|x\right|$;
 \item~$\left|x-1\right|(x+1)=\left|2x-4\right|$.
 \end{enumeratea}
 \end{multicols}
\end{esercizio}

\begin{esercizio}[\Ast]
 \label{ese:7.19}
Risolvi le seguenti equazioni con più valori assoluti.
\begin{multicols}{2}
 \begin{enumeratea}
 \item~$\left|x^2-5x+6\right|=(3-x)\left|x^2+x-2\right|$;
 \item~$\left|x\right|^2-\left|x\right|=2$;
 \item~$\left|x\right|^2+3\left|x\right|+2=0$;
 \item~$\left|x\right|^2-5\left|x\right|+6=0$.
 \end{enumeratea}
 \end{multicols}
\end{esercizio}

\begin{esercizio}[\Ast]
 \label{ese:7.20}
Risolvi le seguenti equazioni con più valori assoluti.
\begin{multicols}{2}
 \begin{enumeratea}
 \item~$\left|4x-x^2\right|-2x=2\left|x^2-9\right|$;
 \item~$(x-1)^2\left|x\right|=x^2-1$;
 \item~$\left|x\right|=3x-\left|x^2-1\right|$;
 \item~$\left|x-2\right|+\left|x\right|=1+x^2$.
 \end{enumeratea}
 \end{multicols}
\end{esercizio}

\begin{esercizio}[\Ast]
 \label{ese:7.21}
Risolvi le seguenti equazioni con più valori assoluti.
\begin{multicols}{2}
 \begin{enumeratea}
 \item~$\left|3x-6\right|+\left|4x-x^2\right|=x+3$;
 \item~$\left|x^2-4\right|+1-2x=2x^2+\left|x+2\right|$;
 \item~$x+\left|x^2+x-6\right|=\frac 1 4(x^2+10x+25)$;
 \item~$x+2\left|-x-1\right|=x^2-\left|x\right|$.
 \end{enumeratea}
 \end{multicols}
\end{esercizio}

\begin{esercizio}[\Ast]
 \label{ese:7.22}
Risolvi le seguenti equazioni con più valori assoluti.
\begin{multicols}{2}
 \begin{enumeratea}
 \item~$\left|x^3-4x\right|=\left|x\right|$;
 \item~$\left|x-2\right|=\left|x^2-4\right|-4$;
 \item~$\left|x-2\right|=\left|x^2-4\right|-\frac 9 4$;
 \item~$\left|x^2-4x\right|=\left|2x^2-3\right|$.
 \end{enumeratea}
 \end{multicols}
\end{esercizio}

\begin{esercizio}[\Ast]
 \label{ese:7.23}
Risolvi le seguenti equazioni con più valori assoluti.
\begin{multicols}{2}
 \begin{enumeratea}
 \item~$\left|x-1\right|^2-\left|x^2-1\right|=1$;
 \item~$\left|9-4x^2\right|=x^2+2\left|x-3\right|$;
 \item~$(\left|x-1\right|-\left|3x-3\right|)^2=0$;
 \item~$\left(\left|x\right|-2\left|17-x^2\right|\right)^3=8$.
 \end{enumeratea}
 \end{multicols}
\end{esercizio}

\begin{esercizio}[\Ast]
 \label{ese:7.24}
Risolvi le seguenti equazioni con più valori assoluti.
\begin{multicols}{2}
 \begin{enumeratea}
 \item~$\left(\left|2x-1\right|-1\right)\left(6-2\left|x^2-9\right|\right)=0$;
 \item~$\left|x-2\right|\left(1-\left|x-1\right|\right)=\frac 1 4$;
 \item~$\frac{\left|x-1\right|+3\left|4x+x^2+3\right|} 2=2$;
 \item~$\left|x-1\right|-\left|x+1\right|=1$.
 \end{enumeratea}
 \end{multicols}
\end{esercizio}
%\newpage
\begin{esercizio}[\Ast]
 \label{ese:7.25}
Risolvi le seguenti equazioni con più valori assoluti.
\begin{multicols}{2}
 \begin{enumeratea}
 \item~$\left|4x^2-4\right|-2\left|x+1\right|=0$;
 \item~$\left|x-4\right|=\left|(x-1)^2-1\right|$;
 \item~$\left|3x^2-\frac 1 2\right|-x=\left|x-1\right|$;
 \item~$(x-1)\left|4-2x\right|=x^2-2$.
 \end{enumeratea}
 \end{multicols}
\end{esercizio}
\pagebreak
\begin{esercizio}[\Ast]
 \label{ese:7.26}
Risolvi le seguenti equazioni con più valori assoluti.
\begin{multicols}{2}
 \begin{enumeratea}
 \item~$(x-1)\left|4-2x\right|=x^2-1$;
 \item~$(x-1)\left|4-2x\right|=x^2+1$;
 \item~$x^2\left|2x+2\right|=4\left|x\right|$;
 \item~$\left|x-2\right|-\left|1-x\right|=(x-1)^2$.
 \end{enumeratea}
 \end{multicols}
\end{esercizio}

\begin{esercizio}[\Ast]
 \label{ese:7.27}
Risolvi le seguenti equazioni con più valori assoluti.
\begin{multicols}{2}
 \begin{enumeratea}
 \item~$2\left|x^2-9\right|+6\left|4x+12\right|=0$;
 \item~$\left|x-2\right|+\left|x\right|=1-x^2$;
 \item~$\left|x-2\right|=\left|x^2-4\right|-2$;
 \item~$\left|5x-x^2\right|=3+2x-\left|x\right|$.
 \end{enumeratea}
 \end{multicols}
\end{esercizio}

\begin{esercizio}[\Ast]
 \label{ese:7.28}
Risolvi le seguenti equazioni con più valori assoluti.
\begin{multicols}{2}
 \begin{enumeratea}
 \item~$2\left|4-x^2\right|=\left|x^2-2x+3\right|$;
 \item~$\left|3-3x\right|+x=8-2\left|16-4x^2\right|$;
 \item~$\left|x-1\right|=\frac 2 {\left|x+1\right|}-1$;
 \item~ $\left|x^2-x\right|+3x=\left|x-1\right|-\left|2x+1\right|$.
 \end{enumeratea}
 \end{multicols}
\end{esercizio}

\begin{esercizio}[\Ast]
 \label{ese:7.29}
Risolvi le seguenti equazioni con più valori assoluti.
 \begin{enumeratea}
\item~$\left|x^2-5x+6\right|+3x+2=\left|x^2-1\right|-\left|x+1\right|$;
\item~$\left(\left|x-3\right|+1\right)^2=\left|x^2+1\right|-3\left|x\right|-\left|-5\right|$.
 \end{enumeratea}
\end{esercizio}

\subsection*{7.4 - Disequazioni in valore assoluto}

\begin{esercizio}[\Ast]
 \label{ese:7.30}
Risolvi le seguenti disequazioni in valore assoluto.
\begin{multicols}{2}
 \begin{enumeratea}
 \item~$\left|x+1\right|<1$;
 \item~$\left|x^2-3x+3\right|<3$;
 \item~$\left|3x^2-1\right|>6$;
 \item~$5-\left|5-x^2\right|\ge 6$.
 \end{enumeratea}
 \end{multicols}
\end{esercizio}

\begin{esercizio}[\Ast]
 \label{ese:7.31}
Risolvi le seguenti disequazioni in valore assoluto.
\begin{multicols}{2}
 \begin{enumeratea}
 \item~$\left|9-16x^2\right|>0$;
 \item~$\left|x^2+6x\right|>2$;
 \item~$\left|5x-x^2\right|>6$;
 \item~$\left|\frac 2 x+\frac 1 3\right|-\frac 1 2>2$.
 \end{enumeratea}
 \end{multicols}
\end{esercizio}

\begin{esercizio}[\Ast]
 \label{ese:7.32}
Risolvi le seguenti disequazioni in valore assoluto.
\begin{multicols}{2}
 \begin{enumeratea}
 \item~$\left|x^2-3\right|\ge \left|-4\right|$;
 \item~$2x^2-7x+3>\left|x^2-2x\right|$;
 \item~$\left|\frac x{x-1}\right|>1$;
 \item~$\left|x^2+3\right|<\left|5-2x\right|$.
 \end{enumeratea}
 \end{multicols}
\end{esercizio}

\begin{esercizio}[\Ast]
 \label{ese:7.33}
Risolvi le seguenti disequazioni in valore assoluto.
\begin{multicols}{2}
 \begin{enumeratea}
 \item~$\left|x^2-1\right|+3x\ge 2\left(x+\left|x^2-1\right|\right)$;
 \item~$\left|x-2\right|+2\left|2-x\right|>(x-2)(x+2)$;
 \item~$\frac{\left|x\right|}{x-1}<\frac{x+1}{\left|x\right|}$;
 \item~$(x-3)^2-\left|x^2-4\right|<10+(x+1)(x-1)-6x$.
 \end{enumeratea}
 \end{multicols}
\end{esercizio}
\pagebreak
\begin{esercizio}[\Ast]
 \label{ese:7.34}
Risolvi le seguenti disequazioni in valore assoluto.
\begin{multicols}{2}
 \begin{enumeratea}
 \item~$\left|\frac{3x+2}{2x}\right|\le 1$;
 \item~$\left|x-1\right|>\left|2x-1\right|$;
 \item~$\left|x^2-4\right|<\left|x^2-2x\right|$;
 \item~$\left|x+1\right|<\left|x\right|-x$.
 \end{enumeratea}
 \end{multicols}
\end{esercizio}

\begin{esercizio}[\Ast]
 \label{ese:7.35}
Risolvi le seguenti disequazioni in valore assoluto.
\begin{multicols}{2}
 \begin{enumeratea}
 \item~$\left|x-1\right|+3x-\left|3-x\right|+1<0$;
 \item~$\left|x-1\right|\ge 4-\left|2x-3\right|$;
 \item~$1-\left|4x^2-1\right|+5x<\left|x^2-9\right|+3x^2$;
 \item~$\left|x^2-x\right|-2\le \left|x-2\right|$.
 \end{enumeratea}
 \end{multicols}
\end{esercizio}

\begin{esercizio}[\Ast]
 \label{ese:7.36}
Risolvi le seguenti disequazioni in valore assoluto.
\begin{multicols}{2}
 \begin{enumeratea}
 \item~$\frac{\left|x-3\right|-x^2}{\left|x-1\right|}\le 1-\left|x-1\right|$;
 \item~$(x-1)^2-\left|x-3\right|<\left|x+5\right|(x-5)+2x$;
 \item~$\left\{\begin{array}{l}{\left|2x-3\right|<6}\\{\left|x^2-2x\right|>3}\end{array}\right.$;
 \item~$\frac{\left|x^2-1\right|-5}{\left|x-1\right|(x^2-x-6)}\ge 0$.
 \end{enumeratea}
 \end{multicols}
\end{esercizio}

\subsection{Risposte}
\paragraph{7.3.} d)~se $x\le -1\vee x\ge 1\to f(x)=2 $; se $-1<x<1 \to f(x)=2x^2$,\protect\\
\quad e)~se $x\ge 0 \to f(x)=\frac 1 x-x$; se $x<0 \to f(x)=x-\frac 1 x$,\protect\\
\quad f)~se $x<-3\vee x>1 \to f(x)=\frac{x+2}{x-1}+x^2+4x+3+1$;\protect\\ se $-3\le x<-2 \to f(x)=\frac{x+2}{x-1}-(x^2+4x+3)+1$;\protect\\ se $-2\le x<-1\to f(x)=-\frac{x+2}{x-1}-(x^2+4x+3)+1$;\protect\\ se $-1\le x<1\to f(x)=-\frac{x+2}{x-1}+x^2+4x+3+1$.

\paragraph{7.4.} a)~$x_1=1\vee x_2=-\frac 1 2$,\quad b)~$x_1=\sqrt 5\vee x_2=-\sqrt 5$,\quad c)~$\emptyset $,\quad d)~$\emptyset $.

\paragraph{7.5.} a)~$x_1=-\frac 3 2\vee x_2=\frac 1 2$,\quad b)~$x_1=0\vee x_2=1\vee x_3=2\vee x_4=3$,\quad c)~$x_{1\text{,}2}=\pm \sqrt 2$,\protect\\
\quad d)~$x_{1\text{,}2}=\pm 2$.

\paragraph{7.6.} a)~$x_{1\text{,}2}=\pm \sqrt{10}\vee x_{3\text{,}4}=\pm 2$,\quad b)~$x_{1\text{,}2}=\pm 1$,\quad c)~$\emptyset $,\quad d)~$x_{1\text{,}2}=\frac{\pm \sqrt 3} 2$.

\paragraph{7.7.} a)~$x_{1\text{,}2}=\pm \sqrt 6$,\quad b)~$x_1=0\vee x_2=\frac 2 3\vee x_{3\text{,}4}=-4\pm 3\sqrt 2$,\quad c)~$x_{1\text{,}2}=\frac{1\pm \sqrt 2} 2\vee x_3=\frac 1 2$,\quad d)~$x_{1\text{,}2}=\frac{1\pm \sqrt{17}} 2$.

\paragraph{7.8.} a)~$x=\frac 1 2$,\quad b)~$x_1=\frac{3+\sqrt{21}} 2\vee x_2=\frac{-3+\sqrt{29}} 2$,\quad c)~$x_1=-1\vee x_2=2$,\quad d)~$\emptyset $.

\paragraph{7.9.} a)~$x_1=1\vee x_2=2$,\quad b)~$\emptyset $,\quad c)~$x_1=-\frac{2+\sqrt{14}} 2\vee x_2=\frac{\sqrt 6} 2$,\quad d)~$x_1=\sqrt 3\vee x_2=4+\sqrt 7$.

\paragraph{7.10.} a)~$x=2$,\quad b)~$x=1$,\quad c)~$x_{1\text{,}2}=2\pm \sqrt 3$,\quad d)~$x_1=-1\vee x_2=2$.

\paragraph{7.11.} a)~$\emptyset $,\quad b)~$ x=0 $,\quad c)~$x_1=-1\vee x_2=\frac{5-\sqrt{21}} 2$,\quad d)~$x_{1\text{,}2}=\pm \sqrt{2+\sqrt 2}\vee x_{3\text{,}4}=\pm \sqrt{1+\sqrt 3}$.

\paragraph{7.12.} a)~$x_{1\text{,}2}=\pm 1\vee x_{3.4}=\pm \sqrt 5$,\quad b)~$x_1=-1\vee x_2=\frac{13} 3\vee x_3=\frac 5 2$,\protect\\
\quad c)~$x_1=\frac{18} 5\vee x_2=-2\vee x_3=\frac{4\pm 2\sqrt 7} 3$,\quad d)~$x_1=2\vee x_2=3$.

\paragraph{7.13.} a)~$x_1=-2\vee x_2=3$,\quad b)~$x_{1\text{,}2}=\frac{1\pm \sqrt 3} 2$,\quad c)~$x_1=-3\vee x_2=4$,\quad d)~$x_{1\text{,}2}=\frac{-1\pm \sqrt{17}} 2$.

\paragraph{7.14.} a)~$x_{1\text{,}2}=\pm 2$,\quad b)~$x_{1\text{,}2}=\frac{\pm \sqrt{2+2\sqrt{37}}} 2\vee x_{3\text{,}4}\pm \frac{\sqrt{-2+2\sqrt{37}}} 2$,\quad c)~$x_{1\text{,}2}=\frac{-1\pm \sqrt{21}}{10}$,\protect\\
\quad d)~$x_1=2\vee x_2=3$.

\paragraph{7.15.} a)~$x\le -1\vee x\ge 1$,\quad b)~$x_1=-3\vee x_2=1$,\quad c)~$x=2$,\quad d)~$x=-\frac{\sqrt{161}+1}{10}$,\quad e)~$\emptyset $.

\paragraph{7.16.} a)~$x_1=2\vee x_2=3$,\quad b)~$x=2$,\quad c)~$\emptyset $,\quad d)~$x_1=\frac 5 2\vee x_2=-\frac 1 4$.

\paragraph{7.17.} a)~$x_1=\frac 9 8\vee x_2=\sqrt 2-\frac 1 2$,\quad b)~$\emptyset $,\quad c)~$x_1=-3\vee x_2=-1\vee x_3=2$,\quad d)~$x_1=-\frac{1+\sqrt{21}} 2\vee x_2=\frac{1-\sqrt{13}} 2$.

\paragraph{7.18.} a)~$x=-2$,\quad b)~$\emptyset $,\quad c)~$x_1=0\vee x_2=\frac 3 2$,\quad d)~$x=\sqrt 6-1$.

\paragraph{7.19.} a)~$x_1=0\vee x_2=3 \vee x_{3\text{,}4}=-1\pm \sqrt 5$,\quad b)~$x_{1\text{,}2}=\pm 2$,\quad c)~$\emptyset $,\quad d)~$x_{1\text{,}2}=\pm 2\vee x_{3\text{,}4}=\pm 3$.

\paragraph{7.20.} a)~$x_1=-3-3\sqrt 3\vee x_2=1-\sqrt 7$,\quad b)~$x_1=1\vee x_2=1+\sqrt 2$,\quad c)~$x_{1\text{,}2}=\sqrt 2\pm 1$,\quad d)~$x_1=1\vee x_2=-1-\sqrt 2$.

\paragraph{7.21.} a)~$x_1=3\vee x_2=\sqrt 3\vee x_3=1+\sqrt{10}$,\quad b)~$x_{1\text{,}2}=\frac{-1\pm \sqrt 5} 2$,\protect\\
\quad c)~$x_{1\text{,}2}=\frac{-5\pm 2\sqrt 5} 5\vee x_{3\text{,}4}=\frac{1\pm 2\sqrt{37}} 3$,\quad d)~$x_1=1-\sqrt 3\vee x_2=2+\sqrt 6$.

\paragraph{7.22.} a)~$x_{1\text{,}2}=\pm \sqrt 5\vee x_{3\text{,}4}=\pm \sqrt 3\vee x=0$,\quad b)~$x_1=3\vee x_2=-\frac{1+\sqrt{41}} 2$,\protect\\
\quad c)~$x_1=\frac 1 2\vee x_2=\frac{1+3\sqrt 2} 2\vee x_3=-\frac{1+\sqrt{34}} 2$,\quad d)~$x_{1\text{,}2}=-2\pm \sqrt 7\vee x_{3\text{,}4}=\frac{2\pm \sqrt{13}} 3$.

\paragraph{7.23.} a)~$x=\frac{1-\sqrt 3} 2$,\quad b)~$x_1=1\vee x_2=-\frac 3 5\vee x_{3\text{,}4}=\frac{-1\pm \sqrt{46}} 3$,\quad c)~$x=1$,\protect\\
\quad d)~$x_{1\text{,}2}=\pm 4\vee x_{3\text{,}4}=\frac{\pm 1+\sqrt{257}} 4$.

\paragraph{7.24.} a)~$x_1=0\vee x_2=1\vee x_{3\text{,}4}=\pm \sqrt 6\vee x_{5\text{,}6}=\pm 2\sqrt 3$,\quad b)~$x_1=\frac 3 2\vee x_2=\frac{2-\sqrt 3} 2$,\protect\\
\quad c)~$x_1=-3\vee x_2=-\frac 4 3\vee x_3=-\frac 2 3$,\quad d)~$x=-\frac 1 2$.

\paragraph{7.25.} a)~$x_1=-1\vee x_2=\frac 1 2\vee x_3=\frac 3 2$,\quad b)~$x_{1\text{,}2}=\frac{1\pm \sqrt{17}} 2$,\quad c)~$x_{1\text{,}2}=\pm \frac{\sqrt 2} 2$,\protect\\
\quad d)~$x_1=3+\sqrt 3\vee x_{2\text{,}3}=\frac{3\pm \sqrt 3} 3$.

\paragraph{7.26.} a)~$x_1=1\vee x_2=5$,\quad b)~$x=3+\sqrt 6$,\quad c)~$x_1=-2\vee x_2=0\vee x_3=1$,\protect\\
\quad d)~$x_1=0\vee x_2=\sqrt 2$.

\paragraph{7.27.} a)~$x=-3$,\quad b)~$\emptyset $,\quad c)~$x_1=0\vee x_2=1\vee x_3=\frac{1+\sqrt{17}} 2\vee x_4=-\frac{1+\sqrt{33}} 2$,\protect\\
\quad d)~$x_1=1\vee x_2=3\vee x_3=2\sqrt 3+3\vee x_4=4-\sqrt{19}$.

\paragraph{7.28.} a)~$x_1=-1\vee x_2=\frac 5 3\vee x_{3\text{,}4}=-1\pm 2\sqrt 3$,\quad b)~$x_1=\frac{-1+\sqrt{87}} 4\vee x_2=\frac{1+\sqrt{43}} 4\vee x_3=\frac{1-3\sqrt{33}} 8\vee x_4=\frac{-1-\sqrt{217}} 8$,\quad c)~$x_1=0\vee x_2=1\vee x_3=\frac{1-\sqrt{17}} 2$,\quad d)~$x_1=0\vee x_2=-2$.

\paragraph{7.29.} a)~$x=10$,\quad b)~$x=8$.

\paragraph{7.30.} a)~$ -2<x<0 $,\quad b)~$ 0<x<3 $,\quad c)~$x<-\frac{\sqrt{21}} 3\vee x>\frac{\sqrt{21}} 3$,\quad d)~$\emptyset $.

\paragraph{7.31.} a)~$x\neq \pm \frac 3 4$,\quad b)~$x<-3-\sqrt{11}\vee -3-\sqrt 7<x<-3+\sqrt 7\vee x>-3+\sqrt{11}$,\protect\\
\quad c)~$x<-1\vee 2<x<3\vee x>6$,\quad d)~$-\frac{12}{17}<x<\frac{12}{13}\wedge x\neq 0$.

\paragraph{7.32.} a)~$x\le -\sqrt 7\vee x\ge \sqrt 7$,\quad b)~$x<\frac{3-\sqrt 5} 2\vee x>\frac{\sqrt{13}+5} 2$,\quad c)~$x>\frac 1 2\wedge x\neq 1$,\protect\\
\quad d)~$-1-\sqrt 3<x<-1+\sqrt 3$.

\paragraph{7.33.} a)~$\frac{\sqrt 5-1} 2\le x\le \frac{\sqrt 5+1} 2$,\quad b)~$-5<x<2$,\quad c)~$x<1$,\quad d)~ $ \forall x \in \insR -\{-2\text{,}2\} $.

\paragraph{7.34.} a)~$-2\le x\le -\frac 2 5$,\quad b)~$0<x<\frac 2 3$,\quad c)~$x<-1$,\quad d)~$x<-\frac 1 3$.

\paragraph{7.35.} a)~ $x<\frac 1 3$,\quad b)~$x\le 0\vee x\ge \frac 8 3$,\quad c)~$\forall x\in R$,\quad d)~$-2\le x\le 2$.

\paragraph{7.36.} a)~$x\ge \frac 5 4$,\quad b)~$x>\frac{29} 5$,\quad c)~$-\frac 3 2<x<-1\vee 3<x<\frac 9 2$,\protect\\
\quad d)~$(x\le -\sqrt 6\vee -2<x\le \sqrt 6\vee x>3)\wedge x\neq -1$.
