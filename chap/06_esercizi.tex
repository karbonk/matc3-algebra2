% (c)~2014 Claudio Carboncini - claudio.carboncini@gmail.com
% (c)~2014 Dimitrios Vrettos - d.vrettos@gmail.com
\section{Esercizi}
\subsection{Esercizi dei singoli paragrafi}
\subsection*{6.1 - Sistemi di secondo grado}

\begin{esercizio}[\Ast]
 \label{ese:6.1}
Risolvere i seguenti sistemi di secondo grado.
\begin{multicols}{2}
 \begin{enumeratea}
 \item~$\left\{\begin{array}{l}x^2+2y^2=3\\x+y=2\end{array}\right.$;
 \item~$\left\{\begin{array}{l}3x^2-4y^2-x=0\\x-2y=1\end{array}\right.$;
 \item~$\left\{\begin{array}{l}4x^2+2y^2-6=0\\x=y\end{array}\right.$;
 \item~$\left\{\begin{array}{l}2x^2-6xy=x\\3x+5y=-2\end{array}\right.$.
 \end{enumeratea}
 \end{multicols}
\end{esercizio}

\begin{esercizio}[\Ast]
 \label{ese:6.2}
Risolvere i seguenti sistemi di secondo grado.
\begin{multicols}{2}
 \begin{enumeratea}
 \item~$\left\{\begin{array}{l}x-2y=1\\x^{2}+y^{2}=2\end{array}\right.$;
 \item~$\left\{\begin{array}{l}x^2+xy=0\\3x+4y=8\end{array}\right.$;
 \item~$\left\{\begin{array}{l}x^2+2xy=25\\x+2y=5\end{array}\right.$;
 \item~$\left\{\begin{array}{l}y+3x=-1\\y^{2}-3x^{2}=1\end{array}\right.$.
 \end{enumeratea}
 \end{multicols}
\end{esercizio}

\begin{esercizio}[\Ast]
\label{ese:6.3}
Risolvere i seguenti sistemi di secondo grado.
\begin{multicols}{2}
 \begin{enumeratea}
 \item~$\left\{\begin{array}{l}y^2-3y=2xy\\y=x-3\end{array}\right.$;
 \item~$\left\{\begin{array}{l}xy-x^2+2y^2=y-2x\\x+y=0\end{array}\right.$;
 \item~$\left\{\begin{array}{l}{x+2y=-1}\\{x+5y^2=23}\end{array}\right.$;
 \item~$\left\{\begin{array}{l}{x-5y=2}\\{x^2+2y^2=4}\end{array}\right.$.
 \end{enumeratea}
 \end{multicols}
\end{esercizio}

\begin{esercizio}[\Ast]
 \label{ese:6.4}
Risolvere i seguenti sistemi di secondo grado.
\begin{multicols}{2}
 \begin{enumeratea}
 \item~$\left\{\begin{array}{l}x^2-2xy+y^2=36\\2x-y=2\end{array}\right.$;
 \item~$\left\{\begin{array}{l}x^2-4x+y^2-2y=0\\2x-y=3\end{array}\right.$;
 \item~$\left\{\begin{array}{l}(x+y)^2+y(1-2x)=18-x\\x-y=1\end{array}\right.$;
 \item~$\left\{\begin{array}{l}3(x-1)-2y=0\\4y^{2}+9x^{2}=5\end{array}\right.$.
 \end{enumeratea}
 \end{multicols}
\end{esercizio}

\begin{esercizio}[\Ast]
 \label{ese:6.5}
Risolvere i seguenti sistemi di secondo grado.
\begin{multicols}{2}
 \begin{enumeratea}
 \item~$\left\{\begin{array}{l}3x-y=2\\x^2+2xy+y^2=0\end{array}\right.$;
 \item~$\left\{\begin{array}{l}x^2-4xy+4y^2-1=0\\x=y+2\end{array}\right.$;
 \item~$\left\{\begin{array}{l}x^2+y^2=1\\x+3y=10\end{array}\right.$;
 \item~$\left\{\begin{array}{l}x^2+y^2=2\\x+y=2\end{array}\right.$.
 \end{enumeratea}
 \end{multicols}
\end{esercizio}

\begin{esercizio}[\Ast]
 \label{ese:6.6}
Risolvere i seguenti sistemi di secondo grado.
\begin{multicols}{2}
 \begin{enumeratea}
 \item~$\left\{\begin{array}{l}x+y=1\\x^2+y^2-3x+2y=3\end{array}\right.$;
 \item~$\left\{\begin{array}{l}3x+y=2\\x^2-y^2=1\end{array}\right.$;
 \item~$\left\{\begin{array}{l}5x^2-y^2+4y-2x+2=0\\x-y=1\end{array}\right.$;
 \item~$\left\{\begin{array}{l}x^2+y^2=25\\4x-3y+7=0\end{array}\right.$.
 \end{enumeratea}
 \end{multicols}
\end{esercizio}
\pagebreak
\begin{esercizio}[\Ast]
\label{ese:6.7}
Risolvere i seguenti sistemi di secondo grado.
\begin{multicols}{2}
 \begin{enumeratea}
 \item~$\left\{\begin{array}{l}x+2y=3\\x^2-4xy+2y^2+x+y-1=0\end{array}\right.$;
 \item~$\left\{\begin{array}{l}x^2-4xy+4y^2-1=0\\x=y+2\end{array}\right.$;
 \item~$\left\{\begin{array}{l}2x^2+xy-7x-2y=-6\\2x+y=3\end{array}\right.$;
 \item~$\left\{\begin{array}{l}x-2y-7=0\\x^2-xy=4\end{array}\right.$.
 \end{enumeratea}
 \end{multicols}
\end{esercizio}

\begin{esercizio}[\Ast]
\label{ese:6.8}
Risolvere i seguenti sistemi di secondo grado.
\begin{multicols}{2}
 \begin{enumeratea}
 \item~$\left\{\begin{array}{l}x+y=0\\x^2+y^2-x-10=0\end{array}\right.$;
 \item~$\left\{\begin{array}{l}x^2+2y^2-3xy-x+2y-4=0\\2x-3y+4=0\end{array}\right.$;
 \item~$\left\{\begin{array}{l}x^2-4y^2=0\\4x-7y=2\end{array}\right.$;
 \item~$\left\{\begin{array}{l}x-2y=1\\x^2+y^2-2x=1\end{array}\right.$.
 \end{enumeratea}
 \end{multicols}
\end{esercizio}

\begin{esercizio}[\Ast]
\label{ese:6.9}
Risolvere i seguenti sistemi di secondo grado.
\begin{multicols}{2}
 \begin{enumeratea}
 \item~$\left\{\begin{array}{l}\dfrac{1}{4}x^{2}+2x-2y^{2}+2y=1\\4x-3y=-4\end{array}\right.$;
 \item~$\left\{\begin{array}{l}x^2+xy-y^2+5x=24\\x-y=-2\end{array}\right.$;
 \item~$\left\{\begin{array}{l}2x-y=2\\x^{2}+\dfrac{y^{2}}{2}=2x+y\end{array}\right.$;
 \item~$\left\{\begin{array}{l}2x-xy+8y=0\\\dfrac{x-y}{2}-\dfrac{x+y}{5}=0\end{array}\right.$.
 \end{enumeratea}
 \end{multicols}
\end{esercizio}

\begin{esercizio}[\Ast]
 \label{ese:6.10}
Risolvere i seguenti sistemi di secondo grado.
 \begin{enumeratea}
 \item~$\left\{\begin{array}{l}x+y=1\\x^2+y^2-2xy-2y-2=0\end{array}\right.$;
 \item~$\left\{\begin{array}{l}9x^2-12xy+4y^2-2x+6y=8\\x-2y=2\end{array}\right.$;
 \item~$\left\{\begin{array}{l}3x+y=4\\x^2-y^2=1\end{array}\right.$;
 \item~$\left\{\begin{array}{l}\frac 1 2(2y-x)(y+x)-(x+y)^2+\frac 3 2x(x+y+1)+2(y-1)=0\\\frac 2 3(x-3)^2+4\left(x-\frac 3 2\right)=2({xy}+1)\end{array}\right.$.
 \end{enumeratea}
\end{esercizio}

\begin{esercizio}[\Ast]
\label{ese:6.11}
Risolvere i seguenti sistemi di secondo grado.
\begin{multicols}{2}
 \begin{enumeratea}
 \item~$\left\{\begin{array}{l}2x+y=3\\x(x-y)=0\end{array}\right.$;
 \item~$\left\{\begin{array}{l}5-(y+2x)^{2}=-(y-3x)^{2}\\x+2y=3\end{array}\right.$;
 \item~$\left\{\begin{array}{l}2-x(x-1)+y(x+1)=0\\5+2(y-x-1)+2x=5\end{array}\right.$;
 \item~$\left\{\begin{array}{l}xy+4=(x-y)^{2}\\2x=y\end{array}\right.$.
 \end{enumeratea}
 \end{multicols}
\end{esercizio}

\begin{esercizio}[\Ast]
\label{ese:6.12}
Risolvere i seguenti sistemi di secondo grado.
\begin{multicols}{2}
 \begin{enumeratea}
 \item~$\left\{\begin{array}{l}15-8(2+x)^{2}=3(5+2y)^{2}\\2x-3=2y\end{array}\right.$;
 \item~$\left\{\begin{array}{l}x+y=-2\\xy=-15\end{array}\right.$;
 \item~$\left\{\begin{array}{l}x+y=14\\\dfrac{1}{x}+\dfrac{1}{y}=\dfrac{7}{24}\end{array}\right.$;
 \item~$\left\{\begin{array}{l}\dfrac{1}{x}+\dfrac{1}{y}=\dfrac{1}{5}\\xy=180\end{array}\right.$.
 \end{enumeratea}
 \end{multicols}
\end{esercizio}

\begin{esercizio}[\Ast]
\label{ese:6.13}
Risolvere i seguenti sistemi di secondo grado.
\begin{multicols}{2}
 \begin{enumeratea}
 \item~$\left\{\begin{array}{l}x^{2}+y^{2}=\dfrac{a^{2}+b^{2}}{9}\\x+y=\dfrac{a+b}{3}\end{array}\right.$;
 \item~$\left\{\begin{array}{l}xy(x+y)=70\\133-x^{3}=y^{3}\end{array}\right.$;
 \item~$\left\{\begin{array}{l}x^{2}+y^{2}=37-xy\\x=7-y\end{array}\right.$;
 \item~$\left\{\begin{array}{l}\dfrac{x}{y}=\dfrac{5}{2}-\dfrac{y}{x}\\x=\dfrac{3}{2}-y\end{array}\right.$.
 \end{enumeratea}
 \end{multicols}
\end{esercizio}

\begin{esercizio}[\Ast]
 \label{ese:6.14}
Risolvere i seguenti sistemi e discuterli rispetto al parametro.
\begin{multicols}{2}
 \begin{enumeratea}
 \item~$\left\{\begin{array}{l}x+y=3 \\x^2+y^2=k \end{array}\right.$;
 \item~$ \left\{\begin{array}{l}ky+2x=4\\xy=2\end{array}\right. $;
 \item~$ \left\{\begin{array}{l}y=kx-1 \\y^2-kx^2+1=0\end{array}\right. $;
 \item~$\left\{\begin{array}{l}y=kx-2k \\x^2-2y-x=2\end{array}\right.$.
 \end{enumeratea}
 \end{multicols}
\end{esercizio}

\begin{esercizio}[\Ast]
 \label{ese:6.15}
Risolvere i seguenti sistemi e discuterli rispetto al parametro.
\begin{multicols}{2}
 \begin{enumeratea}
 \item~$\left\{\begin{array}{l}y=x+k \\y=3x^2+2x\end{array}\right.$;
 \item~$ \left\{\begin{array}{l}y=-x+k \\x^2-y^2-1=0\end{array}\right. $;
 \item~$\left\{\begin{array}{l}y+x-k=0 \\{xy}+2{kx}-3{ky}-6k^2=0\end{array}\right.$;
 \item~$ \left\{\begin{array}{l}y-x+k=0 \\y-x^2+4x-3=0\end{array}\right. $.
 \end{enumeratea}
 \end{multicols}
\end{esercizio}

\subsection*{6.2 - Sistemi frazionari}

\begin{esercizio}[\Ast]
 \label{ese:6.16}
Trovare le soluzioni dei seguenti sistemi frazionari.
\begin{multicols}{2}
 \begin{enumeratea}{\spazielenx
 \item~$\left\{\begin{array}{l}x^2+y^2=4\\\dfrac{x+2y}{x-1}=2\end{array}\right.$;
 \item~$\left\{\begin{array}{l}\dfrac{x+2y}{x-y}=4\\x^2+y^2+3x-2y=1\end{array}\right.$;
 \item~$\left\{\begin{array}{l}\dfrac{2x+y}{x+2y}=3\\xy+3y=1\end{array}\right.$;
 \item~$\left\{\begin{array}{l}\dfrac{3x-2y} x=\dfrac{1-x}{y-1}\\2x-y=1\end{array}\right.$.}
 \end{enumeratea}
 \end{multicols}
\end{esercizio}

\begin{esercizio}[\Ast]
 \label{ese:6.17}
Trovare le soluzioni dei seguenti sistemi frazionari.
\begin{multicols}{2}
 \begin{enumeratea}{\spazielenx
 \item~$\left\{\begin{array}{l}x+y=-1\\\dfrac{3}{x}-\dfrac{2}{y}=-2\end{array}\right.$;
 \item~$\left\{\begin{array}{l}6xy=1\\\dfrac{3x+1}{2-2y}=2\end{array}\right.$;
 \item~$\left\{\begin{array}{l}\dfrac{2(y-x)}{xy}=5\\x-y=-\dfrac{5}{12}\end{array}\right.$;
 \item~$\left\{\begin{array}{l}2x-y=5\\\dfrac{x+y}{x-y}=\dfrac{x-y}{x+y}+\dfrac{48}{7}\end{array}\right.$.}
 \end{enumeratea}
 \end{multicols}
\end{esercizio}

\begin{esercizio}[\Ast]
 \label{ese:6.18}
Trovare le soluzioni dei seguenti sistemi frazionari.
\begin{multicols}{2}
 \begin{enumeratea}{\spazielenx
 \item~$\left\{\begin{array}{l}\dfrac{x+y}{x-2}=y+\dfrac 1 3\\y=2x+2\end{array}\right.$;
 \item~$\left\{\begin{array}{l}\dfrac{2x+1}{y-2}=\dfrac{y-1}{x+1}\\2x+2y=3\end{array}\right.$;
 \item~$\left\{\begin{array}{l}\dfrac{y-1}{x+y}=x\\x-y=0\end{array}\right.$;
 \item~$\left\{\begin{array}{l}\dfrac{x+1}{2y-1}=y\\2y-x=-4\end{array}\right.$.}
 \end{enumeratea}
 \end{multicols}
\end{esercizio}

\begin{esercizio}[\Ast]
 \label{ese:6.19}
Trovare le soluzioni dei seguenti sistemi frazionari.
\begin{multicols}{2}
 \begin{enumeratea}{\spazielenx
 \item~$\left\{\begin{array}{l}x+y=-1\\6\left(\dfrac{1}{x}+\dfrac{1}{y}\right)=x+y\end{array}\right.$;
 \item~$\left\{\begin{array}{l}2x+y=4\\\dfrac{2x+y+2}{x+y}=4-\dfrac{2x+y}{y}\end{array}\right.$;
 \item~$\left\{\begin{array}{l}x+y=4\\\dfrac{x-y}{x+y}=\dfrac{8}{3}+\dfrac{x+y}{x-y}\end{array}\right.$;
 \item~$\left\{\begin{array}{l}\dfrac{4x-6}{2x-2}+\dfrac{4y-6}{y-1}-2=0\\\dfrac{4y+9}{2}-4=3x\end{array}\right.$.}
 \end{enumeratea}
 \end{multicols}
\end{esercizio}

\subsection*{6.3 - Sistemi in più incognite}

\begin{esercizio}
 \label{ese:6.20}
Risolvere i seguenti sistemi di secondo grado in tre incognite.
\begin{multicols}{2}
 \begin{enumeratea}
 \item~$\left\{\begin{array}{l}x-3y-z=-4\\3x+2y+z=6\\4x^2+2xz+y^2=6\end{array}\right.$;
 \item~$\left\{\begin{array}{l}x+y=5\\2x-y+3z=9\\x^2-y+z^2=1\end{array}\right.$;
 \item~$\left\{\begin{array}{l}x-y+z=1\\2x-y+z=0\\x^2-y+z=3\end{array}\right.$;
 \item~$\left\{\begin{array}{l}x-y+2z=3\\2x-2y+z=1\\x^2-y^2+z=12\end{array}\right.$.
 \end{enumeratea}
 \end{multicols}
\end{esercizio}

\begin{esercizio}[\Ast]
 \label{ese:6.21}
Risolvere i seguenti sistemi di secondo grado in tre incognite.
\begin{multicols}{2}
 \begin{enumeratea}
 \item~$\left\{\begin{array}{l}2x-3y=-3\\5y+2z=1\\x^2+y^2+z^2=1\end{array}\right.$;
 \item~$\left\{\begin{array}{l}x-2y+z=3\\x+2y+z=3\\x^2+y^2+z^2=29\end{array}\right.$;
 \item~$\left\{\begin{array}{l}x+y-z=0\\x-y+3z=9\\x^2-y+z=12\end{array}\right.$;
 \item~$\left\{\begin{array}{l}x-y=1\\x+y+z=0\\x^2+xy-z=0\end{array}\right.$;
 \item~$\left\{\begin{array}{l}x-y-z=-1\\x+y+z=1\\x+y^2+z^2=32\end{array}\right.$.
 \end{enumeratea}
 \end{multicols}
\end{esercizio}

\subsection*{6.4 - Sistemi simmetrici}

\begin{esercizio}[\Ast]
 \label{ese:6.22}
Risolvere i seguenti sistemi simmetrici di secondo grado.
\begin{multicols}{2}
 \begin{enumeratea}
 \item~$\left\{\begin{array}{l}x+y=4\\{xy}=3\end{array}\right.$;
 \item~$\left\{\begin{array}{l}x+y=1\\{xy}=7 \end{array}\right.$;
 \item~$\left\{\begin{array}{l}x+y=5\\{xy}=6 \end{array}\right.$;
 \item~$\left\{\begin{array}{l}x+y=-5\\{xy}=-6 \end{array}\right.$.
 \end{enumeratea}
 \end{multicols}
\end{esercizio}
\pagebreak
\begin{esercizio}[\Ast]
 \label{ese:6.23}
Risolvere i seguenti sistemi simmetrici di secondo grado.
\begin{multicols}{2}
 \begin{enumeratea}
 \item~$\left\{\begin{array}{l}x+y=3\\{xy}=2 \end{array}\right.$;
 \item~$\left\{\begin{array}{l}x+y=3\\{xy}=-4\end{array}\right.$;
 \item~$\left\{\begin{array}{l}x+y=-4\\{xy}=4 \end{array}\right.$;
 \item~$\left\{\begin{array}{l}x+y=6\\{xy}=9 \end{array}\right.$.
 \end{enumeratea}
 \end{multicols}
\end{esercizio}

\begin{esercizio}[\Ast]
 \label{ese:6.24}
Risolvere i seguenti sistemi simmetrici di secondo grado.
\begin{multicols}{3}
 \begin{enumeratea}
 \item~$\left\{\begin{array}{l}x+y=2\\{xy}=10 \end{array}\right.$;
 \item~$\left\{\begin{array}{l}x+y=7\\{xy}=12 \end{array}\right.$;
 \item~$\left\{\begin{array}{l}x+y=-1\\{xy}=2 \end{array}\right.$;
 \item~$\left\{\begin{array}{l}x+y=12\\{xy}=-13 \end{array}\right.$;
 \item~$\left\{\begin{array}{l}x+y=\dfrac 6 5\\{xy}=\dfrac 9{25}\end{array}\right.$;
 \item~$\left\{\begin{array}{l}x+y=4\\{xy}=50 \end{array}\right.$.
 \end{enumeratea}
 \end{multicols}
\end{esercizio}

\begin{esercizio}[\Ast]
 \label{ese:6.25}
Risolvere i seguenti sistemi simmetrici di secondo grado.
\begin{multicols}{2}
 \begin{enumeratea}
 \item~$\left\{\begin{array}{l}x+y=-5\\{xy}=-14 \end{array}\right.$;
 \item~$\left\{\begin{array}{l}x+y=5\\{xy}=-14\end{array}\right.$;
 \item~$\left\{\begin{array}{l}x+y=\dfrac 1 4\\{xy}=-\dfrac 3 8\end{array}\right.$;
 \item~$\left\{\begin{array}{l}x+y=2\\{xy}=-10\end{array}\right.$.
 \end{enumeratea}
 \end{multicols}
\end{esercizio}

\begin{esercizio}[\Ast]
 \label{ese:6.26}
Risolvere i seguenti sistemi simmetrici di secondo grado.
\begin{multicols}{2}
 \begin{enumeratea}
 \item~$\left\{\begin{array}{l}x+y=4\\{xy}=0 \end{array}\right.$;
 \item~$\left\{\begin{array}{l}x+y=\dfrac 5 2\\{xy}=-\dfrac 7 2\end{array}\right.$;
 \item~$\left\{\begin{array}{l}x+y=-5\\{xy}=2 \end{array}\right.$;
 \item~$\left\{\begin{array}{l}x+y=\dfrac 4 3\\{xy}=-\dfrac 1 2 \end{array}\right.$.
 \end{enumeratea}
 \end{multicols}
\end{esercizio}

\begin{esercizio}[\Ast]
 \label{ese:6.27}
Risolvere i seguenti sistemi simmetrici di secondo grado.
\begin{multicols}{2}
 \begin{enumeratea}
 \item~$\left\{\begin{array}{l}x+y=\dfrac 5 2\\{xy}=-\dfrac 9 2\end{array}\right.$;
 \item~$\left\{\begin{array}{l}x+y=2\\{xy}=-\dfrac 1 3\end{array}\right.$;
 \item~$\left\{\begin{array}{l}x+y=1\\{xy}=-3\end{array}\right.$;
 \item~$\left\{\begin{array}{l}x+y=4\\{xy}=-50\end{array}\right.$.
 \end{enumeratea}
 \end{multicols}
\end{esercizio}

\begin{esercizio}[\Ast]
\label{ese:6.28}
Risolvere i seguenti sistemi riconducibili al sistema simmetrico fondamentale.
\begin{multicols}{2}
 \begin{enumeratea}
 \item~$\left\{\begin{array}{l}x+y=1 \\x^2+y^2=1\end{array}\right.$;
 \item~$\left\{\begin{array}{l}x+y=2 \\x^2+y^2=2\end{array}\right.$;
 \item~$\left\{\begin{array}{l}x+y=3 \\x^2+y^2=5\end{array}\right.$;
 \item~$\left\{\begin{array}{l}x+y=2 \\x^2+y^2+x+y=1\end{array}\right.$;
 \item~$\left\{\begin{array}{l}x+y=4\\x^2+y^2=8\end{array}\right.$;
 \item~$\left\{\begin{array}{l}x+y=2 \\x^2+y^2-3xy=4\end{array}\right.$.
 \end{enumeratea}
 \end{multicols}
\end{esercizio}

\begin{esercizio}[\Ast]
\label{ese:6.29}
Risolvere i seguenti sistemi riconducibili al sistema simmetrico fondamentale.
\begin{multicols}{2}
 \begin{enumeratea}
 \item~$\left\{\begin{array}{l}{x+y=-12}\\{x^2+y^2=72}\end{array}\right.$;
 \item~$\left\{\begin{array}{l}{2x+2y=-2}\\{(y-x)^2-{xy}=101}\end{array}\right.$;
 \item~$\left\{\begin{array}{l}{-4x-4y=-44}\\{2x^2+2y^2-3{xy}=74}\end{array}\right.$;
 \item~$\left\{\begin{array}{l}x+y=3 \\x^2+y^2-4x-4y=5\end{array}\right.$.
 \end{enumeratea}
 \end{multicols}
\end{esercizio}

\begin{esercizio}[\Ast]
 \label{ese:6.30}
Risolvere i seguenti sistemi riconducibili al sistema simmetrico fondamentale.
\begin{multicols}{2}
 \begin{enumeratea}
 \item~$\left\{\begin{array}{l}x+y=7 \\x^2+y^2=29\end{array}\right.$;
 \item~$\left\{\begin{array}{l}2x+2y=-2\\4x^2+4y^2=52\end{array}\right.$;
 \item~$\left\{\begin{array}{l}\dfrac{x+y} 2=\dfrac 3 4\\3x^2+3y^2=\dfrac{15} 4\end{array}\right.$;
 \item~$\left\{\begin{array}{l}x+y=-3 \\x^2+y^2-5xy=37\end{array}\right.$.
 \end{enumeratea}
 \end{multicols}
\end{esercizio}

\begin{esercizio}[\Ast]
\label{ese:6.31}
Risolvere i seguenti sistemi riconducibili al sistema simmetrico fondamentale.
\begin{multicols}{2}
 \begin{enumeratea}
 \item~$\left\{\begin{array}{l}x+y=-6 \\x^2+y^2-xy=84\end{array}\right.$;
 \item~$\left\{\begin{array}{l}x+y=-5 \\x^2+y^2-4xy+5x+5y=36\end{array}\right.$;
 \item~$\left\{\begin{array}{l}x+y=-7 \\x^2+y^2-6xy-3x-3y=44\end{array}\right.$;
 \item~$\left\{\begin{array}{l}x^2+y^2=-1\\x+y=6 \end{array}\right.$.
 \end{enumeratea}
 \end{multicols}
\end{esercizio}

\begin{esercizio}[\Ast]
\label{ese:6.32}
Risolvere i seguenti sistemi riconducibili al sistema simmetrico fondamentale.
\begin{multicols}{2}
 \begin{enumeratea}
 \item~$\left\{\begin{array}{l}x^2+y^2=1\\x+y=-7\end{array}\right.$;
 \item~$\left\{\begin{array}{l}x^2+y^2=18\\x+y=6 \end{array}\right.$;
 \item~$\left\{\begin{array}{l}x^2+y^2-4xy-6x-6y=1\\x+y=1 \end{array}\right.$;
 \item~$\left\{\begin{array}{l}x^2+y^2=8\\x+y=3\end{array}\right.$.
 \end{enumeratea}
 \end{multicols}
\end{esercizio}

\begin{esercizio}[\Ast]
\label{ese:6.33}
Risolvere i seguenti sistemi riconducibili a sistemi simmetrici.
\begin{multicols}{2}
 \begin{enumeratea}
 \item~$\left\{\begin{array}{l}{x-y=1}\\{x^2+y^2=5}\end{array}\right.$;
 \item~$\left\{\begin{array}{l}{\dfrac 1 x+\dfrac 1 y=-12}\\{{xy}=\dfrac 1{35}}\end{array}\right.$;
 \item~$\left\{\begin{array}{l}{-2x+y=3}\\{{xy}=1}\end{array}\right.$;
 \item~$\left\{\begin{array}{l}{4(x+y)=1}\\4xy=-3\end{array}\right.$.
 \end{enumeratea}
 \end{multicols}
\end{esercizio}

\begin{esercizio}[\Ast]
\label{ese:6.34}
Risolvere i seguenti sistemi riconducibili a sistemi simmetrici.
\begin{multicols}{2}
 \begin{enumeratea}
 \item~$\left\{\begin{array}{l}{4(x+y)=1}\\8xy=-1\end{array}\right.$;
 \item~$\left\{\begin{array}{l}x+y=-\dfrac{13}{10}\\xy=-3\end{array}\right.$;
 \item~$\left\{\begin{array}{l}x+y=4a\\xy=3a^{2}\end{array}\right.$;
 \item~$\left\{\begin{array}{l}x+y=2a+b\\xy=a^{2}+ab\end{array}\right.$.
 \end{enumeratea}
 \end{multicols}
\end{esercizio}

\begin{esercizio}[\Ast]
 \label{ese:6.35}
Risolvere i seguenti sistemi di grado superiore al secondo.
\begin{multicols}{2}
 \begin{enumeratea}
 \item~$\left\{\begin{array}{l}{x+y=-1}\\{x^3+y^3=-1}\end{array}\right.$;
 \item~$\left\{\begin{array}{l}{{xy}=-2}\\{x^2+y^2=13}\end{array}\right.$;
 \item~$\left\{\begin{array}{l}{x+y=-2}\\{x^3+y^3-{xy}=-5}\end{array}\right.$;
 \item~$\left\{\begin{array}{l}{x+y=8}\\{x^3+y^3=152}\end{array}\right.$.
 \end{enumeratea}
 \end{multicols}
\end{esercizio}

\begin{esercizio}[\Ast]
 \label{ese:6.36}
Risolvere i seguenti sistemi di grado superiore al secondo.
\begin{multicols}{2}
 \begin{enumeratea}
 \item~$\left\{\begin{array}{l}x^3+y^3=9\\x+y=3\end{array}\right.$;
 \item~$\left\{\begin{array}{l}x^3+y^3=-342\\x+y=-6\end{array}\right.$;
 \item~$\left\{\begin{array}{l}{x^3-y^3=351}\\{{xy}=-14}\end{array}\right.$;
 \item~$\left\{\begin{array}{l}x^3+y^3=35\\x+y=5\end{array}\right.$.
 \end{enumeratea}
 \end{multicols}
\end{esercizio}

\begin{esercizio}[\Ast]
 \label{ese:6.37}
Risolvere i seguenti sistemi di grado superiore al secondo.
\begin{multicols}{2}
 \begin{enumeratea}
 \item~$\left\{\begin{array}{l}x^4+y^4=2\\x+y=0\end{array}\right.$;
 \item~$\left\{\begin{array}{l}x^4+y^4=17\\x+y=-3\end{array}\right.$;
 \item~$\left\{\begin{array}{l}x^3+y^3=-35\\xy=6\end{array}\right.$;
 \item~$\left\{\begin{array}{l}x^3+y^3=-26\\xy=-3\end{array}\right.$.
 \end{enumeratea}
 \end{multicols}
\end{esercizio}

\begin{esercizio}[\Ast]
 \label{ese:6.38}
Risolvere i seguenti sistemi di grado superiore al secondo.
\begin{multicols}{2}
 \begin{enumeratea}
 \item~$\left\{\begin{array}{l}{x+y=3}\\{x^4+y^4=17}\end{array}\right.$;
 \item~$\left\{\begin{array}{l}{x+y=-1}\\{8x^4+8y^4=41}\end{array}\right.$;
 \item~$\left\{\begin{array}{l}{x+y=3}\\{x^4+y^4=2}\end{array}\right.$;
 \item~$\left\{\begin{array}{l}{x+y=5}\\{x^4+y^4=257}\end{array}\right.$.
 \end{enumeratea}
 \end{multicols}
\end{esercizio}

\begin{esercizio}[\Ast]
 \label{ese:6.39}
Risolvere i seguenti sistemi di grado superiore al secondo.
\begin{multicols}{2}
 \begin{enumeratea}
 \item~$\left\{\begin{array}{l}x^4+y^4=2\\xy=1\end{array}\right.$;
 \item~$\left\{\begin{array}{l}x^4+y^4=17\\xy=-2\end{array}\right.$;
 \item~$\left\{\begin{array}{l}{x+y=-1}\\{x^5+y^5=-211}\end{array}\right.$;
 \item~$\left\{\begin{array}{l}x^5+y^5=64\\x+y=4\end{array}\right.$.
 \end{enumeratea}
 \end{multicols}
\end{esercizio}

\begin{esercizio}[\Ast]
 \label{ese:6.40}
Risolvere i seguenti sistemi di grado superiore al secondo.
\begin{multicols}{2}
 \begin{enumeratea}
 \item~$\left\{\begin{array}{l}x^5+y^5=-\np{2882}\\x+y=-2\end{array}\right.$;
 \item~$\left\{\begin{array}{l}x^5+y^5=2\\x+y=0\end{array}\right.$;
 \item~$\left\{\begin{array}{l}x^5+y^5=31\\xy=-2\end{array}\right.$;
 \item~$\left\{\begin{array}{l}x^4+y^4=337\\xy=12\end{array}\right.$.
 \end{enumeratea}
 \end{multicols}
\end{esercizio}

\begin{esercizio}[\Ast]
 \label{ese:6.41}
Risolvere i seguenti sistemi di grado superiore al secondo.
\begin{multicols}{2}
 \begin{enumeratea}
 \item~$\left\{\begin{array}{l}x^3+y^3=\dfrac{511} 8\\xy=-2\end{array}\right.$;
 \item~$\left\{\begin{array}{l}x^2+y^2=5\\xy=2 \end{array}\right.$;
 \item~$\left\{\begin{array}{l}x^2+y^2=34\\xy=15 \end{array}\right.$;
 \item~$\left\{\begin{array}{l}xy=1 \\x^2+y^2+3xy=5\end{array}\right.$.
 \end{enumeratea}
 \end{multicols}
\end{esercizio}

\begin{esercizio}[\Ast]
 \label{ese:6.42}
Risolvere i seguenti sistemi di grado superiore al secondo.
\begin{multicols}{2}
 \begin{enumeratea}
 \item~$\left\{\begin{array}{l}xy=12 \\x^2+y^2=25\end{array}\right.$;
 \item~$\left\{\begin{array}{l}xy=1 \\x^2+y^2-4xy=-2\end{array}\right.$;
 \item~$\left\{\begin{array}{l}x^2+y^2=5\\xy=3 \end{array}\right.$;
 \item~$\left\{\begin{array}{l}x^2+y^2=18\\xy=9 \end{array}\right.$.
 \end{enumeratea}
 \end{multicols}
\end{esercizio}

\begin{esercizio}[\Ast]
 \label{ese:6.43}
Risolvere i seguenti sistemi di grado superiore al secondo.
\begin{multicols}{2}
 \begin{enumeratea}
 \item~$\left\{\begin{array}{l}x^2+y^2+3xy=10\\xy=6 \end{array}\right.$;
 \item~$\left\{\begin{array}{l}x^2+y^2+5xy-2x-2y=3\\xy=1 \end{array}\right.$;
 \item~$\left\{\begin{array}{l}x^2+y^2-6xy+3x+3y=2\\xy=2 \end{array}\right.$;
 \item~$\left\{\begin{array}{l}x^2+y^2=8\\xy=-3\end{array}\right.$.
 \end{enumeratea}
 \end{multicols}
\end{esercizio}

\begin{esercizio}[\Ast]
\label{ese:6.44}
Risolvere i seguenti sistemi di grado superiore al secondo.
\begin{multicols}{2}
 \begin{enumeratea}
 \item~$\left\{\begin{array}{l}x^2+y^2+5xy+x+y=-6\\xy=-2 \end{array}\right.$;
 \item~$\left\{\begin{array}{l}{x+y=-\dfrac 1 3}\\{x^5+y^5=-\dfrac{31}{243}}\end{array}\right.$;
 \item~$\left\{\begin{array}{l}x^2+y^2+5xy+x+y=-\dfrac{25} 4\\xy=-2 \end{array}\right.$;
 \item~$\left\{\begin{array}{l}{x+y=1}\\{x^5+y^5=-2}\end{array}\right.$.
 \end{enumeratea}
 \end{multicols}
\end{esercizio}

\begin{esercizio}[\Ast]
 \label{ese:6.45}
Risolvere i seguenti sistemi di grado superiore al secondo.
\begin{multicols}{2}
 \begin{enumeratea}
 \item~$\left\{\begin{array}{l}{x+y=1}\\{x^5+y^5+7{xy}=17}\end{array}\right.$;
 \item~$\left\{\begin{array}{l}9xy=2\\9\left(x^{2}+y^{2}\right)=5\end{array}\right.$;
 \item~$\left\{\begin{array}{l}xy=-6\\x^{2}+y^{2}=13\end{array}\right.$;
 \item~$\left\{\begin{array}{l}xy=6\\x^{2}+y^{2}=15\end{array}\right.$.
 \end{enumeratea}
 \end{multicols}
\end{esercizio}

\begin{esercizio}[\Ast]
 \label{ese:6.46}
Determina i punti di intersezione tra retta e parabola, interpreta graficamente le equazioni e le soluzioni del sistema.
\begin{multicols}{2}
 \begin{enumeratea}
 \item~$\left\{\begin{array}{l}y=x^2-4x\\2x-y-3=0 \end{array}\right.$;
 \item~$\left\{\begin{array}{l}x^2-y=1\\y=2x+1 \end{array}\right.$;
 \item~$\left\{\begin{array}{l}y=x-4\\x^{2}+y=4 \end{array}\right.$;
 \item~$\left\{\begin{array}{l}x-y=0\\y=x^{2}\end{array}\right.$.
 \end{enumeratea}
 \end{multicols}
\end{esercizio}

\begin{esercizio}[\Ast]
 \label{ese:6.47}
Determina i punti di intersezione tra retta e circonferenza, interpreta graficamente le equazioni e le soluzioni del sistema.
\begin{multicols}{2}
 \begin{enumeratea}
 \item~$\left\{\begin{array}{l}x^2+y^2=1\\x+y=0 \end{array}\right.$;
 \item~$\left\{\begin{array}{l}x^2+y^2-2x-8=0\\y=x+1 \end{array}\right.$;
 \item~$\left\{\begin{array}{l}x^2+y^2-3y-3=0\\y-2x=3 \end{array}\right.$;
 \item~$\left\{\begin{array}{l}x^2+y^2-2x-4y-4=0\\x+1=2y\end{array}\right.$.
 \end{enumeratea}
 \end{multicols}
\end{esercizio}
\pagebreak
\subsection*{6.5 - Sistemi omogenei di quarto grado}

\begin{esercizio}[\Ast]
 \label{ese:6.48}
Risolvi i seguenti sistemi omogenei.
\begin{multicols}{2}
 \begin{enumeratea}
 \item~$\left\{\begin{array}{l}x^2-2xy+y^2=0\\x^2+3xy-2y^2=0\end{array}\right.$;
 \item~$\left\{\begin{array}{l}3x^2-2xy-y^2=0\\2x^2+xy-3y^2=0\end{array}\right.$;
 \item~$\left\{\begin{array}{l}x^2-6{xy}+8y^2=0 \\x^2+4{xy}-5y^2=0 \end{array}\right.$;
 \item~$\left\{\begin{array}{l}2x^2+xy-y^2=0\\4x^2-2xy-6y^2=0\end{array}\right.$.
 \end{enumeratea}
\end{multicols}
\end{esercizio}

\begin{esercizio}[\Ast]
\label{ese:6.49}
Risolvi i seguenti sistemi omogenei.
\begin{multicols}{2}
 \begin{enumeratea}
 \item~$\left\{\begin{array}{l}x^2-5xy+6y^2=0\\x^2-4xy+4y^2=0\end{array}\right.$;
 \item~$\left\{\begin{array}{l}x^2-5xy+6y^2=0\\x^2+2xy-8y^2=0\end{array}\right.$;
 \item~$\left\{\begin{array}{l}x^2+xy-2y^2=0\\x^2+5xy+6y^2=0\end{array}\right.$;
 \item~$\left\{\begin{array}{l}x^2+7xy+12y^2=0\\2x^2+xy+6y^2=0\end{array}\right.$.
 \end{enumeratea}
\end{multicols}
\end{esercizio}

\begin{esercizio}[\Ast]
\label{ese:6.50}
Risolvi i seguenti sistemi omogenei.
\begin{multicols}{2}
 \begin{enumeratea}
 \item~$\left\{\begin{array}{l}x^2+6xy+8y^2=0\\2x^2+12xy+16y^2=0\end{array}\right.$;
 \item~$\left\{\begin{array}{l}-4x^2-7{xy}+2y^2=0 \\12x^2+21{xy}-6y^2=0 \end{array}\right.$;
 \item~$\left\{\begin{array}{l}x^2+2xy+y^2=0\\x^2+3xy+2y^2=0\end{array}\right.$;
 \item~$\left\{\begin{array}{l}x^2+4xy=0\\x^2+2xy-4y^2-4=0\end{array}\right.$.
 \end{enumeratea}
\end{multicols}
\end{esercizio}

\begin{esercizio}[\Ast]
\label{ese:6.51}
Risolvi i seguenti sistemi omogenei.
\begin{multicols}{2}
 \begin{enumeratea}
 \item~$\left\{\begin{array}{l}x^2-8xy+15y^2=0\\x^2-2xy+y^2=1\end{array}\right.$;
 \item~$\left\{\begin{array}{l}4x^2-y^2=0\\x^2-y^2=-3\end{array}\right.$;
 \item~$\left\{\begin{array}{l}x^2+3xy+2y^2=0\\x^2-3xy-y^2=3\end{array}\right.$;
 \item~$\left\{\begin{array}{l}x^2-4xy+4y^2=0\\2x^2-y^2=-1\end{array}\right.$.
 \end{enumeratea}
\end{multicols}
\end{esercizio}

\begin{esercizio}[\Ast]
\label{ese:6.52}
Risolvi i seguenti sistemi omogenei.
\begin{multicols}{2}
 \begin{enumeratea}
 \item~$\left\{\begin{array}{l}6x^2+5xy+y^2=12\\x^2+4xy+y^2=6\end{array}\right.$;
 \item~$\left\{\begin{array}{l}x^2-xy-2y^2=0\\x^2-4xy+y^2=6\end{array}\right.$;
 \item~$\left\{\begin{array}{l}x^2+y^2=3\\x^2-xy+y^2=3\end{array}\right.$;
 \item~$\left\{\begin{array}{l}x^2-3xy+5y^2=1\\x^2+xy+y^2=1\end{array}\right.$.
 \end{enumeratea}
\end{multicols}
\end{esercizio}

 \begin{esercizio}[\Ast]
\label{ese:6.53}
Risolvi i seguenti sistemi omogenei.
\begin{multicols}{2}
 \begin{enumeratea}
 \item~$\left\{\begin{array}{l}x^2+y^2=5\\x^2-3xy+y^2=11\end{array}\right.$;
 \item~$\left\{\begin{array}{l}x^2+5xy+4y^2=10\\x^2-2xy-3y^2=-11\end{array}\right.$;
 \item~$\left\{\begin{array}{l}4x^2-xy-y^2=-\dfrac 1 2\\x^2+2xy-y^2=\dfrac 1 4\end{array}\right.$;
 \item~$\left\{\begin{array}{l}x^2-xy-8y^2=-8\\x^2-2y^2-xy=16\end{array}\right.$.
 \end{enumeratea}
\end{multicols}
 \end{esercizio}
\pagebreak
\begin{esercizio}[\Ast]
\label{ese:6.54}
Risolvi i seguenti sistemi omogenei.
\begin{multicols}{2}
 \begin{enumeratea}
 \item~$\left\{\begin{array}{l}x^2-6xy-y^2=10\\x^2+xy=-2\end{array}\right.$;
 \item~$\left\{\begin{array}{l}4x^2-3xy+y^2=32\\x^2+3y^2-9xy=85\end{array}\right.$;
 \item~$\left\{\begin{array}{l}x^2+3xy+2y^2=8\\3x^2-y^2+xy=-4\end{array}\right.$;
 \item~$\left\{\begin{array}{l}x^2+5xy-7y^2=-121\\3xy-3x^2-y^2=-7\end{array}\right.$.
 \end{enumeratea}
\end{multicols}
\end{esercizio}

\begin{esercizio}[\Ast]
\label{ese:6.55}
Risolvi i seguenti sistemi particolari.
\begin{multicols}{2}
 \begin{enumeratea}
 \item~$\left\{\begin{array}{l}x^2-5xy-3y^2=27\\-2x^2-2y^2+4xy=-50\end{array}\right.$;
 \item~$\left\{\begin{array}{l}9x^2+5y^2=-3\\x^2+4xy-3y^2=8\end{array}\right.$;
 \item~$\left\{\begin{array}{l}2x^2-4xy-3y^2=18\\xy-2x^2+3y^2=-18\end{array}\right.$;
 \item~$\left\{\begin{array}{l}x^2+2xy=-\dfrac 7 4\\x^2-4xy+4y^2=\dfrac{81} 4\end{array}\right.$.
 \end{enumeratea}
\end{multicols}
\end{esercizio}

\begin{esercizio}[\Ast]
\label{ese:6.56}
Risolvi i seguenti sistemi particolari.
\begin{multicols}{2}
 \begin{enumeratea}
 \item~$\left\{\begin{array}{l}x^2+4xy+4y^2-16=0\\x^2-xy+4y^2-6=0\end{array}\right.$;
 \item~$\left\{\begin{array}{l}x^2-2xy+y^2-1=0\\x^2-2xy-y^2=1\end{array}\right.$.
 \end{enumeratea}
\end{multicols}
\end{esercizio}

\begin{esercizio}[\Ast]
\label{ese:6.57}
Risolvi i seguenti sistemi particolari.
\begin{multicols}{2}
 \begin{enumeratea}
 \item~$\left\{\begin{array}{l}x^2-y^2=0\\2x+y=3\end{array}\right.$;
 \item~$\left\{\begin{array}{l}(x-2y)(x+y-2)=0\\3x+6y=3\end{array}\right.$;
 \item~$\left\{\begin{array}{l}(x+y-1)(x-y+1)=0\\x-2y=1\end{array}\right.$;
 \item~$\left\{\begin{array}{l}(x-3y)(x+5y-2)=0\\(x-2)(x-y+4)=0\end{array}\right.$.
 \end{enumeratea}
\end{multicols}
\end{esercizio}

\begin{esercizio}[\Ast]
\label{ese:6.58}
Risolvi i seguenti sistemi particolari.
\begin{multicols}{2}
 \begin{enumeratea}
 \item~$\left\{\begin{array}{l}(x^2-3x+2)(x+y)=0\\x-y=2\end{array}\right.$;
 \item~$\left\{\begin{array}{l}(x-y)(x+y+1)(2x-y-1)=0\\(x-3y-3)(x+y-2)=0\end{array}\right.$;
 \item~$\left\{\begin{array}{l}(4x^2-9y^2)(x^2-2xy+y^2-9)=0 \\2x-y=2 \end{array}\right.$;
 \item~$\left\{\begin{array}{l}x^2+6xy+9y^2-4=0\\(x^2-y^2)(2x-y-4)=0\end{array}\right.$.
 \end{enumeratea}
\end{multicols}
\end{esercizio}

\begin{esercizio}[\Ast]
 \label{ese:6.59}
Risolvi i seguenti sistemi particolari.
 \begin{enumeratea}
 \item~$\left\{\begin{array}{l}x^2-2xy-8y^2=0\\(x+y)(x-3)=0\end{array}\right.$;
 \item~$\left\{\begin{array}{l}(2x^2-3xy+y^2)(x-y-1)=0 \\(x^2-4xy+3y^2)(12x^2-xy-y^2)=0 \end{array}\right.$;
 \item~$\left\{\begin{array}{l}(x-2y-2)(x^2-9y^2)=0 \\(4x^2-4xy+y^2)(y+2)(x-y)=0 \end{array}\right.$;
 \item~$\left\{\begin{array}{l}x^4-y^4=0 \\x^2-(y^2-6y+9)=0 \end{array}\right.$.
 \end{enumeratea}
\end{esercizio}

\begin{esercizio}[\Ast]
 \label{ese:6.60}
Risolvi i seguenti sistemi particolari.
 \begin{enumeratea}
 \item~$\left\{\begin{array}{l}(y^2-4y+3)(x^2+2x-15)=0 \\(x^2-3xy+2y^2)(9x^2-6xy+y^2)=0 \end{array}\right.$;
 \item~$\left\{\begin{array}{l}(x-y)(x+4y-4)(x+y-1)(3x-5y-2)=0 \\(3x+y-3)(x^2-4y^2)=0 \end{array}\right.$.
 \end{enumeratea}
\end{esercizio}

\subsection*{6.6 - Metodo di addizione}

\begin{esercizio}[\Ast]
 \label{ese:6.61}
Risolvi i seguenti sistemi particolari.
\begin{multicols}{2}
 \begin{enumeratea}
 \item~$\left\{\begin{array}{l}x^2+y^2-2x=0\\x^{2}+y^{2}-4x-2y+4=0\end{array}\right.$;
 \item~$\left\{\begin{array}{l}x^{2}+y^{2}=4 \\x^2+y^2-3y=6 \end{array}\right.$;
 \item~$\left\{\begin{array}{l}x^{2}-y^{2}-2x+3y=2 \\2x^2-2y^2+x+2y=3 \end{array}\right.$;
 \item~$\left\{\begin{array}{l}2x^3+2y^2-8x+3y=3 \\x^3-+y^2-3x+2y=1 \end{array}\right.$.
 \end{enumeratea}
\end{multicols}
\end{esercizio}

\subsection*{6.7 - Sostituzione delle variabili}

\begin{esercizio}[\Ast]
 \label{ese:6.62}
Risolvi i seguenti sistemi particolari.
\begin{multicols}{2}
 \begin{enumeratea}
 \item~$\left\{\begin{array}{l}x^2+y^3=3\\x^{4}-x^{2}y^{3}=20\end{array}\right.$;
 \item~$\left\{\begin{array}{l}x^{2}+y^{3}=10 \\2y^{3}-x^{2}y^{3}=-7 \end{array}\right.$;
 \item~$\left\{\begin{array}{l}x^{6}-3y^{4}+2x^{3}y^{4}=0 \\3x^{3}-2y^{4}=1 \end{array}\right.$;
 \item~$\left\{\begin{array}{l}-x^5+2y^2=-14 \\x^{10}-4x^{5}y^{2}=-128 \end{array}\right.$.
 \end{enumeratea}
\end{multicols}
\end{esercizio}

\subsection*{6.8 - Problemi che si risolvono con sistemi di grado superiore al primo}
\begin{multicols}{2}
\begin{esercizio}[\Ast]
 \label{ese:6.63}
La differenza tra due numeri è $\frac {11} 4$ e il loro prodotto $\frac {21} 8$. Trova i due numeri.
\end{esercizio}

\begin{esercizio}[\Ast]
 \label{ese:6.64}
Trovare due numeri positivi sapendo che la metà del primo supera di $ 1 $ il secondo e che il quadrato del secondo supera di $ 1 $ la sesta parte del quadrato del primo.
\end{esercizio}

\begin{esercizio}[\Ast]
 \label{ese:6.65}
Data una proporzione tra numeri naturali conosciamo i due medi che sono $ 5 $ e $ 16 $. Sappiamo anche che il rapporto tra il prodotto degli estremi e la loro somma è uguale a $\frac {10} 3 $. Trovare i due estremi.
\end{esercizio}

\begin{esercizio}[\Ast]
 \label{ese:6.66}
La differenza tra un numero di due cifre con quello che si ottiene scambiando le cifre è uguale a $ 36 $. La differenza tra il prodotto delle cifre e la loro somma è uguale a $ 11 $. Trovare il numero.
\end{esercizio}

\begin{esercizio}[\Ast]
 \label{ese:6.67}
Determina due numeri sapendo che il loro prodotto è~$72$ e la somma dei loro quadrati è~$180$.
\end{esercizio}

\begin{esercizio}[\Ast]
 \label{ese:6.68}
Determinare i due numeri sapendo che il loro prodotto è~$\frac{1}{2}$ e che la somma dei loro reciproci è~$2$.
\end{esercizio}

\begin{esercizio}[\Ast]
 \label{ese:6.69}
Trovare due numeri consecutivi sapendo che la somma dei loro quadrati diminuita del loro prodotto è uguale a~$43$.
\end{esercizio}

\begin{esercizio}[\Ast]
 \label{ese:6.70}
Trovare due numeri interi positivi sapendo che il loro rapporto è~$\frac{21}{10}$ e la somma dei loro quadrati è~$\frac{551}{36}$.
\end{esercizio}

\begin{esercizio}[\Ast]
 \label{ese:6.71}
Oggi la differenza delle età tra un padre e sua figlia è $ 26 $ anni, mentre due anni fa il prodotto delle loro età era $ 56 $. Determina l'età del padre e della figlia.
\end{esercizio}

\begin{esercizio}[\Ast]
 \label{ese:6.72}
Determinare l'età di un padre e di sua figlia sapendo che il padre aveva $30$~anni quando è nata la figlia e che moltiplicando tra loro le età che hanno attualmente si trova un prodotto uguale a tre volte il quadrato dell'età della figlia.
\end{esercizio}

\begin{esercizio}[\Ast]
 \label{ese:6.73}
Determinare l'età di due ragazzi sapendo che fra tre anni il prodotto delle loro età sarà i~$\frac{39}{4}$ della somma dell'età attuali e che due anni fa l'età del maggiore era doppia di quella del minore.
\end{esercizio}

\begin{esercizio}[\Ast]
 \label{ese:6.74}
La somma delle età di due fratelli oggi è $46$ anni, mentre fra due anni la somma dei quadrati delle loro età sarà $\np{1250}$. Trova l'età dei due fratelli.
\end{esercizio}

\begin{esercizio}[\Ast]
 \label{ese:6.75}
Due podisti partono contemporaneamente per un luogo distante~$90\unit{Km}$. Uno di essi percorrendo ogni ora $1\unit{Km}$ in più dell'altro, arriva un'ora prima. Calcolare la loro velocità.
\end{esercizio}

\begin{esercizio}[\Ast]
 \label{ese:6.76}
Nella produzione di un oggetto la macchina A impiega 5~minuti in più rispetto alla macchina~B. Determinare il numero di oggetti che produce ciascuna macchina in 8~ore se in questo periodo la macchina~A ha prodotto 16~oggetti in meno rispetto alla macchina~B.
\end{esercizio}

\begin{esercizio}[\Ast]
 \label{ese:6.77}
Un serbatoio d'acqua può essere riempito in~$\np{6750}$~secondi utilizzando contemporaneamente due rubinetti. Calcolare quali sarebbero i tempi necessari se si usassero i rubinetti singolarmente, sapendo che uno impiegherebbe due ore meno dell'altro.
\end{esercizio}

\begin{esercizio}[\Ast]
 \label{ese:6.78}
In un rettangolo la differenza tra i due lati è uguale a $2\unit{cm}$. Se si diminuiscono entrambi i lati di $ 1\unit{cm} $ si ottiene un'area di $\np[m^2]{0,1224}$. Calcolare il perimetro del rettangolo.
\end{esercizio}

\begin{esercizio}[\Ast]
 \label{ese:6.79}
Trova due numeri sapendo che la somma tra i loro quadrati è $ 100 $ e il loro rapporto $ \frac 3 4 $.
\end{esercizio}

\begin{esercizio}[\Ast]
 \label{ese:6.80}
Ho comprato due tipi di vino. In tutto 30 bottiglie. Per il primo tipo ho speso \officialeuro~$54$ e per il secondo \officialeuro~$36$. Il prezzo di una bottiglia del secondo tipo costa \officialeuro~$\np{2,5}$ in meno di una bottiglia del primo tipo. Trova il numero delle bottiglie di ciascun tipo che ho acquistato e il loro prezzo unitario.
\end{esercizio}

\begin{esercizio}[\Ast]
 \label{ese:6.81}
In un triangolo rettangolo di area $630\unit{m^2}$, l'ipotenusa misura $53\unit{m}$. Determinare il perimetro.
\end{esercizio}

\begin{esercizio}[\Ast]
 \label{ese:6.82}
Un segmento di $35\unit{cm}$ viene diviso in due parti. La somma dei quadrati costruiti su ciascuna delle due parti è $625\unit{{cm}^2}$. Quanto misura ciascuna parte?
\end{esercizio}

\begin{esercizio}[\Ast]
 \label{ese:6.83}
Determinare le misure dei lati di due quadrati sapendo che la somma delle loro aree è~$89 \unit{m^{2}}$ e che detti lati differisconi di~$3 \unit{m}$.
\end{esercizio}

\begin{esercizio}[\Ast]
 \label{ese:6.84}
Se in un rettangolo il perimetro misura $\np[m]{16,8}$ e l'area $ \np[m^2]{17,28}$, quanto misura la sua diagonale?
\end{esercizio}

\begin{esercizio}[\Ast]
 \label{ese:6.85}
Determinare le misure dei lati di un rettangolo di perimetro~$100 \unit{cm}$ ed equivalente a un quadrato il cui lato è doppio  dell'altezza del rettangolo.
\end{esercizio}

\begin{esercizio}[\Ast]
 \label{ese:6.86}
In un triangolo rettangolo la somma dei cateti misura $\np[cm]{10,5}$, mentre l'ipotenusa è $\np[cm]{7,5}$. Trovare l'area.
\end{esercizio}

\begin{esercizio}[\Ast]
 \label{ese:6.87}
Determinare le misure dei lati di un rettangolo di perimetro~$68 \unit{m}$, sapendo che i punti medi dei suoi lati sono i vertici di un rombo di lato~$13 \unit{m}$.
\end{esercizio}

\begin{esercizio}[\Ast]
 \label{ese:6.88}
Quanto misura un segmento diviso in due parti, tali che una parte è $ \frac 3 4 $ dell'altra, sapendo che la somma dei quadrati costruiti su ognuna delle due parti è uguale a $121\unit{{cm}^2}$?
\end{esercizio}

\begin{esercizio}[\Ast]
 \label{ese:6.89}
Calcolare le diagonali di un rombo di area~$96 \unit{cm^{2}}$, circoscritto a un cerchio di raggio di~$\np{4,8} \unit{cm}$.
\end{esercizio}

\begin{esercizio}[\Ast]
 \label{ese:6.90}
In un trapezio rettangolo con area di $81\unit{m^2}$ la somma della base minore e dell'altezza è $12\unit{m}$ mentre la base minore è $\frac 1 5$ della base maggiore. Trovare il perimetro del rettangolo.
\end{esercizio}

\begin{esercizio}[\Ast]
 \label{ese:6.91}
La differenza tra le diagonali di un rombo è $8\unit{cm}$, mentre la sua area è $24\unit{{cm}^2}$. Determinare il lato del rombo.
\end{esercizio}

\begin{esercizio}[\Ast]
 \label{ese:6.92}
Sappiamo che in un trapezio rettangolo con area di $40\unit{{cm}^2}$ la base minore è $7\unit{cm}$, mentre la somma della base maggiore e dell'altezza è $17\unit{cm}$. Trovare il perimetro del trapezio.
\end{esercizio}

\begin{esercizio}[\Ast]
 \label{ese:6.93}
Determinare la misura delle basi di un trapezio isoscele di area~$15 \unit{m^{2}}$ e altezza~$2 \unit{m}$ sapendo che la differenza tra i quadrati della diagonale e del lato obliquo è~$50 \unit{m^{2}}$.
\end{esercizio}

\begin{esercizio}[\Ast]
 \label{ese:6.94}
Un rettangolo ha l'area uguale a quella di un quadrato. L'altezza del rettangolo è $16\unit{cm}$, mentre la sua base è di $5\unit{cm}$ maggiore del lato del quadrato. Determinare il lato del quadrato.
\end{esercizio}

\begin{esercizio}[\Ast]
 \label{ese:6.95}
Determinare la misura dei raggi di due cerchi, tangenti tra loro esternamente, sapendo che la somma delle due superfici è~$149 \pi \unit{m^{2}}$ e che la distanza dei due centri è~$17 \unit{m}$.
\end{esercizio}

\begin{esercizio}[\Ast]
 \label{ese:6.96}
Determinare base e altezza di un triangolo isoscele sapendo che deve essere inscritto in un cerchio di raggio~$1 \unit{m}$ e che la somma dei valori da ricercare è~$\np{3,2} \unit{m}$.
\end{esercizio}

\begin{esercizio}[\Ast]
 \label{ese:6.97}
La differenza tra i cateti di un triangolo rettangolo è $7k$, mentre la sua area è $60 k^2$. Calcola il perimetro. ($k>0$)
\end{esercizio}

\begin{esercizio}[\Ast]
 \label{ese:6.98}
L'area di un rettangolo che ha come lati le diagonali di due quadrati misura $90 k^2$. La somma dei lati dei due quadrati misura $14k$. Determinare i lati dei due quadrati. ($ k>0 $)
\end{esercizio}

\begin{esercizio}[\Ast]
 \label{ese:6.99}
Nel rettangolo ABCD la differenza tra altezza e base è $4k$. Se prolunghiamo la base AB dalla parte di B di $2k$ fissiamo il punto E e congiungiamo B con E. Trovare il perimetro del trapezio AECD sapendo che la sua area è $28k^2$ con $k>0$.
\end{esercizio}

\begin{esercizio}[\Ast]
 \label{ese:6.100}
In un triangolo isoscele la base è $ \frac 2 3 $ dell'altezza e l'area è $12k^2$. Trova il perimetro del triangolo.
\end{esercizio}
\end{multicols}

\subsection{Risposte}
\paragraph{6.1.} a)~$\left(1;1\right)\vee \left(\frac 5 3;\frac 1 3\right)$,\quad b)~$\left(-1;-1\right)\vee \left(\frac 1 2;-\frac 1 4\right)$,\quad c)~$\left(1;1\right)\vee \left(-1;-1\right)$,\quad d)~$\left(0;-\frac 2 5\right)\vee \left(-\frac 1 4;-\frac 1 4\right)$.

\paragraph{6.2.} a)~$\left(-1;-1\right)\vee \left(\frac{7}{5};\frac{1}{5}\right)$,\quad b)~$\left(0;2\right)\vee (-8;8)$,\quad c)~ $\left(5;0\right)$,\quad d)~$\left(-1;2\right)\vee (0;-1)$.

\paragraph{6.3.} a)~$\left(3;0\right)\vee \left(-6;-9\right)$,\quad b)~$\left(0;0\right)$,\quad c)~ $\left(-\frac{29} 5;\frac{12} 5\right)\vee (3;-2)$,\quad d)~$\left(-\frac{46}{27};-\frac{20}{27}\right)\vee (2;0)$.

\paragraph{6.4.} a)~$(-4;-10)\vee (8;2)$,\quad b)~$(1;-1)\vee (3;3)$,\quad c)~$(3;2)\vee (-3;-4)$,\quad d)~$\left(\frac{1}{3};-1\right)\vee \left(\frac{2}{3};-\frac{1}{2}\right)$.

\paragraph{6.5.} a)~$\left(\frac 1 2;-\frac 1 2\right)$,\quad b)~$\left(3;1\right)\vee \left(5;3\right)$,\quad c)~$\emptyset $,\quad d)~$\left(1;1\right)$.

\paragraph{6.6.} a)~$\left(0;1\right)\vee \left(\frac 7 2;-\frac 5 2\right)$,\quad b)~$\emptyset $,\quad c)~$\left(-\frac 3 2;-\frac 5 2\right)\vee \left(\frac 1 2;-\frac 1 2\right)$,\quad d)~$\left(-4;-3\right)\vee \left(\frac{44}{25};\frac{117}{25}\right)$.

\paragraph{6.7.} a)~$\left(1;1\right)\vee \left(\frac{10} 7;\frac{11}{14}\right)$,\quad b)~$\left(3;1\right)\vee \left(5;-3\right)$,\quad c)~$ \forall (x,y)\in \insR \times \insR:\,y=-2x+3$,\quad d)~ $\left(1;-3\right)\vee \left(-8;-\frac{15} 2\right)$.

\paragraph{6.8.} a)~ $\left(-2;2\right)\vee \left(\frac 5 2;-\frac 5 2\right)$,\quad b)~$\left(4;4\right)\vee \left(-5;-2\right)$,\quad c)~$\left(4;2\right)\vee \left(\frac 4{15};-\frac 2{15}\right)$,\quad d)~$\left(1+\frac{2\sqrt{10}} 5;\frac{\sqrt{10}} 5\right)\vee \left(1-\frac{2\sqrt{10}} 5;-\frac{\sqrt{10}} 5\right)$.

\paragraph{6.9.} a)~ $(-34;-10)\vee (2;2)$,\quad b)~$(-7;-5)\vee (4;6)$,\quad c)~$(2;2)\vee \left(\frac 2{3};-\frac 2{3}\right)$,\quad d)~$(0;0)\vee (16;4)$.

\paragraph{6.10.} a)~$\left(\frac{1+\sqrt{13}} 4;\frac{3-\sqrt{13}} 4\right)\vee \left(\frac{1-\sqrt{13}} 4;\frac{3+\sqrt{13}} 4\right)$,\protect\\
b)~$\left(\frac{-9+\sqrt{241}} 8;\frac{-25+\sqrt{241}}{16}\right)\vee \left(\frac{9-\sqrt{241}} 8;\frac{-25-\sqrt{241}}{16}\right)$,\quad
c)~$\left(\frac{6-\sqrt 2} 4;\frac{-2+3\sqrt 2} 4\right)\vee \left(\frac{6+\sqrt 2} 4;\frac{-2-3\sqrt 2} 4\right)$,\protect\\ d)~$\left(\frac{6-8\sqrt 3}{13};\frac{17+12\sqrt 3}{26}\right)\vee\left(\frac{6+ 8\sqrt 3}{13};\frac{17-12\sqrt 3}{26}\right)$.

\paragraph{6.11.} a)~ $(1;1)\vee (3;0)$,\quad b)~$(1;1)\vee \left(\frac{1}{2};\frac{5}{4}\right)$,\quad c)~$(1;-1)\vee (3;1)$,\quad d)~$\emptyset$.

\paragraph{6.12.} a)~ $\left(1;\frac{5}{2}\right)\vee \left(-\frac{1}{2};-2\right)$,\quad b)~$(3;-5)\vee (-5;3)$,\quad c)~$(8;6)\vee (6;8)$,\quad d)~$(6;30)\vee (30;6)$.

\paragraph{6.13.} a)~ $\left(\frac{a}{3};\frac{b}{3}\right)\vee \left(\frac{b}{3};\frac{a}{3}\right)$,\quad b)~$(2;5)\vee (5;2)$,\quad c)~$(3;4)\vee (4;3)$,\quad d)~$\left(1;\frac{1}{2}\right)\vee \left(\frac{1}{2};1\right)$.

\paragraph{6.14.} a)~$k\ge \frac 9 2.\, \left(\frac{3-\sqrt{2k-9}} 2; \frac{3+\sqrt{2k-9}} 2\right)\vee \left(\frac{3+\sqrt{2k-9}} 2; \frac{3-\sqrt{2k-9}} 2\right)$,\protect\\
\quad d)~$\forall k\in \insR:\, (2; 0)\vee (2k-1; 2k^2-3k)$.

\paragraph{6.15.} a)~$k\ge -\frac 1{12}:\, \left(\frac{-1-\sqrt{12k+1}} 6; \frac{6k-1-\sqrt{12k+1}} 6\right) \vee \left(\frac{-1+\sqrt{12k+1}} 6; \frac{6k-1+\sqrt{12k+1}} 6\right)$,\protect\\
\quad d)~$\forall k\in \insR:\, (3k; -2k)$.

\paragraph{6.16.} a)~$x\neq 1:\, \left(2;0\right)\vee \left(-\frac 6 5;-\frac 8 5\right)$,\quad b)~$x\neq y:\, \left(\frac 2 5;\frac 1 5\right)\vee \left(-2;-1\right)$,\quad c)~$x\neq -2y:\, \emptyset $,\protect\\
\quad d)~$x\neq 0\wedge y\neq 1:\, (4;7)$.

\paragraph{6.17.} a)~$\left(-3;2\right)\vee \left(-\frac 1 2;-\frac 1 2\right)$,\quad b)~$\left(\frac 2 3;\frac 1 4\right)\vee \left(\frac{1}{3};\frac{1}{2}\right)$,\quad c)~$\left(\frac{1}{4};\frac{2}{3}\right)\vee \left(-\frac{2}{3}; -\frac{1}{4}\right)$,\quad d)~$(4;3)\vee \left(\frac{3}{2};-2\right)$.

\paragraph{6.18.} a)~$x\neq 2:\, \left(-1;0\right)\vee \left(\frac{10} 3;\frac{26} 3\right)$,\quad b)~$x\neq -1\wedge y\neq 2:\, \left(-\frac 5 2;4\right)$,\quad c)~$x\neq -y:\, \emptyset $,\protect\\
\quad d)~$y\neq \frac 1 2:\, \left(2;-1\right)\vee \left(9;\frac 5 2\right)$.

\paragraph{6.19.} a)~$\left(3;-2\right)\vee (-2;3)$,\quad b)~$(1;2)\vee (3;-2)$,\quad c)~$(8;4)\vee \left(\frac{4}{3};-\frac{8}{3}\right)$,\quad d)~$\left(\frac{3} 2;2\right)\vee\left(\frac{11}{12};\frac{9}{8}\right)$.

\paragraph{6.20.} a)~$\left(1;2;-1\right)$,\quad b)~$\emptyset $,\quad c)~$\forall z \in \insR (-1;z-2;z)$,\quad d)~$\left(-\frac{47} 3;-\frac{46} 3;\frac 5 3\right)$.

\paragraph{6.21.} a)~$\emptyset $,\quad b)~$\left(5;0;-2\right)\vee \left(-2;0;5\right)$,\quad c)~$\left(-4;\frac{25} 2;\frac{17} 2\right)\vee \left(3;-\frac 3 2;\frac 3 2\right)$,\quad d)~$\left(-1;-2;3\right)\vee \left(\frac 1 2;-\frac 1 2;0\right)$,\quad e)~$\left(0;\frac{3\sqrt 7+1} 2;-\frac{3\sqrt 7-1} 2\right)\vee \left(0;-\frac{3\sqrt 7-1} 2;\frac{3\sqrt 7+1} 2\right)$.

\paragraph{6.22.} a)~$(3;1)\vee(1;3)$,\quad b)~$\emptyset $,\quad c)~$(3;2)\vee(2;3)$,\quad d)~$(1;-6)\vee(-6;1)$.

\paragraph{6.23.} a)~$(2;1)\vee(1;2)$,\quad b)~$(4;-1)\vee(-1;4)$,\quad c)~$(-2;-2)$,\quad d)~$(3;3)$.

\paragraph{6.24.} a)~$\emptyset $,\quad b)~$(4;3)\vee(3;4)$,\quad c)~$\emptyset $,\quad d)~$(13;-1)\vee(-1;13)$,\quad e)~$\left(\frac 3 5;\frac 3 5\right)$,\quad f)~$\emptyset $.

\paragraph{6.25.} a)~$(2;-7)\vee(-7;2)$,\quad b)~$(7;-2)\vee(-2;7)$,\quad c)~$\left(\frac 3 4;-\frac 1 2\right)\vee\left(-\frac 1 2;\frac 3 4\right)$,\protect\\
\quad d)~$\left(1+\sqrt{11};1-\sqrt{11}\right)\vee\left(1-\sqrt{11};1+\sqrt{11}\right)$.

\paragraph{6.26.} a)~$(0;4)\vee(4;0)$,\; b)~$\left(\frac 7 2;-1\right)\vee\left(-1;\frac 7 2\right)$,\; c)~$\left(\frac{-5+\sqrt{17}} 2;\frac{-5-\sqrt{17}} 2\right)\vee \left(\frac{-5-\sqrt{17}} 2\\y=\frac{-5+\sqrt{17}} 2\right)$,\quad d)~$\left(\frac{4+\sqrt{34}} 6;\frac{4-\sqrt{34}} 6\right)\vee \left(\frac{4-\sqrt{34}} 6\\y=\frac{4+\sqrt{34}} 6\right)$.

\paragraph{6.27.} a)~$\left(\frac{5+\sqrt{97}} 4\\y=\frac{5-\sqrt{97}} 4\right)\vee \left(\frac{5-\sqrt{97}} 4\\y=\frac{5+\sqrt{97}} 4\right)$,\quad b)~$\left(\frac{3+{2\sqrt 3}} 3;\frac {3-{2\sqrt 3}} 3\right)\vee \left(\frac {3-{2\sqrt 3}} 3;\frac{3+{2\sqrt 3}} 3\right)$,\quad c)~$\left(\frac{1+\sqrt{13}} 2;\frac{1-\sqrt{13}} 2\right)\vee \left(\frac{1-\sqrt{13}} 2;\frac{1+\sqrt{13}} 2\right)$,\quad d)~$\left(2+3\sqrt 6;2-3\sqrt 6\right)\vee \left(2-3\sqrt 6;2+3\sqrt 6\right)$.

\paragraph{6.28.} a)~$(1;0)\vee(0;1)$,\quad b)~$(1;1)$,\quad c)~$(1;2)\vee(2;1)$,\quad d)~$\emptyset$,\quad e)~$(2;2)$,\quad f)~$(0;2)\vee(2;0)$.

\paragraph{6.29.} a)~$(-6;-6)$,\quad b)~$(-5;4)\vee(4;-5)$,\quad c)~$(3;8)\vee(8;3)$,\quad d)~$(-1;4)\vee(4;-1)$.

\paragraph{6.30.} a)~$(2;5)\vee(5;2)$,\quad b)~$(-3;2)\vee(2;-3)$,\quad c)~$(\frac 1 2;1)\vee(1;\frac 1 2)$,\quad d)~$(-4;1)\vee(1;-4)$.

\paragraph{6.31.} a)~$(-8;2)\vee(2;-8)$,\quad b)~$(-6;1)\vee(1;-6)$,\quad c)~$\left(-\frac 1 2;-\frac{13} 2\right)\vee \left(-\frac{13} 2;-\frac 1 2\right)$,\quad d)~$\emptyset $.

\paragraph{6.32.} a)~$\emptyset $,\; b)~$(3;3)$,\; c)~$\left(\frac{1+\sqrt 5} 2;\frac{1-\sqrt 5} 2\right)\vee \left(\frac{1-\sqrt 5} 2;\frac{1+\sqrt 5} 2\right)$,\; d)~$\left(\frac{3+\sqrt 7} 2;\frac{3-\sqrt 7} 2\right)\vee \left(\frac{3-\sqrt 7} 2;\frac{3+\sqrt 7} 2\right)$.

\paragraph{6.33.} a)~$ (-1; -2)\vee(2;1) $,\quad b)~$\left(-\frac 1 7;-\frac 1 5\right)\vee\left(-\frac 1 5;-\frac 1 7\right)$,\protect\\ c)~$\left(\frac{-3-\sqrt{17}} 4;{\frac{3-\sqrt{17}} 2}\right)\vee \left(\frac{-3+\sqrt{17}} 4;{\frac{3+\sqrt{17}} 2}\right)$,\quad d)~$\left(1;-\frac 3 4\right)\vee\left(-\frac 3 4;1\right)$.

\paragraph{6.34.} a)~$\left(\frac 1 2;-\frac 1 4\right)\vee\left(-\frac 1 4;\frac 1 2\right)$,\quad b)~$\left(-\frac 5 2;\frac 6 5\right)\vee\left(\frac 6 5;-\frac 5 2\right)$,\quad c)~$(a;3a)\vee (3a;a)$,\protect\\ d)~$(m+n;m)\vee(m;m+n)$.

\paragraph{6.35.} a)~$(-1;0)\vee(0;-1)$,\quad b)~$\left(\frac{-3-\sqrt{17}} 2;\frac{-3+\sqrt{17}} 2\right)\vee\left(\frac{3+\sqrt{17}} 2;\frac{3-\sqrt{17}} 2\right)$,\protect\\
\quad c)~$\left(\frac{-5-\sqrt{10}} 5;\frac{-5+\sqrt{10}} 5\right)\vee \left(\frac{-5+\sqrt{10}} 5;\frac{-5-\sqrt{10}} 5\right)$,\quad d)~$(3;5)\vee(5;3)$.

\paragraph{6.36.} a)~$(1;2)\vee(2;1)$,\quad b)~$(-7;1)\vee(1;-7)$,\quad c)~$(2;-7)\vee(7;-2)$,\quad d)~$(2;3)\vee(3;2)$.

\paragraph{6.37.} a)~$(-1;1)\vee(1;-1)$,\;b)~$(-2;-1)\vee(-1;-2)$,\; c)~$(-3;-2)\vee(-2;-3)$,\; d)~$(-3;1)\vee(1;-3)$.

\paragraph{6.38.} a)~$(1;2)\vee(2;1)$,\quad b)~$\left(-\frac 3 2;\frac 1 2\right)\vee\left(\frac 1 2;-\frac 3 2\right)$,\quad c)~$\emptyset$,\quad d)~$(1;4)\vee(4;1)$.

\paragraph{6.39.} a)~$(1;1)$,\quad b)~$(1;-2)\vee(-2;1)\vee(-1;2)\vee(2;-1)$,\quad c)~$(-3;2)\vee(2;-3)$,\quad d)~$(2;2)$.

\paragraph{6.40.} a)~$(-5;3)\vee(3;-5)$,\quad b)~$\emptyset$,\quad c)~$(-1;2)\vee(2;-1)$,\quad d)~$(-4;-3)\vee(-3;-4)\vee(3;4)\vee(4;3)$.

\paragraph{6.41.} a)~$\left(-\frac 1 2;4\right)\vee\left(4;-\frac 1 2\right)$,\quad b)~$(-2;-1)\vee(-1;-2)\vee(1;2)\vee(2;1)$,\protect\\ c)~$(-5;-3)\vee(-3;-5)\vee(3;5)\vee(5;3)$,\quad d)~$(-1;-1)\vee(1;1)$.

\paragraph{6.42.} a)~$(-4;-3)\vee(-3;-4)\vee(3;4)\vee(4;3)$,\quad b)~$(-1;-1)\vee(1;1)$,\quad c)~$\emptyset$,\quad d)~$(-3;-3)\vee(3;3)$.

\paragraph{6.43.} a)~$\emptyset$,\quad b)~$(1;1)$,\quad c)~$(-3-\sqrt 7;-3+\sqrt 7)\vee(-3+\sqrt 7;-3-\sqrt 7)\vee(1;2)\vee(2;1)$,\protect\\
\quad d)~$\left(\frac{\sqrt{14}+\sqrt 2} 2;\frac{\sqrt 2-\sqrt{14}} 2\right)\vee\left(\frac{\sqrt{14}-\sqrt 2} 2;\frac{\sqrt 2+\sqrt{14}} 2\right)\vee\left(\frac{\sqrt{14}-\sqrt 2} 2;-\frac{\sqrt 2+\sqrt{14}} 2\right)\vee\left(\frac{\sqrt{14}+\sqrt 2} 2;-\frac{\sqrt 2-\sqrt{14}} 2\right)$.

\paragraph{6.44.} a)~$(-2;1)\vee(1;-2)\vee(-\sqrt {2};\sqrt{2})\vee(\sqrt {2};-\sqrt{2})$,\quad b)~$\left(-\frac 2 3;\frac 1 3\right)\vee\left(\frac 1 3;-\frac 2 3\right)$,\protect\\
\quad c)~$\left(\frac{-1+\sqrt{33}} 4;\frac{-1-\sqrt{33}} 4\right)\vee \left(\frac{-1-\sqrt{33}} 4;\frac{-1+\sqrt{33}} 4\right)$,\quad d)~$\emptyset$.

\paragraph{6.45.} a)~$(-1;2)\vee(2;-1)$,\quad b)~$\left(\frac{1}{3};\frac{2}{3}\right)\vee\left(\frac{2}{3};\frac{1}{3}\right)\vee\left(-\frac{2}{3};-\frac{1}{3}\right)\vee \left(-\frac{1}{3};-\frac{2}{3}\right)$,\quad 
c)~$(3;-2)\vee (-3;2)\vee (2;-3)\vee (-2;3)$,\quad 
d)~$\left(2\sqrt{3};\sqrt{3}\right)\vee\left(\sqrt{3};2\sqrt{3}\right)\vee\left(-\sqrt{3};-2\sqrt{3}\right)\vee \left(-2\sqrt{3};-\sqrt{3}\right)$.

\paragraph{6.46.} a)~$\left(3+\sqrt{6};3+2\sqrt{6}\right)\vee\left(3-\sqrt{6};3-2\sqrt{6}\right)$,\quad b)~$\left(1+\sqrt{3};3+2\sqrt{3}\right)\vee\left(1-\sqrt{3};3-2\sqrt{3}\right)$,\quad c)~$\left(\frac{-1+\sqrt{33}}{2};\frac{-9+\sqrt{33}}{2}\right)\vee\left(\frac{-1-\sqrt{33}}{2};\frac{-9-\sqrt{33}}{2}\right)$,
\quad d)~$(0;0)\vee(1;1)$.

\paragraph{6.47.} a)~$\left(\frac{\sqrt{2}}{2};-\frac{\sqrt{2}}{2}\right)\vee\left(-\frac{\sqrt{2}}{2};\frac{\sqrt{2}}{2}\right)$,\quad b)~$\left(\frac{\sqrt{14}}{2};\frac{\sqrt{14}}{2}+1\right)\vee\left(-\frac{\sqrt{14}}{2};1-\frac{\sqrt{14}}{2}\right)$,\protect\\ c)~$\left(\frac{1+\sqrt{26}}{5};\frac{7+2\sqrt{26}}{5}\right)\vee\left(\frac{1-\sqrt{26}}{5};\frac{7-2\sqrt{26}}{5}\right)$,\quad d)~$\left(\frac{7+2\sqrt{41}}{5};\frac{6+\sqrt{41}}{5}\right)\vee\left(\frac{7-2\sqrt{41}}{5};\frac{6-\sqrt{41}}{5}\right)$.

\paragraph{6.48.} a)~$(0;0)$,\quad b)~$(t;t)$,\quad c)~$(0;0)$,\quad d)~$(t;-t)$.

\paragraph{6.49.} a)~$(2t;t)$,\quad b)~$(2t;t)$,\quad c)~$(-2t;t)$,\quad d)~$(0;0)$.

\paragraph{6.50.} a)~$(-4t;t)\vee(-2t;t)$,\quad b)~$(k;4k)\vee(k;-\frac{1 2}k)$,\quad c)~$(-t;t)$,\quad d)~$(-4;1)\vee(4;-1)$.

\paragraph{6.51.} a)~$\left(-\frac 3 2;-\frac 1 2\right)\vee\left(\frac 3 2;\frac 1 2\right)\vee\left(-\frac 5 4;-\frac 1 4\right)\vee\left(\frac 5 4;\frac 1 4\right)$,\quad b)~$(1;2)\vee(-1;-2)\vee (-1;2)\vee (1;-2)$,\protect\\
\quad c)~$(-1;1)\vee (1;-1)\left(-\frac{2\sqrt 3} 3;\frac{\sqrt 3} 3\right)\vee\left(\frac{2\sqrt 3} 3;-\frac{\sqrt 3} 3\right)$,\quad d)~$\emptyset$.

\paragraph{6.52.} a)~$(1;1)\vee(-1;-1)\vee\left(\sqrt 6;-4\sqrt 6\right)\vee\left(-\sqrt 6;4\sqrt 6\right)$,\quad b)~$(1;-1)\vee(-1;1)$,\protect\\
\quad c)~$(\sqrt 3;0)\vee (-\sqrt 3;0)\vee (0;\sqrt 3)\vee (0;-\sqrt 3)$,\quad d)~$(1;0)\vee (-1;0)\vee\left(\frac{\sqrt 3} 3;\frac{\sqrt 3} 3\right)\vee\left(-\frac{\sqrt 3} 3;-\frac{\sqrt 3} 3\right)$.

\paragraph{6.53.} a)~$(1;-2)\vee(-1;2)\vee (-2;1)\vee (2;-1)$,\quad b)~$(2;-3)\vee(-2;3)$,\protect\\
\quad c)~$\left(\frac 1 2;1\right)\vee \left(-\frac 1 2;-1\right)$,\quad d)~$(4;-2)\vee(-4;2)\vee (6;2)\vee (-6;-2)$.

\paragraph{6.54.} a)~$(-1;3)\vee(1;-3)$,\quad b)~$(1;-4)\vee(-1;4)\vee (-1;-7)\vee (1;7)$,\protect\\
\quad c)~$(0;2)\vee (0;-2)\vee \left(\frac{10} 3;-\frac{14} 3\right)\vee \left(-\frac{10} 3;\frac{14} 3\right)$,\quad d)~$(2;5)\vee (-2;-5)\vee \left(-\frac{18} 7;-\frac{37} 7\right)\vee \left(\frac{18} 7;\frac{37} 7\right)$.

\paragraph{6.55.} a)~$(3;-2)\vee(-3;2)\vee\left(\frac{34} 7;-\frac 1 7\right)\vee \left(-\frac{34} 7;\frac 1 7\right)$,\quad b)~$\emptyset$,\quad c)~$(-3;0)\vee(3;0)$,\protect\\
\quad d)~$\left(\frac 1 2;-2\right)\vee\left(-\frac 1 2;2\right)\vee\left(\frac 7 4;-\frac{11} 8\right)\vee\left(-\frac 7 4;\frac{11} 8\right)$.

\paragraph{6.56.} a)~$(-2;-1)\vee(2;1)$,\quad b)~$(-1;0)\vee(1;0)$.

\paragraph{6.57.} a)~$(1;1)\vee(3;-3)$,\quad b)~$(3;-1)\vee\left(\frac 1 2;\frac 1 4\right)$,\quad c)~$(1;0)\vee(-3;-2)$,\protect\\
\quad d)~$(2;0)\vee\left(2;\frac 2 3\right)\vee(-6;-2)\vee(-3;1)$.

\paragraph{6.58.} a)~$(1;-1)_\text{doppia} \vee(2;0)$,\quad b)~$(0;-1)_\text{doppia}\vee\left(-\frac 3 2;-\frac 3 2\right)\vee(1;1)_\text{doppia}$,\quad c)~$(5;8)\vee\left(\frac 3 2;1\right)\vee(-1;-4)\vee\left(\frac 3 4;-\frac 1 2\right)$,\quad d)~$(1;-1)\vee(2;0)\vee(-1;1)\vee\left(\frac 1 2;\frac 1 2\right)\vee\left(-\frac 1 2;-\frac 1 2\right)\vee\left(\frac{10} 7;-\frac 8 7\right)$.

\paragraph{6.59.}a)~$(0;0)_\text{doppia}\vee\left(3;-\frac 3 2\right)\vee(3;\frac 3 4)$,\quad b)~$(t;t)\vee\left(\frac 3 2;\frac 1 2\right)\vee\left(\frac 1 5;-\frac 4 5\right)\vee\left(-\frac 1 2;-\frac 3 2\right)$,\protect\\
\quad c)~$(0;0)_\text{tripla}\vee(-2;-2)_\text{doppia}\vee\left(-\frac 2 3;-\frac 4 3\right)_\text{doppia}\vee(6;-2)\vee(-6;-2)$,\quad d)~$\left(\frac 3 2;\frac 3 2\right)\vee\left(-\frac 3 2;\frac 3 2\right)$.

\paragraph{6.60.} a)~$(1;1)\vee(2;1)\vee(3;3)_\text{doppia}\vee(6;3)\vee\left(\frac 1 3;1\right)\vee(1;3)\vee(-5;-5)\vee\left(-5;-\frac 5 2\right)\vee\left(3;\frac 3 2\right)\vee(-5;-15)\vee(3;9)$,\quad b)~$(0;0)_\text{doppia}\vee\left(\frac 3 4;\frac 3 4\right)\vee(1;0)\vee\left(\frac 8{11};\frac 9{11}\right)\vee\left(\frac{17}{18};\frac 1 6\right)\vee\left(\frac 4 3;\frac 2 3\right)\vee\protect\\
(-4;2)(2;-1)\vee\left(\frac 2 3;\frac 1 3\right)\vee\left(\frac 4{11};-\frac 2{11}\right)\vee(4;2)$.

\paragraph{6.61.} a)~$(1;1)\vee(2;0)$,\quad b)~$\left(-\frac{8}{5};-\frac{6}{5}\right)\vee (2;0)$,\quad c)~$\emptyset$,\quad
d)~$(1;3)$.

\paragraph{6.62.} a)~$(2;-1)\vee(-2;-1)$,\quad b)~$(-3;1)\vee(3;1)\vee\left(-\sqrt{3};\sqrt[3]{7}\right)\vee\left(\sqrt{3};\sqrt[3]{7}\right)$,\protect\\ c)~$(1;1)\vee(1;-1)\vee\left(\frac{\sqrt{3}}{2};\frac{1}{2}\right)\vee\left(\frac{\sqrt{3}}{2};-\frac{1}{2}\right)$,
\quad d)~$(2;3)\vee(2;-3)$.

\begin{multicols}{2}

\paragraph{6.63.} $\left(-\frac 3 4;-\frac 7 2\right)\vee \left(\frac 7 2;\frac 3 4\right)$.

\paragraph{6.64.} $(12;5)$.

\paragraph{6.65.} $(4;20)\vee (20;4)$.

\paragraph{6.66.} $73$.

\paragraph{6.67.} $(6;12)$.

\paragraph{6.68.} $\left(\frac{2}{3};\frac{3}{4}\right)$.

\paragraph{6.69.} $(7;6)\vee (-7;-6)$.

\paragraph{6.70.} $\left(\frac{7}{2};\frac{5}{3}\right)$.

\paragraph{6.71.} $(30;4)$.

\paragraph{6.72.} $(45;15)$.

\paragraph{6.73.} $(10;18)$.

\paragraph{6.74.} $(23;23)$.

\paragraph{6.75.} $(10 \unit{Km};9 \unit{Km})$.

\paragraph{6.76.} $(32;48)$.

\paragraph{6.77.} $(3~\unit{ore};5~\unit{ore})$.

\paragraph{6.78.} $2p=144\unit{cm}$.

\paragraph{6.79.} $(-6;-8)\vee (6;8)$.

\paragraph{6.80.} $(12;18)$.

\paragraph{6.81.} $2p=126\unit{m}$.

\paragraph{6.82.} $(15\unit{cm};20\unit{cm})$.

\paragraph{6.83.} $(5\unit{m};8\unit{m})$.

\paragraph{6.84.} $\text{Diagonale }=6\unit{m}$.

\paragraph{6.85.} $(10\unit{cm};40\unit{cm})$.

\paragraph{6.86.} $\Area=\np[{cm}^2]{13,5}$.

\paragraph{6.87.} $(10\unit{m};24\unit{m})$.

\paragraph{6.88.} $\np[cm]{15,4}$.

\paragraph{6.89.} $(12\unit{cm};16\unit{cm})$.

\paragraph{6.90.} $2p_1=42\vee 2p_2=57+3\sqrt{145}$.

\paragraph{6.91.} $2\sqrt{10}\unit{cm}$.

\paragraph{6.92.} $2p=24+2\sqrt{13}$.

\paragraph{6.93.} $(5\unit{m};10\unit{m})$.

\paragraph{6.94.} $20\unit{cm}$.

\paragraph{6.95.} $(7\unit{m};10\unit{m})$.

\paragraph{6.96.} $(\np[m]{1,92};\np[m]{1,28})\vee(\np[m]{1,6};\np[m]{1,6})$.

\paragraph{6.97.} $2p=40k$.

\paragraph{6.98.} $l_1=5k\vee l_2=9k$.

\paragraph{6.99.} $2p=15+k\sqrt{53}$.

\paragraph{6.100.} $2p=4k(1+\sqrt{10})$.
\end{multicols}
